\documentclass[10pt,a5paper]{article}
\usepackage[margin=.75cm,top=.75cm]{geometry}
\usepackage[utf8]{inputenc}
\usepackage[IL2]{fontenc}
\usepackage[czech]{babel}
\usepackage{microtype}
\usepackage{amssymb}
\usepackage{amsthm}
\usepackage{amsmath}
\usepackage{xcolor}
\usepackage{graphicx}
\usepackage{wasysym}
\usepackage{multicol}
\usepackage[inline]{enumitem}

\newcommand{\hint}[1]{{\color{gray}\footnotesize\noindent(Nápověda: #1)}}

\setlist[enumerate]{label={(\alph*)},topsep=\smallskipamount,itemsep=\smallskipamount,parsep=0pt,itemjoin={\quad}}
\setlist[itemize]{topsep=\smallskipamount,noitemsep}

\def\tisk{%
\newbox\shipouthackbox
\pdfpagewidth=2\pdfpagewidth
\let\oldshipout=\shipout
\def\shipout{\afterassignment\zdvojtmp \setbox\shipouthackbox=}%
\def\zdvojtmp{\aftergroup\zdvoj}%
\def\zdvoj{%
    \oldshipout\vbox{\hbox{%
        \copy\shipouthackbox
        \hskip\dimexpr .5\pdfpagewidth-\wd\shipouthackbox\relax
        \box\shipouthackbox
    }}%
}}%

\let\results\newpage
\let\endresults\relax

\def\resultssame{%
    \long\def\results##1\endresults{%
        \vfill
        \noindent\rotatebox{180}{\vbox{##1}}%
    }%
}


\newtheorem*{poz}{Pozorování}

\theoremstyle{definition}
\newtheorem{uloha}{\atr Úloha}
\newtheorem{suloha}[uloha]{\llap{$\star$ }Úloha}
\newtheorem*{bonus}{Bonus}
\newtheorem*{defn}{Definice}

\pagestyle{empty}

\let\ee\expandafter

\def\vysld{}
\let\printvysl\relax
\let\printalphvysl\relax

\makeatletter
\long\def\vyslplain#1{\ee\ee\ee\gdef\ee\ee\ee\vysld\ee\ee\ee{\ee\vysld\ee\printvysl\ee{\the\c@uloha}{#1}}}
\let\vysl\vyslplain

\def\locvysl#1{\ee\gdef\ee\locvysld\ee{\locvysld\item #1}}
\let\lv\locvysl

\newenvironment{ulohav}[1][]{\begin{uloha}[#1]\gdef\locvysld{\begin{enumerate*}}}{\ee\vyslplain\ee{\locvysld\end{enumerate*}}\end{uloha}}
\def\stitem{\@noitemargtrue\@item[$\star$ \@itemlabel]}

\makeatother

\def\atr{}
\def\basic{\def\atr{\llap{\mdseries$\sun$ }\gdef\atr{}}}
\def\interest{\def\atr{\llap{$\star$ }\gdef\atr{}}}
\def\iinterest{\def\atr{\llap{$\star\star$ }\gdef\atr{}}}

\newcommand{\cimage}[2][]{$\vcenter{\hbox{\includegraphics[#1]{#2}}}$}
\newcommand{\pimage}[2][]{\par\vglue\medskipamount\nobreak\hbox to\hsize{\hfil\includegraphics[#1]{kruznice_img/#2.pdf}\hfil}\vfil}

\begin{document}


% \tisk
% \resultssame

\section*{Kružnice}
% všechny obrázky jsou nakresleny v Ipe
% https://ipe.otfried.org/

\def\lc#1{\lv{$#1$}}

\begin{ulohav}
Určete poloměr zvýrazněné kružnice, pokud platí
\vskip-1.5\bigskipamount
\vbox{}
\begin{multicols}{3}
\raggedright
\begin{enumerate}
    \item strana čtverce = 1\pimage{kruznice_v_rohu}\lc{3-2\sqrt2}
    \item strana čtverce = 1\pimage{kruznice_okolo_ctverce}\lc{\sqrt2 + 1}
    \item strana čtverce = 1\pimage{ctverec_5}\lc{\frac12 (\sqrt2 - 1)}
    \item poloměr velké kružnice = 1\pimage{kruznice_3}\lc{2\sqrt3 - 3}
    \item poloměr velké kružnice = 1\pimage{kruznice_pul}\lc{\frac23}
    \item strana rovnostr. trojúhelníka = 1\pimage{rovnostr}\lc{\frac14(\sqrt3 - 1)}
    \item poloměr velké kružnice = 1\pimage{spolecna_tecna}\lc{\frac14}
    \item odvěsna pravoúh. trojúhelníka = 1\pimage{pravouh_2}\lc{\frac1{14}(4 - \sqrt2)}
    \item odvěsna pravoúh. trojúhelníka = 1\pimage{pravouh_3}\lc{\frac12(\sqrt2 - 1)}
\end{enumerate}
\end{multicols}
\noindent \textcolor{red}{Může se hodit:} $\operatorname{tg} 22{,}5^\circ = \sqrt2 - 1$.
\end{ulohav}


\vskip-1.5\bigskipamount
\vbox{}
\begin{multicols}{2}
\begin{uloha}
Bod $X$ je libovolný vnitřní bod průměru půlkružnice.
\begin{enumerate}
    \item Dokažte, že šrafovaná oblast má stejný obsah jako šedý kruh.
    \item Dokažte, že zvýrazněné vepsané kružnice jsou stejně velké (pokud existují).
\end{enumerate}
\end{uloha}
\[\includegraphics{kruznice_img/final.pdf}\]
\end{multicols}


\baselineskip=1.25\baselineskip
\setlist[enumerate]{label=\textbf{(\alph*)},itemjoin={\quad}}
\setlength{\columnseprule}{.4pt}

\results
\parindent=0pt
\parskip=\smallskipamount
\rightskip=0pt plus1fil\relax
\def\printvysl#1#2{\textbf{#1.} #2\par}
\def\printalphvysl#1#2#3{\textbf{#1}(#2)\ #3\par}
\vysld
\endresults

\end{document}