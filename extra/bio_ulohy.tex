\documentclass[9pt,a6paper,landscape]{extarticle}
\usepackage[margin=.5cm]{geometry}
\usepackage[utf8]{inputenc}
\usepackage[IL2]{fontenc}
\usepackage[czech]{babel}
\usepackage{microtype}
\usepackage{amssymb}
\usepackage{amsthm}
\usepackage{amsmath}
\usepackage{xcolor}
\usepackage{graphicx}
\usepackage{wasysym}

\usepackage[inline]{enumitem}

\newcommand{\hint}[1]{{\color{gray}\footnotesize\noindent(Nápověda: #1)}}

\setlist[enumerate]{label={(\alph*)},topsep=\smallskipamount,itemsep=\smallskipamount,parsep=0pt,itemjoin={\quad}}
\setlist[itemize]{topsep=\smallskipamount,noitemsep}

\def\tisk{%
\newbox\shipouthackbox
\pdfpagewidth=2\pdfpagewidth
\let\oldshipout=\shipout
\def\shipout{\afterassignment\zdvojtmp \setbox\shipouthackbox=}%
\def\zdvojtmp{\aftergroup\zdvoj}%
\def\zdvoj{%
    \oldshipout\vbox{\hbox{%
        \copy\shipouthackbox
        \hskip\dimexpr .5\pdfpagewidth-\wd\shipouthackbox\relax
        \box\shipouthackbox
    }}%
}}%
\def\tiskctyr{%
\newbox\shipouthackbox
\pdfpagewidth=2\pdfpagewidth
\pdfpageheight=2\pdfpageheight
\let\oldshipout=\shipout
\def\shipout{\afterassignment\zctyrtmp \setbox\shipouthackbox=}%
\def\zctyrtmp{\aftergroup\zctyr}%
\def\zctyr{%
    \offinterlineskip
    \oldshipout\vbox{\hbox{%
        \copy\shipouthackbox
        \hskip\dimexpr .5\pdfpagewidth-\wd\shipouthackbox\relax
        \copy\shipouthackbox
    }%
    \vskip\dimexpr .5\pdfpageheight-\ht\shipouthackbox\relax
    \hbox{%
        \copy\shipouthackbox
        \hskip\dimexpr .5\pdfpagewidth-\wd\shipouthackbox\relax
        \box\shipouthackbox
    }}%
}}


\newtheorem*{poz}{Pozorování}

\theoremstyle{definition}
\newtheorem{uloha}{\atr Úloha}
\newtheorem{suloha}[uloha]{\llap{$\star$ }Úloha}
\newtheorem*{bonus}{Bonus}
\newtheorem*{defn}{Definice}

\pagestyle{empty}

\let\ee\expandafter

\def\vysld{}
\let\printvysl\relax
\let\printalphvysl\relax

\makeatletter
\long\def\vyslplain#1{\ee\ee\ee\gdef\ee\ee\ee\vysld\ee\ee\ee{\ee\vysld\ee\printvysl\ee{\the\c@uloha}{#1}}}
\let\vysl\vyslplain

\def\locvysl#1{\ee\gdef\ee\locvysld\ee{\locvysld\item #1}}
\let\lv\locvysl

\newenvironment{ulohav}[1][]{\begin{uloha}[#1]\gdef\locvysld{\begin{enumerate*}}}{\ee\vyslplain\ee{\locvysld\end{enumerate*}}\end{uloha}}
\def\stitem{\@noitemargtrue\@item[$\star$ \@itemlabel]}

\makeatother

\def\atr{}
\def\basic{\def\atr{\llap{\mdseries$\sun$ }\gdef\atr{}}}
\def\interest{\def\atr{\llap{$\star$ }\gdef\atr{}}}
\def\iinterest{\def\atr{\llap{$\star\star$ }\gdef\atr{}}}

\let\results\newpage
\let\endresults\relax

\def\resultssame{%
    \long\def\results##1\endresults{%
        \vfill\noindent\rotatebox{180}{\vbox{##1}}%
    }%
}

\begin{document}

% \resultssame
% \tiskctyr

\section*{Úlohy ze života}


Ve všech úlohách se předpokládá, že poměr výskytů genotypů AA:Aa:aa v~populaci je 1:2:1. Dále se předpokládá, že hnědé oči jsou dominantní oproti modrým.

\begin{ulohav}
Jaká je pravděpodobnost, že
\begin{enumerate}
    \item modrooké dítě budou mít dva recesivní rodiče?\lv{1}
    \item modrooké dítě budou mít heterozygotní a recesivní rodič?\lv{$\frac12$}
    \item modrooké dítě budou mít dva heterozygotní rodiče?\lv{$\frac14$}
    \item modrooké dítě budou mít dva hnědoocí rodiče?\lv{$\frac19$}
    \item hnědooké dítě budou mít dva hnědoocí rodiče?\lv{$\frac89$}
\end{enumerate}
\end{ulohav}

\begin{ulohav}
Jaké je pravděpodobnost, že
\begin{enumerate}
    \item rodiče modrookého dítěte jsou oba modroocí?\lv{$\frac14$}
    \item rodiče modrookého dítěte jsou oba hnědoocí?\lv{$\frac14$}
    \item rodiče modrookého dítěte mají různé barvy očí?\lv{$\frac12$}
    \item rodiče hnědookého dítěte jsou oba hnědoocí?\lv{$\frac23$}
\end{enumerate}
\end{ulohav}

\interest
\begin{ulohav}
Jaká je pravděpodobnost, že 
\begin{enumerate}
    \item sourozenec modrookého dítěte bude opět modrooký?\lv{$\frac{9}{16}$}
    \item sourozenec modrookého dítěte bude hnědooký?\lv{$\frac{7}{16}$}
\end{enumerate}
\end{ulohav}


\results
\parindent=0pt
\parskip=\smallskipamount
\rightskip=0pt plus1fil\relax
\def\printvysl#1#2{\textbf{#1.} #2\qquad}
\def\printalphvysl#1#2#3{\textbf{#1}(#2)\ #3\par}
\vysld
\endresults


\end{document}