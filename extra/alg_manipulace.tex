\documentclass[10pt,a5paper]{article}
\usepackage[margin=.7cm]{geometry}
\usepackage[utf8]{inputenc}
\usepackage[IL2]{fontenc}
\usepackage[czech]{babel}
\usepackage{microtype}
\usepackage{amssymb}
\usepackage{amsthm}
\usepackage{amsmath}
\usepackage{xcolor}
\usepackage{graphicx}
\usepackage{wasysym}
\usepackage{multicol}

\usepackage[inline]{enumitem}

\newcommand{\R}{\mathbb{R}}

\newcommand{\hint}[1]{{\color{gray}\footnotesize\noindent(Nápověda: #1)}}

\setlist[enumerate]{label={\textbf{\arabic*.}},topsep=\smallskipamount,itemsep=1.2\medskipamount,parsep=0pt,itemjoin={\quad}}
\setlist[itemize]{topsep=\smallskipamount,noitemsep}

\def\tisk{%
\newbox\shipouthackbox
\pdfpagewidth=2\pdfpagewidth
\let\oldshipout=\shipout
\def\shipout{\afterassignment\zdvojtmp \setbox\shipouthackbox=}%
\def\zdvojtmp{\aftergroup\zdvoj}%
\def\zdvoj{%
    \oldshipout\vbox{\hbox{%
        \copy\shipouthackbox
        \hskip\dimexpr .5\pdfpagewidth-\wd\shipouthackbox\relax
        \box\shipouthackbox
    }}%
}}%



\newtheorem*{poz}{Pozorování}

\theoremstyle{definition}
\newtheorem{uloha}{\atr Úloha}
\newtheorem{suloha}[uloha]{\llap{$\star$ }Úloha}
\newtheorem*{bonus}{Bonus}
\newtheorem*{defn}{Definice}

\pagestyle{empty}

\let\ee\expandafter

\def\vysld{}
\let\printvysl\relax
\let\printalphvysl\relax

\makeatletter
\long\def\vysl#1{\ee\ee\ee\gdef\ee\ee\ee\vysld\ee\ee\ee{\ee\vysld\ee\printvysl\ee{\the\c@uloha}{#1}}}
\let\vyslplain\vysl

\def\locvysl#1{\ee\gdef\ee\locvysld\ee{\locvysld\item #1}}
\let\lv\locvysl

\long\def\itvysl#1{\ee\gdef\ee\vysld\ee{\vysld\printvysl{#1}}}
\def\pr#1#2{\item $#1$\itvysl{#2}}
\def\prs#1#2{\pr{\left\{\begin{aligned}#1\end{aligned}\right.}{#2}}

\newenvironment{ulohav}[1][]{\begin{uloha}[#1]\gdef\locvysld{\begin{enumerate*}}}{\ee\vyslplain\ee{\locvysld\end{enumerate*}}\end{uloha}}

\def\stitem{\@noitemargtrue\@item[$\star$ \@itemlabel]}

\makeatother

\def\atr{}
\def\basic{\def\atr{\llap{\mdseries$\sun$ }\gdef\atr{}}}
\def\interest{\def\atr{\llap{$\star$ }\gdef\atr{}}}
\let\mb\mathbf

\def\R{\mathbb{R}}
\def\N{\mathbb{N}}

\let\results\newpage
\let\endresults\relax

\def\resultssame{%
    \long\def\results##1\endresults{%
        \vfill\noindent\rotatebox{180}{\vbox{##1}}%
    }%
}

\begin{document}

% \resultssame
% \tisk

\section*{Algebraické manipulace pro pokročilé}


\begin{uloha}
Nalezněte všechna řešení soustav a dokažte, že žádná další neexistují.
\begin{multicols}{3}
\everymath{\displaystyle}
\begin{enumerate}
    \rightskip0pt plus1fil\relax
    \prs{x^2 + 1 &= 2y\\ y^2 + 1 &= 2x}{$\{[1;1]\}$}
    \prs{x^2 - 3y + 4 &= z\\y^2 - 3z + 4 &= x\\z^2 - 3x + 4 &= y}{$\{[2;2;2]\}$}
    \prs{x^2 - 3y + 3 &= z\\y^2 - 3z + 4 &= x\\z^2 - 3x + 5 &= y}{$\emptyset$}
    \prs{2x^2 + 2xy + 1 &= 4z\\2y^2 + 2yz + 1 &= 4x\\2z^2 + 2zx + 1 &= 4y}{$\bigl\{\bigl[\frac12;\frac12;\frac12\bigr]\bigr\}$}
    \prs{x^2 + y + z = 2\\y^2 + z + x = 2\\z^2 + x + y = 2}{$\bigl\{[-1-\sqrt3;-1-\sqrt3;-1-\sqrt3];\penalty0 [-1+\sqrt3;-1+\sqrt3;-1+\sqrt3];\penalty0 [1;1;0];\penalty0 [1;0;1]; [0;1;1]\bigr\}$}
    \prs{x^2 + y^2 + z = 2\\y^2 + z^2 + x = 2\\z^2 + x^2 + y = 2}{$\bigl\{\bigl[\frac14(-1-\sqrt{17});\frac14(-1-\sqrt{17});\frac14(-1-\sqrt{17})\bigr];\penalty0 \bigl[\frac14(-1+\sqrt{17});\frac14(-1+\sqrt{17});\frac14(-1+\sqrt{17})\bigr];\penalty0 [1;1;0];\penalty0 [1;0;1];\penalty0 [0;1;1];\penalty0 \bigl[-\frac12; -\frac12; \frac32\bigr];\penalty0 \bigl[-\frac12; \frac32; -\frac12\bigr];\penalty0 \bigl[\frac32; -\frac12; -\frac12\bigr]\bigr\}$}
    %
\end{enumerate}
\end{multicols}
\end{uloha}

\begin{uloha}
Dokažte, že pro všechna reálná čísla $x$, $y$ platí $x^2 + y^2 \geq 2xy$ a rozhodněte, ve kterých případech nastane rovnost.
\end{uloha}

\begin{uloha}
Dokažte, že pro všechna \emph{kladná} reálná čísla $x$, $y$ platí $\frac xy + \frac yx \geq 2$  a rozhodněte, ve kterých případech nastane rovnost.
\end{uloha}

\begin{uloha}
Dokažte, že pro všechna reálná čísla $x$, $y$, $z$ platí $x^2 + y^2 + z^2 \geq xy + yz + zx$ a rozhodněte, ve kterých případech nastane rovnost.
\end{uloha}

\begin{uloha}[66--C--III--4]
Dokažte, že pro všechna kladná reálná čísla $a \leq b \leq c$ platí
\[ (-a+b+c)\left( \frac1a + \frac1b + \frac1c \right) \geq 3 \]
\end{uloha}

\begin{uloha}[61--C--III--1]
Pro libovolná reálná čísla $x$, $y$, $z$ splňující $x<y<z$ dokažte nerovnost
\[ x^2 - y^2 + z^2 > (x-y+z)^2 \]
\end{uloha}

\begin{uloha}[63--C--III--3]
Pro kladná reálná čísla $a$, $b$, $c$ platí $c^2 + ab = a^2 + b^2$. Ukažte, že pak také platí $c^2 + ab \leq ac + bc$.
\end{uloha}

\begin{uloha}[63--C--II--1]
Určete, jakých hodnot může nabývat výraz $V = ab + bc + cd + da$, splňují-li reálná čísla $a$, $b$, $c$, $d$ dvojici podmínek
\begin{align*}
2a - 5b + 2c - 5d &= 4,\\
3a + 4b + 3c + 4d &= 6.
\end{align*}
\end{uloha}

\begin{uloha}[60--C--III--4]
Nechť $x$, $y$, $z$ jsou kladná reálná čísla. Dokažte, že čísla \[x+y+z-xyz\quad \text{a}  \quad xy + yz + zx - 3\] nemohou být současně záporná.
\end{uloha}


\begin{uloha}[65--C--III--1]
Najděte nejmenší možnou hodnotu výrazu $3x^2 - 12xy + y^4$, ve kterém $x$ a $y$ jsou libovolná nezáporná \emph{celá} čísla.
\end{uloha}



\results
\footnotesize
\subsubsection*{Výsledky soustav rovnic}
\def\printvysl#1{\item #1}
\rightskip=0pt plus1fill\relax
\begin{enumerate*}
    \vysld
\end{enumerate*}
\endresults


\end{document}

