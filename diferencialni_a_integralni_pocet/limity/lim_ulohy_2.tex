\documentclass[10pt,a4paper]{article}

\usepackage{amsmath}
\usepackage{amssymb}
%\pagestyle{empty}
\usepackage[margin=.5in]{geometry}

\def\vysl#1{}

\begin{document}

\section*{Limity v nevlastním bodě I}

%Pozor na to, že některé limity jsou pro $x \to \infty$ a jiné pro $x \to -\infty$!

\begin{enumerate}
	\everymath{\displaystyle}
	\parskip\bigskipamount
	%\item $\lim_{x\to\infty} \frac{x^2}{x}$ 
	\item $\lim_{x\to\infty} \frac{x^2-1}{x-1}$
	\item $\lim_{x\to\infty} \frac{x^2-4}{x^2 - 3x + 2}$
	\item $\lim_{x\to\infty} \frac{x - 3x^2}{x - 4x^2}$
	\item $\lim_{x\to\infty} \left( \frac{x}{1+x} - \frac{x}{1-x} \right)$
	\item $\lim_{x\to\infty} \left( \frac{x^2}{1+x} - \frac{x^2}{1-x} \right)$
	\item $\lim_{x \to -\infty} \frac{x^2 - 2x + 5}{2x^3 - x^2 + 4}$
	\item $\lim_{x \to \infty} \frac{\sqrt x - 6x}{3x+1}$
	
	
\end{enumerate}

\newpage

\begin{enumerate}
	\everymath{\displaystyle}
	\parskip\bigskipamount
	\lineskip\medskipamount
	%\item $\lim_{x\to\infty} \frac{x^2}{x}$ 
	\item $\lim_{x\to\infty} \frac{x^2-1}{x-1} = \lim_{x\to\infty} \frac{x - \frac1x}{1-\frac1x} = \frac{\lim\limits_{x\to\infty} x - \lim\limits_{x\to\infty} \frac1x}{\lim\limits_{x\to\infty}1 - \lim\limits_{x\to\infty}\frac1x} = \frac{\infty - 0}{1 - 0} = \infty$
	\item $\lim_{x\to\infty} \frac{x^2-4}{x^2 - 3x + 2} = \lim_{x\to\infty} \frac{1 - \frac{4}{x^2}}{1 - \frac3x + \frac2{x^2}} = \frac{\lim\limits_{x\to\infty}1 - \lim\limits_{x\to\infty}\frac{4}{x^2}}{\lim\limits_{x\to\infty}1 - \lim\limits_{x\to\infty}\frac3x + \lim\limits_{x\to\infty}\frac2{x^2}} = \frac{1 - 0}{1 - 0 + 0} = 1$
	\item (už trochu rychleji) $\lim_{x\to\infty} \frac{x - 3x^2}{x - 4x^2} = \lim_{x\to\infty} \frac{\frac1x - 3}{\frac1x - 4} = \frac{-3}{-4} = \frac34$
	\item $\lim_{x\to\infty} \left( \frac{x}{1+x} - \frac{x}{1-x} \right) = \left(\lim_{x\to\infty} \frac{x}{1+x}\right) - \left(\lim_{x\to\infty}\frac{x}{1-x}\right) = \left(\lim_{x\to\infty} \frac{1}{\frac1x+1}\right) - \left(\lim_{x\to\infty}\frac{1}{\frac1x-1}\right) = 1 - (-1) = 2$\\
	alternativně: $\lim_{x\to\infty} \left( \frac{x}{1+x} - \frac{x}{1-x} \right) = \lim_{x\to\infty} \frac{2x^2}{x^2 - 1} = \lim_{x\to\infty} \frac{2}{1 - \frac1{x^2}} = 2$
	\item $\lim_{x\to\infty} \left( \frac{x^2}{1+x} - \frac{x^2}{1-x} \right) = \lim_{x\to\infty} x\left( \frac{x}{1+x} - \frac{x}{1-x} \right) = \left(\lim_{x\to\infty} x\right) \cdot  \lim_{x\to\infty} \left( \frac{x}{1+x} - \frac{x}{1-x} \right) = \infty \cdot 2 = \infty$\\
	(případně lze využít i postupy jako v minulém bodě)
	\item $\lim_{x \to -\infty} \frac{x^2 - 2x + 5}{2x^3 - x^2 + 4} = \lim_{x \to -\infty} \frac{1 - \frac2x + \frac5{x^2}}{2x - 1 + \frac4{x^2}} = \frac{1 - 0 + 0}{-\infty -1 + 0} = 0$
	\item $\lim_{x \to \infty} \frac{\sqrt x - 6x}{3x+1} = \lim_{x \to \infty}\frac{\frac{1}{\sqrt x} - 6}{3 + \frac1x} = \frac{0 - 6}{3 + 0} = -2$ (využívám toho, že $\lim_{x \to \infty}\tfrac{1}{\sqrt x} = 0$, protože $\lim_{x \to \infty}\sqrt x = \infty$)
	
	
\end{enumerate}

\end{document}