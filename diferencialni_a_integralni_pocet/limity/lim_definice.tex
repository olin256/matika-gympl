\documentclass[handout]%
{beamer}
%\usetheme{Execushares}
\usetheme{AnnArbor}
\usecolortheme{beaver}
%\setbeamercolor{title}{parent=structure,bg=green!50!black,fg=white}
%\usecolortheme{dolphin}
\setbeamertemplate{navigation symbols}{}%remove navigation symbols

\usepackage{amsmath}
\usepackage{amssymb}
\usepackage{amsthm}
%\usepackage[utf8]{inputenc}
\usepackage[czech]{babel}
\usepackage{tikz-cd}
\usepackage[mathscr]{euscript}
\usepackage[IL2]{fontenc}
\usepackage{mathtools}

\usetikzlibrary{calc,shapes.callouts,shapes.arrows}

%\usepackage{beamerarticle}

%\usepackage{bbm}

\def\rllap#1{\hbox to0pt{\hss#1\hss}}

\newcommand{\bubblethis}[2]{
        \tikz[remember picture,baseline]{\node[anchor=base,inner sep=0,outer sep=0]%
        (#1) {\underline{#1}};\node[overlay,cloud callout,callout relative pointer={(-0.2cm,+0.7cm)},%
        aspect=2.5,fill=yellow!90] at ($(#1.north)+(-0.5cm,1.6cm)$) {#2};}%
    }%
		
\newcommand{\speechthis}[2]{
        \tikz[remember picture,baseline]{\node[anchor=base,inner sep=0,outer sep=0]%
        (pom) {#1};\node[overlay,ellipse callout,fill=blue!50] 
        at ($(pom.north)+(1cm,+0.8cm)$) {#2};}%
    }%
		
\newcommand{\R}{\mathbb R}

\title{Limity funkcí}
\author{Alexander Slávik} %  and J. Trlifaj
\subtitle{Úvod a definice}
\institute{Gymnázium Voděradská}
\date{2. 10. 2020}

\begin{document}

%
%\frame{\titlepage}

\section{Definice}


\begin{frame}
	\frametitle{Okolí}
	
	\begin{block}{Definice}
	Pro reálná čísla $c$, $\varepsilon$, kde $\varepsilon > 0$, definujeme \emph{$\varepsilon$-okolí bodu $c$} jako
	\[ B(c; \varepsilon) = (c-\varepsilon; c+\varepsilon) \]
	(tj. otevřený interval s konci $c-\varepsilon$ a $c + \varepsilon$).
	\end{block}
	
	\pause\bigskip
	\begin{block}{Definice}
	Pro reálná čísla $c$, $\varepsilon$, kde $\varepsilon > 0$, definujeme \emph{prstencové $\varepsilon$-okolí bodu $c$} jako
	\[ P(c; \varepsilon) = B(c; \varepsilon) \setminus \{c\} =  (c-\varepsilon; c) \cup (c; c+\varepsilon). \]
	\end{block}
\end{frame}



\begin{frame}
	\frametitle{Limita}
	
	\begin{alertblock}{Stěžejní Definice}
	Řekneme, že \emph{funkce $f$ má v bodě $c$ limitu $A$}, pokud platí
	\[ \textcolor{blue}{\forall \varepsilon \in \R, \varepsilon>0} \  \textcolor{red}{\exists \delta \in \R, \delta > 0}\  \textcolor{green!50!black}{\forall x \in P(c; \delta)\colon f(x) \in B(A; \varepsilon)}.  \]
	\pause
	Jinak řečeno:\pause
	\begin{itemize}
		\item Pro jakkoliv malé okolí bodu $A$ je možné nalézt dostatečně malé prstencové okolí bodu $c$ takové, že funkční hodnoty bodů z tohoto prstencového okolí budou všechny v onom okolí $A$.\pause
		\item Pro každé kladné $\varepsilon$ existuje kladné $\delta$ takové, že pokud se $x$ liší od $c$ o~méně jak $\delta$ (ovšem $x \neq c$), tak se $f(x)$ liší od $A$ o méně jak $\varepsilon$.
	\end{itemize}
	
	\end{alertblock}
	\pause
	%Uvedenou skutečnost zapisujeme jako
	To, že $f$ má v $c$ limitu $A$, zapisujeme jako
	\[ \lim_{x \to c} f(x) = A. \]
	
	%\pause\bigskip
	%\begin{block}{Definice}
	%Pro reálná čísla $c$, $\varepsilon$, kde $\varepsilon > 0$, definujeme \emph{prstencové $\varepsilon$-okolí bodu $c$} jako
	%\[ P(c; \varepsilon) = B(c; \varepsilon) \setminus \{c\} =  (c-\varepsilon; c) \cup (c; c+\varepsilon). \]
	%\end{block}
\end{frame}


\section{Příklad výpočtu}


\begin{frame}
	\frametitle{Důkaz, že $\lim_{x\to1}x = 1$}
	\begin{itemize}
		\item Chceme dokázat, že
		\[ \textcolor{blue}{\forall \varepsilon \in \R, \varepsilon>0} \  \textcolor{red}{\exists \delta \in \R, \delta > 0}\  \textcolor{green!50!black}{\forall x \in P(\overset{c}{1}; \delta)\colon \overset{\rllap{$\scriptstyle f(x)$}}{x} \in B(\overset{A}{1}; \varepsilon)}.  \]
		\item Ideálně bychom chtěli nějaký \emph{předpis}, jak pro zadané $\varepsilon$ najít patřičné $\delta$; \pause ovšem zde si vždycky stačí vzít za $\delta$ přímo $\varepsilon$:
		\[ \forall x \in P(1; \varepsilon)\colon x \in B(1; \varepsilon). \]\pause
		\item Je možné si zvolit i menší $\delta$. \pause
		\item Kdybychom chtěli dokázat, že $\lim_{x\to1}2x = 2$, budeme postupovat stejně, jen použijeme $\delta = \frac{\varepsilon}{2}$ (nebo menší).
		
	\end{itemize}
	
\end{frame}


\section{Poznámky}

\begin{frame}
	\frametitle{Poznámky}
	\begin{itemize}
		\item Limita funkce v bodě \emph{nijak nezávisí} na funkční hodnotě v onom bodě \pause (prstencové okolí!).
		\item Limita funkce v nějakém bodě vůbec nemusí existovat; např. následující limity \alert{ne}existují:\pause
		\[ \lim_{x \to 0}\frac{|x|}{x}, \qquad \pause \lim_{x \to 0}\frac{1}{x}, \qquad \pause \lim_{x \to 0}\sin\left(\frac{1}{x}\right). \] \pause
		\item Pokud funkce má v nějakém bodě limitu, pak je tato určena jednoznačně (tj. nemůže se nám např. stát, že v jednom bodě bude limita funkce jak $1$, tak $2$).
	\end{itemize}
\end{frame}


\end{document}
