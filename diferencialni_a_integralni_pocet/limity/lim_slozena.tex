\documentclass%[handout]%
{beamer}
%\usetheme{Execushares}
\usetheme{AnnArbor}
\usecolortheme{beaver}
%\setbeamercolor{title}{parent=structure,bg=green!50!black,fg=white}
%\usecolortheme{dolphin}
\setbeamertemplate{navigation symbols}{}%remove navigation symbols

\usepackage{amsmath}
\usepackage{amssymb}
\usepackage{amsthm}
%\usepackage[utf8]{inputenc}
\usepackage[czech]{babel}
\usepackage{tikz-cd}
\usepackage[mathscr]{euscript}
\usepackage[IL2]{fontenc}
\usepackage{mathtools}
\usepackage[normalem]{ulem}

\usetikzlibrary{calc,shapes.callouts,shapes.arrows}

%\usepackage{beamerarticle}

%\usepackage{bbm}

\def\rllap#1{\hbox to0pt{\hss#1\hss}}

%\newcommand{\bubblethis}[2]{
        %\tikz[remember picture,baseline]{\node[anchor=base,inner sep=0,outer sep=0]%
        %(#1) {\underline{#1}};\node[overlay,cloud callout,callout relative pointer={(-0.2cm,+0.7cm)},%
        %aspect=2.5,fill=yellow!90] at ($(#1.north)+(-0.5cm,1.6cm)$) {#2};}%
    %}%
		%
%\newcommand{\speechthis}[2]{
        %\tikz[remember picture,baseline]{\node[anchor=base,inner sep=0,outer sep=0]%
        %(pom) {#1};\node[overlay,ellipse callout,fill=blue!50] 
        %at ($(pom.north)+(1cm,+0.8cm)$) {#2};}%
    %}%
		
\newcommand{\R}{\mathbb R}

\title{Limity funkcí}
\author{Alexander Slávik} %  and J. Trlifaj
\subtitle{Limita složené funkce}
\institute{Gymnázium Voděradská}
\date{11. 11. 2020}

\begin{document}

%
%\frame{\titlepage}

%
\begin{frame}
	\frametitle{Limita složené funkce}
	
	Alias \uv{substituce v limitách}.
	\pause
	
	\begin{block}{Věta o limitě složené funkce}
	Nechť funkce $f$ a $g$ splňují
	\[ \lim_{x \to c}g(x) = D \quad \text{a} \quad \lim_{y \to D}f(y) = A, \]
	kde $c, D, A \in \R^*$. \pause Dále nechť je splněna jedna z těchto podmínek:\pause
	\begin{itemize}
		\item[(P)] Existuje $\delta > 0$ takové, že $g(x) \neq D$ pro $x \in P(c; \delta)$;\pause
		\item[(S)] $f$ je spojitá v bodě $D$ \pause (neboli $\lim\limits_{y \to D}f(y) = f(D)$).\pause
	\end{itemize}
	Potom $\lim\limits_{x \to c}f(g(x)) = A$.
	
	\end{block}
\end{frame}


\end{document}
