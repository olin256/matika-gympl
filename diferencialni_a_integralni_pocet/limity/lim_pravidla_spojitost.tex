\documentclass[handout]%
{beamer}
%\usetheme{Execushares}
\usetheme{AnnArbor}
\usecolortheme{beaver}
%\setbeamercolor{title}{parent=structure,bg=green!50!black,fg=white}
%\usecolortheme{dolphin}
\setbeamertemplate{navigation symbols}{}%remove navigation symbols

\usepackage{amsmath}
\usepackage{amssymb}
\usepackage{amsthm}
%\usepackage[utf8]{inputenc}
\usepackage[czech]{babel}
\usepackage{tikz-cd}
\usepackage[mathscr]{euscript}
\usepackage[IL2]{fontenc}
\usepackage{mathtools}

\usetikzlibrary{calc,shapes.callouts,shapes.arrows}

%\usepackage{beamerarticle}

%\usepackage{bbm}

\def\rllap#1{\hbox to0pt{\hss#1\hss}}

%\newcommand{\bubblethis}[2]{
        %\tikz[remember picture,baseline]{\node[anchor=base,inner sep=0,outer sep=0]%
        %(#1) {\underline{#1}};\node[overlay,cloud callout,callout relative pointer={(-0.2cm,+0.7cm)},%
        %aspect=2.5,fill=yellow!90] at ($(#1.north)+(-0.5cm,1.6cm)$) {#2};}%
    %}%
		%
%\newcommand{\speechthis}[2]{
        %\tikz[remember picture,baseline]{\node[anchor=base,inner sep=0,outer sep=0]%
        %(pom) {#1};\node[overlay,ellipse callout,fill=blue!50] 
        %at ($(pom.north)+(1cm,+0.8cm)$) {#2};}%
    %}%
		
\newcommand{\R}{\mathbb R}

\title{Limity funkcí}
\author{Alexander Slávik} %  and J. Trlifaj
\subtitle{Pravidla pro počítání}
\institute{Gymnázium Voděradská}
\date{7. 10. 2020}

\begin{document}

%
%\frame{\titlepage}



\section{Základní pravidla}


\begin{frame}
\frametitle{Terminologie}

\begin{itemize}
	\item Funkce $f$ má v bodě $c$ limitu $A$.\pause
	\item $\lim\limits_{x \to c} f(x) = A$.\pause
	\item Limita (z) $f(x)$ pro $x$ jdoucí k $c$ je $A$.
\end{itemize}


\end{frame}



\begin{frame}
	\frametitle{Počítání s limitami}
	Mějme dvě funkce $f$, $g$, které mají v bodě $c \in \R$ limity:
	\[ \lim_{x\to c}f(x) = A, \qquad \lim_{x\to c}g(x) = B. \]
	Potom:\pause
	\begin{itemize}
		\item Limita funkce $f + g$ v bodě $c$ existuje a je rovna $A + B$.\pause
		\item Limita funkce $f \cdot g$ v bodě $c$ existuje a je rovna $A \cdot B$.\pause
		\item Je-li \alert<6->{$B \neq 0$}, pak limita funkce $\frac fg$ v bodě $c$ existuje a je rovna $\frac AB$.
	\end{itemize}
	\pause
	Symbolicky:
	\begin{align*}
	\lim_{x \to c}(f(x) + g(x)) &= \lim_{x \to c}f(x) + \lim_{x \to c}g(x),\\
	\lim_{x \to c}(f(x) \cdot g(x)) &= \bigl(\lim_{x \to c}f(x)\bigr) \cdot \bigl(\lim_{x \to c}g(x)\bigr),\\
	\lim_{x \to c}\frac{f(x)}{g(x)} &= \frac{\lim\limits_{x \to c}f(x)}{\alert<6->{\lim\limits_{x \to c}g(x)}}
	\end{align*}\pause
	%\[ \lim_{x \to c}(f(x) + g(x)) =  \]
	
\end{frame}


\begin{frame}
	\frametitle{Důsledky}
	

	\begin{block}{Pozorování}
	Je-li $f$ funkce, $k$ reálné číslo a platí $\lim\limits_{x \to c} f(x) = A$, pak $\lim\limits_{x \to c} kf(x) = kA$.
	\end{block}
	
	\bigskip
	\pause
	Např. $\lim\limits_{x \to 0} \frac{10 \sin x}{x} = 10 \cdot \lim\limits_{x \to 0} \frac{\sin x}{x} = 10 \cdot 1 = 10$.
	
\end{frame}


\section{Spojitost}


\begin{frame}
	\frametitle{Spojité funkce}
	\begin{block}{Definice}
	Řekneme, že funkce $f$ je \emph{spojitá} v bodě $c \in D_f$, pokud $f(c) = \lim\limits_{x \to c} f(x)$.
	\end{block}
	\pause \medskip
	Z pravidel pro limity pak pro dvě funkce $f$ a $g$, které jsou spojité v bodě $c \in D_f \cap D_g$, plyne:\pause
	\begin{itemize}
		\item funkce $f + g$ je spojitá v $c$,\pause
		\item funkce $f \cdot g$ je spojitá v $c$,\pause
		\item pokud $g(c)\neq 0$, tak funkce $\frac fg$ je spojitá v $c$.
	\end{itemize}
	
	\pause
	\begin{exampleblock}{Příklady}
	Následující funkce jsou spojité ve všech bodech svých definičních oborů:
	\pause
	\begin{itemize}
		\item polynomiální funkce \pause% jsou spojité (ve všech bodech $\R$).\pause
		\item exponenciální a logaritmické funkce \pause
		\item goniometrické funkce
	\end{itemize}
	\end{exampleblock}
	
\end{frame}

\section{Početní techniky}

\begin{frame}
	\frametitle{Jak počítat \uv{$\frac00$}}
	Pokud $\lim\limits_{x \to c}f(x) = 0$ a stejně tak $\lim\limits_{x \to c}g(x) = 0$, jak může dopadnout
	\[ \lim\limits_{x \to c}\frac{f(x)}{g(x)}? \]
	\pause
	Vcelku \emph{libovolně}. \pause Při praktických výpočtech nám pomůže zejména následující
	
	\begin{block}{Pozorování}\pause
	Shodují-li se funkce $f$ a $g$ na nějakém prstencovém okolí bodu $c \in \R$, pak mají v $c$ tutéž limitu (pokud existuje).
	\end{block}

\end{frame}

%
%\begin{frame}
	%\frametitle{Standardní limity}
	%
	%\begin{itemize}
		%\parskip\medskipamount
		%\item $\displaystyle\lim_{x \to 0} \frac{\sin x}{x} = 1$
		%\item $\displaystyle\lim_{x \to 0} \frac{\mathrm{e}^x - 1}{x} = 1$
		%\item $\displaystyle\lim_{x \to 0} \frac{\operatorname{ln}(x+1)}{x} = 1$
		%\item $\displaystyle\lim_{x \to 0} \frac{1 - \cos x}{x^2} = \frac12$
	%\end{itemize}
%\end{frame}



\end{document}
