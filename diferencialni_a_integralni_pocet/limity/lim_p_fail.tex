\documentclass[12pt,a4paper]{article}

\usepackage{amsmath}
\usepackage{amssymb}
%\pagestyle{empty}
\usepackage[margin=1in]{geometry}
\usepackage{hyperref}
\usepackage[utf8]{inputenc}
\usepackage[IL2]{fontenc}
\usepackage[czech]{babel}
\usepackage{graphicx}

\def\vysl#1{}

\def\Z{\mathbb Z}

%\parindent 0pt
\def\graf#1{\[\includegraphics[width=.65\hsize]{#1.pdf}\]}

\begin{document}

\section*{Selhání podmínky (P)}

Ve Větě o limitě složené funkce chce podmínka (P), aby se vnitřní funkce \uv{vyhýbala} své limitní hodnotě na nějakém prstencovém okolí. Standardní příklad, kdy toto může selhat, je funkce $g(x) = x \sin \frac 1x$:
\graf{grafy/sincrazy}
Platí, že $\lim_{x \to 0} g(x) = 0$, ovšem $g$ přitom nabývá hodnoty $0$ na \emph{každém} prstencovém okolí nuly (protože $g(\frac{1}{k\pi}) = 0$ pro všechna $k \in \Z \setminus \{0\}$).

Uvažme nyní funkci $f$, která nabývá ve všech nenulových reálných číslech hodnotu $0$, jenom v $0$ je to $1$:
\graf{grafy_p/dira}
Pro tuto funkci platí, že $\lim_{x \to 0} f(x) = 0$.

Budeme-li se dívat na složenou funkci $f(g(x))$, tak ta bude nabývat hodnoty $1$ právě pro $x = \frac{1}{k\pi}$, $k \in \Z \setminus \{0\}$, v ostatních nenulových číslech bude $0$ (a v nule nebude definována). Její graf tedy vypadá cca takto:
\graf{grafy_p/mnoho_der}
Tato funkce na libovolně malém prstencovém okolí nuly nabývá hodnot jak $0$, tak $1$, proto $\lim_{x \to 0} f(g(x))$ neexistuje.
%
%\everymath{\displaystyle}
%
%\begin{enumerate}
	%\item $\lim_{x \to \infty} \sin x$ neexistuje (hodnoty \uv{oscilují} a k ničemu se neblíží) \graf{sin}
	%\item $\lim_{x \to \infty} x^2 = \infty$ (definice: pro každé okolí nekonečna $B(\varepsilon, \infty)$ najdeme (prstencové) okolí nekonečna $P(\delta, \infty)$, že pro $x \in P(\delta, \infty)$ je $x^2 \in B(\varepsilon, \infty)$; hodnoty se \uv{blíží nekonečnu} a funkce \uv{přeroste všechny meze}) \graf{kvadr}
	%\item $\lim_{x \to \infty} \operatorname{ln}x = \infty$ (pro logaritmus je situace stejná, jen se logaritmus blíží k~nekonečnu \uv{mnohem méně ochotně}) \graf{ln}
	%\item $\lim_{x \to \infty}\tfrac1x = 0$ (definice: pro každé okolí nuly $B(\varepsilon, 0)$ najdeme (prstencové) okolí nekonečna $P(\delta, \infty)$, že pro $x \in P(\delta, \infty)$ je $\tfrac1x \in B(\varepsilon, 0)$; hodnoty se \uv{blíží nule}) \graf{inv}
	%\item $\lim_{x \to \infty}x(1+\sin x)$ neexistuje (sice pro vyšší hodnoty $x$ umíme dostat čím dál vyšší hodnoty funkce, ale limita neexistuje, protože se funkce vždycky \uv{vrátí} zpátky do nuly; z definice, např. nelze najít žádné $\delta$ takové, aby pro hodnoty $x \in P(\delta, \infty)$ platilo $x(1+\sin x) \in B(1,\infty)$) \graf{xsin}
	%\item $\lim_{x \to \infty}(x + 10\sin x) = \infty$ (graf sice trochu skáče nahoru a dolů, ale postupně \uv{přeroste všechny meze}) \graf{sinplus}
	%\item $\lim_{x \to \infty} \tfrac{2x^3 + x}{x^3 + 1} = 2$ (funkce toho typu, který jsme počítali na hodině; o výsledku \uv{rozhodují} jen koeficienty u $x^3$) \graf{rational}
%\end{enumerate}
%
%
%\section*{Příklady a ne-příklady nevlastních limit}
%
%\begin{enumerate}
	%\item $\lim_{x \to 0} \tfrac{1}{x^2} = \infty$ (definice: pro každé okolí nekonečna $B(\varepsilon, \infty)$ najdeme prstencové okolí nuly $P(\delta, 0)$, že pro $x \in P(\delta, 0)$ je $\tfrac1{x^2} \in B(\varepsilon, \infty)$) \graf{invkvadr}
	%\item $\lim_{x \to 0} \tfrac{1}{x}$ neexistuje (zatímco \uv{zleva} se hodnoty \uv{blíží $-\infty$}, \uv{zprava} se \uv{blíží $\infty$}) \graf{invzero}
	%\item $\lim_{x \to 0} \operatorname{cotg} x$ neexistuje (ze stejných důvodů jako předchozí případ) \graf{cotg}
	%\item $\lim_{x \to 0} \operatorname{ln}\mathopen|x\mathclose| = -\infty$ \graf{logabs}
%\end{enumerate}
%
%\section*{Jiné příklady}
%
%\begin{enumerate}
	%\item $\lim_{x \to 0} x\sin\tfrac1x = 0$ \graf{sincrazy} 
%\end{enumerate}


\end{document}