\documentclass%[handout]%
{beamer}
%\usetheme{Execushares}
\usetheme{AnnArbor}
\usecolortheme{beaver}
%\setbeamercolor{title}{parent=structure,bg=green!50!black,fg=white}
%\usecolortheme{dolphin}
\setbeamertemplate{navigation symbols}{}%remove navigation symbols

\usepackage{amsmath}
\usepackage{amssymb}
\usepackage{amsthm}
%\usepackage[utf8]{inputenc}
\usepackage[czech]{babel}
\usepackage{tikz-cd}
\usepackage[mathscr]{euscript}
\usepackage[IL2]{fontenc}
\usepackage{mathtools}
\usepackage[normalem]{ulem}

\usetikzlibrary{calc,shapes.callouts,shapes.arrows}

%\usepackage{beamerarticle}

%\usepackage{bbm}

\def\rllap#1{\hbox to0pt{\hss#1\hss}}

%\newcommand{\bubblethis}[2]{
        %\tikz[remember picture,baseline]{\node[anchor=base,inner sep=0,outer sep=0]%
        %(#1) {\underline{#1}};\node[overlay,cloud callout,callout relative pointer={(-0.2cm,+0.7cm)},%
        %aspect=2.5,fill=yellow!90] at ($(#1.north)+(-0.5cm,1.6cm)$) {#2};}%
    %}%
		%
%\newcommand{\speechthis}[2]{
        %\tikz[remember picture,baseline]{\node[anchor=base,inner sep=0,outer sep=0]%
        %(pom) {#1};\node[overlay,ellipse callout,fill=blue!50] 
        %at ($(pom.north)+(1cm,+0.8cm)$) {#2};}%
    %}%
		
\newcommand{\R}{\mathbb R}
\newcommand{\N}{\mathbb N}
\newcommand{\Z}{\mathbb Z}
\def\d{\text{d}}

%\title{Limity funkcí}
\author{Alexander Slávik} %  and J. Trlifaj
\title{Derivace funkce}
\subtitle{Pravidla a aplikace}
\institute{Gymnázium Voděradská}
\date{9. 12. 2020}

\begin{document}



\begin{frame}
	\frametitle{Úvodní varování}
	Derivace je nějaká limita, takže:
	
	\[ \text{Derivace funkce} \pause \begin{cases} \text{je vlastní (tj. reálné číslo),}  \\ \text{je nevlastní (tj. $\pm \infty$) -- \alert{to vůbec nebudeme uvažovat}},  \\ \text{vůbec neexistuje.} \end{cases} \]
	\pause
	Bude potřeba si dávat (trochu) pozor na definiční obory funkce a její derivace \pause (ten může být menší).
	
	\pause 
	Platí následující věta:
	\begin{block}{Věta}
	Má-li funkce $f$ v nějakém bodě derivaci, je v tom bodě spojitá.
	\end{block}
	\pause
	\dots ovšem ne všechny spojité funkce musí mít derivaci všude! (Např. $|x|$ nemá derivaci v $0$).
\end{frame}



\section{Pravidla pro derivování}

\begin{frame}
	\frametitle{Pravidla pro derivování}
	Už víme
	\begin{itemize}
		\item $(f+g)' = f' + g'$
		\item $(f \cdot g)' = f' \cdot g + f \cdot g'$
	\end{itemize}
	\pause
	Kromě toho v bodech, kde je $g$ nenulová, platí:
	\begin{itemize}
		\item $\bigl(\frac fg\bigr)' = \frac{f' \cdot g - f \cdot g'}{g^2}$
	\end{itemize}
	\pause
Přesné znění:	
\begin{block}{Věta}
Mají-li funkce $f$, $g$ v bodě $x_0$ derivaci, pak v tomto bodě mají derivace i $f+g$, $f\cdot g$ a je-li $g(x_0)\neq0$, tak také $\frac fg$ a platí $(f+g)'(x_0) = f'(x_0) + g'(x_0)$,\\ $(f\cdot g)'(x_0) = f'(x_0)g(x_0) + f(x_0)g'(x_0)$ \\ a $\bigl(\frac fg\bigr)'(x_0) = \frac{f'(x_0) \cdot g(x_0) - f(x_0) \cdot g'(x_0)}{g^2(x_0)}$
\end{block}

\end{frame}


\begin{frame}
	\frametitle{Derivace mocninných funkcí podruhé}
	Už víme, že $(x^n)' = n x^{n-1}$ pro všechna $n \in \N$ (a $x \in \R$).

	\pause
	Pro záporná $n$ to platí také, jen pozor na definiční obory;
	\pause položíme $m = -n$ a máme
	\[ (x^n)' = \pause \left(\frac{1}{x^m}\right)' = \pause \frac{(1)'\cdot x^m - 1 \cdot (x^m)'}{(x^m)^2} = \pause \frac{-mx^{m-1}}{x^{2m}} = \pause -mx^{-m-1} = \pause nx^{n-1}, \]\pause
	což platí pro všechna $x \in \R \setminus \{0\}$.
\end{frame}


\begin{frame}
	\frametitle{Derivace dalších elementárních funkcí}
	
	\begin{itemize}
		\item $(\mathrm{e}^x)' = \mathrm{e}^x$ (\uv{$\mathrm{e}^x$ nelze zderivovat}) \pause
		\item $(\sin x)' = \cos x$ \pause
		\item $(\cos x)' = -\sin x$ \pause
		\item $(\ln x)' = \frac 1x$ \alert{pro $x \in \R^+$}  \pause
		\item $(\operatorname{tg} x)' = \dfrac{1}{\cos^2 x}$ pro $x \in \R \setminus \{\tfrac12 \pi + k \pi; k \in \Z\}$  \pause
		\item $(\operatorname{cotg} x)' = -\dfrac{1}{\sin^2 x}$ pro $x \in \R \setminus \{k \pi; k \in \Z\}$
	\end{itemize}
\end{frame}




\begin{frame}
	\frametitle{Derivace složené funkce}
	\begin{block}{Věta}
	Jestliže funkce $g$ má derivaci v bodě $x_0$ a funkce $f$ v bodě $z_0 = g(x_0)$, pak také \emph{složená funkce} $f \circ g$ má derivaci v bodě $x_0$ a platí
	\[ (f \circ g)'(x_0) = f'(g(x_0)) \cdot g'(x_0). \]
	\end{block}
	\pause
	Též nazývané řetízkové pravidlo (chain rule).
\end{frame}


\end{document}


%\section{Extrémy a monotonie}



\begin{frame}
	\frametitle{Extrémy funkce}
	\pause
	\begin{itemize}
		\item extrémy $\displaystyle \begin{cases}\text{maxima} \\ \text{minima} \end{cases}$ \pause
		\item extrémy $\displaystyle \begin{cases}\text{globální} \\ \text{lokální} \end{cases}$ \pause
		\item extrémy $\displaystyle \begin{cases}\text{ostré} \\ \text{(neostré)} \end{cases}$ 
	\end{itemize}
	
	%\pause
	%\begin{block}{Definice}
	%\end{block}
	
\end{frame}


\begin{frame}
	\frametitle{Extrémy -- definice}
	%\renewcommand{\baselinestretch}{1.5}
	
  Řekneme, že funkce $f$ má v bodě $x_0 \in D_f$	
	\begin{center}
	\begin{tabular}{c|c}
	
	\vtop{\advance\baselineskip15pt \hsize.4\hsize \alert{(neostré) globální maximum}, pokud
	pro všechna $x \in D_f$ platí $f(x) \leq f(x_0)$.} 
		& \vtop{\advance\baselineskip15pt \hsize.4\hsize \alert{(neostré) lokální maximum}, pokud
	existuje okolí $B(x_0; \varepsilon)$ takové, že pro všechna $x \in D_f \cap B(x_0; \varepsilon)$
		platí $f(x) \leq f(x_0)$.
	} \\ \hline
	\vtop{\advance\baselineskip15pt \hsize.4\hsize \alert{ostré globální maximum}, pokud
	pro všechna $x \in D_f$, $x \neq x_0$ platí $f(x) < f(x_0)$. }
	
	& \vtop{\advance\baselineskip15pt \hsize.4\hsize \alert{ostré lokální maximum}, pokud
	existuje okolí $B(x_0; \varepsilon)$ takové, že pro všechna $x \in D_f \cap B(x_0; \varepsilon)$,
	$x \neq x_0$ platí $f(x) < f(x)$.}
	\end{tabular}
\end{center}
\pause
Pro minima to je úplně stejné, jen se všude otočí znaménko nerovnosti.
\end{frame}


\begin{frame}
\frametitle{Extrémy a derivace}

Klíčový vztah:
\begin{block}{Věta}
Má-li funkce $f$ v bodě $a$ (neostrý) lokální extrém a existuje-li derivace $f'(a)$, pak je $f'(a) = 0$.
\end{block}
\pause
\begin{alertblock}{Častý omyl}
\alert{Neplatí} $f'(a) = 0 \Rightarrow$ $f$ má v $a$ extrém. 
\end{alertblock}
\pause
Příklad hodný zapamatování: $x^3$ (v nule je derivace nulová, ale extrém tam není).

\medskip\pause
Body, kde je derivace nulová, nazýváme \emph{stacionární} nebo též \emph{podezřelé z~extrému}.
\end{frame}


%\begin{frame}
%\frametitle{Extrémy a monotonie}
%
%Klíčové vztahy:
%\begin{block}{Věta}
%Má-li funkce $f$ v bodě $a$ (neostrý) lokální extrém a existuje-li derivace $f'(a)$, pak je $f'(a) = 0$.
%\end{block}
%\pause
%\begin{alertblock}{Častý omyl}
%\alert{Neplatí} $f'(a) = 0 \Rightarrow$ $f$ má v $a$ extrém. 
%\end{alertblock}
%\pause
%Příklad hodný zapamatování: $x^3$ (v nule je derivace nulová, ale extrém tam není).
%
%\medskip\pause
%Body, kde je derivace nulová, nazýváme \emph{stacionární} nebo též \emph{podezřelé z~extrému}.
%\end{frame}


%\begin{frame}
	%\frametitle{Možné výsledky}
	%
	%\[ \text{Derivace funkce} \pause \begin{cases} \text{je vlastní (tj. reálné číslo),} \pause \\ \text{je nevlastní (tj. $\pm \infty$)}, \pause \\ \text{vůbec neexistuje.} \end{cases} \]
	%
%\end{frame}




\end{document}
