\documentclass[12pt,a4paper]{article}

\usepackage{amsmath}
\usepackage{amssymb}
\pagestyle{empty}
\usepackage[margin=1in]{geometry}
\usepackage{hyperref}
\usepackage[utf8]{inputenc}
\usepackage[IL2]{fontenc}
\usepackage[czech]{babel}
\usepackage{amsthm}
\usepackage{enumitem}
\usepackage{microtype}

\theoremstyle{definition}

\newtheorem{uloha}{Úloha}

\DeclareMathOperator{\tg}{tg}
\def\ee{\mathrm{e}}

\begin{document}

\section*{Snad trochu zajímavější úlohy na derivace}

\begin{uloha}
Nalezněte předpis kubické funkce $f$ (tj. polynomu stupně 3), která bude splňovat $f(0) = 0$, $f(1) = 1$, $f'(0) = 0$, $f'(1) = 0$. Ověřte si v nějakém kreslítku grafů, že výsledek opravdu vypadá tak, jak má.
\end{uloha}

\begin{uloha}
Nalezněte příklad polynomiální funkce, která bude mít dva stacionární body, ale žádné lokální extrémy.
\end{uloha}

\begin{uloha}
Dokažte, že rovnice $\ee^x = x$ nemá v reálných číslech řešení, zatímco $\ee^{x} = \ee x$ má jedno řešení. (Nápověda: uvažte extrémy a monotonii funkcí $\ee^x - x$ a $\ee^x - \ee x$).
\end{uloha}


\end{document}