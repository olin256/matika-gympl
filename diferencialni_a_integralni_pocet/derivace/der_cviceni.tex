\documentclass[12pt,a4paper]{article}

\usepackage{amsmath}
\usepackage{amssymb}
%\pagestyle{empty}
\usepackage[margin=1in]{geometry}
\usepackage{hyperref}
\usepackage[utf8]{inputenc}
\usepackage[IL2]{fontenc}
\usepackage[czech]{babel}

\DeclareMathOperator{\tg}{tg}
\def\ee{\mathrm{e}}

\begin{document}

\section*{Prostě derivování}

%Výsledky jsou na druhé straně.

Zderivujte uvedené funkce. V mnoha případech je možné nejprve výraz trochu upravit, čímž si člověk zjednoduší derivování, případně využít toho, co už má spočtené.

\begin{enumerate}
	\everymath{\displaystyle}
	\parskip\bigskipamount
	\item $\frac{\sqrt x + \sqrt[3] x}{x}$
	\item $\sqrt{1-x^2}$
	\item $\frac{x}{x+1}$
	\item $\frac{x^2}{x^2+1}$
	\item $\cos\left(\frac{x^2}{x^2+1}\right)$
	\item $\sin(x^3 - 2x^2)$
	\item $\cos(x \cdot \cos x)$
	\item $\cos x \cdot \tg x$
	\item $\ln(x^2)$
	\item $(2^x)^3$
	\item $x \cdot \ln x$
	\item $x^x$ (nápověda: přepište výraz do tvaru $\ee^{\text{něco}}$)
\end{enumerate}


\end{document}