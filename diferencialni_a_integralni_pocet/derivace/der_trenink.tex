\documentclass[12pt,a4paper]{article}

\usepackage{amsmath}
\usepackage{amssymb}
\pagestyle{empty}
\usepackage[margin=1in]{geometry}
\usepackage{hyperref}
\usepackage[utf8]{inputenc}
\usepackage[IL2]{fontenc}
\usepackage[czech]{babel}
\usepackage{amsthm}
\usepackage{enumitem}

\theoremstyle{definition}
\newtheorem{uloha}{Úloha}

\DeclareMathOperator{\tg}{tg}
\def\ee{\mathrm{e}}

\setlist[enumerate]{label={(\alph*)}}

\begin{document}

\section*{Derivace -- trénink na písemku}
%\bigskip

\emph{Výsledky jsou na druhé straně.}

%Zderivujte uvedené funkce. V mnoha případech je možné nejprve výraz trochu upravit, čímž si člověk zjednoduší derivování, případně využít toho, co už má spočtené.
\begin{uloha}
Nalezněte rovnici tečny %k funkci
\begin{enumerate}
	\item k funkci $2^x$ v bodě $0$,
	\item k funkci $\sin x$ v bodě $\pi$.
\end{enumerate}
\end{uloha}


\begin{uloha}
U následujících funkcí určete maximální (tj. co největší) intervaly, na kterých je funkce rostoucí či klesající, a nalezněte všechna lokální maxima a minima.
\begin{enumerate}
	\item $2 x^3 - x^2 - 8 x - 4$
	\item $x^4 - 8 x^3 + 18 x^2 - 5$
	\item $\displaystyle \frac3x - \frac{1}{x^2}$
	\item $\dfrac{1-2x^2}{x^2+3}$
	\item $x^2 - \ln(x^2)$
\end{enumerate}
\end{uloha}

\begin{uloha}
Nalezněte globální extrémy následujících funkcí na zadaných intervalech:
\begin{enumerate}
	\item $\dfrac{x}{x^2+1}$ na $\langle0; 2\rangle$
	\item $(x^2+1)\ee^{x}$ na $\langle-2; 0\rangle$
	\item $2 x^3 - x^2 - 8 x - 4$ na $\langle -3; 2)$
\end{enumerate}
\end{uloha}


%\end{document}


\newpage



\subsection*{Výsledky}

\setcounter{uloha}{0}

\begin{uloha}
(a) $y = (\ln 2) x + 1$; (b) $y = -x + \pi$
\end{uloha}


\begin{uloha}
\begin{enumerate}
	\item Rostoucí na $(-\infty; -1\rangle$ a $\langle \frac43; \infty)$, klesající na $\langle -1; \frac43 \rangle$, lok. maximum v~$-1$, lok. minimum v $\frac43$.
	\item Rostoucí na $\langle0; \infty)$, klesající na $(-\infty; 0\rangle$, lok. minimum v $0$. (Stacionární bod $3$ není extrémem.)
	\item Rostoucí na $(0; \frac23\rangle$, klesající na $(-\infty; 0)$ a $\langle \frac23; \infty)$, lok. maximum v $\frac23$. (Pozor na to, že je potřeba vzít v potaz i bod $0$, ve kterém funkce není definována.)
	\item Rostoucí na $(-\infty; 0\rangle$, klesající na $\langle0; \infty)$, lok. maximum v $0$.
	\item Rostoucí na $\langle-1; 0)$ a $\langle1; \infty)$, klesající na $(-\infty; 1\rangle$ a $(0; 1\rangle$, lok. minimum v $-1$ a $1$.
\end{enumerate}
\end{uloha}


\begin{uloha}
\begin{enumerate}
	\item Minimum v 0, maximum v 1.
	\item Minimum v $-1$, maximum v $0$.
	\item Minimum v $-3$, maximum v $-1$.
\end{enumerate}
\end{uloha}

\end{document}