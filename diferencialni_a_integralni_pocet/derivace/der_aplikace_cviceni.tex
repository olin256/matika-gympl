\documentclass[12pt,a4paper]{article}

\usepackage{amsmath}
\usepackage{amssymb}
\pagestyle{empty}
\usepackage[margin=1in]{geometry}
\usepackage{hyperref}
\usepackage[utf8]{inputenc}
\usepackage[IL2]{fontenc}
\usepackage[czech]{babel}
\usepackage{amsthm}
\usepackage{enumitem}

\theoremstyle{definition}
\newtheorem{uloha}{Úloha}

\DeclareMathOperator{\tg}{tg}
\def\ee{\mathrm{e}}

\setlist[enumerate]{label={(\alph*)}}

\begin{document}

\section*{Aplikace derivací}
%\bigskip

Výsledky jsou na druhé straně.

%Zderivujte uvedené funkce. V mnoha případech je možné nejprve výraz trochu upravit, čímž si člověk zjednoduší derivování, případně využít toho, co už má spočtené.
\begin{uloha}
Nalezněte rovnici tečny k funkci
\begin{enumerate}
	\item k funkci $2^x$ v bodě $0$,
	\item k funkci $\sin x$ v bodě $\pi$.
\end{enumerate}
\end{uloha}


\begin{uloha}
U následujících funkcí určete maximální (tj. co největší) intervaly, na kterých je funkce rostoucí či klesající, a nalezněte všechna lokální maxima a minima.
\begin{enumerate}
	\item $2x^2 + 6x + 1$
	\item $x^3-9 x^2+24 x+2$
	\item $\ee^{x^3-9 x^2+24 x+2}$
	\item $x + \frac1x$
\end{enumerate}
\end{uloha}

\newpage

\subsection*{Výsledky}

\setcounter{uloha}{0}

\begin{uloha}
(a) $y = (\ln 2) x + 1$; (b) $y = -x + \pi$
\end{uloha}


\begin{uloha}
\begin{enumerate}
	\item Klesající na $(-\infty; -\frac32\rangle$, rostoucí $\langle-\frac32; \infty)$, v $-\frac32$ je lokální minimum.
	\item Klesající na $\langle2; 4\rangle$, rostoucí na $(-\infty; 2\rangle$ a $\langle4; \infty)$, ve $2$ je lokální maximum a v $4$ lokální minimum.
	\item Klesající na $\langle2; 4\rangle$, rostoucí na $(-\infty; 2\rangle$ a $\langle4; \infty)$, ve $2$ je lokální maximum a v $4$ lokální minimum.
	\item Klesající na $\langle-1; 0)$ a $(0; 1\rangle$, rostoucí na $(-\infty; -1\rangle$ a $\langle1; \infty$, v $-1$ je lokální maximum a v $1$ je lokální minimum.
\end{enumerate}
\end{uloha}

\end{document}