\documentclass%[handout]%
{beamer}
%\usetheme{Execushares}
\usetheme{AnnArbor}
\usecolortheme{beaver}
%\setbeamercolor{title}{parent=structure,bg=green!50!black,fg=white}
%\usecolortheme{dolphin}
\setbeamertemplate{navigation symbols}{}%remove navigation symbols

\usepackage{amsmath}
\usepackage{amssymb}
\usepackage{amsthm}
%\usepackage[utf8]{inputenc}
\usepackage[czech]{babel}
\usepackage{tikz-cd}
\usepackage[mathscr]{euscript}
\usepackage[IL2]{fontenc}
\usepackage{mathtools}
\usepackage[normalem]{ulem}

\usetikzlibrary{calc,shapes.callouts,shapes.arrows}

%\usepackage{beamerarticle}

%\usepackage{bbm}

\def\rllap#1{\hbox to0pt{\hss#1\hss}}

%\newcommand{\bubblethis}[2]{
        %\tikz[remember picture,baseline]{\node[anchor=base,inner sep=0,outer sep=0]%
        %(#1) {\underline{#1}};\node[overlay,cloud callout,callout relative pointer={(-0.2cm,+0.7cm)},%
        %aspect=2.5,fill=yellow!90] at ($(#1.north)+(-0.5cm,1.6cm)$) {#2};}%
    %}%
		%
%\newcommand{\speechthis}[2]{
        %\tikz[remember picture,baseline]{\node[anchor=base,inner sep=0,outer sep=0]%
        %(pom) {#1};\node[overlay,ellipse callout,fill=blue!50] 
        %at ($(pom.north)+(1cm,+0.8cm)$) {#2};}%
    %}%
		
\newcommand{\R}{\mathbb R}
\def\d{\text{d}}

%\title{Limity funkcí}
\author{Alexander Slávik} %  and J. Trlifaj
\title{Derivace funkce}
\subtitle{Úvod}
\institute{Gymnázium Voděradská}
\date{25. 11. 2020}

\begin{document}



\section{Úvod do derivace}

\begin{frame}
	\frametitle{Co je derivace?}
	Derivace $\approx$ jak rychle funkce roste či klesá v daném bodě.
	
	\medskip
	\pause
	Trochu přesněji: Derivace $\approx$ směrnice tečny v daném bodě.
	
	\pause
	\begin{block}{Definice}
	Funkce $f$ má v bodě $x \in \R$ derivaci $c \in \R^*$, pokud platí
	\[ c = \lim_{h \to 0} \frac{f(x+h) - f(x)}{h} \]
	\end{block}
	\pause
	\bigskip
	Značení pro derivaci $f$ v $x$:
	\[ f'(x), \qquad \frac{\d f(x)}{\d x} \]
	
\end{frame}



\begin{frame}
	\frametitle{Možné výsledky}
	
	\[ \text{Derivace funkce} \pause \begin{cases} \text{je vlastní (tj. reálné číslo),} \pause \\ \text{je nevlastní (tj. $\pm \infty$)}, \pause \\ \text{vůbec neexistuje.} \end{cases} \]
	
\end{frame}




\end{document}
