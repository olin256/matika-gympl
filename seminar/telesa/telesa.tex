\documentclass[12pt,a5paper]{article}
\usepackage[margin=.8cm]{geometry}
\usepackage[utf8]{inputenc}
\usepackage[IL2]{fontenc}
\usepackage[czech]{babel}
\usepackage{microtype}
\usepackage{amssymb}
\usepackage{amsthm}
\usepackage{amsmath}
\usepackage{xcolor}
\usepackage{graphicx}

\usepackage[inline]{enumitem}

\newcommand{\hint}[1]{{\color{gray}\footnotesize\noindent(Nápověda: #1)}}

\setlist[enumerate]{label={(\alph*)},topsep=\smallskipamount,itemsep=\smallskipamount,parsep=0pt}
\setlist[itemize]{topsep=\smallskipamount,noitemsep}

\def\tisk{%
\newbox\shipouthackbox
\pdfpagewidth=2\pdfpagewidth
\let\oldshipout=\shipout
\def\shipout{\afterassignment\zdvojtmp \setbox\shipouthackbox=}%
\def\zdvojtmp{\aftergroup\zdvoj}%
\def\zdvoj{%
    \oldshipout\vbox{\hbox{%
        \copy\shipouthackbox
        \hskip\dimexpr .5\pdfpagewidth-\wd\shipouthackbox\relax
        \box\shipouthackbox
    }}%
}}%


\newtheorem*{poz}{Pozorování}

\theoremstyle{definition}
\newtheorem{uloha}{Úloha}
\newtheorem{suloha}[uloha]{\llap{$\star$ }Úloha}
\newtheorem*{bonus}{Bonus}
\newtheorem*{defn}{Definice}

\pagestyle{empty}

\let\ee\expandafter

\def\vysld{}
\let\printvysl\relax
\let\printalphvysl\relax

\makeatletter
\def\vyslplain#1{\ee\ee\ee\gdef\ee\ee\ee\vysld\ee\ee\ee{\ee\vysld\ee\printvysl\ee{\the\c@uloha}{#1}}}
\let\vysl\vyslplain

\def\locvysl#1{\ee\gdef\ee\locvysld\ee{\locvysld\item #1}}
\let\lv\locvysl

\newenvironment{ulohav}[1][]{\begin{uloha}[#1]\gdef\locvysld{\begin{enumerate*}}}{\ee\vyslplain\ee{\locvysld\end{enumerate*}}\end{uloha}}
\newenvironment{sulohav}[1][]{\begin{suloha}[#1]\gdef\locvysld{\begin{enumerate*}}}{\ee\vyslplain\ee{\locvysld\end{enumerate*}}\end{suloha}}

\makeatother

\begin{document}

%\tisk

\section*{Tělesa}

\begin{defn}
\emph{Platónské těleso} je konvexní mnohostěn splňující
\begin{enumerate}[label={(\arabic*)}]
    \item všechny stěny jsou shodné pravidelné mnohoúhelníky,
    \item z každého vrcholu vede stejný počet hran.
\end{enumerate}
\end{defn}

\begin{uloha}
Nalezněte příklad tělesa, které bude splňovat podmínku (1) z~definice platónských těles, ale ne (2).\vysl{třeba slepíme dva pravidelné čtyřstěny podél jedné stěny}
\end{uloha}

\begin{uloha}\label{moznosti}
Jaké vlastně stěny mohou platónská tělesa mít? Kolik nejméně/\penalty0nejvíce mohou mít ony stěnové mnohoúhelníky mít stran? (Proč?)\vysl{mohou to být max. pětiúhelníky}
\end{uloha}

\begin{ulohav}\label{vzorce}
Máme-li platónské těleso, jehož stěny jsou pravidelné $n$-úhelníky a z každého vrcholu vychází $d$ hran, vyjádřete
\begin{enumerate}
    \item počet hran $e$ pomocí počtu stěn $f$, $n$ a $d$,\lv{$e = \frac12 n f$}
    \item počet vrcholů $v$ pomocí $f$, $n$ a $d$.\lv{$v = \frac{fn} d$}
\end{enumerate}
(Vyjádření nemusí využít všechny neznámé.)
\end{ulohav}

\begin{uloha}
Pomocí Eulerova vzorce $v-e+f=2$ a výsledků úloh \ref{moznosti} a \ref{vzorce} odvoďte, jaká přesně mohou existovat platónská tělesa; ke každému určete, kolik má stěn, hran a vrcholů, stupeň vrcholů $d$ a počet stran stěn $n$.
\vysl{\begin{tabular}{ccccc}
$f$ & $v$ & $e$ & $n$ & $d$ \\
4 &   4 &    6  &  3 &   3 \\
6 &   8 &   12  &  4 &   3 \\
8 &   6 &   12  &  3 &   4 \\
12&  20 &   30  &  5 &   3 \\
20&  12 &   30  &  3 &   5
\end{tabular}}
\end{uloha}

\begin{uloha}
Proč jsou ve výsledné tabulce určité \uv{symetrie}?\vysl{jistě jste na to přišli} % dualita
\end{uloha}


\begin{ulohav}
% jde o archimedovska telesa
% https://en.wikipedia.org/wiki/Archimedean_solid
Zkuste nalézt/popsat těleso (spočíst, kolik bude mít jakých stěn), v jehož každém vrcholu se stýkají
\begin{enumerate}
    \item dva šestiúhelníky a jeden trojúhelník,\lv{čtyři šestiúhelníky, čtyři trojúhelníky}
        % https://en.wikipedia.org/wiki/Truncated_tetrahedron
    \item dva osmiúhelníky a jeden trojúhelník,\lv{šest osmiúhelníků, osm trojúhelníků}
        % https://en.wikipedia.org/wiki/Truncated_cube
    \item dva čtverce a dva trojúhelníky,\lv{šest čtverců, osm trojúhelníků}
        % https://en.wikipedia.org/wiki/Cuboctahedron
    \item tři čtverce a jeden trojúhelník.\lv{osmnáct čtverců, osm trojúhelníků}
        % https://en.wikipedia.org/wiki/Rhombicuboctahedron
\end{enumerate}
(Všechny mnohoúhelníky uvažujeme \emph{pravidelné}.)
\end{ulohav}


\newpage
\parindent=0pt
\parskip=\bigskipamount
\def\printvysl#1#2{\textbf{#1.}\ #2\par}
\def\printalphvysl#1#2#3{\textbf{#1}(#2)\ #3\par}
\vysld

\end{document}