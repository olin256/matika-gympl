\documentclass[11pt,a4paper]{article}
\usepackage[margin=.5in]{geometry}
\usepackage[utf8]{inputenc}
\usepackage[IL2]{fontenc}
\usepackage[czech]{babel}
\usepackage{microtype}
\usepackage{amssymb}
\usepackage{amsthm}
\usepackage{amsmath}
\usepackage{xcolor}
\usepackage{graphicx}

\usepackage[inline]{enumitem}

\newcommand{\R}{\mathbb{R}}

\setlist[enumerate]{label={(\alph*)},topsep=\smallskipamount,noitemsep}
\setlist[itemize]{topsep=\smallskipamount,noitemsep}

\theoremstyle{definition}
\newtheorem{uloha}{Úloha}
\newtheorem*{bonus}{Bonus}

\pagestyle{empty}

\def\graf{}

\begin{document}

\section*{Úvodní skupinová zábava}

\begin{uloha}
Vyřešte následující kvíz. Každá otázka má právě jednu správnou odpověď. (Stačí uvést výsledky.)\par
\medskip
\hbox to\hsize{%
\begin{minipage}[t]{.3\hsize}
\noindent
\textbf{1.} Kolik je v kvízu celkem správných odpovědí (A)?
\begin{enumerate}[noitemsep,label={(\Alph*)}]
    \item 0
    \item 1
    \item 2
    \item 3
\end{enumerate}
\end{minipage}%
\hfil
\begin{minipage}[t]{.3\hsize}
\noindent
\textbf{2.} Kolik je v kvízu celkem správných odpovědí (B)?
\begin{enumerate}[noitemsep,label={(\Alph*)}]
    \item 0
    \item 1
    \item 2
    \item 3
\end{enumerate}
\end{minipage}%
\hfil
\begin{minipage}[t]{.3\hsize}
\noindent
\textbf{3.} Která je správná odpověď na otázku 2?\par
\begin{enumerate}[noitemsep,label={(\Alph*)}]
    \item (A)
    \item (B)
    \item (C)
    \item (D)
\end{enumerate}
\end{minipage}%
}
\end{uloha}


\begin{uloha}
Kolik čtverců je na obrázku? (Stačí odpověď.)
\[\includegraphics{ctverce.pdf}\]
\end{uloha}


\begin{uloha}
Čtverec a pravidelný pětiúhelník sdílí jednu stranu a každý ještě sdílí jednu stranu s pravidelným $n$-úhelníkem, jehož čtyři strany jsou na obrázku znázorněny tučně. Kolik je $n$? (Uveďte stručně svůj postup.)
\[\includegraphics{polygony.pdf}\]
\end{uloha}


\begin{uloha}
Do tabulky $4 \times 4$ doplňte čísla $1$ či $2$ tak, aby součet ve všech čtvercích $3\times 3$ byl dělitelný čtyřmi, součet celé tabulky nebyl dělitelný čtyřmi a zároveň byl tento celkový součet (a) co nejmenší, (b) co největší. Musí být zaplněna všechna pole. (Stačí uvést výsledek.)
\end{uloha}

\begin{uloha}
Tři dvojice bratr\,+\,sestra se potřebují dostat na druhou stranu řeky. K dispozici mají loďku, ve které se mohou plavit jedna nebo dvě osoby (bez posádky loďka plout nemůže a všichni umí loďku ovládat). Jak mají postupovat, aby se všichni dostali na druhou stranu řeky, ale v žádný okamžik nenastala situace, že by některá z žen byla ve společnosti muže jiného než svého bratra a její bratr by u toho nebyl?
\end{uloha}

\begin{uloha}
Zvětšíme-li čitatel i jmenovatel jistého zlomku o 1, dostaneme zlomek, jehož hodnota je o $1/20$ větší. Provedeme-li totéž s tímto novým zlomkem, dostaneme zlomek, jehož hodnota je o $1/12$ větší, než byla hodnota zlomku na počátku. Určete všechny tři zlomky. (Uveďte výpočty.)
\end{uloha}

\begin{uloha}
Věž vysoká $50$\,m má tvar válce, na který je nahoře posazen kužel. Jak válec, tak kužel mají poloměr podstavy $15$\,m. Jak vysoká je válcová část, pokud je celkový objem věže $10500\pi\,\mathrm{m}^3$?  (Uveďte výpočty.)
\end{uloha}

\begin{uloha}
Ruda a Jiřina běhají po atletickém oválu o délce 320\,m. Ruda je pecivál, takže \uv{běží} rychlostí 7\,km/h, zatímco Jiřina si cválá 12\,km/h. Jestliže vyrazí ze stejného místa a běží po oválu stejným směrem, za jak dlouho se znovu setkají (v sekundách)? (Uveďte výpočty.)
\end{uloha}

\begin{uloha}
Jirka a Mirka musí otestovat na covid celou firmu o 127 zaměstnancích. Jirka provede jeden odběr za 5 minut, Mirka za 6, ovšem Mirku po ránu bolela hlava a začala s odběry až půl hodiny po Jirkovi. Za jak dlouho (od Jirkova startu) mají otestovanou celou firmu? (Uveďte výpočty.)
\end{uloha}

\begin{uloha}
Vymyslete soustavu dvou rovnic o dvou neznámých, která bude mít právě dvě řešení, a to $[1; 1]$ a $[1; 2]$.
\end{uloha}

\begin{uloha}
Napište báseň o čtyřech verších na matematické téma; použijte buď obkročný (\emph{abba}), nebo střídavý (\emph{abab}) rým.
\end{uloha}

\end{document}