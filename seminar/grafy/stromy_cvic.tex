\documentclass[9pt,a6paper,landscape]{extarticle}
\usepackage[margin=.7cm]{geometry}
\usepackage[utf8]{inputenc}
\usepackage[IL2]{fontenc}
\usepackage[czech]{babel}
\usepackage{microtype}
\usepackage{amssymb}
\usepackage{amsthm}
\usepackage{amsmath}
\usepackage{xcolor}

\usepackage[inline]{enumitem}

\newcommand{\hint}[1]{{\color{gray}\footnotesize\noindent(Nápověda: #1)}}

\setlist[enumerate]{label={(\alph*)},topsep=\smallskipamount,itemsep=\smallskipamount,parsep=0pt,itemjoin={\quad}}
\setlist[itemize]{topsep=\smallskipamount,noitemsep}

\def\tisk{%
\newbox\shipouthackbox
\pdfpagewidth=2\pdfpagewidth
\let\oldshipout=\shipout
\def\shipout{\afterassignment\zdvojtmp \setbox\shipouthackbox=}%
\def\zdvojtmp{\aftergroup\zdvoj}%
\def\zdvoj{%
    \oldshipout\vbox{\hbox{%
        \copy\shipouthackbox
        \hskip\dimexpr .5\pdfpagewidth-\wd\shipouthackbox\relax
        \box\shipouthackbox
    }}%
}}%

\def\tiskctyr{%
\newbox\shipouthackbox
\pdfpagewidth=2\pdfpagewidth
\pdfpageheight=2\pdfpageheight
\let\oldshipout=\shipout
\def\shipout{\afterassignment\zctyrtmp \setbox\shipouthackbox=}%
\def\zctyrtmp{\aftergroup\zctyr}%
\def\zctyr{%
    \offinterlineskip
    \oldshipout\vbox{\hbox{%
        \copy\shipouthackbox
        \hskip\dimexpr .5\pdfpagewidth-\wd\shipouthackbox\relax
        \copy\shipouthackbox
    }%
    \vskip\dimexpr .5\pdfpageheight-\ht\shipouthackbox\relax
    \hbox{%
        \copy\shipouthackbox
        \hskip\dimexpr .5\pdfpagewidth-\wd\shipouthackbox\relax
        \box\shipouthackbox
    }}%
}}

\newtheorem*{poz}{Pozorování}

\theoremstyle{definition}
\newtheorem{uloha}{Úloha}
\newtheorem{suloha}[uloha]{\llap{$\star$ }Úloha}
\newtheorem*{bonus}{Bonus}
\newtheorem*{defn}{Definice}

\pagestyle{empty}


\let\ee\expandafter

\def\vysld{}
\let\printvysl\relax
\let\printalphvysl\relax

\makeatletter
\def\vyslplain#1{\ee\ee\ee\gdef\ee\ee\ee\vysld\ee\ee\ee{\ee\vysld\ee\printvysl\ee{\the\c@uloha}{#1}}}
\let\vysl\vyslplain

\def\locvysl#1{\ee\gdef\ee\locvysld\ee{\locvysld\item #1}}
\let\lv\locvysl

\newenvironment{ulohav}[1][]{\begin{uloha}[#1]\gdef\locvysld{\begin{enumerate*}}}{\ee\vyslplain\ee{\locvysld\end{enumerate*}}\end{uloha}}
\newenvironment{sulohav}[1][]{\begin{suloha}[#1]\gdef\locvysld{\begin{enumerate*}}}{\ee\vyslplain\ee{\locvysld\end{enumerate*}}\end{suloha}}

\makeatother

\begin{document}

% \tisk
% \tiskctyr

\section*{Stromořadí}

\begin{ulohav}
Rozhodněte, zda následující skóre musí, mohou, nebo nemohou přináležet k nějakému stromu; pokud mohou, zkonstruujte ho, a pokud nemusí, tak zkonstruujte i protipříklad.
\begin{flushleft}
\begin{enumerate*}
    \item $(2, 2, 2, 2)$\lv{nemůže -- nemá listy}
    \item $(1, 1, 1, 1, 4)$\lv{musí -- \uv{hvězdička} se 4 rameny}
    \item $(1, 1, 2, 3, 3)$\lv{nemůže}
    \item $(1, 1, 2, 2, 3)$\lv{takový graf vůbec neexistuje}
    \item $(1, 1, 2, 2, 2)$\lv{může a nemusí (cesta vs. trojúhelník a samostatná hrana)}
    \item $(1, 1, 1, 2, 2, 3)$\lv{může a nemusí (cesta s odbočkou vs. trojúhelník s ocasem a samostatná hrana)}
    \item $(1,1,1,1,1,1,1, 2,2, 3,3,3, 4)$\lv{nemůže -- je tam moc hran}
    \item $(1, 1, 1, 3, 3, 3)$\lv{nemůže -- je tam moc hran}
    \item $(1, 1, 1, 1, 1, 1, 4, 4)$\lv{musí -- dva spojené vrcholy stupně 4 s~napojenými listy}
\end{enumerate*}
\end{flushleft}
\end{ulohav}

\begin{ulohav}
Nalezněte příklad souvislého grafu,
\begin{enumerate}
    \item který bude mít jedinou kostru,\lv{můžeme vzít přímo strom (dokonce musíme)}
    \item který bude mít vícero koster, ale jehož každé dvě kostry budou isomorfní,\lv{např. kružnice}
    \item který bude mít alespoň dvě neisomorfní kostry.\lv{např. čtverec s~úhlopříčkou}
\end{enumerate}
\end{ulohav}

\begin{ulohav}
Nechť $G$ je souvislý graf, který ale není strom.
\begin{enumerate}
    \item Musí mít všechny jeho kostry stejný počet hran?\lv{ano -- počet hran bude o jedna menší, než počet vrcholů}
    \item Musí mít všechny jeho kostry stejný počet listů?\lv{ne -- opět např. čtverec s~úhlopříčkou}
\end{enumerate}
Vysvětlete proč, nebo nalezněte protipříklad.
\end{ulohav}

\begin{uloha}
Najděte (souvislý) graf, který bude mít přesně 7 různých koster. \hint{Existuje příklad s přesně 7 hranami.}
\vysl{např. kružnice na 7 vrcholech}
\end{uloha}


\begin{suloha}
Ukažte, že souvislý graf, který není strom, už nutně musí mít aspoň \emph{tři} různé kostry. \hint{Uvažte, co se stane, když do kostry přidáme nějakou nepoužitou hranu z původního grafu.}
\vysl{Jelikož nejde o strom, nějaká hrana v kostře chybí; po přidání tého hrany ke kostře dostaneme (jediný) cyklus. Odebráním kterékoliv hrany z onoho cyklu (musí tam být ještě aspoň dvě další) dostaneme pokaždé nějakou jinou kostru.}
\end{suloha}



\newpage
\parindent=0pt
\parskip=\smallskipamount
\def\printvysl#1#2{\textbf{#1.}\ #2\par}
\def\printalphvysl#1#2#3{\textbf{#1}(#2)\ #3\par}
\vysld


\end{document}

