\documentclass[10pt,a5paper]{article}
\usepackage[margin=1.2cm]{geometry}
\usepackage[utf8]{inputenc}
\usepackage[IL2]{fontenc}
\usepackage[czech]{babel}
\usepackage{microtype}
\usepackage{amssymb}
\usepackage{amsthm}
\usepackage{amsmath}
\usepackage{xcolor}
\usepackage{graphicx}

\usepackage[inline]{enumitem}

\newcommand{\R}{\mathbb{R}}

\DeclareMathOperator{\tg}{tg}
\DeclareMathOperator{\cotg}{cotg}

\setlist[enumerate]{label={(\alph*)},topsep=\smallskipamount,itemsep=\smallskipamount,parsep=0pt}
\setlist[itemize]{topsep=\smallskipamount,noitemsep}

\def\tisk{%
\newbox\shipouthackbox
\pdfpagewidth=2\pdfpagewidth
\let\oldshipout=\shipout
\def\shipout{\afterassignment\zdvojtmp \setbox\shipouthackbox=}%
\def\zdvojtmp{\aftergroup\zdvoj}%
\def\zdvoj{%
    \oldshipout\vbox{\hbox{%
        \copy\shipouthackbox
        \hskip\dimexpr .5\pdfpagewidth-\wd\shipouthackbox\relax
        \box\shipouthackbox
    }}%
}}%



\newtheorem*{poz}{Pozorování}

\theoremstyle{definition}
\newtheorem{uloha}{Úloha}
\newtheorem{suloha}[uloha]{\llap{$\star$ }Úloha}
\newtheorem*{bonus}{Bonus}
\newtheorem*{defn}{Definice}

\pagestyle{empty}

\DeclareMathOperator{\arctg}{arctg}

\let\=\doteq
\let\ee\expandafter

\def\vysld{}
\let\printvysl\relax
\let\printalphvysl\relax

\makeatletter
\def\vyslplain#1{\ee\ee\ee\gdef\ee\ee\ee\vysld\ee\ee\ee{\ee\vysld\ee\printvysl\ee{\the\c@uloha}{#1}}}
\def\vyslalph#1{\edef\tmpuloha{{\the\c@uloha}{\alph{enumi}}}\ee\ee\ee\gdef\ee\ee\ee\vysld\ee\ee\ee{\ee\vysld\ee\printalphvysl\tmpuloha{#1}}}

\def\locvysl#1{\ee\gdef\ee\locvysld\ee{\locvysld\item #1}}
\let\lv\locvysl

\newenvironment{ulohav}[1][]{\begin{uloha}[#1]\gdef\locvysld{\begin{enumerate*}}}{\ee\vyslplain\ee{\locvysld\end{enumerate*}}\end{uloha}}
\newenvironment{sulohav}[1][]{\begin{suloha}[#1]\gdef\locvysld{\begin{enumerate*}}}{\ee\vyslplain\ee{\locvysld\end{enumerate*}}\end{suloha}}

\makeatother

\begin{document}

% \tisk

\section*{Determinanty všeho druhu}


\begin{ulohav}\label{nuda}
Spočtete následující velmi zajímavé determinanty:
\begin{flushleft}
\begin{enumerate*}[itemjoin=\quad]
    \item $\begin{vmatrix}3 & 4 \\ 2 & -5\end{vmatrix}$\lv{$-23$}
    \item $\begin{vmatrix}2 &3 & 1 \\ -1 & 2 & 3 \\ 3 & 2 & -1\end{vmatrix}$\lv{$0$}
    \item $\begin{vmatrix}1 &1 & 2 \\ 2 & 3 & 1 \\ 3 & 4 & -5\end{vmatrix}$\lv{$-8$}
    \item $\begin{vmatrix}\sqrt2&\sqrt3&\sqrt6 \\ \sqrt6&\sqrt2&\sqrt3 \\ \sqrt3&\sqrt6&\sqrt3\end{vmatrix}$\lv{$5 \sqrt{3}+3 \sqrt{6}-12$}
    \item $\begin{vmatrix}0&0&3&3 \\ 3&0&1&2 \\ 1&0&2&4 \\ 2&1&3&4\end{vmatrix}$\lv{$-15$}
    \item $\begin{vmatrix}1&2&3&4 \\ -2&1&-4&3 \\ 3&-4&-1&2 \\ 4&3&-2&-1\end{vmatrix}$\lv{$900$}
    \item $\begin{vmatrix}1&3&2&5&0 \\ 0&-2&5&2&-1 \\ 0&0&1&2&-4 \\ 0&0&0&4&-1 \\ 0&0&0&0&5\end{vmatrix}$\lv{$-40$}\label{trojuh}
    \item $\begin{vmatrix} 1 & 2 & 3 &  4 & 1\\ 0 & -1 & 2 & 4 & 2\\ 0 & 0 & 4 & 0 & 0\\ -3 & -6 & -9 & -12 & 4\\ 0 & 0 & 1 & 1 & 1\end{vmatrix}$\lv{$28$}
    \item $\begin{vmatrix}0&0&1&0&0&0\\1&0&0&0&0&0\\0&0&0&0&1&0\\0&0&0&0&0&1\\0&0&0&1&0&0\\0&1&0&0&0&0\end{vmatrix}$\lv{$-1$}
\end{enumerate*}
\end{flushleft}
\end{ulohav}

\begin{uloha}
Jak se spočte determinant matice, která vypadá jako v Úloze \ref{nuda}, části \ref{trojuh}?
\vyslplain{Prostě se vynásobí čísla na diagonále.}
\end{uloha}

\begin{ulohav}
Nalezněte všechna reálná $x$, aby platilo
\begin{flushleft}
\begin{enumerate*}[itemjoin=\quad]
    \item $\begin{vmatrix}6 & 2 \\ 3 & x\end{vmatrix} = 0$\lv{$x=1$}
    \item $\begin{vmatrix}x-1 & x & x+2 \\ 1 & 2 & 1 \\ 1 & x & 2\end{vmatrix} = 0$\lv{$x=2$}
    \item $\begin{vmatrix}8 & 6 & 4 \\ x & 5 & 5 \\ 7 & x & x-2\end{vmatrix} = 0$\lv{$x \in \{1;5\}$}
\end{enumerate*}
\end{flushleft}
\end{ulohav}


\begin{uloha}
Spočtěte obsah trojúhelníka, který má vrcholy v bodech o souřadnicích $[1; 2]$, $[2; 5]$ a $[5; 6]$.\vyslplain{4}
\end{uloha}


\newpage
\parindent=0pt
\parskip=\smallskipamount
\def\printvysl#1#2{\textbf{#1.}\ #2\par}
\def\printalphvysl#1#2#3{\textbf{#1}(#2)\ #3\par}
\vysld


\end{document}

