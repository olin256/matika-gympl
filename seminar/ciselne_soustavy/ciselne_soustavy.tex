\documentclass[10pt,a5paper]{article}
\usepackage[margin=1cm]{geometry}
\usepackage[utf8]{inputenc}
\usepackage[IL2]{fontenc}
\usepackage[czech]{babel}
\usepackage{microtype}
\usepackage{amssymb}
\usepackage{amsthm}
\usepackage{amsmath}
\usepackage{xcolor}
\usepackage{graphicx}

\usepackage[inline]{enumitem}

\newcommand{\hint}[1]{{\color{gray}\footnotesize\noindent(Nápověda: #1)}}

\setlist[enumerate]{label={(\alph*)},topsep=\smallskipamount,itemsep=\smallskipamount,parsep=0pt}
\setlist[itemize]{topsep=\smallskipamount,noitemsep}

\def\tisk{%
\newbox\shipouthackbox
\pdfpagewidth=2\pdfpagewidth
\let\oldshipout=\shipout
\def\shipout{\afterassignment\zdvojtmp \setbox\shipouthackbox=}%
\def\zdvojtmp{\aftergroup\zdvoj}%
\def\zdvoj{%
    \oldshipout\vbox{\hbox{%
        \copy\shipouthackbox
        \hskip\dimexpr .5\pdfpagewidth-\wd\shipouthackbox\relax
        \box\shipouthackbox
    }}%
}}%



\newtheorem*{poz}{Pozorování}

\theoremstyle{definition}
\newtheorem{uloha}{Úloha}
\newtheorem{suloha}[uloha]{\llap{$\star$ }Úloha}
\newtheorem*{bonus}{Bonus}
\newtheorem*{defn}{Definice}

\pagestyle{empty}

\let\ee\expandafter

\def\vysld{}
\let\printvysl\relax
\let\printalphvysl\relax

\makeatletter
\long\def\vyslplain#1{\ee\ee\ee\gdef\ee\ee\ee\vysld\ee\ee\ee{\ee\vysld\ee\printvysl\ee{\the\c@uloha}{#1}}}
\let\vysl\vyslplain

\def\locvysl#1{\ee\gdef\ee\locvysld\ee{\locvysld\item #1}}
\let\lv\locvysl

\newenvironment{ulohav}[1][]{\begin{uloha}[#1]\gdef\locvysld{\begin{enumerate*}}}{\ee\vyslplain\ee{\locvysld\end{enumerate*}}\end{uloha}}
\newenvironment{sulohav}[1][]{\begin{suloha}[#1]\gdef\locvysld{\begin{enumerate*}}}{\ee\vyslplain\ee{\locvysld\end{enumerate*}}\end{suloha}}

\makeatother

\begin{document}

% \tisk

\section*{Číselné soustavy}

\begin{uloha}
V ASCII\footnote{American Standard Code for Information Interchange} jsou velkým písmenům anglické abecedy přiřazena čísla 65--90 a malým písmenům čísla 97--122. Proč? \hint{Dvojková soustava.}
\vysl{pro převod mezi velkými a malými písmeny stačí přepínat jediný bit (šestý zprava, tj. odpovídající $2^5$)}
\end{uloha}

\begin{ulohav}
Převeďte následující čísla do desítkové soustavy:
\begin{enumerate}
    \item $1100101_2$\lv{101}
    \item $1021_3$\lv{34}
    \item $1\mathrm A2\mathrm A3_{11}$\lv{28306}
    \item $\mathrm{DEAD}_{16}$\lv{57005}
\end{enumerate}
\end{ulohav}

\begin{ulohav}
Nalezněte přirozené číslo $n$ takové, že bude platit
\begin{enumerate}
    \item $12423_n = 3300$\lv{7}
    \item $12423_n = 17812$\lv{11}
    \item $\mathrm{MAMBA}_n = 8764035$\lv{25}
\end{enumerate}
\end{ulohav}

\begin{ulohav}
Převeďte do soustavy o základu $n$ následující čísla $x$ (zadaná v desítkové soustavě):
\begin{enumerate}
    \item $x = 100$, $n = 3$\lv{10201}
    \item $x = 100$, $n = 5$\lv{400}
    \item $x = 1234$, $n = 16$\lv{4D2}
    \item $x = 209890$, $n = 27$\lv{AHOJ}
\end{enumerate}
\end{ulohav}

\begin{ulohav}
Napište si čísla 0--15
\begin{enumerate}
    \item ve dvojkové soustavě,\lv{0, 1, 10, 11, 100, 101, 110, 111, 1000, 1001, 1010, 1011, 1100, 1101, 1110, 1111}
    \item ve čtyřkové soustavě.\lv{0, 1, 2, 3, 10, 11, 12, 13, 20, 21, 22, 23, 30, 31, 32, 33}
\end{enumerate}
Co pozorujete?
\end{ulohav}

\begin{uloha}
Vymyslete rychlý způsob, jak mezi sebou převádět zápisy čísel ve dvojkové a čtyřkové soustavě (a posléze také šestnáctkové).
\end{uloha}

\begin{suloha}
Aniž byť jen pomyslíte na desítkovou soustavu, převeďte $101011_2$ do trojkové soustavy.\vyslplain{1121}
\end{suloha}

%\end{document}

\newpage
\parindent=0pt
\parskip=\smallskipamount
\def\printvysl#1#2{\textbf{#1.}\ #2\par}
\def\printalphvysl#1#2#3{\textbf{#1}(#2)\ #3\par}
\vysld

\end{document}

