\documentclass[11pt,a4paper]{article}
\usepackage[margin=.5in]{geometry}
\usepackage[utf8]{inputenc}
\usepackage[IL2]{fontenc}
\usepackage[czech]{babel}
\usepackage{microtype}
\usepackage{amssymb}
\usepackage{amsthm}
\usepackage{amsmath}
\usepackage{xcolor}
\usepackage{graphicx}

\usepackage[inline]{enumitem}

\newcommand{\R}{\mathbb{R}}

\DeclareMathOperator{\tg}{tg}
\DeclareMathOperator{\cotg}{cotg}

\setlist[enumerate]{label={(\alph*)},topsep=\smallskipamount,itemsep=\smallskipamount}
\setlist[itemize]{topsep=\smallskipamount,noitemsep}


\theoremstyle{definition}
\newtheorem{uloha}{Úloha}
\newtheorem*{bonus}{Bonus}

\pagestyle{empty}

\let\ee\expandafter

\def\vysld{}
\let\printvysl\relax
\let\printalphvysl\relax

\newenvironment{amatrix}[1]{%
  \left(\begin{array}{@{}*{#1}{c}|c@{}}
}{%
  \end{array}\right)
}

\makeatletter
\def\vyslplain#1{\ee\ee\ee\gdef\ee\ee\ee\vysld\ee\ee\ee{\ee\vysld\ee\printvysl\ee{\the\c@uloha}{#1}}}
\let\vysl\vyslplain

\def\locvysl#1{\ee\gdef\ee\locvysld\ee{\locvysld\item #1}}
\let\lv\locvysl

\newenvironment{ulohav}[1][]{\begin{uloha}[#1]\gdef\locvysld{\begin{enumerate*}}}{\ee\vyslplain\ee{\locvysld\end{enumerate*}}\end{uloha}}
\newenvironment{sulohav}[1][]{\begin{suloha}[#1]\gdef\locvysld{\begin{enumerate*}}}{\ee\vyslplain\ee{\locvysld\end{enumerate*}}\end{suloha}}

\makeatother

\begin{document}

\section*{Cvičení na soustavy rovnic}

\begin{uloha}\label{soustavy}
Nalezněte množiny řešení následujících soustav rovnic:
\[
\text{(a)}\ 
\begin{amatrix}{2}3&5&2 \\ 300&500&200\end{amatrix},\quad
\text{(b)}\ 
\begin{amatrix}{3}0&1&1&2 \\1&2&3&1\\1&1&2&1\end{amatrix},\quad
\text{(c)}\ 
\begin{amatrix}{4}1&1&3&2&2 \\1&0&1&1&1\\2&1&-1&1&0\\2&2&1&2&1\end{amatrix} ,\quad
\text{(d)}\ 
\begin{amatrix}{6}1&1&0&1&2&-1&1 \\2&2&1&0&-1&1&3\end{amatrix}.
\]
\vysl{(a) $\{(\frac23 - \frac53t; t) \mid t \in \R\}$ (b) $\emptyset$, (c) $\bigl\{(\frac25 -\frac35t; -\frac15 -\frac15t; \frac35 - \frac25t; t) \mid t \in \R\bigr\}$,\\ (d) $\bigl\{(1-a-b-2c+d; a; 1+2b+5c-3d; b; c; d) \mid a, b, c, d \in \R\bigr\}$}
\end{uloha}


\begin{ulohav}
Rozhodněte, pro které dvojice reálných čísel $(p, q)$ bude mít soustava daná maticí
\[ \begin{amatrix}{3}
-1 & 2 & 0 & 2 \\
1 & 0 & 3 & -1 \\
1 & 2 & p & q
\end{amatrix}\]
\begin{enumerate*}
    \item nekonečně mnoho řešení,\lv{$p = 6$, $q = 0$,}
    \item žádné řešení,\lv{$p = 6$, $q \neq 0$,}
    \item právě jedno řešení.\lv{$p \neq 6$, $q$ může být cokoliv}
\end{enumerate*}
\end{ulohav}


\begin{ulohav} % tezka uloha, asi spatne pochopitelna bez analyticke geometrie
Rozhodněte, zda se množiny $K$ a $L$ rovnají, pokud
\begin{enumerate}
    \item $K = \{(1; t) \mid t \in \R\}$, $L = \{(1; -s) \mid s \in \R\}$,\lv{ano}
    \item $K = \{(t; t) \mid t \in \R\}$, $L = \{(2s; 2s) \mid s \in \R\}$,\lv{ano}
    \item $K = \{(1 + t; 1 - t) \mid t \in \R\}$, $L = \{(2 + t; -t) \mid t \in \R\}$,\lv{ano}
    \item $K = \{(2; 3-t; t) \mid t \in \R\}$, $L = \{(3; 2-t; t) \mid t \in \R\}$,\lv{ne}
    \item $K = \{(-1 + 2t; t; 2+2t) \mid t \in \R\}$, $L = \{(3-p; 2-\frac p2; 6 - p) \mid p \in \R \}$.\lv{ne}
\end{enumerate}
\end{ulohav}


\begin{uloha}
Nalezněte zápis množiny řešení části (a) Úlohy \ref{soustavy} takový, že bude obsahovat jen celá čísla.
\vysl{Např. $\{(-1-5t; 1+3t) \mid t \in \R\}$}
\end{uloha}


\begin{uloha}
Z celkem jednadvaceti dvorních dam některé hrají na loutnu a některé na harfu.
Deset jich hraje na loutnu. Těch, co hrají na harfu, je dvakrát více než těch, co hrají \emph{pouze} na loutnu. Určete, kolik jich hraje na harfu; nalezněte všechny možnosti.
\vysl{$h \in \{8; 10; 12; 14\}$}
\end{uloha}


\begin{ulohav}
Vymyslete soustavu $n$ lineárních rovnic, jejíž množina řešení bude $K$, pokud
\begin{enumerate}
    \item $n = 2$, $K = \{(1+t; 1-2t) \mid t \in \R\}$,\lv{např. $\begin{amatrix}{2} 2 & 1 & 3 \\ 2 & 1 & 3 \end{amatrix}$,}
    \item $n = 3$, $K = \{(1-2s; 2; 3s) \mid s \in \R \}$,\lv{např. $\begin{amatrix}{3} 0 & 1 & 0 & 2 \\ 3 & 0 & 2 & 3 \\ 0 & 0 & 0 & 0 \end{amatrix}$,}
    \item $n = 2$, $K = \{(t-u; t+u; u; t) \mid t, u \in \R\}$\lv{např. $\begin{amatrix}{3} 0 & 1 & 0 & 2 \\ 3 & 0 & 2 & 3 \\ 0 & 0 & 0 & 0 \end{amatrix}$.}
\end{enumerate}
\end{ulohav}


\begin{uloha}
Pavouk má 8 očí, 8 nohou a žádné rohy.
Nosorožec má 2 oči, 4 nohy a 1 roh.
Hlemýžď má 2 oči, žádné nohy a 2 rohy.
Jacksonův chameleon má 2 oči, 4 nohy a 3 rohy.
V matfyzácké zoo mají dohromady 14 očí, 20 nohou a 9 rohů. Kolik kterých druhů zvířat se tam nachází? Najděte všechny možnosti. Předpokládáme, že žádná jiná než zmiňovaná zvířata se zde nevyskytují a že počet zvířat je nezáporné celé číslo.
\vysl{$(p, n, h, c) \in \{ (0, 5, 2, 0),\> (1, 0, 0, 3) \}$}
\end{uloha}


\begin{uloha}
Nalezněte předpis kvadratické funkce $y=ax^2+bx+c$, jejíž graf prochází body $(-1;0)$, $(1;5)$ a $(2;2)$.
\vysl{$y = -\frac{1}{6}(11x^2-15x-26)$}
\end{uloha}


\newpage
\parindent=0pt
\parskip=\smallskipamount
\def\printvysl#1#2{\textbf{#1.}\ #2\par}
\def\printalphvysl#1#2#3{\textbf{#1}(#2)\ #3\par}
\vysld


\end{document}