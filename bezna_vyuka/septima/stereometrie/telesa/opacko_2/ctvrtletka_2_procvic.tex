\documentclass[10pt,a5paper]{extarticle}
\usepackage[margin=1cm]{geometry}
\usepackage[utf8]{inputenc}
\usepackage[IL2]{fontenc}
\usepackage[czech]{babel}
\usepackage{microtype}
\usepackage{amssymb}
\usepackage{amsthm}
\usepackage{amsmath}
\usepackage{xcolor}
\usepackage{graphicx}

\usepackage[inline]{enumitem}

\newcommand{\R}{\mathbb{R}}

\DeclareMathOperator{\arctg}{arctg}

\setlist[enumerate]{label={(\alph*)},topsep=\smallskipamount,itemsep=\smallskipamount,parsep=0pt,itemjoin={\qquad}}
\setlist[itemize]{topsep=\smallskipamount,noitemsep}

\def\tisk{%
\newbox\shipouthackbox
\pdfpagewidth=2\pdfpagewidth
\let\oldshipout=\shipout
\def\shipout{\afterassignment\zdvojtmp \setbox\shipouthackbox=}%
\def\zdvojtmp{\aftergroup\zdvoj}%
\def\zdvoj{%
    \oldshipout\vbox{\hbox{%
        \copy\shipouthackbox
        \hskip\dimexpr .5\pdfpagewidth-\wd\shipouthackbox\relax
        \box\shipouthackbox
    }}%
}}%

\newtheorem*{poz}{Pozorování}

\theoremstyle{definition}
\newtheorem{uloha}{Úloha}
\newtheorem{suloha}[uloha]{\llap{$\star$ }Úloha}
\newtheorem*{bonus}{Bonus}
\newtheorem*{defn}{Definice}

\pagestyle{empty}

\newcommand{\sitem}{\item[\addtocounter{enumi}{1}{\footnotesize$\star$}(\alph{enumi})]}

\let\ee\expandafter

\def\vysld{}
\let\printvysl\relax
\let\printalphvysl\relax

\makeatletter
\long\def\vyslplain#1{\ee\ee\ee\gdef\ee\ee\ee\vysld\ee\ee\ee{\ee\vysld\ee\printvysl\ee{\the\c@uloha}{#1}}}

\def\locvysl#1{\ee\gdef\ee\locvysld\ee{\locvysld\item #1}}
\let\lv\locvysl

\newenvironment{ulohav}[1][]{\begin{uloha}[#1]\gdef\locvysld{\begin{enumerate}}}{\ee\vyslplain\ee{\locvysld\end{enumerate}}\end{uloha}}
\newenvironment{sulohav}[1][]{\begin{suloha}[#1]\gdef\locvysld{\begin{enumerate*}}}{\ee\vyslplain\ee{\locvysld\end{enumerate*}}\end{suloha}}

\makeatother

\let\vysl\vyslplain

\begin{document}

\section*{11. Opáčko na již druhou čtvrtletku aneb Vesměs vykrádačka Petákové}


\mathcode`\,="013B


\begin{uloha}% pet 74
Určete rozměry válcové nádoby o objemu 5 litrů, jestliže výška nádoby se rovná polovině průměru podstavy.\vysl{$v = r = \sqrt[3]{\frac 5\pi}$\,dm}
\end{uloha}

\begin{uloha}% pet 68
Kužel o objemu $V$ rozdělíme na 3 tělesa podle tří rovin rovnoběžných s~podstavou tak, že výsledná tělesa mají stejnou výšku (tj. třetinu původní výšky). Vypočítejte, v jakém poměru jsou jejich objemy.\vysl{$19:7:1$}
\end{uloha}


\begin{uloha}
Kolmý čtyřboký jehlan má za podstavu obdélník o rozměrech $4 \times 5$. Jestliže boční stěna, jejíž spodní hrana má délku 4, svírá s podstavou úhel $60^\circ$, jaký úhel svírá boční stěna se spodní hranou délky 5?\vysl{$\arctg\frac{\frac{5\sqrt3}{2}}{2} \doteq 65^\circ13'$}
\end{uloha}


\begin{uloha}
Rotační kužel má výšku rovnou průměru podstavy a jeho objem je $\frac{125\pi}{12}$. Určete jeho povrch.\vysl{$r = \frac52$, $v = 5$, takže $s = \frac{5\sqrt5}{2}$ a $S = \pi \frac52 (\frac52 + \frac{5\sqrt5}{2}) = \frac{25}{4} \left(\sqrt{5}+1\right) \pi$}
\end{uloha}

\begin{ulohav}
Kulová úseč o výšce $3$ má objem $54 \pi$.
\begin{enumerate}
    \item Určete poloměr koule, ze které úseč vzešla.\lv{7}
    \item Určete, kolik procent povrchu \uv{původní} koule se nacházelo na této úseči.\lv{$\frac{3}{14} \doteq 21,4\,\%$}
\end{enumerate}
\end{ulohav}

\begin{uloha}
Komolý jehlan o výšce 3 má za obě podstavy čtverce, přičemž horní má poloviční délku hrany oproti spodnímu. Určete ony délky hran, pokud jeho objem je $63$.\vysl{$3$ a $6$}
\end{uloha}


\begin{uloha}
Podstavou pravidelného čtyřbokého jehlanu $ABCDV$ je čtverec o straně~$1$; výška jehlanu je $9$. Určete objem jehlanu $ABCDS_{AV}$.\vysl{$\frac{3}{2}$}
\end{uloha}

\begin{uloha}% pet 81
Vypočítejte povrch a objem čočky, která vznikne průnikem dvou koulí o poloměrech 3 a 4. Vzdálenost středů koulí je 5.
\vysl{Výška úseče z menší koule je $\frac65$, z větší $\frac45$, tudíž povrchy vrchlíků jsou $\frac{36}{5}\pi$ a $\frac{32}5\pi$, dohromady povrch $\frac{68}{5}\pi$, objemy jsou $\frac{468}{125} \pi$ a $\frac{896}{375}\pi$, dohromady $\frac{92}{15}\pi$.}
\end{uloha}


\newpage
\parindent=0pt
\parskip=\smallskipamount
\def\printvysl#1#2{\textbf{#1.}\ #2\par}
\vysld


\end{document}