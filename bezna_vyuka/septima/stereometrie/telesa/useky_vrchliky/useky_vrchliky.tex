\documentclass[10pt,a5paper]{article}
\usepackage[margin=1.2cm]{geometry}
\usepackage[utf8]{inputenc}
\usepackage[IL2]{fontenc}
\usepackage[czech]{babel}
\usepackage{microtype}
\usepackage{amssymb}
\usepackage{amsthm}
\usepackage{amsmath}
\usepackage{xcolor}
\usepackage{graphicx}

\usepackage[inline]{enumitem}

\newcommand{\R}{\mathbb{R}}

\DeclareMathOperator{\tg}{tg}
\DeclareMathOperator{\cotg}{cotg}

\setlist[enumerate]{label={(\alph*)},topsep=\smallskipamount,itemsep=\smallskipamount,parsep=0pt}
\setlist[itemize]{topsep=\smallskipamount,noitemsep}

\def\tisk{%
\newbox\shipouthackbox
\pdfpagewidth=2\pdfpagewidth
\let\oldshipout=\shipout
\def\shipout{\afterassignment\zdvojtmp \setbox\shipouthackbox=}%
\def\zdvojtmp{\aftergroup\zdvoj}%
\def\zdvoj{%
    \oldshipout\vbox{\hbox{%
        \copy\shipouthackbox
        \hskip\dimexpr .5\pdfpagewidth-\wd\shipouthackbox\relax
        \box\shipouthackbox
    }}%
}}%



\newtheorem*{poz}{Pozorování}

\theoremstyle{definition}
\newtheorem{uloha}{Úloha}
\newtheorem{suloha}[uloha]{\llap{$\star$ }Úloha}
\newtheorem*{bonus}{Bonus}
\newtheorem*{defn}{Definice}

\pagestyle{empty}

\let\ee\expandafter

\def\vysld{}
\let\printvysl\relax
\let\printalphvysl\relax

\makeatletter
\long\def\vyslplain#1{\ee\ee\ee\gdef\ee\ee\ee\vysld\ee\ee\ee{\ee\vysld\ee\printvysl\ee{\the\c@uloha}{#1}}}

\def\locvysl#1{\ee\gdef\ee\locvysld\ee{\locvysld\item #1}}
\let\lv\locvysl

\newenvironment{ulohav}[1][]{\begin{uloha}[#1]\gdef\locvysld{\begin{enumerate*}}}{\ee\vyslplain\ee{\locvysld\end{enumerate*}}\end{uloha}}
\newenvironment{sulohav}[1][]{\begin{suloha}[#1]\gdef\locvysld{\begin{enumerate*}}}{\ee\vyslplain\ee{\locvysld\end{enumerate*}}\end{suloha}}

\makeatother
\def\cm{\mathrm{cm}}

\begin{document}

%\tisk

\section*{5. Úseky, vrchlíky a spol.}

Pro připomenutí vzorce z Wikipedie: 
\begin{itemize}
    \item objem kulové úseče $V = \frac13 \pi v^2 (3r - v) = \frac16 \pi v (3\varrho^2 + v^2)$,
    \item povrch kulového vrchlíku $S = 2\pi r v = \pi(\varrho^2 + v^2)$.
\end{itemize}


\mathcode`\,="013B


\begin{uloha}
Jaký povrch a objem bude mít činka, která vznikla spojením dvou koulí o poloměru 5\,cm válcem o poloměru podstavy 3\,cm a výšce 15\,cm? Jakou bude mít hmotnost, jestliže je celá z železa o hustotě $7,874\,\mathrm{g}\cdot\cm^{-3}$?
\vyslplain{povrch: $270\pi\,\cm^2 \doteq 848,23\,\cm^2$, objem: $459\pi\,\cm^3 \doteq 1441,99\,\cm^3$, hmotnost: cca 11\,354\,g}
\[ \includegraphics{cinka.pdf} \]
\end{uloha}


\begin{uloha} % realisticky.cz s jinými čísly
Vodojem kulového tvaru je naplněn z jedné poloviny svého objemu a obsahuje $315\,\mathrm m^{3}$ vody. Určete jeho poloměr. Jaký je povrch vodojemu? Kolik bude stát jeho natření barvou, jestliže 4,5\,kg barvy o vydatnosti $26\,\mathrm{m}^2 \cdot \mathrm{kg}^{-1}$ stojí 780\,Kč?
\vyslplain{poloměr $3 \sqrt[3]{\frac{35}{2\pi}} \doteq 5,32\,\mathrm m$, povrch $18\sqrt[3]{35^2 \cdot 2\pi} \doteq 355,4\,\mathrm m^2$, cena cca 2369\,Kč (pokud můžeme kupovat barvu jen po baleních po 4,5\,kg, tak potom jsou potřeba 4 balení celkem za 3120\,Kč)}
\end{uloha}


\begin{uloha}
Jaký poloměr měl kulový pomeranč, pokud objem úseče o výšce 2\,cm je $50\,\cm^3$?
\vyslplain{$\frac{75+4 \pi }{6 \pi } \doteq 4,65\, \cm$}
\end{uloha}


\begin{ulohav} % realisticky.cz
Země je přibližně koule s poloměrem 6378\,km.
\begin{enumerate}
    \item Určete plochu zemského povrchu ležícího v severním mírném pásmu (mezi obratníkem $23^\circ 27'$ a polárním kruhem $66^\circ 33'$).\lv{cca $1,3277\cdot 10^8\,\mathrm{km}^2$}
    \item Kolik procent zemského povrchu onen mírný pás tvoří?\lv{cca 25,973\,\%, pokud počítáme jen severní mírný pás, jinak dvakrát tolik}
\end{enumerate}
\end{ulohav}


\begin{uloha}[k této úloze se mi nechtělo vymýšlet čísla]
Rozmyslete si, jak by se počítal objem létajícího talíře:
\[ \includegraphics{ufo.pdf} \]
\vyslplain{Spočteme objem úseče, která tvoří \uv{trup}, od ní odečteme tu malou úseč, která zasahuje do \uv{kabiny}, a k tomu by se přičetl objem \uv{kabiny}.}
\end{uloha}


\newpage
\parindent=0pt
\parskip=\smallskipamount
\def\printvysl#1#2{\textbf{#1.}\ #2\par}
\vysld


\end{document}