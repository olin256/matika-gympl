\documentclass[10pt,a5paper]{article}
\usepackage[margin=1cm]{geometry}
\usepackage[utf8]{inputenc}
\usepackage[IL2]{fontenc}
\usepackage[czech]{babel}
\usepackage{microtype}
\usepackage{amssymb}
\usepackage{amsthm}
\usepackage{amsmath}
\usepackage{xcolor}
\usepackage{graphicx}

\usepackage[inline]{enumitem}

\newcommand{\R}{\mathbb{R}}

\DeclareMathOperator{\tg}{tg}
\DeclareMathOperator{\cotg}{cotg}

\setlist[enumerate]{label={(\alph*)},topsep=\smallskipamount,itemsep=\smallskipamount,parsep=0pt}
\setlist[itemize]{topsep=\smallskipamount,noitemsep}

\def\tisk{%
\newbox\shipouthackbox
\pdfpagewidth=2\pdfpagewidth
\let\oldshipout=\shipout
\def\shipout{\afterassignment\zdvojtmp \setbox\shipouthackbox=}%
\def\zdvojtmp{\aftergroup\zdvoj}%
\def\zdvoj{%
    \oldshipout\vbox{\hbox{%
        \copy\shipouthackbox
        \hskip\dimexpr .5\pdfpagewidth-\wd\shipouthackbox\relax
        \box\shipouthackbox
    }}%
}}%



\newtheorem*{poz}{Pozorování}

\theoremstyle{definition}
\newtheorem{uloha}{Úloha}
\newtheorem{suloha}[uloha]{\llap{$\star$ }Úloha}
\newtheorem*{bonus}{Bonus}
\newtheorem*{defn}{Definice}

\pagestyle{empty}

\DeclareMathOperator{\arctg}{arctg}

\let\=\doteq
\let\ee\expandafter

\def\vysld{}
\let\printvysl\relax
\let\printalphvysl\relax

\makeatletter
\def\vyslplain#1{\ee\ee\ee\gdef\ee\ee\ee\vysld\ee\ee\ee{\ee\vysld\ee\printvysl\ee{\the\c@uloha}{#1}}}

\def\locvysl#1{\ee\gdef\ee\locvysld\ee{\locvysld\item #1}}
\let\lv\locvysl

\newenvironment{ulohav}[1][]{\begin{uloha}[#1]\gdef\locvysld{\begin{enumerate*}}}{\ee\vyslplain\ee{\locvysld\end{enumerate*}}\end{uloha}}
\newenvironment{sulohav}[1][]{\begin{suloha}[#1]\gdef\locvysld{\begin{enumerate*}}}{\ee\vyslplain\ee{\locvysld\end{enumerate*}}\end{suloha}}

\def\stitem{\@noitemargtrue\@item[$\star$ \@itemlabel]}

\makeatother

\begin{document}

%\tisk

\section*{1. Odchylky přímek od rovin}

Ve všech úlohách značí $S_{XY}$ střed úsečky $XY$.


\begin{ulohav}[Motivační]
V krychli $ABCDEFGH$ určete odchylku následujících přímek a rovin:
\begin{enumerate*}
    \item $EG$ a $FGH$,\lv{$0^\circ$}
    \item $AB$ a $FGH$,\lv{$0^\circ$}
    \item $CD$ a $EDH$,\lv{$90^\circ$}
    \item $BF$ a $ABC$,\lv{$90^\circ$}
    \item $BD$ a $EDH$,\lv{$45^\circ$}
    \item $GS_{BF}$ a $ABE$,\lv{$\arctg(2) \doteq 63^\circ 26'$}
    \item $GS_{BF}$ a $FGH$.\lv{$\arctg\bigl(\frac12\bigr) \doteq 26^\circ 34'$}
\end{enumerate*}
\end{ulohav}


\begin{ulohav}[Z pondělí]
V krychli $ABCDEFGH$ určete odchylku následujících přímek a rovin:
\begin{enumerate*}
    \item $S_{EC}$ a $ABC$,\lv{$\arctg\bigl(\frac1{\sqrt2}\bigr) \= 35^\circ 16'$}
    \item $CF$ a $BCH$,\lv{$30^\circ$}
    \item $FH$ a $ACH$,\lv{$\arctg(\sqrt2) \= 54^\circ 44'$}
    \item $CH$ a $ADH$,\lv{$45^\circ$}
    \item $S_{EH}C$ a $ABC$.\lv{$\arctg\bigl(\frac2{\sqrt5}\bigr) \= 41^\circ 49'$}
\end{enumerate*}
\end{ulohav}


\begin{sulohav}[Skutečné krychlové výzvy]
V krychli $ABCDEFGH$ určete odchylku následujících přímek a rovin:
\begin{enumerate*}
    \item $AG$ a $EBC$,\lv{$\arctg(2) \= 63^\circ 26'$}
    \item $S_{AE}G$ a $EBC$.\lv{$45^\circ$}
\end{enumerate*}
\end{sulohav}

\begin{uloha}
V kvádru $ABCDEFGH$, kde $|AB| = 5$, $|BC|=4$, $|AE|=3$ je $S$ střed hodní podstavy ($EFGH$). Určete odchylku přímky $BS$ od roviny $BCG$.\vyslplain{$\=34^\circ44'$}
\end{uloha}


\begin{ulohav}
Je dán jehlan $ABCDV$, kde $|AB| = a = 4$, $|BC| = b = 3$ a $v = 5$. Určete odchylku
\begin{enumerate*}
    \item $BV$ od $ABC$,\lv{$\arctg2 \= 63^\circ26'$}
    \item $AS_{CV}$ a $ABC$,\lv{$\arctg \frac23 \= 33^\circ 41'$}
    \item $AB$ a $BCV$,\lv{$\arctg\frac52 \= 68^\circ 12'$}
    \item[\refstepcounter{enumi}$\star$ (\alph{enumi})] $AV$ a $BCV$.\lv{$\arcsin \frac{8}{\sqrt{145}} \= 41^\circ 38'$} % https://www.geogebra.org/3d/z77ebkjw
\end{enumerate*}
\end{ulohav}


\begin{suloha} % 3d důkaz toho, že se výšky protínají v jednom bodě https://www.cut-the-knot.org/Curriculum/Geometry/GeoGebra/3DAltitudes.shtml
V krychli $ABCDEFGH$ zvolme body $X$, $Y$, $Z$ libovolně uvnitř stran $AB$, $AD$ a $AE$. Nechť $O$ je kolmý průmět $A$ do roviny $XYZ$.
\begin{enumerate*}
    \item Dokažte, že přímka $XO$ je kolmá na přímku $YZ$ (a podobně pro další dva body).
    \item Co je zač $O$ v~trojúhelníku $XYZ$? Co jsme právě dokázali?
\end{enumerate*}
\end{suloha}


\begin{suloha}[Nijak nesouvisející s předchozím]
Uzavřená lomená čára, která sama sebe neprotíná, prochází všemi vrcholy určité krychle a láme se pouze v nich. Dokažte, že alespoň jeden segment oné čáry se shoduje s hranou oné krychle.
\end{suloha}


\newpage
\parindent=0pt
\parskip=\smallskipamount
\rightskip=0pt plus1fil\relax
\def\printvysl#1#2{\textbf{#1.}\ #2\par}
\vysld


\end{document}