\documentclass[10pt,a5paper]{extarticle}
\usepackage[margin=1cm]{geometry}
\usepackage[utf8]{inputenc}
\usepackage[IL2]{fontenc}
\usepackage[czech]{babel}
\usepackage{microtype}
\usepackage{amssymb}
\usepackage{amsthm}
\usepackage{amsmath}
\usepackage{xcolor}
\usepackage{graphicx}
\usepackage{wasysym}
\usepackage{tikz}

\newcommand{\centeredgraphics}[2][]{\[\includegraphics[#1]{#2}\]}

\usepackage[inline]{enumitem}

\newcommand{\R}{\mathbb{R}}

\newcommand{\hint}[1]{{\color{gray}\footnotesize\noindent(Nápověda: #1)}}

\setlist[enumerate]{label={(\alph*)},topsep=\smallskipamount,itemsep=\smallskipamount,parsep=0pt,itemjoin={\qquad}}
\setlist[itemize]{topsep=\smallskipamount,noitemsep}

\def\tisk{%
\newbox\shipouthackbox
\pdfpagewidth=2\pdfpagewidth
\let\oldshipout=\shipout
\def\shipout{\afterassignment\zdvojtmp \setbox\shipouthackbox=}%
\def\zdvojtmp{\aftergroup\zdvoj}%
\def\zdvoj{%
    \oldshipout\vbox{\hbox{%
        \copy\shipouthackbox
        \hskip\dimexpr .5\pdfpagewidth-\wd\shipouthackbox\relax
        \box\shipouthackbox
    }}%
}}%



\newtheorem*{poz}{Pozorování}

\theoremstyle{definition}
\newtheorem{uloha}{\atr Úloha}
\newtheorem{suloha}[uloha]{\llap{$\star$ }Úloha}
\newtheorem*{bonus}{Bonus}
\newtheorem*{defn}{Definice}

\pagestyle{empty}

\let\ee\expandafter

\def\vysld{}
\let\printvysl\relax

\makeatletter
\long\def\vyslplain#1{\ee\ee\ee\gdef\ee\ee\ee\vysld\ee\ee\ee{\ee\vysld\ee\printvysl\ee{\the\c@uloha}{#1}}}
\let\vysl\vyslplain

\def\locvysl#1{\ee\gdef\ee\locvysld\ee{\locvysld\item #1}}
\let\lv\locvysl

\newenvironment{ulohav}[1][]{\begin{uloha}[#1]\gdef\locvysld{\begin{enumerate*}}}{\ee\vyslplain\ee{\locvysld\end{enumerate*}}\end{uloha}}
\def\stitem{\@noitemargtrue\@item[$\star$ \@itemlabel]}

\makeatother

\def\atr{}
\def\basic{\def\atr{\llap{\mdseries$\sun$ }\gdef\atr{}}}
\def\interest{\def\atr{\llap{$\star$ }\gdef\atr{}}}
\def\iinterest{\def\atr{\llap{$\star\star$ }\gdef\atr{}}}

\let\results\newpage
\let\endresults\relax

\def\resultssame{%
    \long\def\results##1\endresults{%
        \vfill\noindent\rotatebox{180}{\vbox{##1}}%
    }%
}

\begin{document}

%\resultssame
%\tisk

\section*{C3$\frac{\mathbf 1}{\mathbf 2}$. Uspořádání i neuspořádání -- appendix}


\begin{ulohav}
Kolik různých (kladných) dělitelů má číslo $2^5 \cdot 3^6 \cdot 5^3$, které
\begin{enumerate}
    \item jsou dělitelné 6?\lv{$5\cdot 6 \cdot 4 = 120$}
    \item nejsou dělitelné 24?\lv{$6\cdot 7 \cdot 4 - 3 \cdot 6 \cdot 4 = 96$}
    \item jsou dělitelné alespoň dvěma z prvočísel?\lv{$6\cdot 7 \cdot 4 - 5 - 6 - 3 - 1 = 153$}% ($-1$ protože jedničku jsme odečetli $3\times$)}
\end{enumerate}
\end{ulohav}


\begin{uloha}
Identifikátor každého videa na YouTube je řetězec jedenácti znaků z~mno\-žiny a--z, A--Z, 0--9 a - nebo \_ (celkem 64 možností). Kolik existuje takovýchto identifikátorů, které neobsahují souvislý podřetězec \texttt{youtube} (s jakoukoliv velikostí písmen)? Tedy např. \texttt{12yOuTuBE34} nebo \texttt{yYouTubeou} nejsou povolené identifikátory, ale \texttt{youX\_\_Xtube} ano.\vysl{$64^{11} - 5 \cdot 2^7 \cdot 64^4 = 73\,786\,976\,284\,100\,788\,224$}
\end{uloha}


\addtocounter{uloha}{1}


\begin{ulohav}
Proces výběru vlajky pro Tramtárii nakonec proběhne následovně: ve všelidovém hlasování se všechen lid rozhodne pro trojici různých barev (z osmi). Posléze expertní komise tyto barvy seřadí do vlajky.
\begin{enumerate}
    \item Kolik voleb bude v hlasování?\lv{$\frac{8\cdot7\cdot6}{3!} = 56$}
    \item Kolik možností sestavení vlajky bude potom mít komise?\lv{$3! = 6$}
\end{enumerate}
\end{ulohav}

\setcounter{uloha}{7}

\begin{ulohav}
Alice nakonec nebude mít tolik času, kolik plánovala, a ze dvanácti památek stihne jenom šest. Chce si tedy nejprve vybrat oněch šest bez ohledu na to, v jakém pořadí je potom navštíví.
\begin{enumerate}
    \item Kolika způsoby to může provést?\lv{$\frac{12\cdot11\cdot10\cdot9\cdot8\cdot7}{6!} = 924$}
    \item Co když chce navštívit tři (ze čtyř) v Čechách a pak tři (z osmi) na Moravě?\lv{$4 \cdot \frac{8 \cdot7\cdot6}{3!} = 224$}
    \item Co když chce navštívit \emph{aspoň} tři na Moravě?\lv{$224+ 420+ 224+ 28 = 896$}
\end{enumerate}
\end{ulohav}


\results
\parindent=0pt
\parskip=\smallskipamount
\rightskip=0pt plus1fil\relax
\def\printvysl#1#2{\textbf{#1.} #2\par}
\vysld
\endresults


\end{document}