\documentclass[12pt,a5paper]{extarticle}
\usepackage[margin=1cm]{geometry}
\usepackage[utf8]{inputenc}
\usepackage[IL2]{fontenc}
\usepackage[czech]{babel}
\usepackage{microtype}
\usepackage{amssymb}
\usepackage{amsthm}
\usepackage{amsmath}
\usepackage{xcolor}
\usepackage{graphicx}
\usepackage{wasysym}
\usepackage{multicol}

\usepackage[inline]{enumitem}

\newcommand{\R}{\mathbb{R}}

\newcommand{\hint}[1]{{\color{gray}\footnotesize\noindent(Nápověda: #1)}}

\setlist[enumerate]{label={(\alph*)},topsep=\smallskipamount,itemsep=\smallskipamount,parsep=0pt}
\setlist[itemize]{topsep=\smallskipamount,noitemsep}

\def\tisk{%
\newbox\shipouthackbox
\pdfpagewidth=2\pdfpagewidth
\let\oldshipout=\shipout
\def\shipout{\afterassignment\zdvojtmp \setbox\shipouthackbox=}%
\def\zdvojtmp{\aftergroup\zdvoj}%
\def\zdvoj{%
    \oldshipout\vbox{\hbox{%
        \copy\shipouthackbox
        \hskip\dimexpr .5\pdfpagewidth-\wd\shipouthackbox\relax
        \box\shipouthackbox
    }}%
}}%


\newtheorem*{poz}{Pozorování}

\theoremstyle{definition}
\newtheorem{uloha}{\atr Úloha}
\newtheorem{suloha}[uloha]{\llap{$\star$ }Úloha}
\newtheorem*{bonus}{Bonus}
\newtheorem*{defn}{Definice}

\pagestyle{empty}

\let\ee\expandafter

\def\vysld{}
\let\printvysl\relax

\makeatletter
\long\def\vyslplain#1{\ee\ee\ee\gdef\ee\ee\ee\vysld\ee\ee\ee{\ee\vysld\ee\printvysl\ee{\the\c@uloha}{#1}}}
\let\vysl\vyslplain

\def\locvysl#1{\ee\gdef\ee\locvysld\ee{\locvysld\item #1}}
\let\lv\locvysl

\newenvironment{ulohav}[1][]{\begin{uloha}[#1]\gdef\locvysld{\begin{enumerate}}}{\ee\vyslplain\ee{\locvysld\end{enumerate}}\end{uloha}}
\def\stitem{\@noitemargtrue\@item[$\star$ \@itemlabel]}

\makeatother

\def\atr{}
\def\basic{\def\atr{\llap{\mdseries$\sun$ }\gdef\atr{}}}
\def\interest{\def\atr{\llap{$\star$ }\gdef\atr{}}}
\def\iinterest{\def\atr{\llap{$\star\star$ }\gdef\atr{}}}

\begin{document}

% \tisk


\section*{16. O jednom trojúhelníku}

\emph{Všechny úlohy se postupně odkazují na tytéž body, přímky atd.}

\medskip
\noindent Mějme body $A[-3;2]$, $B[5;4]$ a přímky $p\colon x + 3y - 3 = 0$ a $q\colon x=2+t,\ y=-2+2t$, $t \in \R$.

\begin{uloha}
Ověřte, že bod $A$ leží na přímce $p$ a bod $B$ na přímce $q$.\vysl{Ano, leží.}
\end{uloha}

\begin{uloha}
Nalezněte souřadnice průsečíku přímek $p$ a $q$; nadále se na něj budeme odkazovat jako na bod $C$.\vysl{$C[3;0]$}
\end{uloha}

\begin{uloha}
Určete obecnou rovnici přímky $AB$.\vysl{např. $x-4y+11=0$}
\end{uloha}

\begin{uloha}\label{opsiste}
Určete souřadnice středu kružnice opsané trojúhelníku $ABC$. Návod: nalezněte rovnice (obecné či parametrické, to je na vás) os dvou stran (i volba stran je na vás) a spočtěte jejich průsečík. (Výsledky obsahují pro kontrolu obecné rovnice všech tří os.)\vysl{osa $AB$: $4x+y-7=0$; osa $BC$: $x+2y-8=0$; osa $CA$: $3x-y+1=0$\\[\smallskipamount]střed kružnice opsané má souřadnice $\bigl[\frac67; \frac{25}{7}\bigr]$}
\end{uloha}

\begin{uloha}
Spočtěte poloměr kružnice opsané trojúhelníku $ABC$.\vysl{$\frac57 \sqrt{34}$}
\end{uloha}

\begin{uloha}
Určete souřadnice průsečíku výšek trojúhelníku $ABC$ (postup je podobný jako v Úloze \ref{opsiste}).\vysl{$\bigl[\frac{23}7; -\frac{8}{7}\bigr]$}
\end{uloha}

\interest
\begin{uloha}\label{osa-uhlu}
Určete rovnici (parametrickou nebo obecnou) osy úhlu $BAC$.\vysl{parametrická: $x = -3 + \bigl(\frac{4}{\sqrt{17}} + \frac{3}{\sqrt{10}}\bigr)t,\ y = 2 + \bigl(\frac{1}{\sqrt{17}} - \frac{1}{\sqrt{10}}\bigr)t$\\[\smallskipamount]obecná: $\bigl(\frac{1}{\sqrt{17}} - \frac{1}{\sqrt{10}}\bigr)x - \bigl(\frac{4}{\sqrt{17}} + \frac{3}{\sqrt{10}}\bigr) y + \bigl(\frac{11}{\sqrt{17}} + \frac{3}{\sqrt{10}}\bigr) = 0$}
\end{uloha}

\interest
\begin{uloha}[Pokud jste řešili Úlohu \ref{osa-uhlu}]
Rozmyslete si, jak by se počítaly souřadnice středu kružnice vepsané trojúhelníku $ABC$ (počítat je nemusíte).\vysl{Jedna možnost je metodou jako v Úloze \ref{osa-uhlu} nalézt rovnici ještě jedné osy a pak spočítat průsečík (na papíře dost děsivá představa, ale s dobrým softwarem by to problém nebyl). Jinou možností může být např. si pomocí nějakých vzorců spočítat poloměr kružnice vepsané a pak nalézt rovnice dvou rovnoběžek se stranami v dané vzdálenosti. Popravdě nevím, jaká metoda je vlastně nejlepší.}
\end{uloha}

\begin{uloha}
Rozhodněte, které z následujících bodů se nachází uvnitř trojúhelníka $ABC$:
\[ K[7;3],\quad L[-1;3],\quad M[0;2],\quad N[1;-1],\quad O[2;3],\quad P[\pi;\sqrt[3]{44}] \]
\vysl{$M$, $O$ a $P$}
\end{uloha}

\newpage
\parindent=0pt
\parskip=\smallskipamount
\def\printvysl#1#2{\textbf{#1.} #2\par}
\vysld

\end{document}