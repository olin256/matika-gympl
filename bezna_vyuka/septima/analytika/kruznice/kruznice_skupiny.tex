\documentclass[10pt,a5paper]{extarticle}
\usepackage[margin=1cm]{geometry}
\usepackage[utf8]{inputenc}
\usepackage[IL2]{fontenc}
\usepackage[czech]{babel}
\usepackage{microtype}
\usepackage{amssymb}
\usepackage{amsthm}
\usepackage{amsmath}
\usepackage{xcolor}
\usepackage{graphicx}
\usepackage{wasysym}
\usepackage{multicol}

\usepackage[inline]{enumitem}

\newcommand{\R}{\mathbb{R}}

\newcommand{\hint}[1]{{\color{gray}\footnotesize\noindent(Nápověda: #1)}}

\setlist[enumerate]{label={(\alph*)},topsep=\smallskipamount,itemsep=\smallskipamount,parsep=0pt}
\setlist[itemize]{topsep=\smallskipamount,noitemsep}

\def\tisk{%
\newbox\shipouthackbox
\pdfpagewidth=2\pdfpagewidth
\let\oldshipout=\shipout
\def\shipout{\afterassignment\zdvojtmp \setbox\shipouthackbox=}%
\def\zdvojtmp{\aftergroup\zdvoj}%
\def\zdvoj{%
    \oldshipout\vbox{\hbox{%
        \copy\shipouthackbox
        \hskip\dimexpr .5\pdfpagewidth-\wd\shipouthackbox\relax
        \box\shipouthackbox
    }}%
}}%



\newtheorem*{poz}{Pozorování}

\theoremstyle{definition}
\newtheorem{uloha}{\atr Úloha}
\newtheorem{suloha}[uloha]{\llap{$\star$ }Úloha}
\newtheorem*{bonus}{Bonus}
\newtheorem*{defn}{Definice}

\pagestyle{empty}

\let\ee\expandafter

\def\vysld{}
\let\printvysl\relax
\let\printalphvysl\relax

\makeatletter
\long\def\vyslplain#1{\ee\ee\ee\gdef\ee\ee\ee\vysld\ee\ee\ee{\ee\vysld\ee\printvysl\ee{\the\c@uloha}{#1}}}
\let\vysl\vyslplain

\def\locvysl#1{\ee\gdef\ee\locvysld\ee{\locvysld\item #1}}
\let\lv\locvysl

\newenvironment{ulohav}[1][]{\begin{uloha}[#1]\gdef\locvysld{\begin{enumerate}}}{\ee\vyslplain\ee{\locvysld\end{enumerate}}\end{uloha}}
\def\stitem{\@noitemargtrue\@item[$\star$ \@itemlabel]}

\makeatother

\def\atr{}
\def\basic{\def\atr{\llap{\mdseries$\sun$ }\gdef\atr{}}}
\def\interest{\def\atr{\llap{$\star$ }\gdef\atr{}}}
\def\iinterest{\def\atr{\llap{$\star\star$ }\gdef\atr{}}}
\let\mb\mathbf

\begin{document}

%\tisk

\section*{Standardní úlohy na kružnice}


\begin{uloha}
Vyřeště úlohu \uv{Nalezněte rovnici kružnice procházející třemi body}.
\begin{enumerate}
    \item Popiště obecně, \uv{jak na to}.
    \item Vyřešte úlohu pro konkrétní body $R[-3;2]$, $S[-1;4]$, $T[3;0]$ (jde o body z~minulé hodiny). Vyjde $x^2+(y-1)^2=10$.%\lv{$$}
    \item Při řešení příslušné soustavy kvadratických rovnic dostaneme jako mezikrok(y) jisté lineární rovnice obsahující souřadnice středu jakožto neznámé. Tyto lineární rovnice můžeme interpretovat jako (obecné) rovnice jistých přímek. O jaké přímky se jedná?
\end{enumerate}
\end{uloha}

\begin{uloha}
Vyřešte úlohu \uv{Nalezněte průsečík(y) přímky a kružnice}.
\begin{enumerate}
    \item Popiště obecně, \uv{jak na to}.
    \item Vyřešte úlohu pro konkrétní přímku danou obecnou rovnicí $3x + y - 5 = 0$ a kružnici se středem v bodě $[2;1]$ a poloměrem 2. Vyjdou body $[2;-1]$ a $\bigl[\frac45; \frac{13}{5}\bigr]$.\label{zadani-kruznice}
    \item Kolik může mít tato úloha obecně řešení? Co znamenají různé počty geometricky a jak se to projeví v řešení příslušné soustavy rovnic?
    \item Uveďte příklad přímky, která nebude mít s kružnicí z bodu \ref{zadani-kruznice} žádný společný bod.
    \item (Kdyby zbyl čas) Jak bychom postupovali v případě, kdy by přímka byla zadaná parametricky?
\end{enumerate}
\end{uloha}


\begin{uloha}
Vyřešte úlohu \uv{Nalezněte průsečík(y) dvou kružnic}.
\begin{enumerate}
    \item Popiště obecně, \uv{jak na to}.
    \item Vyřešte úlohu pro kružnici $k_1$ se středem $[2;1]$ a poloměrem $2$ a kružnici $k_2$ se středem $[5;3]$ a poloměrem $\sqrt5$. Vyjdou body $[4;1]$ a $\bigl[\frac{36}{13};\frac{37}{13}\bigr]$.
    \item Kolik může mít tato úloha obecně řešení? Co znamenají různé počty geometricky a jak se to projeví v řešení příslušné soustavy rovnic?
    \item (Kdyby zbyl čas) Při řešení příslušné soustavy kvadratických rovnic dostaneme jako mezikrok jistou lineární rovnici, kterou můžeme interpretovat jako (obecnou) rovnici jisté přímky. O jakou přímku se jedná?
\end{enumerate}
\end{uloha}



\end{document}