\documentclass[11pt,a5paper]{extarticle}
\usepackage[margin=1cm]{geometry}
\usepackage[utf8]{inputenc}
\usepackage[IL2]{fontenc}
\usepackage[czech]{babel}
\usepackage{microtype}
\usepackage{amssymb}
\usepackage{amsthm}
\usepackage{amsmath}
\usepackage{xcolor}
\usepackage{graphicx}
\usepackage{wasysym}
\usepackage{multicol}

\usepackage[inline]{enumitem}

\newcommand{\R}{\mathbb{R}}

\newcommand{\hint}[1]{{\color{gray}\footnotesize\noindent(Nápověda: #1)}}

\DeclareMathOperator{\tg}{tg}
\DeclareMathOperator{\cotg}{cotg}

\setlist[enumerate]{label={(\alph*)},topsep=\smallskipamount,itemsep=\smallskipamount,parsep=0pt}
\setlist[itemize]{topsep=\smallskipamount,noitemsep}

\def\tisk{%
\newbox\shipouthackbox
\pdfpagewidth=2\pdfpagewidth
\let\oldshipout=\shipout
\def\shipout{\afterassignment\zdvojtmp \setbox\shipouthackbox=}%
\def\zdvojtmp{\aftergroup\zdvoj}%
\def\zdvoj{%
    \oldshipout\vbox{\hbox{%
        \copy\shipouthackbox
        \hskip\dimexpr .5\pdfpagewidth-\wd\shipouthackbox\relax
        \box\shipouthackbox
    }}%
}}%

\let\results\newpage
\let\endresults\relax

\def\resultssame{%
    \long\def\results##1\endresults{%
        \vfill\noindent\rotatebox{180}{\vbox{##1}}%
    }%
}

\newtheorem*{poz}{Pozorování}

\theoremstyle{definition}
\newtheorem{uloha}{\atr Úloha}
\newtheorem{suloha}[uloha]{\llap{$\star$ }Úloha}
\newtheorem*{bonus}{Bonus}
\newtheorem*{defn}{Definice}

\pagestyle{empty}

\let\ee\expandafter

\def\vysld{}
\let\printvysl\relax

\makeatletter
\long\def\vyslplain#1{\ee\ee\ee\gdef\ee\ee\ee\vysld\ee\ee\ee{\ee\vysld\ee\printvysl\ee{\the\c@uloha}{#1}}}
\let\vysl\vyslplain

\def\locvysl#1{\ee\gdef\ee\locvysld\ee{\locvysld\item #1}}
\let\lv\locvysl

\newenvironment{ulohav}[1][]{\begin{uloha}[#1]\gdef\locvysld{\begin{enumerate}}}{\ee\vyslplain\ee{\locvysld\end{enumerate}}\end{uloha}}
\def\stitem{\@noitemargtrue\@item[$\star$ \@itemlabel]}

\makeatother

\def\atr{}
\def\basic{\def\atr{\llap{\mdseries$\sun$ }\gdef\atr{}}}
\def\interest{\def\atr{\llap{$\star$ }\gdef\atr{}}}
\def\iinterest{\def\atr{\llap{$\star\star$ }\gdef\atr{}}}
\let\mb\mathbf


\begin{document}

% \tisk
% \resultssame

\section*{18. Vzdálenosti v rovině}


\begin{uloha}
Rozmyslete si, že přímky
$p\colon 2x - y + 4 = 0$ a $q \colon {-}4x + 2y + 7 = 0$
jsou rovnoběžné, a určete jejich vzdálenost.\vysl{$\frac{3\sqrt5}{2}$}
\end{uloha}

\begin{uloha}
Nalezněte všechny možné hodnoty $m \in \R$ takové, že bod $M[4; m]$ bude mít od přímky $p \colon 3x-4y+2=0$ vzdálenost $2$.\vysl{$1$ a $6$}
\end{uloha}

\begin{uloha}
Nalezněte obecné rovnice všech přímek, které budou rovnoběžné s přímkou $p \colon 3x-4y+2=0$ a jejich vzdálenost od oné přímky bude $2$.\vysl{$3x-4y-8=0$ a $3x-4y+12=0$}
\end{uloha}

\begin{uloha}
Nalezněte všechny body na přímce $q\colon x = 1 + 2t$, $y = 2 + t$, $t \in \R$, které budou mít od přímky $p \colon 3x-4y+2=0$ vzdálenost $3$.\vysl{$[-11;-4]$ a $[19;11]$}
\end{uloha}


\begin{uloha}\label{pata}
Na přímce $p \colon 3x-4y+2=0$ nalezněte bod, který je nejblíže bodu $D[3; 9]$. \hint{Oním nejbližším bodem je pata kolmice na $p$ spuštěné z bodu $D$. Přijde mi lehce praktičtější si onu kolmici popsat parametricky.}\vysl{$[6;5]$}
\end{uloha}


\interest
\begin{uloha}
Postupem podobným jako v Úloze \ref{pata} nalezněte obecný vzorec pro souřadnice bodu na přímce $ax + by + c = 0$, který bude nejblíže bodu $P[p_1; p_2]$.\vysl{$[p_1;p_2] - \frac{ap_1 + bp_2 + c}{a^2 + b^2} (a;b)$ neboli $\bigl[p_1 - \frac{ap_1 + bp_2 + c}{a^2 + b^2}\, a;\ p_2 - \frac{ap_1 + bp_2 + c}{a^2 + b^2}\, b\bigr]$}
\end{uloha}


\begin{uloha}
Množina všech bodů v rovině, které mají stejnou vzdálenost od přímek $p \colon 3x-4y+2=0$ a $q \colon 5x+12y-7=0$, je sjednocením dvou přímek; určete jejich obecné rovnice. (Rozmyslete si, že toto je další možná metoda, jak hledat rovnice os úhlů.)\vysl{$14x - 112y + 61 = 0$ a $64x + 8y - 9 = 0$}
\end{uloha}


\interest
\begin{uloha}
Nalezněte obecné rovnice všech přímek, které budou procházet bodem $A[4;7]$ a jejich vzdálenost od počátku $O[0;0]$ bude rovna $1$. \hint{Může být praktické si zvolit $c = 1$ v obecné rovnici, což je možné kdykoliv, když ty přímky neprochází počátkem.}\vysl{$4x-3y+5 = 0$ a $12x-5y-13 = 0$}
\end{uloha}



\results
\parindent=0pt
\parskip=\smallskipamount
\rightskip=0pt plus1fil\relax
\def\printvysl#1#2{\textbf{#1.} #2\par}
\vysld
\endresults


\end{document}