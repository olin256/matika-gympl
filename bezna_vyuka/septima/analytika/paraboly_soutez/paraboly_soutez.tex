\documentclass[12pt,a5paper]{article}
\usepackage[margin=.7cm]{geometry}
\usepackage[utf8]{inputenc}
\usepackage[IL2]{fontenc}
\usepackage[czech]{babel}
\usepackage{microtype}
\usepackage{amssymb}
\usepackage{amsthm}
\usepackage{amsmath}
\usepackage{xcolor}
\usepackage{graphicx}
\usepackage{wasysym}
\usepackage{multicol}

\usepackage[inline]{enumitem}

\newcommand{\R}{\mathbb{R}}

\newcommand{\hint}[1]{{\color{gray}\footnotesize\noindent(Nápověda: #1)}}

\setlist[enumerate]{label={(\alph*)},topsep=\smallskipamount,itemsep=\smallskipamount,parsep=0pt,itemjoin={\quad}}
\setlist[itemize]{topsep=\smallskipamount,noitemsep}

\def\tisk{%
\newbox\shipouthackbox
\pdfpagewidth=2\pdfpagewidth
\let\oldshipout=\shipout
\def\shipout{\afterassignment\zdvojtmp \setbox\shipouthackbox=}%
\def\zdvojtmp{\aftergroup\zdvoj}%
\def\zdvoj{%
    \oldshipout\vbox{\hbox{%
        \copy\shipouthackbox
        \hskip\dimexpr .5\pdfpagewidth-\wd\shipouthackbox\relax
        \box\shipouthackbox
    }}%
}}%

\let\results\newpage
\let\endresults\relax

\def\resultssame{%
    \long\def\results##1\endresults{%
        %\vfill
        \noindent\rotatebox{180}{\vbox{##1}}%
    }%
}

\newtheorem*{poz}{Pozorování}

\theoremstyle{definition}
\newtheorem{uloha}{\atr Úloha}
\newtheorem{suloha}[uloha]{\llap{$\star$ }Úloha}
\newtheorem*{bonus}{Bonus}
\newtheorem*{defn}{Definice}

\pagestyle{empty}

\let\ee\expandafter

\def\vysld{}
\let\printvysl\relax
\let\printalphvysl\relax

\def\body#1{{\leavevmode\unskip\nobreak\hfil\penalty50\hskip2em
  \hbox{}\nobreak\hfil{\footnotesize(#1)}%
  \parfillskip=0pt \finalhyphendemerits=0 \endgraf}}


\makeatletter
\long\def\vyslplain#1{\ee\ee\ee\gdef\ee\ee\ee\vysld\ee\ee\ee{\ee\vysld\ee\printvysl\ee{\the\c@uloha}{#1}}}
\let\vysl\vyslplain

\def\locvysl#1{\ee\gdef\ee\locvysld\ee{\locvysld\item #1}}
\let\lv\locvysl

\newenvironment{ulohav}[1][]{\begin{uloha}[#1]\gdef\locvysld{\begin{enumerate*}}}{\ee\vyslplain\ee{\locvysld\end{enumerate*}}\end{uloha}}
\def\stitem{\@noitemargtrue\@item[$\star$ \@itemlabel]}

\makeatother

\def\atr{}
\def\basic{\def\atr{\llap{\mdseries$\sun$ }\gdef\atr{}}}
\def\interest{\def\atr{\llap{$\star$ }\gdef\atr{}}}
\def\iinterest{\def\atr{\llap{$\star\star$ }\gdef\atr{}}}


\begin{document}

% \tisk

\section*{Parabolická soutěž}

% úlohy převzaty ze sbírky ke standardní učebnici Stereometrie

\emph{V závorkách jsou uvedeny počty bodů.}

\begin{uloha}[1,5]
Napište rovnice paraboly, která má ohnisko $F[2;5]$ a řídící přímku $x = 0$.\vysl{4(x-1)=(y-5)^2} % 5.60 e
\end{uloha}

\begin{uloha}[1,5]
Napište rovnice paraboly, která má ohnisko $F[-6;4]$ a řídící přímku $y = 6$.\vysl{-4(y-5)=(x+6)^2} % 5.60 c
\end{uloha}

\begin{uloha}[2]
Určete ohnisko, vrchol a řídící přímku paraboly dané rovnicí $x^2 + 4y - 6x + 3 = 0$.\vysl{$F[3;\frac12]$, $V[3;\frac32]$, $y=\frac52$} % 5.62 a
\end{uloha}


\begin{uloha}[2,5]
Určete ohnisko, vrchol a řídící přímku paraboly dané rovnicí $y^2 + 8y + 3x - 6 = 0$.\vysl{$F[\frac{79}{12};-4]$, $V[\frac{22}{3};-4]$, $x=\frac{97}{12}$} % 5.62 c
\end{uloha}

\begin{ulohav}
Určete vrcholové rovnice všech parabol, které mají osu rovnoběžnou s osou $y$ a
\begin{enumerate}
    \item mají vrchol $V[3;5]$ a prochází bodem $[0;2]$.\body{1,5}
    \item mají vrchol $V[6;-2]$ a prochází bodem $[3;-5]$.\body{1,5}
    \item prochází body $[0;-1]$, $[-2;7]$ a $[5;14]$.\body{2,5}
    \item jejichž vrcholová tečna má rovnici $y=\frac52$ a prochází body $[3;4]$ a $[6;7]$.\body{2,5}
\end{enumerate}
\end{ulohav}


\begin{uloha}[2]
Určete čísla $a$, $b$, $c$ tak, aby parabola s rovnicí $y = ax^2 + bx + c$ procházela body $[1;1]$, $[0;-4]$, $[-2;-2]$. % 5.68
\end{uloha}

\begin{uloha}[1,5]
Jak dlouhou tětivu vytíná parabola o rovnici $y^2 - 8x = 0$ na přímce dané rovnicí $x-y-2=0$? % 5.73
\end{uloha}


\begin{ulohav}
Napište rovnici tečny k parabole
\begin{enumerate}
    \item $y^2 = 2x$ v bodě $[2;-2]$.\body{2}
    \item $3y^2+x-12y+14=0$ v bodě $[-2;2]$.\body{3}
\end{enumerate}
\end{ulohav}



\end{document}