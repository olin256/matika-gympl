\documentclass[9pt,a6paper,landscape]{extarticle}
\usepackage[margin=.8cm,top=3mm]{geometry}
\usepackage[utf8]{inputenc}
\usepackage[IL2]{fontenc}
\usepackage[czech]{babel}
\usepackage{microtype}
\usepackage{amssymb}
\usepackage{amsthm}
\usepackage{amsmath}
\usepackage{xcolor}
\usepackage{graphicx}
\usepackage{wasysym}
\usepackage{multicol}

\usepackage[inline]{enumitem}

\newcommand{\R}{\mathbb{R}}

\newcommand{\hint}[1]{{\color{gray}\footnotesize\noindent(Nápověda: #1)}}

\DeclareMathOperator{\tg}{tg}
\DeclareMathOperator{\cotg}{cotg}

\setlist[enumerate]{label={(\alph*)},topsep=\smallskipamount,itemsep=\smallskipamount,parsep=0pt,itemjoin={\quad}}
\setlist[itemize]{topsep=\smallskipamount,noitemsep}

\def\tisk{%
\newbox\shipouthackbox
\pdfpagewidth=2\pdfpagewidth
\let\oldshipout=\shipout
\def\shipout{\afterassignment\zdvojtmp \setbox\shipouthackbox=}%
\def\zdvojtmp{\aftergroup\zdvoj}%
\def\zdvoj{%
    \oldshipout\vbox{\hbox{%
        \copy\shipouthackbox
        \hskip\dimexpr .5\pdfpagewidth-\wd\shipouthackbox\relax
        \box\shipouthackbox
    }}%
}}%
\def\tiskctyr{%
\newbox\shipouthackbox
\pdfpagewidth=2\pdfpagewidth
\pdfpageheight=2\pdfpageheight
\let\oldshipout=\shipout
\def\shipout{\afterassignment\zctyrtmp \setbox\shipouthackbox=}%
\def\zctyrtmp{\aftergroup\zctyr}%
\def\zctyr{%
    \offinterlineskip
    \oldshipout\vbox{\hbox{%
        \copy\shipouthackbox
        \hskip\dimexpr .5\pdfpagewidth-\wd\shipouthackbox\relax
        \copy\shipouthackbox
    }%
    \vskip\dimexpr .5\pdfpageheight-\ht\shipouthackbox\relax
    \hbox{%
        \copy\shipouthackbox
        \hskip\dimexpr .5\pdfpagewidth-\wd\shipouthackbox\relax
        \box\shipouthackbox
    }}%
}}

\let\results\newpage
\let\endresults\relax

\def\resultssame{%
    \long\def\results##1\endresults{%
        %\vfill
        \noindent\rotatebox{180}{\vbox{##1}}%
    }%
}


\newtheorem*{poz}{Pozorování}

\theoremstyle{definition}
\newtheorem{uloha}{\atr Úloha}
\newtheorem{suloha}[uloha]{\llap{$\star$ }Úloha}
\newtheorem*{bonus}{Bonus}
\newtheorem*{defn}{Definice}

\pagestyle{empty}

\let\ee\expandafter

\def\vysld{}
\let\printvysl\relax
\let\printalphvysl\relax

\makeatletter
\long\def\vyslplain#1{\ee\ee\ee\gdef\ee\ee\ee\vysld\ee\ee\ee{\ee\vysld\ee\printvysl\ee{\the\c@uloha}{#1}}}
\let\vysl\vyslplain

\def\locvysl#1{\ee\gdef\ee\locvysld\ee{\locvysld\item #1}}
\let\lv\locvysl

\newenvironment{ulohav}[1][]{\begin{uloha}[#1]\gdef\locvysld{\begin{enumerate*}}}{\ee\vyslplain\ee{\locvysld\end{enumerate*}}\end{uloha}}
\def\stitem{\@noitemargtrue\@item[$\star$ \@itemlabel]}

\makeatother

\def\atr{}
\def\basic{\def\atr{\llap{\mdseries$\sun$ }\gdef\atr{}}}
\def\interest{\def\atr{\llap{$\star$ }\gdef\atr{}}}
\def\iinterest{\def\atr{\llap{$\star\star$ }\gdef\atr{}}}
\let\mb\mathbf

\begin{document}

% \tiskctyr
% \resultssame

\section*{22. Základní úlohy o hyperbole}

\begin{ulohav}
Následující rovnice jsou (možná) \uv{zamaskované} rovnice hyperbol; převeďte je do středového tvaru, určete souřadnice středu, délky poloos, excentricitu, souřadnice ohnisek, rovnice asymptot a průsečíky s osami souřadnic. Hyperbolu \emph{načrtněte}.
\begin{enumerate}
    \item $4x^2 - 9y^2 + 18y - 45 = 0$\lv{$\frac{x^2}{9} + \frac{(y-1)^2}{4} = 1$; střed $[0;1]$, hlavní poloosa $a = 3$, vedlejší poloosa $b = 2$, excentricita $\sqrt{13}$, ohniska $[\pm\sqrt{13}; 1]$, asymptoty $y = \pm\frac23x + 1$, průsečíky s osou $x$ $\bigl[\pm\frac32\sqrt5; 0\bigr]$, s osou $y$ nejsou}
    \item $x^2 - 4y^2 + 4x - 4y + 2 = 0$\lv{$\frac{(x+2)^2}{1} - \frac{(y+\frac12)^2}{(\frac12)^2} = 1$; střed $[-2;-\frac12]$, hlavní poloosa $a = 1$, vedlejší poloosa $b = \frac12$, excentricita $\frac{\sqrt{5}}2$, ohniska $\bigl[-2\pm\frac{\sqrt{5}}2; -\frac12\bigr]$, asymptoty $y = \pm\frac12(x + 2) - \frac12$, průsečíky s osou $x$ $\bigl[-2\pm\sqrt2; 0\bigr]$, s osou $y$ $\bigl[0;-\frac12\pm\frac12\sqrt3\bigr]$}
\end{enumerate}
\end{ulohav}


\begin{uloha}
Napište rovnice všech možných hyperbol, jejichž osy splývají s osami souřadnic a které prochází body $K[2;1]$ a $L[8;-2]$.\vysl{$-\frac{x^2}{16} + \frac{y^2}{\frac45} = 1$}
\end{uloha}

\begin{ulohav}
Pro hyperbolu $h$ danou rovnicí $x^2 - y^2 = 1$ nalezněte všechny přímky rovnoběžné s~přímkou $p$, které budou mít s $h$ právě jeden společný bod, jestliže $p$ je
\begin{enumerate*}
    \item $x = 0$,\lv{$x = \pm 1$}
    \item $y = 0$,\lv{žádná neexistuje}
    \item $y = 2x$,\lv{$y = 2x \pm \sqrt3$}
    \item $y = x$,\lv{$y = x + c$, kde $c \in \R \setminus\{0\}$}
\end{enumerate*}
\end{ulohav}

\baselineskip=1.25\baselineskip
\setlist[enumerate]{label=\textbf{(\alph*)},itemjoin={\quad}}

\results
\parindent=0pt
\parskip=\smallskipamount
\rightskip=0pt plus1fil\relax
\def\printvysl#1#2{\textbf{#1.} #2\par}
\vysld
\endresults



\end{document}