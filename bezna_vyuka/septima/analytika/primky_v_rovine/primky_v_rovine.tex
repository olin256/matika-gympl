\documentclass[10pt,a6paper,landscape]{extarticle}
\usepackage[margin=.7cm]{geometry}
\usepackage[utf8]{inputenc}
\usepackage[IL2]{fontenc}
\usepackage[czech]{babel}
\usepackage{microtype}
\usepackage{amssymb}
\usepackage{amsthm}
\usepackage{amsmath}
\usepackage{xcolor}
\usepackage{graphicx}
\usepackage{wasysym}
\usepackage{multicol}

\usepackage[inline]{enumitem}

\newcommand{\R}{\mathbb{R}}

\newcommand{\hint}[1]{{\color{gray}\footnotesize\noindent(Nápověda: #1)}}



\setlist[enumerate]{label={(\alph*)},topsep=\smallskipamount,itemsep=\smallskipamount,parsep=0pt}
\setlist[itemize]{topsep=\smallskipamount,noitemsep}

\def\tisk{%
\newbox\shipouthackbox
\pdfpagewidth=2\pdfpagewidth
\let\oldshipout=\shipout
\def\shipout{\afterassignment\zdvojtmp \setbox\shipouthackbox=}%
\def\zdvojtmp{\aftergroup\zdvoj}%
\def\zdvoj{%
    \oldshipout\vbox{\hbox{%
        \copy\shipouthackbox
        \hskip\dimexpr .5\pdfpagewidth-\wd\shipouthackbox\relax
        \box\shipouthackbox
    }}%
}}%
\def\tiskctyr{%
\newbox\shipouthackbox
\pdfpagewidth=2\pdfpagewidth
\pdfpageheight=2\pdfpageheight
\let\oldshipout=\shipout
\def\shipout{\afterassignment\zctyrtmp \setbox\shipouthackbox=}%
\def\zctyrtmp{\aftergroup\zctyr}%
\def\zctyr{%
    \offinterlineskip
    \oldshipout\vbox{\hbox{%
        \copy\shipouthackbox
        \hskip\dimexpr .5\pdfpagewidth-\wd\shipouthackbox\relax
        \copy\shipouthackbox
    }%
    \vskip\dimexpr .5\pdfpageheight-\ht\shipouthackbox\relax
    \hbox{%
        \copy\shipouthackbox
        \hskip\dimexpr .5\pdfpagewidth-\wd\shipouthackbox\relax
        \box\shipouthackbox
    }}%
}}

\let\ee\expandafter

\newtheorem*{poz}{Pozorování}

\theoremstyle{definition}
\newtheorem{uloha}{\atr Úloha}
\newtheorem{suloha}[uloha]{\llap{$\star$ }Úloha}
\newtheorem*{bonus}{Bonus}
\newtheorem*{defn}{Definice}

\pagestyle{empty}

\def\vysld{}
\let\printvysl\relax

\makeatletter
\long\def\vyslplain#1{\ee\ee\ee\gdef\ee\ee\ee\vysld\ee\ee\ee{\ee\vysld\ee\printvysl\ee{\the\c@uloha}{#1}}}
\let\vysl\vyslplain

\def\locvysl#1{\ee\gdef\ee\locvysld\ee{\locvysld\item #1}}
\let\lv\locvysl

\newenvironment{ulohav}[1][]{\begin{uloha}[#1]\gdef\locvysld{\begin{enumerate}}}{\ee\vyslplain\ee{\locvysld\end{enumerate}}\end{uloha}}
\def\stitem{\@noitemargtrue\@item[$\star$ \@itemlabel]}

\makeatother

\def\atr{}
\def\basic{\def\atr{\llap{\mdseries$\sun$ }\gdef\atr{}}}
\def\interest{\def\atr{\llap{$\star$ }\gdef\atr{}}}
\def\iinterest{\def\atr{\llap{$\star\star$ }\gdef\atr{}}}

\begin{document}

%\tisk

\section*{17. Mix rovinných přímek}


\begin{ulohav}
Napište obecnou rovnici přímky, která
\begin{enumerate}
    \item prochází bodem $A[1;-4]$ a je kolmá k~přímce $BC$, kde $B[3;-7]$, $C[3;2]$.\lv{$y+4=0$} %3.34 a
    \item má parametrické vyjádření $x = 3-2t$, $y = -4+t$, $t \in \R$.\lv{$x+2y+5=0$} % 3.38 a
    \item prochází bodem $M[3;5]$ a je rovnoběžná s přímkou $p\colon 7x-3y+2=0$.\lv{$7x-3y-6=0$} % 3.44
    \item prochází bodem $M[4;-7]$ a je kolmá k přímce $p\colon 2x-5y+10=0$.\lv{$5x+2y-6=0$} % 3.45
    \item prochází bodem $A[-3;5]$ a průsečíkem přímek $p\colon x+2y-3=0$ a $q\colon 2x-3y+8=0$.\lv{$3x+2y-1=0$} % 3.46
    \item prochází bodem $A[-4;2]$ a je rovnoběžná s osou $x$.\lv{$y-2=0$} %3.47 c
    \item prochází bodem $A[-4;2]$ a je rovnoběžná s osou $y$.\lv{$x+4=0$} %3.47 d
    \item prochází bodem $A[-4;2]$ a je rovnoběžná s osou I. a III. kvadrantu.\lv{$x-y+6=0$} %3.47 e
\end{enumerate}
\end{ulohav}

\begin{ulohav}
Určete reálné číslo $m$ tak, aby přímka $p$ procházela bodem $A$, pokud
\begin{enumerate}
    \item $p\colon mx+(m-5)y+m+2=0$,\quad $A[-1;1]$\lv{$m = 3$} %3.49 a
    \item $p\colon (6m-5)x-2(3-7m)y+2m-13=0$,\quad $A[-5;2]$\lv{$m$ může být jakékoliv reálné číslo} %3.49 d
\end{enumerate}
\end{ulohav}

\begin{uloha} %3.50 a
Určete reálné číslo $m$ tak, aby přímky
\[ p\colon (m-1)x + 3my + 2 = 0 \quad \text{a} \quad q\colon 2x+9y+5=0 \]
byly rovnoběžné.\vysl{$m=3$}
\end{uloha}


\newpage
\parindent=0pt
\parskip=\smallskipamount
\def\printvysl#1#2{\textbf{#1.} #2\par}
\vysld


\end{document}