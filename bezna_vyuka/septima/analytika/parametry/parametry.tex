\documentclass[12pt,a5paper]{extarticle}
\usepackage[margin=.8cm]{geometry}
\usepackage[utf8]{inputenc}
\usepackage[IL2]{fontenc}
\usepackage[czech]{babel}
\usepackage{microtype}
\usepackage{amssymb}
\usepackage{amsthm}
\usepackage{amsmath}
\usepackage{xcolor}
\usepackage{graphicx}
\usepackage{wasysym}
\usepackage{multicol}

\usepackage[inline]{enumitem}

\newcommand{\R}{\mathbb{R}}

\newcommand{\hint}[1]{{\color{gray}\footnotesize\noindent(Nápověda: #1)}}

\setlist[enumerate]{label={(\alph*)},topsep=\smallskipamount,itemsep=\smallskipamount,parsep=0pt}
\setlist[itemize]{topsep=\smallskipamount,noitemsep}

\def\tisk{%
\newbox\shipouthackbox
\pdfpagewidth=2\pdfpagewidth
\let\oldshipout=\shipout
\def\shipout{\afterassignment\zdvojtmp \setbox\shipouthackbox=}%
\def\zdvojtmp{\aftergroup\zdvoj}%
\def\zdvoj{%
    \oldshipout\vbox{\hbox{%
        \copy\shipouthackbox
        \hskip\dimexpr .5\pdfpagewidth-\wd\shipouthackbox\relax
        \box\shipouthackbox
    }}%
}}%



\newtheorem*{poz}{Pozorování}

\theoremstyle{definition}
\newtheorem{uloha}{\atr Úloha}
\newtheorem{suloha}[uloha]{\llap{$\star$ }Úloha}
\newtheorem*{bonus}{Bonus}
\newtheorem*{defn}{Definice}

\pagestyle{empty}

\let\ee\expandafter

\def\vysld{}
\let\printvysl\relax
\let\printalphvysl\relax

\makeatletter
\long\def\vyslplain#1{\ee\ee\ee\gdef\ee\ee\ee\vysld\ee\ee\ee{\ee\vysld\ee\printvysl\ee{\the\c@uloha}{#1}}}
\let\vysl\vyslplain

\def\locvysl#1{\ee\gdef\ee\locvysld\ee{\locvysld\item #1}}
\let\lv\locvysl

\newenvironment{ulohav}[1][]{\begin{uloha}[#1]\gdef\locvysld{\begin{enumerate}}}{\ee\vyslplain\ee{\locvysld\end{enumerate}}\end{uloha}}
\def\stitem{\@noitemargtrue\@item[$\star$ \@itemlabel]}

\makeatother

\def\atr{}
\def\basic{\def\atr{\llap{\mdseries$\sun$ }\gdef\atr{}}}
\def\interest{\def\atr{\llap{$\star$ }\gdef\atr{}}}
\let\mb\mathbf

\newcommand\at[2]{\begin{array}{l}x=#1\\y=#2\end{array}}

\begin{document}

%\tisk

\section*{15. Parametry}


\begin{uloha}\label{shodnost}
Rozhodněte, které z následujících parametrických vyjádření popisují tutéž přímku (vždy $t \in \R$):
\[ p\colon\at{2 - t}{-3 + 2t} \quad q\colon\at{3-t}{-2 + 2t} \quad r\colon\at{-1 - t}{3 - 2t} \quad s\colon\at{1-2t}{-1+4t} \]
\vysl{$p$ a $s$ jsou tatáž přímka, $q$ je s nimi pouze rovnoběžná, $r$ je různoběžná}
\end{uloha}


\begin{uloha}\label{primka}
Určete parametrickou rovnici přímky $AB$, jsou-li souřadnice bodů $A[-6; 3]$, $B[-2; 1]$.\vysl{např. $\at{-6+4t}{3-2t}$, $t \in \R$, nebo \uv{jednodušeji} $\at{-6+2t}{3-t}$}
\end{uloha}

\begin{ulohav}
U následujících bodů
\[ C[-10;5],\quad D[0;0], \quad E[-4;2], \quad F[2; 1], \quad G\bigl[-7;\tfrac72\bigr]\]
rozhodněte, zda leží na
\begin{enumerate}
    \item přímce $AB$ (z Úlohy \ref{primka}),\lv{všechny kromě $F$}
    \item polopřímce $AB$,\lv{$D$ a $E$}
    \item polopřímce $BA$,\lv{$C$, $E$ a $G$}
    \item úsečce $AB$.\lv{$E$}
\end{enumerate}
\end{ulohav}


\begin{uloha}
Doplňte na místa otazníků čísla tak, aby body $H[11;?]$ a $I[?;3]$ ležely na přímce $AB$ z Úlohy \ref{primka}.\vysl{$H\bigl[11; -\frac{11}{2}\bigr]$, $I[-6;3]$ ($I$ je prostě bod $A$)}
\end{uloha}


\begin{uloha}
Určete souřadnice průsečíku přímky $AB$ z Úlohy \ref{primka} a přímky $p$ z Úlohy~\ref{shodnost}.\vysl{$\bigl[\frac23;-\frac13\bigr]$}
\end{uloha}

\begin{uloha}
Určete parametrickou rovnici osy úsečky $AB$.\vysl{např. $\at{-4+2t}{2+4t}$, $t \in \R$}
\end{uloha}

\begin{uloha}
V krychli $ABCDEFGH$ (souřadnice jsou na papíru 14) určete parametrickou rovnici přímky $S_{CD}S_{EB}$.\vysl{např. $x=\frac12$, $y = 1 - t$, $z = \frac12 t$, $t \in \R$}
\end{uloha}

\interest
\begin{uloha}
Určete parametrickou rovnici osy úhlu $KLM$ ($L$ je vrchol), jestliže souřadnice bodů jsou $K[4;5]$, $L[1;1]$, $M[2;1]$. \hint{Směrový vektor osy úhlu by mohl být součet směrových vektorů ramen -- kdybychom si dali pozor na velikosti oněch vektorů.}\vysl{např. $\at{1+2t}{1+t}$, $t \in \R$}
\end{uloha}


\newpage
\parindent=0pt
\parskip=\smallskipamount
\def\printvysl#1#2{\textbf{#1.} #2\par}
\vysld


\end{document}