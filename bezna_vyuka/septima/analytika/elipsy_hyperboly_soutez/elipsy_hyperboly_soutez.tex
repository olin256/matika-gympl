\documentclass[12pt,a4paper]{extarticle}
\usepackage[margin=.7cm]{geometry}
\usepackage[utf8]{inputenc}
\usepackage[IL2]{fontenc}
\usepackage[czech]{babel}
\usepackage{microtype}
\usepackage{amssymb}
\usepackage{amsthm}
\usepackage{amsmath}
\usepackage{xcolor}
\usepackage{graphicx}
\usepackage{wasysym}
\usepackage{multicol}

\usepackage[inline]{enumitem}

\newcommand{\R}{\mathbb{R}}

\newcommand{\hint}[1]{{\color{gray}\footnotesize\noindent(Nápověda: #1)}}

\setlist[enumerate]{label={(\alph*)},topsep=\smallskipamount,itemsep=\smallskipamount,parsep=0pt,itemjoin={\quad}}
\setlist[itemize]{topsep=\smallskipamount,noitemsep}

\def\tisk{%
\newbox\shipouthackbox
\pdfpagewidth=2\pdfpagewidth
\let\oldshipout=\shipout
\def\shipout{\afterassignment\zdvojtmp \setbox\shipouthackbox=}%
\def\zdvojtmp{\aftergroup\zdvoj}%
\def\zdvoj{%
    \oldshipout\vbox{\hbox{%
        \copy\shipouthackbox
        \hskip\dimexpr .5\pdfpagewidth-\wd\shipouthackbox\relax
        \box\shipouthackbox
    }}%
}}%

\let\results\newpage
\let\endresults\relax

\def\resultssame{%
    \long\def\results##1\endresults{%
        %\vfill
        \noindent\rotatebox{180}{\vbox{##1}}%
    }%
}


\newtheorem*{poz}{Pozorování}

\theoremstyle{definition}
\newtheorem{uloha}{\atr Úloha}
\newtheorem{suloha}[uloha]{\llap{$\star$ }Úloha}
\newtheorem*{bonus}{Bonus}
\newtheorem*{defn}{Definice}

\pagestyle{empty}

\let\ee\expandafter

\def\vysld{}
\let\printvysl\relax
\let\printalphvysl\relax

\makeatletter
\long\def\vyslplain#1{\ee\ee\ee\gdef\ee\ee\ee\vysld\ee\ee\ee{\ee\vysld\ee\printvysl\ee{\the\c@uloha}{#1}}}
\let\vysl\vyslplain

\def\locvysl#1{\ee\gdef\ee\locvysld\ee{\locvysld\item #1}}
\let\lv\locvysl

\newenvironment{ulohav}[1][]{\begin{uloha}[#1]\gdef\locvysld{\begin{enumerate*}}}{\ee\vyslplain\ee{\locvysld\end{enumerate*}}\end{uloha}}
\def\stitem{\@noitemargtrue\@item[$\star$ \@itemlabel]}

\makeatother

\def\atr{}
\def\basic{\def\atr{\llap{\mdseries$\sun$ }\gdef\atr{}}}
\def\interest{\def\atr{\llap{$\star$ }\gdef\atr{}}}
\def\iinterest{\def\atr{\llap{$\star\star$ }\gdef\atr{}}}
\let\mb\mathbf



\begin{document}

%\tisk
%\resultssame


\section*{Hyperbolicko-eliptická soutěž}

\emph{V závorkách jsou uvedeny počty bodů.}


\begin{uloha}[1,5]
Nalezněte všechny průsečíky přímky $p$ a hyperboly $h$, jestliže jejich rovnice jsou \[p\colon 3x - 2y + 2 = 0, \quad h\colon {-}(x-2)^2 + (y-5)^2 = 1.\]\vysl{$[2;4]$ a $\bigl[\frac{22}{5};\frac{38}{5}\bigr]$}
\end{uloha}

\begin{uloha}[2]
Nalezněte rovnice všech hyperbol, jejichž asymptoty mají rovnice $y = 2x + 4$ a $y = -2x - 2$ a délka hlavní poloosy je $2$.\vysl{$\frac{(x+\frac32)^2}{4} - \frac{(y-1)^2}{16} = 1$ a $-\frac{(x+\frac32)^2}{1} + \frac{(y-1)^2}{4} = 1$}
\end{uloha}

\begin{uloha}[2]
Elipsa má ohniska v bodech $[-3;1]$ a $[5;1]$, přičemž délka vedlejší poloosy je $2$. Určete rovnici oné elipsy.\vysl{$\frac{(x-1)^2}{20} + \frac{(y-1)^2}{4} = 1$}
\end{uloha}

\begin{uloha}[2]
Hyperbola má ohniska v bodech $[-3;1]$ a $[5;1]$, přičemž délka vedlejší poloosy je $2$. Určete rovnici oné hyperboly.\vysl{$\frac{(x+1)^2}{12} - \frac{(y-1)^2}{4} = 1$}
\end{uloha}

\begin{uloha}[3]
Elipsa má ohniska v bodech $[-1;-2]$ a $[-1;4]$ a prochází bodem $[0;5]$. Určete rovnici oné elipsy.\vysl{$\frac{(x+1)^2}{9} + \frac{(y-1)^2}{18} = 1$}
\end{uloha}

\begin{uloha}[3]
Hyperbola má ohniska v bodech $[-1;-2]$ a $[-1;4]$ a prochází bodem $[0;5]$. Určete rovnici oné hyperboly.\vysl{$-\frac{(x+1)^2}{1} + \frac{(y-1)^2}{8} = 1$}
\end{uloha}

\begin{uloha}[5]
Nalezněte rovnice všech tečen k elipse dané rovnicí $4x^2 + y^2 = 4$ procházejících bodem $[-2;0]$.\vysl{$y = \pm \frac{2}{\sqrt3}(x+2)$}
\end{uloha}

\begin{uloha}[4]
Nalezněte rovnice všech elips, které budou současně splňovat:
\begin{itemize}
    \item jejich osy budou rovnoběžné s osami souřadnic,
    \item osy souřadnic budou jejich tečny,
    \item střed bude ležet na přímce $y = x+1$,
    \item délka hlavní poloosy bude $7$.
\end{itemize}
\vysl{$\frac{(x-6)^2}{6^2} + \frac{(y-7)^2}{7^2} = 1$ a $\frac{(x+7)^2}{7^2} + \frac{(y+6)^2}{6^2} = 1$}
\end{uloha}

\begin{uloha}[4,5]
Na elipse o rovnici $\frac{x^2}{9} + \frac{y^2}{4} = 1$ nalezněte bod nejblíže přímce o rovnici $y = x-6$.\vysl{$\bigl[\frac{9}{\sqrt13};-\frac{4}{\sqrt13}\bigr]$}
\end{uloha}

\begin{uloha}[1,5]
Určete všechny hodnoty parametru $c \in \R$, pro který je rovnice \[3x^2 + 2y^2 - 6x + 8y = c\] rovnicí nějaké elipsy v rovině.\vysl{$c > -11$}
\end{uloha}


\begin{uloha}[1,5]
Určete všechny hodnoty parametru $c \in \R$, pro který je rovnice \[3x^2 - 2y^2 - 6x + 8y = c\] rovnicí nějaké hyperboly v rovině.\vysl{$c \neq 5$}
\end{uloha}


\begin{uloha}[2,5]
Je dána hyperbola $\frac{x^2}{16} - \frac{(y+2)^2}9 = 1$. Vypočítejte délku takové její tětivy, která je kolmá na osu $x$ a prochází ohniskem hyperboly.\vysl{$\frac92$}
\end{uloha}


\begin{uloha}[2]
Určete odchylku asymptot hyperboly dané rovnicí $2x^2 - x - 3y^2 - 7y + 13 = 0$.\vysl{$\arccos\frac15 \doteq 78^\circ 28'$}
\end{uloha}


\begin{uloha}[4]
Množina všech bodů, jejichž vzdálenosti od bodu $[5;0]$ a od přímky $x = \frac{16}{5}$ jsou v~poměru $5:4$, je jistá hyperbola; určete souřadnice jejího středu a délky poloos.\vysl{střed $[0;0]$, hlavní poloosa 3, vedlejší poloosa 4}
\end{uloha}


\baselineskip=1.25\baselineskip
\setlist[enumerate]{label=\textbf{(\alph*)},itemjoin={\quad}}

\results
\parindent=0pt
\parskip=\smallskipamount
\rightskip=0pt plus1fil\relax
\def\printvysl#1#2{\textbf{#1.} #2\par}
\vysld
\endresults


\end{document}