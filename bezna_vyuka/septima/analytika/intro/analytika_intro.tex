\documentclass[10pt,a6paper,landscape]{article}
\usepackage[margin=.8cm]{geometry}
\usepackage[utf8]{inputenc}
\usepackage[IL2]{fontenc}
\usepackage[czech]{babel}
\usepackage{microtype}
\usepackage{amssymb}
\usepackage{amsthm}
\usepackage{amsmath}
\usepackage{xcolor}
\usepackage{graphicx}
\usepackage{wasysym}

\usepackage[inline]{enumitem}

\newcommand{\R}{\mathbb{R}}

\newcommand{\hint}[1]{{\color{gray}\footnotesize\noindent(Nápověda: #1)}}

\setlist[enumerate]{label={(\alph*)},topsep=\smallskipamount,itemsep=\smallskipamount,parsep=0pt}
\setlist[itemize]{topsep=\smallskipamount,noitemsep}

\def\tisk{%
\newbox\shipouthackbox
\pdfpagewidth=2\pdfpagewidth
\let\oldshipout=\shipout
\def\shipout{\afterassignment\zdvojtmp \setbox\shipouthackbox=}%
\def\zdvojtmp{\aftergroup\zdvoj}%
\def\zdvoj{%
    \oldshipout\vbox{\hbox{%
        \copy\shipouthackbox
        \hskip\dimexpr .5\pdfpagewidth-\wd\shipouthackbox\relax
        \box\shipouthackbox
    }}%
}}%
\def\tiskctyr{%
\newbox\shipouthackbox
\pdfpagewidth=2\pdfpagewidth
\pdfpageheight=2\pdfpageheight
\let\oldshipout=\shipout
\def\shipout{\afterassignment\zctyrtmp \setbox\shipouthackbox=}%
\def\zctyrtmp{\aftergroup\zctyr}%
\def\zctyr{%
    \offinterlineskip
    \oldshipout\vbox{\hbox{%
        \copy\shipouthackbox
        \hskip\dimexpr .5\pdfpagewidth-\wd\shipouthackbox\relax
        \copy\shipouthackbox
    }%
    \vskip\dimexpr .5\pdfpageheight-\ht\shipouthackbox\relax
    \hbox{%
        \copy\shipouthackbox
        \hskip\dimexpr .5\pdfpagewidth-\wd\shipouthackbox\relax
        \box\shipouthackbox
    }}%
}}

\let\results\newpage
\let\endresults\relax

\def\resultssame{%
    \long\def\results##1\endresults{%
        %\vfill
        \noindent\rotatebox{180}{\vbox{##1}}%
    }%
}

\newtheorem*{poz}{Pozorování}

\theoremstyle{definition}
\newtheorem{uloha}{\atr Úloha}
\newtheorem{suloha}[uloha]{\llap{$\star$ }Úloha}
\newtheorem*{bonus}{Bonus}
\newtheorem*{defn}{Definice}

\pagestyle{empty}

\let\ee\expandafter

\def\vysld{}
\let\printvysl\relax

\makeatletter
\long\def\vyslplain#1{\ee\ee\ee\gdef\ee\ee\ee\vysld\ee\ee\ee{\ee\vysld\ee\printvysl\ee{\the\c@uloha}{#1}}}

\def\locvysl#1{\ee\gdef\ee\locvysld\ee{\locvysld\item #1}}
\let\lv\locvysl

\newenvironment{ulohav}[1][]{\begin{uloha}[#1]\gdef\locvysld{\begin{enumerate*}}}{\ee\vyslplain\ee{\locvysld\end{enumerate*}}\end{uloha}}
\def\stitem{\@noitemargtrue\@item[$\star$ \@itemlabel]}

\makeatother

\def\atr{}
\def\basic{\def\atr{\llap{\mdseries$\sun$ }\gdef\atr{}}}

\begin{document}

% \tiskctyr
% \resultssame

\section*{7. Analytická geometrie -- průzkum bojem}


\begin{ulohav}
Pokračujme s body $A[2; 2]$ a $B[6; 5]$.
\begin{enumerate}
    \item Určete souřadnice bodu $Y$ tak, aby $B$ byl střed úsečky $AY$.\lv{$[10; 8]$}
    \item Určete souřadnice bodu $Z$ tak, aby $A$ byl střed úsečky $BZ$.\lv{$[-2; -1]$}
    \item Určete souřadnice bodů $C$, $D$ tak, aby $ABCD$ byl čtverec.\lv{dvě řešení: (1) $C_1[3; 9]$ a $D_1[-1; 6]$, (2) $C_2[9; 1]$ a $D_2[5; -2]$}
    \item Je-li $T[-1; -2]$, určete délky stran trojúhelníka $ABT$.\lv{$|AB| = 5$, $|AT| = 5$, $|BT| = 7\sqrt2$}
    \stitem Nalezněte souřadnice bodu $R$ tak, aby trojúhelník $ABR$ byl rovnostranný.\lv{dvě řešení: (1) $R_1[4-\frac{3 \sqrt{3}}{2}; 2 \sqrt{3}+\frac{7}{2}]$, (2) $R_2[\frac{3 \sqrt{3}}{2}+4; \frac{7}{2}-2 \sqrt{3}]$}
\end{enumerate}
\end{ulohav}

\begin{uloha}
Jsou dány body $A[1; 3]$, $B[-1; x]$. Určete $x$ tak, aby $|AB| = \sqrt5$.\vyslplain{dvě řešení, $x_1 = 4$, $x_2 = 2$}
\end{uloha}

\begin{ulohav}
Určete číslo $p \in \R$ tak, aby platilo $|AB| = d$:
\begin{enumerate}
    \item $A[3; p; 2]$, $B[-1; 0; p]$, $d = 3\sqrt2$\lv{1}
    \item $A[2+p; 2; 1]$, $B[3; -p; 2]$, $d = \sqrt{10}$\lv{dvě řešení, $1$ a $-2$}
\end{enumerate}
\end{ulohav}


\baselineskip=1.25\baselineskip
\setlist[enumerate]{label=\textbf{(\alph*)},itemjoin={\quad}}

\results
\parindent=0pt
\parskip=\smallskipamount
\rightskip=0pt plus1fil\relax
\def\printvysl#1#2{\textbf{#1.} #2\par}
\vysld
\endresults



\end{document}

