\documentclass[10pt,a5paper]{article}
\usepackage[margin=1cm]{geometry}
\usepackage[utf8]{inputenc}
\usepackage[IL2]{fontenc}
\usepackage[czech]{babel}
\usepackage{microtype}
\usepackage{amssymb}
\usepackage{amsthm}
\usepackage{amsmath}
\usepackage{xcolor}
\usepackage{graphicx}
\usepackage{wasysym}
\usepackage{multicol}
\usepackage{tikz}

\usepackage[inline]{enumitem}

\newcommand{\R}{\mathbb{R}}

\newcommand{\hint}[1]{{\color{gray}\footnotesize\noindent(Nápověda: #1)}}

\setlist[enumerate]{label={(\alph*)},topsep=\smallskipamount,itemsep=\smallskipamount,parsep=0pt}
\setlist[itemize]{topsep=\smallskipamount,noitemsep}

\def\tisk{%
\newbox\shipouthackbox
\pdfpagewidth=2\pdfpagewidth
\let\oldshipout=\shipout
\def\shipout{\afterassignment\zdvojtmp \setbox\shipouthackbox=}%
\def\zdvojtmp{\aftergroup\zdvoj}%
\def\zdvoj{%
    \oldshipout\vbox{\hbox{%
        \copy\shipouthackbox
        \hskip\dimexpr .5\pdfpagewidth-\wd\shipouthackbox\relax
        \box\shipouthackbox
    }}%
}}%



\newtheorem*{poz}{Pozorování}

\theoremstyle{definition}
\newtheorem{uloha}{\atr Úloha}
\newtheorem{suloha}[uloha]{\llap{$\star$ }Úloha}
\newtheorem*{bonus}{Bonus}
\newtheorem*{defn}{Definice}

\pagestyle{empty}

\let\ee\expandafter

\def\vysld{}
\let\printvysl\relax

\makeatletter
\long\def\vyslplain#1{\ee\ee\ee\gdef\ee\ee\ee\vysld\ee\ee\ee{\ee\vysld\ee\printvysl\ee{\the\c@uloha}{#1}}}
\let\vysl\vyslplain

\def\locvysl#1{\ee\gdef\ee\locvysld\ee{\locvysld\item #1}}
\let\lv\locvysl

\newenvironment{ulohav}[1][]{\begin{uloha}[#1]\gdef\locvysld{\begin{enumerate}}}{\ee\vyslplain\ee{\locvysld\end{enumerate}}\end{uloha}}
\def\stitem{\@noitemargtrue\@item[$\star$ \@itemlabel]}

\makeatother

\def\atr{}
\def\basic{\def\atr{\llap{\mdseries$\sun$ }\gdef\atr{}}}
\def\interest{\def\atr{\llap{$\star$ }\gdef\atr{}}}
\let\mb\mathbf

\begin{document}

%\tisk


\section*{12. Lineární kombinace}


\begin{ulohav}
Vyjádřete vektor $\mb z$ jako lineární kombinaci vektorů $\mb u$, $\mb v$, případně $\mb w$, jestliže
\begin{enumerate}
    \item $\mb z = (2; 10)$, $\mb u = (1; 3)$, $\mb v = (-2; 2)$,\lv{$\mb z = 3\mb u + \frac12 \mb v$}
    \item $\mb z = (2; 10; 0)$, $\mb u = (1; 3; 0)$, $\mb v = (-2; 2; 0)$, $\mb w = (0;0;1)$,\lv{$\mb z = 3\mb u + \frac12 \mb v \mathbin{(+} 0\mb w)$}
    \item $\mb z = (2; -2; -10)$, $\mb u = (2; 1; -1)$, $\mb v = (2; 3; 2)$, $\mb w = (4; 5; -2)$,\lv{$\mb z = 2\mb u - 3\mb v + \mb w$}
    \item $\mb z = (1; 0)$, $\mb u = \bigl(\frac{\sqrt2}{2}; \frac{\sqrt2}{2}\bigr)$, $\mb v = \bigl(\frac{\sqrt2}{2}; -\frac{\sqrt2}{2}\bigr)$,\lv{$\mb z = \frac{\sqrt2}{2}\mb u + \frac{\sqrt2}{2}\mb v$}
\end{enumerate}
\end{ulohav}

\begin{uloha}
Nalezněte všechna reálná čísla $x$ taková, že vektor $\mb z = (1; 5; x)$ je lineární kombinací vektorů $\mb u = (1; -1; 2)$ a $\mb v = (1; 2; -1)$.\vysl{$x = -4$}
\end{uloha}

\begin{ulohav}
Mějme trojúhelník $ABC$ a označme $\mb u = C-B$, $\mb v = C-A$. Zapište jako lineární kombinaci vektorů $\mb u $ a $\mb v$ vektor
\begin{enumerate}
    \item $\mb w_1 = B-A$,\lv{$\mb w_1 = -\mb u + \mb v$}
    \item $\mb w_2 = S_{BC} - A$,\lv{$\mb w_2 = -\frac12 \mb u + \mb v$}
    \item $\mb w_3 = T - A$, kde $T$ je těžiště $\triangle ABC$. \hint{Těžiště se nachází ve dvou třetinách těžnice.}\lv{$\mb w_3 = -\frac13 \mb u + \frac23 \mb v$}
\end{enumerate}
\end{ulohav}

\begin{ulohav}
Mějme krychli $ABCDEFGH$. Zapište vektory $\mb x_1 = G-H$, $\mb x_2 = G-A$, $\mb x_3 = B-S_{AH}$ jako lineární kombinaci vektorů
\begin{enumerate}
    \item $\mb e_1 = A-D$, $\mb e_2 = C-D$, $\mb e_3 = H-D$,\lv{$\mb x_1 = \mb e_2$,\quad $\mb x_2 = -\mb e_1 + \mb e_2 + \mb e_3$,\quad $\mb x_3 = \frac12 \mb e_1 + \mb e_2 - \mb e_3$}
    \item $\mb i = B-A$, $\mb j = C-A$, $\mb k = H-A$.\lv{$\mb x_1 = \mb i$,\quad $\mb x_2 = \mb i + \mb k$,\quad $\mb x_3 = \mb i - \frac12 \mb k$}
\end{enumerate}
\end{ulohav}

\interest
\begin{uloha}
Nalezněte všechny vektory $\mb w$, které budou lineárními kombinacemi vektorů $\mb u = (2; 1; 1)$ a $\mb v = (1; 2; 1)$, budou kolmé na vektor $(1; 1; 1)$ a jejich velikost bude rovna $2$.\vysl{$\mb w_1 = (\sqrt 2; -\sqrt2; 0) = \sqrt 2 \mb u - \sqrt 2 \mb v$, $\mb w_2 = (-\sqrt 2; \sqrt2; 0) = -\sqrt 2 \mb u + \sqrt 2 \mb v$.}
\end{uloha}

\begin{ulohav}
\emph{Osel} je figurka, která umí táhnout cca jako jezdec, ale jen \uv{na jednu stranu}:
\[ \begin{tikzpicture}[scale=.75]
\foreach \y in {1,3}{
    \foreach \x in {0,2,4}{
        \fill[gray!50!white] (\x,\y) rectangle (1+\x,1+\y);}}
\foreach \y in {0,2,4}{
    \foreach \x in {1,3}{
        \fill[gray!50!white] (\x,\y) rectangle (1+\x,1+\y);}}
\draw (0, 0) grid (5, 5);
\node[circle] (osel) at (2.5, 2.5) {\includegraphics[height=.6cm]{osel.png}};
\begin{scope}[thick]
\draw[->] (osel) -- ++(-1,2);
\draw[->] (osel) -- ++(1,-2);
\draw[->] (osel) -- ++(2,1);
\draw[->] (osel) -- ++(-2,-1);
\end{scope}
\end{tikzpicture} \]
\begin{enumerate}
    \item Jak musíme s oslem táhnout, abychom ho posunuli přesně o 10 polí \uv{západně}?\lv{4 tahy \uv{doleva dolů} a 2 tahy \uv{doleva nahoru}}
    \item Je možné s oslem doskákat z jednoho rohu šachovnice $8\times8$ do protějšího?\lv{ne}
    \stitem Zkuste popsat pole, na která se osel (ne)může dostat.
\end{enumerate}
\end{ulohav}

\newpage
\parindent=0pt
\parskip=\smallskipamount
\def\printvysl#1#2{\textbf{#1.} #2\par}
\vysld


\end{document}