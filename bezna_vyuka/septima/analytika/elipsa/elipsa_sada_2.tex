\documentclass[8pt,a5paper]{extarticle}
\usepackage[margin=.6cm]{geometry}
\usepackage[utf8]{inputenc}
\usepackage[IL2]{fontenc}
\usepackage[czech]{babel}
\usepackage{microtype}
\usepackage{amssymb}
\usepackage{amsthm}
\usepackage{amsmath}
\usepackage{xcolor}
\usepackage{graphicx}
\usepackage{wasysym}
\usepackage{multicol}
\usepackage{tikz}

\usepackage[inline]{enumitem}

\newcommand{\R}{\mathbb{R}}

\newcommand{\hint}[1]{{\color{gray}\footnotesize\noindent(Nápověda: #1)}}

\setlist[enumerate]{label={(\alph*)},topsep=\smallskipamount,itemsep=\smallskipamount,parsep=0pt,itemjoin={\quad}}
\setlist[itemize]{topsep=\smallskipamount,noitemsep}

\def\tisk{%
\newbox\shipouthackbox
\pdfpagewidth=2\pdfpagewidth
\let\oldshipout=\shipout
\def\shipout{\afterassignment\zdvojtmp \setbox\shipouthackbox=}%
\def\zdvojtmp{\aftergroup\zdvoj}%
\def\zdvoj{%
    \oldshipout\vbox{\hbox{%
        \copy\shipouthackbox
        \hskip\dimexpr .5\pdfpagewidth-\wd\shipouthackbox\relax
        \box\shipouthackbox
    }}%
}}%

\let\results\newpage
\let\endresults\relax

\def\resultssame{%
    \long\def\results##1\endresults{%
        \vfill
        \noindent\rotatebox{180}{\vbox{##1}}%
    }%
}


\newtheorem*{poz}{Pozorování}

\theoremstyle{definition}
\newtheorem{uloha}{\atr Úloha}
\newtheorem{suloha}[uloha]{\llap{$\star$ }Úloha}
\newtheorem*{bonus}{Bonus}
\newtheorem*{defn}{Definice}

\pagestyle{empty}

\let\ee\expandafter

\def\vysld{}
\let\printvysl\relax
\let\printalphvysl\relax

\makeatletter
\long\def\vyslplain#1{\ee\ee\ee\gdef\ee\ee\ee\vysld\ee\ee\ee{\ee\vysld\ee\printvysl\ee{\the\c@uloha}{#1}}}
\let\vysl\vyslplain

\def\locvysl#1{\ee\gdef\ee\locvysld\ee{\locvysld\item #1}}
\let\lv\locvysl

\newenvironment{ulohav}[1][]{\begin{uloha}[#1]\gdef\locvysld{\begin{enumerate*}}}{\ee\vyslplain\ee{\locvysld\end{enumerate*}}\end{uloha}}
\def\stitem{\@noitemargtrue\@item[$\star$ \@itemlabel]}

\makeatother

\def\atr{}
\def\basic{\def\atr{\llap{\mdseries$\sun$ }\gdef\atr{}}}
\def\interest{\def\atr{\llap{$\star$ }\gdef\atr{}}}
\def\iinterest{\def\atr{\llap{$\star\star$ }\gdef\atr{}}}
\let\mb\mathbf



\begin{document}

%\tisk
%\resultssame

\section*{21. Další úlohy o elipse}

\begin{uloha}\label{elipsy}
Jaké rovnice budou mít níže znázorněné elipsy?
\[ \begin{tikzpicture}[scale=.5]
    \draw[help lines, color=gray, dashed] (-10,-4) grid[step=1] (10,4);
    \draw[->] (-10,0)--(10,0) node[right]{$x$};
    \draw[->] (0,-4)--(0,4) node[below left]{$y$};
    %\node[below left] at (0,0) {$O$};
    \foreach \ii in {-10,...,10} {
        \ifnum \ii=0 \else
            \node[below] at (\ii,0) {$\scriptstyle\ii$};
        \fi
    }
    \foreach \ii in {-4,...,4} {
        \ifnum \ii=0 \else
            \node[right] at (0,\ii) {$\scriptstyle\ii$};
        \fi
    }
    \begin{scope}[thick]
        %
        \draw (-6,-.5) ellipse (2 and 3.5);
        \node[above] at (-6,3) {(a)};
        %
        \draw (-2,1) ellipse (2 and 1);
        \node[above] at (-2,2) {(b)};
        %
        \draw (6,1) ellipse (4 and 3);
        \node at (2,3) {(c)};
        %
        \draw (3.5,-2.5) ellipse (6.5 and .5);
        \node[below] at (3.5,-3) {(d)};
        %
    \end{scope}
\end{tikzpicture} \]
\vysl{\begin{enumerate*}\everymath{\displaystyle}
    \item $\dfrac{(x+6)^2}{2^2} + \dfrac{\left(y+\frac12\right)^2}{\left(\frac72\right)^2} = 1$
    \item $\dfrac{(x+2)^2}{2^2} + \dfrac{(y-1)^2}{1^2} = 1$
    \item $\dfrac{(x-6)^2}{4^2} + \dfrac{(y-1)^2}{3^2} = 1$
    \item $\dfrac{\left(x-\frac72\right)^2}{\left(\frac{13}2\right)^2} + \dfrac{\left(y+\frac52\right)^2}{\left(\frac12\right)^2} = 1$
\end{enumerate*}}
\end{uloha}

\begin{uloha}
Zkuste si nejprve tipnout, kde budou mít elipsy z Úlohy \ref{elipsy} ohniska (na základě \uv{provázkové} konstrukce nebo jakkoliv jinak). Potom pro ony elipsy spočtěte přesně excentricitu a porovnejte ji se svými tipy.
\end{uloha}

\begin{uloha}
Tipněte si, co by to tak mohly být \emph{hlavní vrcholy} a \emph{vedlejší vrcholy} elipsy (nebo to najděte na netu), a vyznačte si je do obrázku v Úloze \ref{elipsy}.
\end{uloha}

\interest
\begin{uloha}\label{obsah}
Zkuste vymyslet vzorec pro obsah elipsy, jejíž poloosy mají délky $a$ a $b$. (Poznámka: pro obvod žádný \uv{hezký} vzorec není.) \hint{Elipsa je jenom \uv{roztažená} kružnice.}\vysl{$S = \pi a b$}
\end{uloha}

\begin{ulohav}
Země je Slunci nejblíže cca $147{,}09\cdot10^6\,\mathrm{km}$ (tzv. \emph{perihelium}) a nejdále cca $152{,}10\cdot10^6\,\mathrm{km}$ (tzv. \emph{afelium}). Podle 1. Keplerova zákona se Země pohybuje po elipe, v jejímž jednom ohnisku je Slunce.
\begin{enumerate}
    \item Ze zadaných údajů dopočtěte délky poloos a excentricitu trajektorie Země.\lv{$a \doteq 149{,}60\cdot10^6\,\mathrm{km}$; $b \doteq 149{,}57\cdot10^6\,\mathrm{km}$; $e \doteq 2{,}51 \cdot10^6\,\mathrm{km}$}
    \stitem Země oběhne Slunce přesně za jeden (astronomický) rok. Jak dlouho Zemi trvá dostat se z perihelia do vedlejšího vrcholu elipsy? Jak dlouho z afelia? Využijte vzorec z Úlohy~\ref{obsah} a
toho, že podle 2. Keplerova zákona jsou plochy opsané průvodičem za jednotku času stejné.\lv{Z perihelia je to cca $0{,}2473$ roku $\doteq 90{,}3$ dní a z afelia cca $0{,}2527$ roku $\doteq 92{,}3$ dní.}
    \stitem Jak dlouho by trval jeden oběh Země kolem Slunce, pokud by Země obíhala po kružnici o poloměru $147{,}09\cdot10^6\,\mathrm{km}$? Použijte 3. Keplerův zákon.\lv{cca $0{,}975$ roku $\doteq 356{,}1$ dní}
\end{enumerate}
\end{ulohav}

\begin{uloha}
Popište množinu všech bodů v rovině, jejichž vzdálenost od osy $y$ je dvojnásobná oproti vzdálenosti od bodu $[3;0]$.\vysl{jde o elipsu se středem v bodě $[4;0]$, hlavní osa je totožná s osou $x$, $a = 2$, $b = \sqrt3$}
\end{uloha}

%\interest
\begin{uloha}
Do elipsy je vepsán čtverec; určete délky stran onoho čtverce v závislosti na délkách poloos elipsy.\vysl{$2\sqrt{\frac{a^2b^2}{a^2+b^2}}$}
\end{uloha}

\begin{uloha}
Určete rovnice rovnoběžek s přímkou $y = 2x$, které jsou tečnami elipsy o rovnici $x^2 + 6y^2 = 6$.\vysl{$y = 2x+5$ a $y = 2x-5$}
\end{uloha}

\begin{ulohav}
Kolmý řez bagetou je elipsa o délkách poloos $a = 4\,\mathrm{cm}$ a $b = 2{,}5\,\mathrm{cm}$.
\begin{enumerate}
    \item Co bude řezem, pokud bagetu rozkrojíme pod úhlem $45^\circ$ tak, že řez bude procházet hlavní osou?\lv{elipsa o rozměrech $a = 4$, $b = 2{,}5\cdot\sqrt2$}
    \item Co bude řezem, pokud bagetu rozkrojíme pod úhlem $45^\circ$ tak, že řez bude procházet vedlejší osou?\lv{elipsa o rozměrech $a = 4\sqrt2$, $b = 2{,}5$}
    \item Ve kterém případě se na výsledný kus pečiva vejde více pomazánky? Použijte výsledek Úlohy \ref{obsah}.\lv{vyjde to nastejno}
\end{enumerate}
\end{ulohav}


\baselineskip=1.25\baselineskip
\setlist[enumerate]{label=\textbf{(\alph*)},itemjoin={\quad}}

\results
\parindent=0pt
\parskip=\smallskipamount
\rightskip=0pt plus1fil\relax
\def\printvysl#1#2{\textbf{#1.} #2\par}
\vysld
\endresults

\end{document}