\documentclass[10pt,a4paper]{article}
\usepackage[utf8]{inputenc}
\usepackage[IL2]{fontenc}
\usepackage[czech]{babel}
\usepackage{amsmath}
\usepackage{amsthm}
\usepackage[margin=1cm]{geometry}
\usepackage[inline]{enumitem}

\setlist[enumerate]{label={(\alph*)},itemjoin={\quad},topsep=\smallskipamount,parsep=0pt,itemsep=\smallskipamount}

\usepackage{tikz,pgfplots}
\pgfplotsset{compat=1.18,width=.8\hsize}

\theoremstyle{definition}
\newtheorem{uloha}{Úloha}
\newtheorem{suloha}[uloha]{\llap{$\star$ }Úloha}

\pagestyle{empty}

\let\ee\expandafter

\def\vysld{}
\let\printvysl\relax

\let\results\newpage
\let\endresults\relax

\def\resultssame{%
    \long\def\results##1\endresults{%
        \vfill
        \noindent\rotatebox{180}{\vbox{##1}}%
    }%
}

\makeatletter
\long\def\vyslplain#1{\ee\ee\ee\gdef\ee\ee\ee\vysld\ee\ee\ee{\ee\vysld\ee\printvysl\ee{\the\c@uloha}{#1}}}
\let\vysl\vyslplain

\def\locvysl#1{\ee\gdef\ee\locvysld\ee{\locvysld\item #1}}
\let\lv\locvysl

\newenvironment{ulohav}[1][]{\begin{uloha}[#1]\gdef\locvysld{\begin{enumerate*}}}{\ee\vyslplain\ee{\locvysld\end{enumerate*}}\end{uloha}}
\def\stitem{\@noitemargtrue\@item[$\star$ \@itemlabel]}

\makeatother

\newcommand{\staritem}{\item[\addtocounter{enumi}{1}$\star$ (\alph{enumi})]}

\newcommand{\hint}[1]{{\footnotesize\textcolor{gray}{(Nápověda: #1)}}}

\begin{document}

\section*{23. Parabolická všehochuť}

\begin{uloha}
Určete rovnice (ve vrcholovém tvaru) parabol znázorněných níže.
\end{uloha}
\vysl{\begin{enumerate*}\item $-(y-4) = (x+3)^2$ \item $9(x-1) = (y+3)^2$ \item $-\frac13(x+7) = (y+4)^2$ \item $4(y-5) = (x-3)^2$ \end{enumerate*}}

\hbox to\hsize{\hfil
\begin{tikzpicture}
  \begin{axis}[grid=both,ymin=-6,ymax=6,xmax=10,xmin=-10,xlabel={$x$},ylabel={$y$},axis lines=middle,unit vector ratio=1 1,ytick distance=1,xtick distance=1]
  %
  \addplot[smooth, domain=-3.2:3.2] (\x-3,-{(\x)^2+4});
  \node[above] at (-3,4) {(a)};
  %
  \addplot[smooth, domain=-1:3] ({(\x)^2+1},{3*(\x-1)});
  \node[left] at (1,-3) {(b)};
  %
  \addplot[smooth, domain=-1:1] ({-3*(\x)^2-7},\x-4);
  \node[right] at (-7,-4) {(c)};
  %
  \addplot[smooth, domain=-2:2] (\x+3,{((\x)^2)/4+5});
  \node[below] at (3,5) {(d)};
  %%%% zadane paraboly
%   \addplot[smooth, domain=-2:2] (\x-7,{(\x)^2+3});
%   \draw (-10,2.75) -- (-4,2.75);
%   \addplot[mark=*] coordinates {(-7,3.25)};
%   %
%   \addplot[smooth, domain=-3:3] ({2*((\x)^2)-1},\x-1);
%   \draw (-1.125,-3) -- (-1.125,1);
%   \addplot[mark=*] coordinates {(-.875,-1)};
%   %
%   \addplot[smooth, domain=-6:2] ({-((\x)^2)/2-2},\x+4);
%   \draw (-1.5,2) -- (-1.5,6);
%   \addplot[mark=*] coordinates {(-2.5,4)};
%   %
  \end{axis}
\end{tikzpicture}%
\hfil}

\begin{uloha}
Pro paraboly z předchozí úlohy určete hodnoty parametru, souřadnice ohniska a rovnici řídící přímky.
\end{uloha}
\vysl{\begin{enumerate*}
\item $p=\frac12$, $F[-3;3{,}75]$, $y=4{,}25$ %$x-4 = -(y+3)^2$
\item $p=\frac92$, $F[3{,}25;-3]$, $x = -1{,}25$%$9(y+3) = (x-1)^2$
\item $p=\frac16$, $F\bigl[-\frac{85}{12}; -4\bigr]$, $x = -\frac{83}{12}$%$\frac13(x+7) = -(y+4)^2$
\item $p=2$, $F[3;6]$, $y=4$ %$4(y-5) = (x-3)^2$
\end{enumerate*}}

\begin{uloha}
Načrtněte paraboly (včetně řídící přímky a ohniska) o rovnicích \begin{enumerate*} \item $y - 3 = (x+7)^2$, \item $x-2 y^2-4 y-6 = 0$, \item $2 x+y^2-8 y+18 = 0$ \end{enumerate*} (asi bude nejprve nutné najít souřadnice vrcholu a ohniska a rovnici řídící přímky).
\end{uloha}
\setbox1=\hbox{%
\begin{tikzpicture}
    \begin{axis}[grid=both,ymin=-6,ymax=6,xmax=10,xmin=-10,xlabel={$x$},ylabel={$y$},axis lines=middle,unit vector ratio=1 1,ytick distance=1,xtick distance=1]
  %%%% zadane paraboly
  \addplot[smooth, domain=-2:2] (\x-7,{(\x)^2+3});
  \draw (-10,2.75) -- (-4,2.75);
  \addplot[mark=*] coordinates {(-7,3.25)};
  %
  \addplot[smooth, domain=-3:3] ({2*((\x)^2)-1},\x-1);
  \draw (-1.125,-3) -- (-1.125,1);
  \addplot[mark=*] coordinates {(-.875,-1)};
  %
  \addplot[smooth, domain=-6:2] ({-((\x)^2)/2-2},\x+4);
  \draw (-1.5,2) -- (-1.5,6);
  \addplot[mark=*] coordinates {(-2.5,4)};
  %
  \end{axis}
\end{tikzpicture}}
\vysl{\box1\relax}

\begin{uloha}
Vyberte správnou volbu z podtržených: Čím je parametr větší, tím je parabola \underline{více} / \underline{méně} \uv{špičatá}.
\end{uloha}
\vysl{méně}

\begin{uloha}
Nalezněte rovnici paraboly, která
\begin{enumerate}
    \item má ohnisko v $[-2;1]$ a řídící přímku $x = 0$,
    \item má ohnisko v $[0;1]$ a vrchol v $[0;5]$,
    \item má vrchol v $[-4;3]$, prochází bodem $[-2;2]$ a její řídící přímka je rovnoběžná s osou $y$,
    \item má vrchol v $[1;1]$, přímka $y=2x+2$ je její tečnou a její řídící přímka je rovnoběžná s osou $x$,
    \stitem má ohnisko v $[0;0]$, prochází bodem $[3;4]$ a její řídící přímka je rovnoběžná s osou $y$.
\end{enumerate}
\end{uloha}
\vysl{\begin{enumerate*}
\item $-4(x+1) = (y-1)^2$
\item $-16(y-5) = x^2$
\item $\frac12(x+4) = (y-3)^2$
\item $-3(y-1) = (x-1)^2$
\item $4(x+1) = y^2$ a $-16(x-4) = y^2$ (dvě řešení)
\end{enumerate*}}

\begin{suloha}
Nalezněte poloměr co největší kružnice, která se dotýká paraboly $y = x^2$ pouze v jejím vrcholu a jejíž střed leží na kladné části osy $y$.
\end{suloha}
\vysl{$\frac12$}

\begin{ulohav}
Na parabole $y=x^2$ nalezněte bod nejblíže k
\begin{enumerate}
\item přímce $y = 2x-2$, \hint{Hledejte rovnoběžnou tečnu.}\lv{$[1;1]$}
\stitem bodu $[9;6]$. \hint{Derivujte. Derivací je kubický polynom, jehož jedním kořenem je $-1$.}\lv{$\bigl[\frac12 + \frac{\sqrt{19}}{2}; 5+\frac{\sqrt{19}}{2}\bigr]$}
\end{enumerate}
\end{ulohav}



\baselineskip=1.25\baselineskip
\setlist[enumerate]{label=\textbf{(\alph*)},itemjoin={\quad}}

\results
\parindent=0pt
\parskip=\smallskipamount
\rightskip=0pt plus1fil\relax
\def\printvysl#1#2{\textbf{#1.} #2\par}
\vysld
\endresults


\end{document}