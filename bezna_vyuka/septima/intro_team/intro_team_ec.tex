\documentclass[12pt,a4paper]{article}
\usepackage[margin=.5in]{geometry}
\usepackage[utf8]{inputenc}
\usepackage[IL2]{fontenc}
\usepackage[czech]{babel}
\usepackage{microtype}
\usepackage{amssymb}
\usepackage{amsthm}
\usepackage{amsmath}
\usepackage{xcolor}
\usepackage{graphicx}

\usepackage[inline]{enumitem}

\newcommand{\R}{\mathbb{R}}

\setlist[enumerate]{label={(\alph*)},topsep=\smallskipamount,itemsep=\smallskipamount}
\setlist[itemize]{topsep=\smallskipamount,noitemsep}

\theoremstyle{definition}
\newtheorem{uloha}{Úloha}
\newtheorem*{bonus}{Bonus}

\pagestyle{empty}

\begin{document}


\section*{EC, úvodní skupinová práce}


\begin{uloha}
Nalezněte všechna reálná čísla $x$ splňující $5^x \cdot \bigl(\sqrt5\bigr)^{x+1} \cdot \bigl(\sqrt[3]5\bigr)^{x+2} = \sqrt[4]5$.
\end{uloha}

\begin{uloha}
Nalezněte všechna reálná řešení rovnice s neznámou $x$
\[ \left(\frac{x}{x+2}\right)^2 - \frac{3x}{x+2} = 4. \]
\end{uloha}

\begin{uloha}
Určete hodnotu $x$, pokud platí
\begin{enumerate}
    \item $\log_3 x = 4$,
    \item $\log_{\sqrt2} x = 4$,
    \item $\log_x 27 = 3$,
    \item $\log_x 5 = -1$,
    \item $x = \log_2 \log_2 16$.
\end{enumerate}
\end{uloha}

\begin{uloha}
Určete reálná čísla $a$, $b$ tak, aby pro funkci $h\colon y = a \log_2 x + b$ platilo $h(4) = 5$ a $h\bigl(\frac14\bigr) = -7$.
\end{uloha}


\begin{uloha}
Jsou-li $a$, $b$, $c$ kladná reálná čísla, vyjádřete pomocí $\log a$, $\log b$ a $\log c$ výraz
\[ \log \sqrt{\frac{10a}{bc}}. \]
\end{uloha}


\begin{uloha}
Určete všechna reálná čísla $x$ z intervalu $\langle0; \pi\rangle$, která splňují $\sin x = \frac{1}{10}$. Uveďte jednak \uv{přesné} výsledky, jednak výsledky zaokrouhlené na dvě desetinná místa.
\end{uloha}


\begin{uloha}
Určete hodnoty $\cos x$, $\cos 2x$ a $\cos \frac x2$, víte-li, že $\sin x = \frac27$ a $x \in \bigl\langle\frac\pi2; \frac{3\pi}{2}\bigr\rangle$.
\end{uloha}


\begin{uloha}
Určete, pro která reálná čísla $x$ má smysl výraz $\log(|2 - x| - 2)$.
\end{uloha}


\begin{uloha}
Je dána rovnice
\[ x^2 - 2 d x + 2 d^2 - 9 = 0 \]
s neznámou $x$ a reálným parametrem $d$. Určete všechny hodnoty $d$, pro které rovnice nemá v reálných číslech řešení.
\end{uloha}


\begin{uloha}
Napište báseň o čtyřech verších na téma \emph{logaritmus}; použijte buď obkročný (\emph{abba}), nebo střídavý (\emph{abab}) rým.
\end{uloha}


\end{document}