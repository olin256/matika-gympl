\documentclass[12pt,a5paper]{article}
\usepackage[margin=1cm]{geometry}
\usepackage[utf8]{inputenc}
\usepackage[IL2]{fontenc}
\usepackage[czech]{babel}
\usepackage{microtype}
\usepackage{amssymb}
\usepackage{amsthm}
\usepackage{amsmath}
\usepackage{xcolor}
\usepackage{graphicx}
\usepackage{wasysym}
\usepackage{multicol}

\usepackage[inline]{enumitem}

\newcommand{\R}{\mathbb{R}}

\newcommand{\hint}[1]{{\color{gray}\footnotesize\noindent(Nápověda: #1)}}

\DeclareMathOperator{\tg}{tg}
\DeclareMathOperator{\cotg}{cotg}

\setlist[enumerate]{label={(\alph*)},topsep=\smallskipamount,itemsep=\medskipamount,parsep=0pt}
\setlist[itemize]{topsep=\smallskipamount,noitemsep}

\def\tisk{%
\newbox\shipouthackbox
\pdfpagewidth=2\pdfpagewidth
\let\oldshipout=\shipout
\def\shipout{\afterassignment\zdvojtmp \setbox\shipouthackbox=}%
\def\zdvojtmp{\aftergroup\zdvoj}%
\def\zdvoj{%
    \oldshipout\vbox{\hbox{%
        \copy\shipouthackbox
        \hskip\dimexpr .5\pdfpagewidth-\wd\shipouthackbox\relax
        \box\shipouthackbox
    }}%
}}%



\newtheorem*{poz}{Pozorování}

\theoremstyle{definition}
\newtheorem{uloha}{\atr Úloha}
\newtheorem{suloha}[uloha]{\llap{$\star$ }Úloha}
\newtheorem*{bonus}{Bonus}
\newtheorem*{defn}{Definice}

\pagestyle{empty}

\let\ee\expandafter

\def\vysld{}
\let\printvysl\relax
\let\printalphvysl\relax

\makeatletter
\long\def\vysl#1{\ee\ee\ee\gdef\ee\ee\ee\vysld\ee\ee\ee{\ee\vysld\ee\printvysl\ee{\the\c@uloha}{#1}}}
\let\vyslplain\vysl

\def\locvysl#1{\ee\gdef\ee\locvysld\ee{\locvysld\item #1}}
\let\lv\locvysl

\newenvironment{ulohav}[1][]{\begin{uloha}[#1]\gdef\locvysld{\begin{enumerate}}}{\ee\vyslplain\ee{\locvysld\end{enumerate}}\end{uloha}}

\def\stitem{\@noitemargtrue\@item[$\star$ \@itemlabel]}

\makeatother

\def\atr{}
\def\basic{\def\atr{\llap{\mdseries$\sun$ }\gdef\atr{}}}
\def\interest{\def\atr{\llap{$\star$ }\gdef\atr{}}}
\let\mb\mathbf

\let\la\langle
\let\ra\rangle
\def\bla{\bigl\la}
\def\bra{\bigr\ra}

\def\R{\mathbb{R}}

\begin{document}

% \tisk


\section*{Nerovnice a jejich soustavy}

\begin{ulohav}
\everymath{\displaystyle}
Vyřešte nerovnice:
\begin{enumerate}
    \item $5(x-1) - x(7-x) \leq x^2$\lv{$\bla-\frac52; \infty\bigr)$}
    \item $\frac{37-2x}{2} + 9 \leq \frac{3x-8}{4} - x$\lv{$\bigl\la \frac{118}{3}; \infty\bigr)$}
\end{enumerate}
\end{ulohav}


\begin{ulohav}
\everymath{\displaystyle}
Vyřešte následující soustavy nerovnic:
\begin{enumerate}
    \item $x \leq 3$,\quad $x \geq 6$\lv{$\emptyset$}
    \item $x \leq 3$,\quad $x \geq 3$\lv{$\{3\}$}
    \item $7 - 7x < 3x + 4$,\quad $7-4x > 3 + 3x$\lv{$\bigl(\frac{3}{10};\frac{4}{7}\bigr)$}
    \item $\frac{1-2x}{3} < \frac{1+3x}{4}$,\quad $1-7x \geq -6x$\lv{$\bigl(\frac{1}{17}; 1\bra$}
\end{enumerate}
\end{ulohav}


\begin{ulohav}
\everymath{\displaystyle}
Vyřešte následující nerovnice:
\begin{enumerate}
    \item $(2-x)(3-x)(4-x) > 0$\lv{$(-\infty;2) \cup (3;4)$}
    \item $(x - \pi)(x - \sqrt{10})(x-\sqrt[3]{31})(7x - 22) \geq 0$\lv{$(-\infty; \sqrt[3]{31}\rangle \cup \bla\pi;\frac{22}{7}\bra \cup \langle \sqrt{10}; \infty)$}
    \item $x^2 - x < 0$\lv{$(0;1)$}
    \item $2 - x - x^2 > 0$\lv{$(-2; 1)$}
    \item $4x^2 - 3 \leq 0$\lv{$\bla-\frac{\sqrt3}{2};\frac{\sqrt3}{2}\bra$}
    \item $(x - 2) (x - 3) \geq 2x$\lv{$(-\infty; 1\ra\cup \la 6; \infty)$}
    \item $(x-1)^2 (x-2)^3 \geq 0$\lv{$\{1\} \cup \la2;\infty)$}
    \item $x^6 \leq x^5$\lv{$\la0;1\ra$}
\end{enumerate}
\end{ulohav}


\interest
\begin{uloha}
Součet druhých mocnin tří po sobě jdoucích celých čísel je menší než 110. O jaká čísla mohlo jít?\vysl{možnosti jsou 123, 234, 345 nebo 456}
\end{uloha}


\newpage
\parindent=0pt
\parskip=\smallskipamount
\def\printvysl#1#2{\textbf{#1.}\ #2\par}
\vysld


\end{document}