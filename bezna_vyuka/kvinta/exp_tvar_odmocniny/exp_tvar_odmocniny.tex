\documentclass[9pt,a5paper]{extarticle}
\usepackage[margin=1cm]{geometry}
\usepackage[utf8]{inputenc}
\usepackage[IL2]{fontenc}
\usepackage[czech]{babel}
\usepackage{microtype}
\usepackage{amssymb}
\usepackage{amsthm}
\usepackage{amsmath}

\usepackage[inline]{enumitem}

\newcommand{\R}{\mathbb{R}}
\newcommand{\Z}{\mathbb{Z}}
\newcommand{\N}{\mathbb{N}}
\newcommand{\Q}{\mathbb{Q}}

\DeclareMathOperator{\tg}{tg}
\DeclareMathOperator{\cotg}{cotg}

\setlist[enumerate]{label={(\alph*)},topsep=\smallskipamount,itemsep=\medskipamount,parsep=0pt}
\setlist[itemize]{topsep=\smallskipamount,noitemsep}

\def\tisk{%
\newbox\shipouthackbox
\pdfpagewidth=2\pdfpagewidth
\let\oldshipout=\shipout
\def\shipout{\afterassignment\zdvojtmp \setbox\shipouthackbox=}%
\def\zdvojtmp{\aftergroup\zdvoj}%
\def\zdvoj{%
    \oldshipout\vbox{\hbox{%
        \copy\shipouthackbox
        \hskip\dimexpr .5\pdfpagewidth-\wd\shipouthackbox\relax
        \box\shipouthackbox
    }}%
}}%


\theoremstyle{definition}
\newtheorem{uloha}{Úloha}
\newtheorem{suloha}[uloha]{\llap{$\star$ }Úloha}
\newtheorem{xuloha}[uloha]{\llap{\mdseries$\sun$ }Úloha}
\newtheorem*{bonus}{Bonus}

\pagestyle{empty}

\let\ee\expandafter

\def\vysld{}
\let\printvysl\relax

\makeatletter
\def\vyslplain#1{\ee\ee\ee\gdef\ee\ee\ee\vysld\ee\ee\ee{\ee\vysld\ee\printvysl\ee{\the\c@uloha}{#1}}}
\let\vysl\vyslplain

\makeatother

\newenvironment{ulohav}[1][]{\begin{uloha}[#1]\gdef\locvysld{\begin{enumerate}}}{\ee\vyslplain\ee{\locvysld\end{enumerate}}\end{uloha}}
\newenvironment{sulohav}[1][]{\begin{suloha}[#1]\gdef\locvysld{\begin{enumerate*}}}{\ee\vyslplain\ee{\locvysld\end{enumerate*}}\end{suloha}}
\def\locvysl#1{\ee\gdef\ee\locvysld\ee{\locvysld\item #1}}
\let\lv\locvysl

\def\exp#1{\cdot 10^{#1}}

\begin{document}

%\tisk

\section{Stručné opáčko exponenciálního tvaru}

\mathcode`\,="013B

\everymath{\displaystyle}

\begin{ulohav}
Vyjádřete v exponenciálním tvaru:
\begin{enumerate}
    \item $310000 \cdot 4000000$\lv{$1,24 \exp{12}$}
    \item $0,00077 \cdot 0,0012$\lv{$9,24 \exp{-7}$}
    \item $\frac{0,0003}{6000}$\lv{$5\exp{-8}$}
    \item $0,25 \cdot \frac{300}{0,00005}$\lv{$1,5\exp{6}$}
\end{enumerate}
\end{ulohav}

\begin{ulohav}
Uveďte následující čísla $x$ v exp. tvaru s přesností na $n$ platných cifer:
\begin{enumerate}
    \item $x = 123456$, $n = 3$\lv{$1,23 \exp{5}$}
    \item $x = 123456$, $n = 5$\lv{$1,2346 \exp{5}$}
    \item $x = 0,007007$, $n = 3$\lv{$7,01 \exp{-3}$}
    \item $x = 0,007007$, $n = 2$\lv{$7,0 \exp{-3}$}
\end{enumerate}
\end{ulohav}


\section{Úpravy výrazů s odmocninami}

Pravidla pro počítání s odmocninami (která platí, pokud mají obě strany rovnosti smysl):
\begin{itemize}
    \item $\sqrt[n]{a \cdot b} = \sqrt[n]a \cdot \sqrt[n]{b}$
    \item $\sqrt[n]{\frac ab} = \frac{\sqrt[n]a}{\sqrt[n]b}$
    \item $\sqrt[n]{a^m} = \bigl(\!\sqrt[n]a\bigr)^m$
    \item $\sqrt[n]{\!\sqrt[m]{a}} = \sqrt[m\cdot n]{a}$
    \item $\sqrt[n]{a^m} = \sqrt[kn]{a^{km}}$
\end{itemize}
$a,\, b \in \R$, $m,\, n,\, k \in \N$.

\begin{ulohav}
Vyjádřete jako jednu odmocninu z mocniny:
\begin{enumerate}
    \item $\sqrt{3 \cdot \sqrt3}$\lv{$\sqrt[4]{3^3}$}
    \item $\sqrt{8 \cdot \sqrt{4 \cdot \sqrt2}}$\lv{$\sqrt[8]{2^{17}}$}
    \item $\sqrt[3]{5^2 \cdot \sqrt 5}$\lv{$\sqrt[6]{5^5}$}
    \item $\sqrt[4]{11} \cdot \bigl(\sqrt[4]{11}\bigr)^3$\lv{11}
    \item $\sqrt{5 \cdot \sqrt[3]{\frac15}\cdot \sqrt[4]5}$\lv{$\sqrt[24]{5^{11}}$}
\end{enumerate}
\end{ulohav}

\begin{ulohav}
Vyjádřete jako jednu odmocninu z mocniny:
\begin{enumerate}
    \item $\sqrt{x \cdot \sqrt x}$\lv{$\sqrt[4]{x^3}$}
    \item $\frac{\sqrt[5]{u \cdot \sqrt[6]{u^2}}}{\sqrt u}$\lv{$\sqrt[30]{u^{-7}}$}
\end{enumerate}
\end{ulohav}


\newpage
\setlist[enumerate]{itemjoin={\quad},label={(\alph*)}}
\parindent=0pt
\parskip=\smallskipamount
\def\printvysl#1#2{\textbf{#1.}\ #2\par}
\vysld


\end{document}