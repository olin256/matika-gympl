\documentclass[11pt,a5paper]{article}
\usepackage[margin=1cm]{geometry}
\usepackage[utf8]{inputenc}
\usepackage[IL2]{fontenc}
\usepackage[czech]{babel}
\usepackage{microtype}
\usepackage{amssymb}
\usepackage{amsthm}
\usepackage{amsmath}
\usepackage{xcolor}
\usepackage{graphicx}
\usepackage{wasysym}
\usepackage{multicol}

\usepackage[inline]{enumitem}

\newcommand{\R}{\mathbb{R}}

\newcommand{\hint}[1]{{\color{gray}\footnotesize\noindent(Nápověda: #1)}}

\DeclareMathOperator{\tg}{tg}
\DeclareMathOperator{\cotg}{cotg}

\setlist[enumerate]{label={(\alph*)},topsep=\smallskipamount,itemsep=\smallskipamount,parsep=0pt}
\setlist[itemize]{topsep=\smallskipamount,noitemsep}

\def\tisk{%
\newbox\shipouthackbox
\pdfpagewidth=2\pdfpagewidth
\let\oldshipout=\shipout
\def\shipout{\afterassignment\zdvojtmp \setbox\shipouthackbox=}%
\def\zdvojtmp{\aftergroup\zdvoj}%
\def\zdvoj{%
    \oldshipout\vbox{\hbox{%
        \copy\shipouthackbox
        \hskip\dimexpr .5\pdfpagewidth-\wd\shipouthackbox\relax
        \box\shipouthackbox
    }}%
}}%



\newtheorem*{poz}{Pozorování}

\theoremstyle{definition}
\newtheorem{uloha}{\atr Úloha}
\newtheorem{suloha}[uloha]{\llap{$\star$ }Úloha}
\newtheorem*{bonus}{Bonus}
\newtheorem*{defn}{Definice}

\pagestyle{empty}

\let\ee\expandafter

\def\vysld{}
\let\printvysl\relax
\let\printalphvysl\relax

\makeatletter
\long\def\vysl#1{\ee\ee\ee\gdef\ee\ee\ee\vysld\ee\ee\ee{\ee\vysld\ee\printvysl\ee{\the\c@uloha}{#1}}}
\let\vyslplain\vysl

\def\locvysl#1{\ee\gdef\ee\locvysld\ee{\locvysld\item #1}}
\let\lv\locvysl

\newenvironment{ulohav}[1][]{\begin{uloha}[#1]\gdef\locvysld{\begin{enumerate}}}{\ee\vyslplain\ee{\locvysld\end{enumerate}}\end{uloha}}

\def\stitem{\@noitemargtrue\@item[$\star$ \@itemlabel]}

\makeatother

\def\atr{}
\def\basic{\def\atr{\llap{\mdseries$\sun$ }\gdef\atr{}}}
\def\interest{\def\atr{\llap{$\star$ }\gdef\atr{}}}
\let\mb\mathbf

\def\R{\mathbb{R}}

\begin{document}

% \tisk

\section{Definiční obory}

\begin{uloha}
Co nejefektivněji určete, která z čísel $-3$, $0$, $1$, $4$ patří do definičního oboru výrazu $\dfrac{1}{\sqrt{x^2+x-6}}$.\vysl{jenom 4}
\end{uloha}

\begin{ulohav}\everymath{\displaystyle}
Určete definiční obory výrazů:\par
\vskip-\bigskipamount
\vskip0pt
\begin{flushleft}
\baselineskip=2\baselineskip
\begin{enumerate*}[itemjoin=\qquad]
    \item $\frac{x^2 - 3x + 2}{x^2 - 5x + 6}$\lv{$\R\setminus\{2; 3\}$}
    \item $\frac{1}{\sqrt x} + \sqrt{2-x}$\lv{$(0;2\rangle$}
    \item $|x| \cdot \sqrt{3-|x+1|}$\lv{$\langle-4;2\rangle$}
    \item $\left| \sqrt x + 1 \right|$\lv{$\langle0; \infty)$}
    \item $\frac{1}{\sqrt{x^2}}$\lv{$\R\setminus\{0\}$}
\end{enumerate*}
\end{flushleft}
\end{ulohav}

\begin{ulohav}
Vymyslete výraz, jehož definiční obor bude
\begin{enumerate}
    \item $\R\setminus\{1;2;3\}$,\lv{např. $\frac{1}{(x-1)(x-2)(x-3)}$}
    \item $(-2; 2)$,\lv{např. $\frac{1}{\sqrt{2-|x|}}$ nebo třeba $\frac{1}{\sqrt{x+2}} - \sqrt{\frac1{2-x}}$ apod.}
    \item $\langle-1; \infty) \setminus\{2\}$\lv{např. $\frac{\sqrt{x+1}}{x-2}$}.
\end{enumerate}
\end{ulohav}


\section{Vyjadřování neznámých}
% vykradacka realisticky.cz

\begin{ulohav}\everymath{\displaystyle}
Vyjádřete
\begin{enumerate}
    \item ze vzorce pro velikost magnetické indukce $B = \mu \frac{NI}{l}$ počet závitů $N$,\lv{$N = \frac{Bl}{\mu I}$}
    \item ze stavové rovnice plynu $\frac{p_1V_1}{T_1} = \frac{p_2V_2}{T_2}$ teplotu $T_2$,\lv{$T_2 = \frac{p_2V_2T_1}{p_1T_1}$}
    \item ze vzorce pro zvětšení mikroskopu $\frac{\tau'}{\tau} = \frac{\Delta}{f_1} \cdot \frac{d}{f_2}$ ohniskovou vzdálenost $f_1$,\lv{$f_1 = \frac{\Delta d \tau}{f_2 \tau'}$}
    \item ze vzorce zrychlení rovnoměrně zrychleného pohybu $a = \frac{v-v_0}{t}$ poč. rychlost $v_0$,\lv{$v_0 = v - at$}
    \item ze vzorce pro objemovou roztažnost kapalin $V = V_0(1 + \beta \cdot \Delta t)$ poč. objem $V_0$,\lv{$V_0 = \frac{V}{1+\beta \cdot \Delta t}$}
    \item z předchozího vzorce změnu teploty $\Delta t$,\lv{$\Delta t = \frac{V-V_0}{V_0 \beta}$}
    \item ze vzorce pro výšku svislého vrhu $h = v_0 t - \frac12 g t^2$ poč. rychlost $v_0$,\lv{$v_0 = \frac{2h + gt^2}{2t}$}
    \item z předchozího vzorce gravitační zrychlení $g$,\lv{$g = \frac{2v_0t - 2h}{t^2}$}
    \item ze vzorce pro povrch kvádru $S = 2ab + 2bc + 2ac$ délku strany $b$.\lv{$b = \frac{S - 2ac}{2a + 2c}$}
\end{enumerate}
\end{ulohav}


\newpage
\parindent=0pt
\parskip=\smallskipamount
\def\printvysl#1#2{\textbf{#1.}\ #2\par}
\def\printalphvysl#1#2#3{\textbf{#1}(#2)\ #3\par}
\vysld


\end{document}