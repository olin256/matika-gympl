\documentclass[9pt,a5paper]{extarticle}
\usepackage[margin=1cm]{geometry}
\usepackage[utf8]{inputenc}
\usepackage[IL2]{fontenc}
\usepackage[czech]{babel}
\usepackage{microtype}
\usepackage{amssymb}
\usepackage{amsthm}
\usepackage{amsmath}
\usepackage{xcolor}

\usepackage[inline]{enumitem}

\newcommand{\R}{\mathbb{R}}

\setlist[enumerate]{label={(\alph*)},topsep=\smallskipamount,noitemsep}
\setlist[itemize]{topsep=\smallskipamount,noitemsep}

\def\tisk{%
\newbox\shipouthackbox
\pdfpagewidth=2\pdfpagewidth
\let\oldshipout=\shipout
\def\shipout{\afterassignment\zdvojtmp \setbox\shipouthackbox=}%
\def\zdvojtmp{\aftergroup\zdvoj}%
\def\zdvoj{%
    \oldshipout\vbox{\hbox{%
        \copy\shipouthackbox
        \hskip\dimexpr .5\pdfpagewidth-\wd\shipouthackbox\relax
        \box\shipouthackbox
    }}%
}}%


\theoremstyle{definition}
\newtheorem{uloha}{Úloha}
\newtheorem*{bonus}{Bonus}

\pagestyle{empty}

\let\ee\expandafter

\def\vysld{}
\let\printvysl\relax
\let\printalphvysl\relax

\makeatletter
\def\vyslplain#1{\ee\ee\ee\gdef\ee\ee\ee\vysld\ee\ee\ee{\ee\vysld\ee\printvysl\ee{\the\c@uloha}{#1}}}

\def\locvysl#1{\ee\gdef\ee\locvysld\ee{\locvysld\item #1}}
\let\lv\locvysl

\newenvironment{ulohav}[1][]{\begin{uloha}[#1]\gdef\locvysld{\begin{enumerate*}}}{\ee\vyslplain\ee{\locvysld\end{enumerate*}}\end{uloha}}

\makeatother

\begin{document}

%\tisk


\section*{Úlohy na Vennovy diagramy}

\let\vysl\vyslplain


\begin{ulohav}
Konečná množina $A$ má podmnožiny $X$ a $Y$. Víme, že platí $|X \cap Y| = 7$, $|Y \setminus X| = 4$, $|X'_{A}| = 15$ (doplněk) a $|X \cup Y| = 18$. Určete \begin{enumerate*} \item $|A|$,\lv{29} \item $|X \setminus Y|$.\lv{7} \end{enumerate*} (\uv{Absolutními hodnotami} zde značíme počet prvků množiny.)
\end{ulohav}

\begin{ulohav} % realisticky.cz
29 studentů v jisté třídě dostalo na začátku školního roku klíč od skříňky a 
přístupový čip. Alespoň jednu věc ztratilo do konce roku 12 studentů, 
právě jednu věc pak 9 studentů. Studentů, kteří ztratili klíč, bylo o pět 
více než studentů, kteří ztratili obě věci. Kolik studentů ztratilo:
\begin{enumerate*}
    \item klíč,\lv{8}
    \item čip,\lv{7}
    \item nic?\lv{17}
\end{enumerate*}
\end{ulohav}

\begin{ulohav} % realisticky.cz
Děti ve školách nejsou v bezpečí. Je potěšující, že 610 studentů se 
nezranilo při pádu z~židličky a 605 se jich nezranilo při pádu na 
chodbě. Přesto alespoň jedno zranění utrpělo 45 dětí. Počet studentů, 
kteří spadli ze židle byl o jednu menší než počet těch, kteří jenom 
padli na chodbě. Kolik studentů spadlo:
\begin{enumerate*}
    \item ze židličky,\lv{22}
    \item  na chodbě?\lv{27}
\end{enumerate*}
\end{ulohav}

\begin{ulohav} % realisticky.cz
Písemná práce z matematiky, které se zúčastnilo 35 studentů, obsahovala tři úlohy. 
Dva studenti vyřešili jenom první úlohu a tři studenti jenom druhou úlohu. První a 
druhou úlohu vyřešilo 16 studentů, druhou a třetí 14 studentů. Všechny úlohy
vyřešilo 10 studentů, první nebo třetí 31 studentů a 3 studenti nevyřešili ani první ani 
druhou úlohu. Kolik studentů vyřešilo
\begin{enumerate*}
    \item aspoň dvě úlohy,\lv{27}
    \item aspoň jednu úlohu?\lv{34}
\end{enumerate*}
\end{ulohav}

\begin{ulohav} % realisticky.cz
Hodina matematiky je v plném proudu. Tahák na češtinu si píše 9 
studentů, na mobilu si hraje 10 přítomných a 12 studentů se 
baví se spolusedícím. Celkem je ve třídě 30 studentů, ale pouze 6 jich 
dělá dvě z uvedených činností najednou. Studentů, kteří si pouze 
píšou tahák je třikrát více než těch kteří stihnou vedle psaní taháku 
ještě mobilovat. Ani jeden ze studentů nezvládne všechny tři činnosti, 
8 studentů píše tahák a nepovídá si při tom. \begin{enumerate*}
    \item Kolik studentů nedělá ani jednu z činností a dává pozor?\lv{5}
    \item Kolik studentů je zticha a nebaví se?\lv{18}
\end{enumerate*}
\end{ulohav}

\begin{uloha} % vlastní úloha; myslím, že může být celkem povznášející objev, že takovéto výpočty fungují pro "libovolnou míru"
Severína a Zlatan oba kandidují na pozici starosty Horní Vsi u Dolního Potoka. Z průzkumu mezi elektorátem vyplynulo, že
\begin{itemize}
    \item alespoň jeden z kandidátů je přípustný pro čtyři pětiny voličů,
    \item těch, co by mohli volit Severínu, je dvakrát více než těch, co budou určitě volit Zlatana,
    \item těch, co mají jasno, koho budou volit (a budou volit), je třikrát více než těch, kterým není po chuti ani jeden kandidát.
\end{itemize}
Určete
\begin{enumerate}
    \item jaká část ze všech oprávněných voličů zvažuje oba kandidáty,\lv{$\frac15$}
    \item jakou část z odevzdaných hlasů může maximálně dostat Severína (počítáme, že ti, kteří nechtějí ani jednoho, k volbám nepůjdou).\lv{$\frac23$}
\end{enumerate}
\end{uloha}


\newpage
\parindent=0pt
\parskip=\smallskipamount
\def\printvysl#1#2{\textbf{#1.}\ #2\par}
\vysld


\end{document}