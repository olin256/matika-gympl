\documentclass[10pt,a5paper]{article}
\usepackage[margin=1cm]{geometry}
\usepackage[utf8]{inputenc}
\usepackage[IL2]{fontenc}
\usepackage[czech]{babel}
\usepackage{microtype}
\usepackage{amssymb}
\usepackage{amsthm}
\usepackage{amsmath}
\usepackage{xcolor}

\usepackage[inline]{enumitem}

\newcommand{\R}{\mathbb{R}}

\setlist[enumerate]{label={(\alph*)},topsep=\smallskipamount,noitemsep}
\setlist[itemize]{topsep=\smallskipamount,noitemsep}

\def\tisk{%
\newbox\shipouthackbox
\pdfpagewidth=2\pdfpagewidth
\let\oldshipout=\shipout
\def\shipout{\afterassignment\zdvojtmp \setbox\shipouthackbox=}%
\def\zdvojtmp{\aftergroup\zdvoj}%
\def\zdvoj{%
    \oldshipout\vbox{\hbox{%
        \copy\shipouthackbox
        \hskip\dimexpr .5\pdfpagewidth-\wd\shipouthackbox\relax
        \box\shipouthackbox
    }}%
}}%


\theoremstyle{definition}
\newtheorem{uloha}{Úloha}
\newtheorem*{bonus}{Bonus}

\pagestyle{empty}

\let\ee\expandafter

\def\vysld{}
\let\printvysl\relax
\let\printalphvysl\relax

\makeatletter
\def\vyslplain#1{\ee\ee\ee\gdef\ee\ee\ee\vysld\ee\ee\ee{\ee\vysld\ee\printvysl\ee{\the\c@uloha}{#1}}}

\makeatother

\begin{document}

%\tisk

\section*{Domácí úkol na Vennovy diagramy}


\begin{uloha}
Frisbee tým se zúčastnil mistrovství Evropy (ME) a mistrovství světa (MS), ale ne všichni členové se zúčastnili všeho. Těch, co byli na ME, bylo dvakrát více než těch, co nebyli nikde. Oproti tomu těch, co \textbf{ne}byli na MS, bylo dvakrát víc než těch, kteří se zúčastnili pouze MS. Dále bylo třikrát víc těch, co byli na MS, než těch, co byli jenom na ME. Konečně účastníků pouze MS bylo o tři více, než účastníků pouze ME.
\begin{enumerate}
    \item Kolik členů týmu se zúčastnilo aspoň jednoho mistrovství?
    \item Kolik se jich zúčastnilo obou mistrovství?
    \item Kolik se jich \textbf{ne}zúčastnilo mistrovství Evropy?
\end{enumerate}
\end{uloha}


\begin{uloha}
Průzkum o využití metra mezi pražskými učiteli ukázal, že
\begin{itemize}
    \item linku A užívá 65, B 135 a C 55 učitelů,
    \item metrem vůbec nejezdí 80 učitelů,
    \item nikdo nevyužívá všechny tři linky,
    \item linkou A i B jezdí 40 učitelů,
    \item linkou A i C jezdí 5 učitelů,
    \item linku B nebo C užívá 155 učitelů.
\end{itemize}
Určete, kolik učitelů
\begin{enumerate}
    \item jezdí pouze linkou B,
    \item jezdí linkou A nebo B,
    \item používá právě dvě linky,
    \item se zúčastnilo průzkumu.
\end{enumerate}
\end{uloha}


\end{document}