\documentclass[10pt,a5paper]{extarticle}
\usepackage[margin=1cm]{geometry}
\usepackage[utf8]{inputenc}
\usepackage[IL2]{fontenc}
\usepackage[czech]{babel}
\usepackage{microtype}
\usepackage{amssymb}
\usepackage{amsthm}
\usepackage{amsmath}

\usepackage[inline]{enumitem}

\newcommand{\R}{\mathbb{R}}
\newcommand{\Z}{\mathbb{Z}}
\newcommand{\N}{\mathbb{N}}
\newcommand{\Q}{\mathbb{Q}}

\setlist[enumerate]{label={(\alph*)},topsep=\smallskipamount,itemsep=\medskipamount,parsep=0pt}
\setlist[itemize]{topsep=\smallskipamount,noitemsep}

\def\tisk{%
\newbox\shipouthackbox
\pdfpagewidth=2\pdfpagewidth
\let\oldshipout=\shipout
\def\shipout{\afterassignment\zdvojtmp \setbox\shipouthackbox=}%
\def\zdvojtmp{\aftergroup\zdvoj}%
\def\zdvoj{%
    \oldshipout\vbox{\hbox{%
        \copy\shipouthackbox
        \hskip\dimexpr .5\pdfpagewidth-\wd\shipouthackbox\relax
        \box\shipouthackbox
    }}%
}}%


\theoremstyle{definition}
\newtheorem{uloha}{Úloha}
\newtheorem{suloha}[uloha]{\llap{$\star$ }Úloha}
\newtheorem{xuloha}[uloha]{\llap{\mdseries$\sun$ }Úloha}
\newtheorem*{bonus}{Bonus}

\pagestyle{empty}

\let\ee\expandafter

\def\vysld{}
\let\printvysl\relax
\let\printalphvysl\relax

\makeatletter
\def\vyslplain#1{\ee\ee\ee\gdef\ee\ee\ee\vysld\ee\ee\ee{\ee\vysld\ee\printvysl\ee{\the\c@uloha}{#1}}}
\let\vysl\vyslplain

\makeatother

\newenvironment{ulohav}[1][]{\begin{uloha}[#1]\gdef\locvysld{\begin{enumerate}}}{\ee\vyslplain\ee{\locvysld\end{enumerate}}\end{uloha}}
\newenvironment{sulohav}[1][]{\begin{suloha}[#1]\gdef\locvysld{\begin{enumerate*}}}{\ee\vyslplain\ee{\locvysld\end{enumerate*}}\end{suloha}}
\def\locvysl#1{\ee\gdef\ee\locvysld\ee{\locvysld\item #1}}
\let\lv\locvysl

\let\imp\Rightarrow
\let\equ\Leftrightarrow
\let\aand\wedge
\let\oor\vee


\begin{document}

% \tisk

\section*{Procvičování rovnic}

\let\AA\forall
\let\EE\exists

\begin{ulohav}
Rozhodněte, které z výroků platí (přeskočte ($\star$c) a vraťte se k němu, jen když vám zbude čas po ostatních úlohách):
\begin{enumerate}
    \item $\AA a, b, c \in \R\ \EE x \in \R \colon ax^2 + bx + c = 0$\lv{0}
    \item $\EE a, b, c \in \R\ \AA x \in \R \colon ax^2 + bx + c = 0$\lv{1}
    \item[($\star$c)] $\AA a, b \in \R \ \EE c, x \in \R \colon ax^2 + bx + c = 0$\lv{1}
\end{enumerate}
\end{ulohav}

\begin{ulohav}
Nalezněte všechna řešení následujících rovnic; kde je to potřeba, určete \textbf{podmínky}, za kterých jsou výrazy definovány:
\everymath{\displaystyle}
\def\lm#1{\lv{$\{#1\}$}}
\begin{enumerate}
    \item $\bigl(2x+\tfrac54\bigr)(2x+1)(x+2) = 0$\lm{-\frac58; -\frac12; -2}
    \item $\left(\frac1x - \frac2{x+1}\right)\left(\frac2x - \frac1{x+1}\right) = 0$\lv{$\{1; -2\}$, $x\neq0,-1$}
    \item $2x^2 - 16x + 32 = 0$\lm{4}
    \item $3x^2 + 3x + 1 = 0$\lv{$\emptyset$}
    \item $\frac{3x-2}{2x+1} = \frac32$\lv{$\emptyset$, $x \neq -\frac12$}
    \item $\frac2{x-3} + 1 = \frac{x-1}{x-3}$\lv{$\R\setminus\{3\}$, $x \neq 3$}
    \item $\frac x{x-2}-\frac{x+1}{2-x} = 0$\lv{$\{-\frac12\}$, $x \neq 2$}
    \item $x^2 = (x+1)(x+2)$\lm{-\frac23}
    \item $x^2 + 4x - 2 = 0$\lm{-2-\sqrt6; -2+\sqrt6}
    \item $x^2 = 50$\lm{5\sqrt2; -5\sqrt2}
    \item $(x^2 + 3x - 4)(x^2 + 3x + 4) = 0$\lm{-1; 4}
    \item $\frac{2}{x} = \frac{1}{x+1} + \frac{3}{1-x}$\lv{$\{\frac12(-1-\sqrt3); \frac12(-1+\sqrt3)\}$, $x\neq\pm1, 0$}
    \item $-\frac{1}{x} = \frac{1}{x+1} + \frac{3}{1-x}$\lv{$\{-2-\sqrt3; -2+\sqrt3\}$, $x\neq\pm1, 0$}
\end{enumerate}
\end{ulohav}

\begin{ulohav}
Vymyslete kvadratickou rovnici, jejíž kořeny budou
\begin{enumerate}
    \item $3$ a $4$\lv{$x^2 - 7x + 12 = 0$}
    \item $-3$ a $-4$\lv{$x^2 + 7x + 12 = 0$}
    \item jenom $-3$\lv{$x^2 + 6x + 9 = 0$}
    \item $-1 + \sqrt3$ a $-1 - \sqrt3$\lv{$x^2+2x-2 = 0$}
    \item $\sqrt 2$ a $\sqrt 3$\lv{$x^2 - (\sqrt2 + \sqrt3) x + \sqrt6 = 0$}
\end{enumerate}
\end{ulohav}


\newpage
\renewcommand{\baselinestretch}{1.5}
\setlist[enumerate]{itemjoin={\quad},label={(\alph*)}}
\parindent=0pt
\parskip=\smallskipamount
\def\printvysl#1#2{\textbf{#1.}\ #2\par}
\vysld


\end{document}