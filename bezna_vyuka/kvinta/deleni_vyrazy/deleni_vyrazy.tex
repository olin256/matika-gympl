\documentclass[11pt,a5paper]{article}
\usepackage[margin=1.2cm]{geometry}
\usepackage[utf8]{inputenc}
\usepackage[IL2]{fontenc}
\usepackage[czech]{babel}
\usepackage{microtype}
\usepackage{amssymb}
\usepackage{amsthm}
\usepackage{amsmath}
\usepackage{xcolor}
\usepackage{graphicx}
\usepackage{wasysym}
\usepackage{multicol}

\usepackage[inline]{enumitem}

\newcommand{\R}{\mathbb{R}}

\newcommand{\hint}[1]{{\color{gray}\footnotesize\noindent(Nápověda: #1)}}

\setlist[enumerate]{label={(\alph*)},topsep=\smallskipamount,itemsep=\smallskipamount,parsep=0pt}
\setlist[itemize]{topsep=\smallskipamount,noitemsep}

\def\tisk{%
\newbox\shipouthackbox
\pdfpagewidth=2\pdfpagewidth
\let\oldshipout=\shipout
\def\shipout{\afterassignment\zdvojtmp \setbox\shipouthackbox=}%
\def\zdvojtmp{\aftergroup\zdvoj}%
\def\zdvoj{%
    \oldshipout\vbox{\hbox{%
        \copy\shipouthackbox
        \hskip\dimexpr .5\pdfpagewidth-\wd\shipouthackbox\relax
        \box\shipouthackbox
    }}%
}}%



\newtheorem*{poz}{Pozorování}

\theoremstyle{definition}
\newtheorem{uloha}{\atr Úloha}
\newtheorem{suloha}[uloha]{\llap{$\star$ }Úloha}
\newtheorem*{bonus}{Bonus}
\newtheorem*{defn}{Definice}

\pagestyle{empty}

\let\ee\expandafter

\def\vysld{}
\let\printvysl\relax
\let\printalphvysl\relax

\makeatletter
\long\def\vyslplain#1{\ee\ee\ee\gdef\ee\ee\ee\vysld\ee\ee\ee{\ee\vysld\ee\printvysl\ee{\the\c@uloha}{#1}}}

\def\locvysl#1{\ee\gdef\ee\locvysld\ee{\locvysld\item #1}}
\let\lv\locvysl

\newenvironment{ulohav}[1][]{\begin{uloha}[#1]\gdef\locvysld{\begin{enumerate}}}{\ee\vyslplain\ee{\locvysld\end{enumerate}}\end{uloha}}

\def\stitem{\@noitemargtrue\@item[$\star$ \@itemlabel]}

\makeatother

\def\atr{}
\def\basic{\def\atr{\llap{\mdseries$\sun$ }\gdef\atr{}}}
\def\interest{\def\atr{\llap{$\star$ }\gdef\atr{}}}
\let\mb\mathbf

\begin{document}

%\tisk

\section{Dělení polynomů}

\begin{ulohav}
Vydělte následující polynomy se zbytkem:
\begin{enumerate}
    \item $\bigl(14 t^5 + 4 t^4 - t^3 + 2 t^2 + 3 t + 5\bigr) : \bigl(2 t^2 - 1\bigr)$\lv{$7 t^3+2 t^2+3 t+2$ zb. $6t+7$}
    \item $\bigl(5s^4 + s^2 + 2s + 21\bigr) : s^2$\lv{$5s^2+1$ zb. $2s+21$}
    \item $\bigl(x^3+\sqrt{3} x-3 \sqrt{3}\bigr) : \bigl(x+\sqrt3\bigr)$\lv{$x^2-\sqrt{3} x+\sqrt{3}+3$ zb. $-6 \sqrt{3}-3$}
    \stitem $\bigl(x^{2022} + x^{2021} + \dots + x^2 + x^1 + 1\bigr) : (x-1)$\lv{$x^{2021} + 2x^{2020} + 3x^{2019} + \dots + 2020x^2 + 2021x + 2022$ zb. $2023$}
\end{enumerate}
\end{ulohav}

\interest
\begin{uloha}
Rozmyslete si, že pokud nějaký polynom $p$ (s jedinou neznámou~$x$) dělíme dvojčlenem $x - a$ (kde $a$ je nějaké číslo), tak zbytek po tomto dělení bude stejný, jako když za $x$ dosadíme $a$.
\end{uloha}

\section{Úpravy polynomů}

\begin{ulohav}
Rozložte následující polynomy \uv{maximálně} na součin:
\begin{enumerate}
    \item $16 x^2-24 x y+9 y^2$\lv{$(4x-3y)^2$}
    \item $2 r s t + t^2 + r^2 s^2$\lv{$(r s + t)^2$}
    \item $4 x^4 y^6 - x^2 y^8$\lv{$x^2  y^6 (2 x - y) (2 x + y)$}
    \item $4 x^2 y^4 - x^4 y^8$\lv{$x^2y^4 (2-xy^2)(2+xy^2)$}
    \item $(x+y)^2 - (x+2y)^2$\lv{$-y(2x+3y)$}
    \item $a^2 - 3$\lv{$(a+\sqrt3)(a-\sqrt3)$}
    \item $x^3 - y^6$\lv{$\left(x-y^2\right) \left(x^2+x y^2+y^4\right)$}
    \item $27a^3 + \frac{b^3}{27}$\lv{$\bigl(3a+\frac b3\bigr)\bigl(9a^2 + ab + \frac{b^2}{9}\bigr)$}
    \item $x^4 - y^4$\lv{$(x-y)(x+y)(x^2+y^2)$}
\end{enumerate}
\end{ulohav}

\begin{uloha}
Vymyslete \uv{vzorec} pro $(a+b+c)^2$.\vyslplain{$(a+b+c)^2 = a^2+b^2+c^2+2 b c + 2 a b+2 a c$}
\end{uloha}

\interest
\begin{uloha}
Rozmyslete si, že pro libovolné přirozené číslo $n > 1$ platí
\[ a^n - b^n = (a-b)\bigl(a^{n-1} + a^{n-2}b + a^{n-3}b^2 +  \dots + ab^{n-2} + b^{n-1}\bigr), \]
tedy např. $a^4 - b^4 = (a-b)\bigl(a^3 + a^2b + ab^2 + b^3\bigr)$.
\end{uloha}


\newpage
\parindent=0pt
\parskip=\smallskipamount
\def\printvysl#1#2{\textbf{#1.}\ #2\par}
\vysld


\end{document}