\documentclass[9pt,a5paper]{extarticle}
\usepackage[margin=1cm]{geometry}
\usepackage[utf8]{inputenc}
\usepackage[IL2]{fontenc}
\usepackage[czech]{babel}
\usepackage{microtype}
\usepackage{amssymb}
\usepackage{amsthm}
\usepackage{amsmath}
\usepackage{xcolor}
\usepackage{graphicx}
\usepackage{wasysym}

\usepackage[inline]{enumitem}

\newcommand{\R}{\mathbb{R}}

\setlist[enumerate]{label={(\alph*)},topsep=\smallskipamount,noitemsep}
\setlist[itemize]{topsep=\smallskipamount,noitemsep}

\def\tisk{%
\newbox\shipouthackbox
\pdfpagewidth=2\pdfpagewidth
\let\oldshipout=\shipout
\def\shipout{\afterassignment\zdvojtmp \setbox\shipouthackbox=}%
\def\zdvojtmp{\aftergroup\zdvoj}%
\def\zdvoj{%
    \oldshipout\vbox{\hbox{%
        \copy\shipouthackbox
        \hskip\dimexpr .5\pdfpagewidth-\wd\shipouthackbox\relax
        \box\shipouthackbox
    }}%
}}%


\theoremstyle{definition}
\newtheorem{uloha}{Úloha}
\newtheorem{suloha}[uloha]{\llap{$\star$ }Úloha}
\newtheorem*{bonus}{Bonus}

\pagestyle{empty}

\let\ee\expandafter

\def\vysld{}
\let\printvysl\relax
\let\printalphvysl\relax

\makeatletter
\def\vyslplain#1{\ee\ee\ee\gdef\ee\ee\ee\vysld\ee\ee\ee{\ee\vysld\ee\printvysl\ee{\the\c@uloha}{#1}}}
\let\vysl\vyslplain

\makeatother


\newenvironment{ulohav}[1][]{\begin{uloha}[#1]\gdef\locvysld{\begin{enumerate*}}}{\ee\vyslplain\ee{\locvysld\end{enumerate*}}\end{uloha}}
\newenvironment{sulohav}[1][]{\begin{suloha}[#1]\gdef\locvysld{\begin{enumerate*}}}{\ee\vyslplain\ee{\locvysld\end{enumerate*}}\end{suloha}}
\def\locvysl#1{\ee\gdef\ee\locvysld\ee{\locvysld\item #1}}
\let\lv\locvysl

\let\imp\Rightarrow
\let\equ\Leftrightarrow
\let\aand\wedge
\let\oor\vee


\begin{document}

% \tisk


\section*{Cvičení na výroky}

\begin{uloha}\label{basic}
Sestavte tabulku pravdivostních hodnot pro složené výroky
\begin{enumerate}
    \item $(x \aand y) \imp (y \oor z)$,
    \item $(x \equ \neg y) \oor (x \equ y)$,
    \item $(x \aand y \aand \neg z) \oor (\neg x \aand y \aand \neg z)$. \label{cnd}
\end{enumerate}
\vysl{%
\[\begin{array}{c|c|c|c|c|c}
x & y & z & (x \aand y) \imp (y \oor z) & (x \equ \neg y) \oor (x \equ y) & (x \aand y \aand \neg z) \oor (\neg x \aand y \aand \neg z) \\\hline
0 & 0 & 0 & 1 & 1 & 0 \\
0 & 0 & 1 & 1 & 1 & 0 \\
0 & 1 & 0 & 1 & 1 & 1 \\
0 & 1 & 1 & 1 & 1 & 0 \\
1 & 0 & 0 & 1 & 1 & 0 \\
1 & 0 & 1 & 1 & 1 & 0 \\
1 & 1 & 0 & 1 & 1 & 1 \\
1 & 1 & 1 & 1 & 1 & 0 
%
\end{array}\]
}
\end{uloha}

\begin{uloha}
Výroky z Úlohy \ref{basic} znegujte (tak, aby ve výsledku se negovaly pouze $x$, $y$ či $z$, nikoliv složené výroky).
\vysl{\begin{enumerate*}\item $(x \aand y) \aand (\neg y \aand \neg z)$, \item $(x \equ y) \aand (\neg x \equ y)$, \item $(\neg x \oor \neg y \oor z) \aand (x \oor \neg y \oor z)$ \end{enumerate*}}
\end{uloha}

\begin{ulohav}\label{kecy}
Znegujte následující výroky:
\begin{enumerate}
    \item Přišel jsem, viděl jsem, zvítězil jsem.\lv{Nepřišel jsem, nebo jsem neviděl, nebo jsem nezvítězil.}
    \item Nebude-li pršet, nezmoknem.\label{prset}\lv{Nebude pršet a zmoknem.}
    \item Bude-li každý z nás z křemene, bude celý národ z kvádru.\label{kvadr}\lv{Bude každý z nás z křemene a celý národ z kvádru nebude.}
    \item Půjdu do školy právě tehdy, když nebudu nemocný a nebude sněžit.\lv{Nepůjdu do školy právě tehdy, když nebudu nemocný a nebude sněžit.}
    \item Když půjdu do školy, tak nebudu nemocný a nebude sněžit.\lv{Půjdu do školy a budu nemocný nebo bude sněžit.}
\end{enumerate}
\end{ulohav}

\begin{uloha}
Výroky \ref{prset} a \ref{kvadr} z Úlohy \ref{kecy} přepište pomocí \emph{obměněné implikace} (tj. s využitím \uv{pravidla} $(a \imp b) \equ (\neg b \imp \neg a)$).
\vysl{\begin{enumerate*}\item[\ref{prset}] Zmokneme-li, bude pršet. \item[\ref{kvadr}] Nebude-li celý národ z kvádru, nebude každý z nás z~křemene. \end{enumerate*}}
\end{uloha}


\begin{uloha} % https://en.wikipedia.org/wiki/Disjunctive_normal_form
Vymyslete nějaký složený výrok $S$ obsahující $x$, $y$ a $z$, jehož pravdivostní tabulka bude
\[\begin{array}{c|c|c|c|c|c}
x & y & z & S \\ \hline
0 & 0 & 0 &  1 \\
0 & 0 & 1 &  1 \\
0 & 1 & 0 &  1 \\
0 & 1 & 1 &  0 \\
1 & 0 & 0 &  0 \\
1 & 0 & 1 &  0 \\
1 & 1 & 0 &  1 \\
1 & 1 & 1 &  1 
%
\end{array}\]
Dokážete vymyslet recept pro \emph{jakýkoliv} poslední sloupec?
\par\smallskip
\noindent\emph{Nápověda:} Inspirujte se Úlohou \ref{basic} \ref{cnd}.
\vysl{např. $(\neg x \aand \neg y \aand \neg z) \oor (\neg x \aand \neg y \aand z) \oor (\neg x \aand y \aand \neg z) \oor (x \aand y \aand \neg z) \oor (x \aand y \aand z)$}
\end{uloha}


\begin{suloha} % https://en.wikipedia.org/wiki/Sheffer_stroke
Ukažte, že pomocí operace $x \uparrow y$, která je definována jako $\neg(x \aand y)$, lze dostat všechny ostatní logické operace.
\vysl{$\neg x = x \uparrow x$, $x \aand y = (x \uparrow y) \uparrow (x \uparrow y)$, $x \oor y = (x \uparrow x) \uparrow (y \uparrow y)$, $x \imp y = x \uparrow (y \uparrow y)$, $x \equ y = (x \uparrow y) \uparrow ((x \uparrow x) \uparrow (y \uparrow y))$}
\end{suloha}


\newpage
\parindent=0pt
\parskip=\smallskipamount
\def\printvysl#1#2{\textbf{#1.}\ #2\par}
\vysld


\end{document}

