\documentclass[12pt,a5paper]{extarticle}
\usepackage[margin=1cm]{geometry}
\usepackage[utf8]{inputenc}
\usepackage[IL2]{fontenc}
\usepackage[czech]{babel}
\usepackage{microtype}
\usepackage{amssymb}
\usepackage{amsthm}
\usepackage{amsmath}
\usepackage{xcolor}
\usepackage{wasysym}
\usepackage{multicol}

\usepackage[inline]{enumitem}

\newcommand{\R}{\mathbb{R}}

\newcommand{\hint}[1]{{\color{gray}\footnotesize\noindent(Nápověda: #1)}}

\setlist[enumerate]{label={(\alph*)},topsep=\smallskipamount,itemsep=\smallskipamount,parsep=0pt}
\setlist[itemize]{topsep=\smallskipamount,noitemsep}

\def\tisk{%
\newbox\shipouthackbox
\pdfpagewidth=2\pdfpagewidth
\let\oldshipout=\shipout
\def\shipout{\afterassignment\zdvojtmp \setbox\shipouthackbox=}%
\def\zdvojtmp{\aftergroup\zdvoj}%
\def\zdvoj{%
    \oldshipout\vbox{\hbox{%
        \copy\shipouthackbox
        \hskip\dimexpr .5\pdfpagewidth-\wd\shipouthackbox\relax
        \box\shipouthackbox
    }}%
}}%



\newtheorem*{poz}{Pozorování}

\theoremstyle{definition}
\newtheorem{uloha}{\atr Úloha}
\newtheorem{suloha}[uloha]{\llap{$\star$ }Úloha}
\newtheorem*{bonus}{Bonus}
\newtheorem*{defn}{Definice}

\pagestyle{empty}

\let\ee\expandafter

\def\vysld{}
\let\printvysl\relax
\let\printalphvysl\relax

\makeatletter
\long\def\vyslplain#1{\ee\ee\ee\gdef\ee\ee\ee\vysld\ee\ee\ee{\ee\vysld\ee\printvysl\ee{\the\c@uloha}{#1}}}
\let\vysl\vyslplain

\def\locvysl#1{\ee\gdef\ee\locvysld\ee{\locvysld\item #1}}
\let\lv\locvysl

\newenvironment{ulohav}[1][]{\begin{uloha}[#1]\gdef\locvysld{\begin{enumerate}}}{\ee\vyslplain\ee{\locvysld\end{enumerate}}\end{uloha}}
\def\stitem{\@noitemargtrue\@item[$\star$ \@itemlabel]}

\makeatother

\def\atr{}
\def\basic{\def\atr{\llap{\mdseries$\sun$ }\gdef\atr{}}}
\def\interest{\def\atr{\llap{$\star$ }\gdef\atr{}}}

\begin{document}

% \tisk

\let\ra\rangle
\let\la\langle

\section*{Lineární nerovnice}

\begin{ulohav}
Nalezněte všechna řešení (tj. množinu $K$ všech řešení) následujících nerovnic:
\begin{multicols}{2}
\begin{enumerate}
    \item $-x > 1$\lv{$(-\infty; -1)$}
    \item $0\cdot x \geq 0$\lv{$\R$}
    \item $0\cdot x < 0$\lv{$\emptyset$}
    \item $0\cdot x \leq -3$\lv{$\emptyset$}
    \item $3\cdot x < 0$\lv{$(-\infty;0)$}
    \item $-5\cdot x \leq 0$\lv{$\la0;\infty)$}
    \item $2x-\sqrt2 < x\sqrt2 - 2$\lv{$(-\infty;-1)$}
    \item $x+3 \leq 2x-7$\lv{$\la10;\infty)$}
    \item $y+1 \geq y$\lv{$\R$}
    \item $2y+5 < 2y+3$\lv{$\emptyset$}
    \item $2z+1 > 1 - z$\lv{$(0;\infty)$}
    \item $2\pi z > 1$\lv{$(\frac{1}{2\pi}; \infty)$}
    \item $2 - \dfrac{x+2}3 > x - \dfrac{x+3}{3}$\lv{$(-\infty;\frac73)$}
    \item $(4-t)t > 24 - t^2$\lv{$(6; \infty)$}
    \item $x\sqrt2 + 1 > \sqrt 3 + x$\lv{$(\sqrt6 + \sqrt3 - \sqrt2 - 1 ; \infty)$}
\end{enumerate}
\end{multicols}
\end{ulohav}


\begin{uloha}
Prospěchové stipendium může být každému studentu zvýšeno o 300\,Kč, nebo o 5\,\%. Pro Karla by byla výhodnější druhá varianta. Co z toho plyne pro výši jeho stipendia?\vysl{Je vyšší než 6000\,Kč.}
\end{uloha}


\begin{ulohav}
Řešte soustavy nerovnic:
\begin{enumerate}
    \item $2x-7 \leq 0$, $3x+1 > 0$\lv{$(-\frac13; \frac72)$}
    \item $2x-7 \geq 0$, $3x+1 < 0$\lv{$\emptyset$}
    \item $2x-7 \leq 0$, $3x+1 < 0$\lv{$(-\infty;-\frac13)$}
\end{enumerate}
\end{ulohav}

\newpage
\parindent=0pt
\parskip=\smallskipamount
\def\printvysl#1#2{\textbf{#1.} #2\par}
\vysld

\end{document}