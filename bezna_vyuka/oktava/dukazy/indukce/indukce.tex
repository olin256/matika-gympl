\documentclass[8pt,a5paper]{extarticle}
\usepackage[margin=.8cm]{geometry}
\usepackage[utf8]{inputenc}
\usepackage[IL2]{fontenc}
\usepackage[czech]{babel}
\usepackage{microtype}
\usepackage{amssymb}
\usepackage{amsthm}
\usepackage{amsmath}
\usepackage{xcolor}
\usepackage{graphicx}
\usepackage{wasysym}
\usepackage{tikz}

\usepackage[inline]{enumitem}

\newcommand{\R}{\mathbb{R}}
\newcommand{\Z}{\mathbb{Z}}
\newcommand{\N}{\mathbb{N}}

\newcommand{\hint}[1]{{\color{gray}\footnotesize\noindent(Nápověda: #1)}}

\setlist[enumerate]{label={(\alph*)},topsep=\smallskipamount,itemsep=\smallskipamount,parsep=0pt,itemjoin={\quad}}
\setlist[itemize]{topsep=\smallskipamount,noitemsep}

\def\tisk{%
\newbox\shipouthackbox
\pdfpagewidth=2\pdfpagewidth
\let\oldshipout=\shipout
\def\shipout{\afterassignment\zdvojtmp \setbox\shipouthackbox=}%
\def\zdvojtmp{\aftergroup\zdvoj}%
\def\zdvoj{%
     \oldshipout\vbox{\hbox{%
        \copy\shipouthackbox
        \hskip\dimexpr .5\pdfpagewidth-\wd\shipouthackbox\relax
        \box\shipouthackbox
    }}%
}}%

\let\results\newpage
\let\endresults\relax

\def\resultssame{%
    \long\def\results##1\endresults{%
        \vfill
        \noindent\rotatebox{180}{\vbox{##1}}%
    }%
}


\newtheorem*{poz}{Pozorování}

\theoremstyle{definition}
\newtheorem{uloha}{\atr Úloha}
\newtheorem{suloha}[uloha]{\llap{$\star$ }Úloha}
\newtheorem*{bonus}{Bonus}
\newtheorem*{defn}{Definice}

\pagestyle{empty}

\let\ee\expandafter

\def\vysld{}
\let\printvysl\relax
\let\printalphvysl\relax

\makeatletter
\long\def\vyslplain#1{\ee\ee\ee\gdef\ee\ee\ee\vysld\ee\ee\ee{\ee\vysld\ee\printvysl\ee{\the\c@uloha}{#1}}}
\let\vysl\vyslplain

\def\locvysl#1{\ee\gdef\ee\locvysld\ee{\locvysld\item #1}}
\let\lv\locvysl

\newenvironment{ulohav}[1][]{\begin{uloha}[#1]\gdef\locvysld{\begin{enumerate*}}}{\ee\vyslplain\ee{\locvysld\end{enumerate*}}\end{uloha}}
\def\stitem{\@noitemargtrue\@item[$\star$ \@itemlabel]}
\def\ststitem{\@noitemargtrue\@item[$\star\star$ \@itemlabel]}

\makeatother

\def\atr{}
\def\basic{\def\atr{\llap{\mdseries$\sun$ }\gdef\atr{}}}
\def\interest{\def\atr{\llap{$\star$ }\gdef\atr{}}}
\def\iinterest{\def\atr{\llap{$\star\star$ }\gdef\atr{}}}
\let\mb\mathbf



\begin{document}

% \tisk
%\resultssame

\section*{26. Matematická indukce}

\begin{uloha}
Dokažte matematickou indukcí následující vztahy platné pro všechna $n \in \N$:
\begin{enumerate}
    \item $1 + 2 + \dots + n = \frac12n(n+1)$
    \item $1 + 3 + 5 + \dots + (2n-1) = n^2$
    \item $1^2 + 2^2 + \dots + n^2 = \frac16n(n+1)(2n+1)$
    \item $1^3 + 2^3 + \dots + n^3 = (1 + 2 + \dots + n)^2$
    \item $1 \cdot 2^0 + 2 \cdot 2^1 + 3 \cdot 2^2 + \dots + n \cdot 2^{n-1} = n \cdot 2^n - 2^n + 1$
\end{enumerate}
\end{uloha}

\begin{ulohav}
U následujících součtů stanovte hypotézu, jakému (jednoduššímu) výrazu by se mohly rovnat, a dokažte ji matematickou indukcí:
\begin{enumerate}
    \item $2^0 + 2^1 + \dots + 2^n$,\lv{$2^{n+1}-1$}
    \item $\displaystyle\frac{1}{1\cdot2} + \frac{1}{2\cdot3} + \frac{1}{3\cdot4} + \dots + \frac{1}{n(n+1)}$.\lv{$\frac{n}{n+1}$}
\end{enumerate}
\end{ulohav}

\begin{uloha}
Dokažte matematickou indukcí, že pro každé přirozené číslo $n$ jsou následující výrazy dělitelné šesti:
\begin{enumerate}
    \item $2n^3 + 3n^2 + n$,
    \item $4 n^3-3 n^2-n$.
\end{enumerate}
\end{uloha}

\begin{uloha}
Pro každé přirozené číslo $n$ dokažte
\[ \frac{1}{n+1} + \frac{1}{n+2} + \dots + \frac{1}{2n} \geq \frac12. \]
\end{uloha}

\begin{uloha}
Dokažte, že pro každé přirozené $n$ je možné pokrýt tabulku $2^n \times 2^n$, ze které odstraníme jedno rohové pole, triminy tvaru L. Obrázek znázorňuje jedno takové pokrytí pro $n=2$.
\[ \begin{tikzpicture}[scale=.3]
\def\zaklad{%
\draw (0,0) -- (0,2) -- (1,2) -- (1,1) -- (2,1) -- (2,0) -- cycle;%
}
\zaklad
\begin{scope}[cm={-1,0,0,1,(2,2)}] \zaklad \end{scope}
\begin{scope}[cm={-1,0,0,1,(4,0)}] \zaklad \end{scope}
\begin{scope}[cm={-1,0,0,-1,(4,4)}] \zaklad \end{scope}
\end{tikzpicture} \]
\hint{Útvar o straně $2^{k+1}$ se dá \uv{skoro} poskládat z těch o straně $2^k$, pak už stačí jen něco málo přidat.}
\end{uloha}


\begin{uloha}
Mějme v rovině $n \in \N$ různoběžných přímek takových, že žádné tři se neprotínají v~jednom bodě. Dokažte, že tyto přímky dělí rovinu na $1 + \frac12 n(n+1)$ oblastí. Obrázek znázorňuje situaci pro $n=3$ s rozdělením na $7 = 1 + \frac 12 \cdot 3 \cdot 4$ oblastí.
\[ \begin{tikzpicture}[scale=.5]
\draw (0,0) -- (7,1);
\draw (0,2) -- (7,0);
\draw (2,-.5) -- (2,2.5);
\node at (2.5,.8) {1};
\node at (1,1) {2};
\node at (1,2.2) {3};
\node at (1,-.2) {4};
\node at (5,0) {5};
\node at (5,1.5) {6};
\node at (6.5,.6) {7};
\end{tikzpicture} \]
\end{uloha}


\interest
\begin{uloha}[Bernoulliho nerovnost]
Dokažte, že pro všechna reálná $x > -1$ a všechna přirozená $n$ platí
$(1+x)^n  \geq 1+nx$.\vysl{indukční krok: $(1+x)^{k+1} = (1+x)^k(1+x) \stackrel{\text{IP}}\geq (1+kx)(1+x) = 1+kx+x+kx^2 = 1+(k+1)x + kx^2 \geq 1+(k+1)x$}
\end{uloha}


\interest
\begin{uloha}
Pro všechna $n \in \N$ a $x \in \R$, $x \neq 2k\pi$ ($k \in \Z$) dokažte
\[ \sin x + \sin 2x + \dots + \sin nx = \frac{\sin \bigl(\frac{n+1}{2} x\bigr) \sin \bigl(\frac n2 x\bigr)}{\sin \frac x2}. \]
\end{uloha}

\interest
\begin{uloha}
Dokažte, že pro každé $n \in\N$ existuje $n$-ciferné přirozené číslo, které je dělitelné $2^n$ a jeho cifry jsou pouze $1$ nebo $2$.
\end{uloha}

\iinterest
\begin{uloha}
Dokažte pro všechna $n \in \N$, že mezi libovolnými $3^{n+1}$ přirozenými čísly lze najít $3^n$ čísel, jejichž součet je dělitelný $3^n$.
\hint{V indukčním kroku nahlédněte, že mezi $3^{k+2}$ čísly lze nalézt pět disjunktních skupin, jejichž součet je dělitelný $3^k$, a rozdělte je podle zbytku po dělení $3^{k+1}$.}
\end{uloha}



\end{document}