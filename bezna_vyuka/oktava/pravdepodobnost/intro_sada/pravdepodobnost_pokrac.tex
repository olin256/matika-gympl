\documentclass[9pt,a5paper]{extarticle}
\usepackage[margin=.6cm]{geometry}
\usepackage[utf8]{inputenc}
\usepackage[IL2]{fontenc}
\usepackage[czech]{babel}
\usepackage{microtype}
\usepackage{amssymb}
\usepackage{amsthm}
\usepackage{amsmath}
\usepackage{xcolor}
\usepackage{graphicx}
\usepackage{wasysym}
\usepackage{multicol}

\usepackage[inline]{enumitem}

\newcommand{\R}{\mathbb{R}}

\newcommand{\hint}[1]{{\color{gray}\footnotesize\noindent(Nápověda: #1)}}

\setlist[enumerate]{label={(\alph*)},topsep=\smallskipamount,itemsep=\smallskipamount,parsep=0pt,itemjoin={\quad}}
\setlist[itemize]{topsep=\smallskipamount,noitemsep}

\def\tisk{%
\newbox\shipouthackbox
\pdfpagewidth=2\pdfpagewidth
\let\oldshipout=\shipout
\def\shipout{\afterassignment\zdvojtmp \setbox\shipouthackbox=}%
\def\zdvojtmp{\aftergroup\zdvoj}%
\def\zdvoj{%
    \oldshipout\vbox{\hbox{%
        \copy\shipouthackbox
        \hskip\dimexpr .5\pdfpagewidth-\wd\shipouthackbox\relax
        \box\shipouthackbox
    }}%
}}%

\let\results\newpage
\let\endresults\relax

\def\resultssame{%
    \long\def\results##1\endresults{%
        \vfill
        \noindent\rotatebox{180}{\vbox{##1}}%
    }%
}


\newtheorem*{poz}{Pozorování}

\theoremstyle{definition}
\newtheorem{uloha}{\atr Úloha}
\newtheorem{suloha}[uloha]{\llap{$\star$ }Úloha}
\newtheorem*{bonus}{Bonus}
\newtheorem*{defn}{Definice}

\pagestyle{empty}

\let\ee\expandafter

\def\vysld{}
\let\printvysl\relax

\makeatletter
\long\def\vyslplain#1{\ee\ee\ee\gdef\ee\ee\ee\vysld\ee\ee\ee{\ee\vysld\ee\printvysl\ee{\the\c@uloha}{#1}}}
\let\vysl\vyslplain

\def\locvysl#1{\ee\gdef\ee\locvysld\ee{\locvysld\item #1}}
\let\lv\locvysl

\newenvironment{ulohav}[1][]{\begin{uloha}[#1]\gdef\locvysld{\begin{enumerate*}}}{\ee\vyslplain\ee{\locvysld\end{enumerate*}}\end{uloha}}
\def\stitem{\@noitemargtrue\@item[$\star$ \@itemlabel]}

\makeatother

\def\atr{}
\def\basic{\def\atr{\llap{\mdseries$\sun$ }\gdef\atr{}}}
\def\interest{\def\atr{\llap{$\star$ }\gdef\atr{}}}
\def\iinterest{\def\atr{\llap{$\star\star$ }\gdef\atr{}}}



\begin{document}

%\tisk
%\resultssame

\section*{32$\frac{\mathbf 1}{\mathbf 2}$. Pravděpodobnost aneb Další jinak formulovaná kombinatorika}

\emph{Ve všech úlohách předpokládáme, že všechny volby, uspořádání atd. jsou stejně pravděpodobné.}


\begin{ulohav}\label{koule}
V pytli, do kterého nevidíme, je pět červených a osm zelených koulí. Karel postupně z pytle vytáhl tři koule, přičemž každou po vytažení \emph{vrátil}. Jaká je pravděpodobnost, že
\begin{enumerate}
    \item první vytažená koule je červená?\lv{$\frac{5}{13}$}
    \item druhá vytažená koule je červená?\lv{$\frac{5}{13}$}
    \item jsou všechny vytažené koule červené?\lv{$\left(\frac{5}{13}\right)^3$}
    \item je právě jedna z vytažených koulí červená?\lv{$3\cdot\frac{5}{13}\cdot\frac{8}{13}\cdot\frac{8}{13}$}
    \item pouze ta první z vytažených koulí je červená?\lv{$\frac{5}{13}\cdot\frac{8}{13}\cdot\frac{8}{13}$}
    \item pouze ta první z vytažených koulí je zelená?\lv{$\frac{8}{13}\cdot\frac{5}{13}\cdot\frac{5}{13}$}
\end{enumerate}
\end{ulohav}



\begin{uloha}
Co je pravděpodobnější: hodit při 4 hodech kostkou aspoň jednou 6, nebo při 24 hodech dvěma kostkami hodit aspoň jednou dvě 6?
\vysl{První jev má pravděpodobnost $1-\frac{5^4}{6^4} \doteq 0{,}52$, druhý $1-\frac{35^{24}}{36^{24}} \doteq 0{,}49$, takže pravděpodobnější je první.}
\end{uloha}


\begin{uloha}
Kolikrát nejméně musíme hodit kostkou, aby pravděpodobnost, že aspoň jednou hodíme šestku, byla alespoň $99\,\%$?
\vysl{26-krát}
\end{uloha}


\begin{ulohav}
Ve třídě OC je 17 dívek a 11 chlapců. Určete pravděpodobnost, že
\begin{enumerate}
    \item při náhodné volbě dvojčlenné služby budou vybrány dvě dívky.\lv{$\frac{\binom{17}2}{\binom{28}2}$}
    \item při náhodné volbě dvojčlenné služby budou zastoupena obě pohlaví.\lv{$\frac{17\cdot 11}{\binom{28}2}$}
    \item při náhodném vylosování čtyř lidí na zkoušení budou vylosováni samí hoši.\lv{$\frac{\binom{11}4}{\binom{28}4}$}
    \item při náhodném vylosování čtyř lidí na zkoušení bude vylosován Max a tři další hoši.\lv{$\frac{\binom{10}3}{\binom{28}4}$}
    \item při náhodném vylosování čtyř lidí na zkoušení budou vylosovány alespoň dvě Kačky.\lv{$\frac{\binom42 \cdot \binom{24}2 + \binom43 \cdot \binom{24}1 + \binom44 \cdot \binom{24}0}{\binom{28}4}$}
\end{enumerate}
\end{ulohav}



\baselineskip=1.25\baselineskip
\setlist[enumerate]{label=\textbf{(\alph*)},itemjoin={\quad}}

\results
\parindent=0pt
\parskip=\smallskipamount
\rightskip=0pt plus1fil\relax
\def\printvysl#1#2{\textbf{#1.} #2\par}
\vysld
\endresults



\end{document}