\documentclass[9pt,a5paper]{extarticle}
\usepackage[margin=.6cm]{geometry}
\usepackage[utf8]{inputenc}
\usepackage[IL2]{fontenc}
\usepackage[czech]{babel}
\usepackage{microtype}
\usepackage{amssymb}
\usepackage{amsthm}
\usepackage{amsmath}
\usepackage{xcolor}
\usepackage{graphicx}
\usepackage{wasysym}
\usepackage{multicol}

\usepackage[inline]{enumitem}

\newcommand{\R}{\mathbb{R}}

\newcommand{\hint}[1]{{\color{gray}\footnotesize\noindent(Nápověda: #1)}}

\DeclareMathOperator{\tg}{tg}
\DeclareMathOperator{\cotg}{cotg}

\setlist[enumerate]{label={(\alph*)},topsep=\smallskipamount,itemsep=\smallskipamount,parsep=0pt,itemjoin={\quad}}
\setlist[itemize]{topsep=\smallskipamount,noitemsep}

\def\tisk{%
\newbox\shipouthackbox
\pdfpagewidth=2\pdfpagewidth
\let\oldshipout=\shipout
\def\shipout{\afterassignment\zdvojtmp \setbox\shipouthackbox=}%
\def\zdvojtmp{\aftergroup\zdvoj}%
\def\zdvoj{%
    \oldshipout\vbox{\hbox{%
        \copy\shipouthackbox
        \hskip\dimexpr .5\pdfpagewidth-\wd\shipouthackbox\relax
        \box\shipouthackbox
    }}%
}}%

\let\results\newpage
\let\endresults\relax

\def\resultssame{%
    \long\def\results##1\endresults{%
        \vfill
        \noindent\rotatebox{180}{\vbox{##1}}%
    }%
}


\newtheorem*{poz}{Pozorování}

\theoremstyle{definition}
\newtheorem{uloha}{\atr Úloha}
\newtheorem{suloha}[uloha]{\llap{$\star$ }Úloha}
\newtheorem*{bonus}{Bonus}
\newtheorem*{defn}{Definice}

\pagestyle{empty}

\let\ee\expandafter

\def\vysld{}
\let\printvysl\relax

\makeatletter
\long\def\vyslplain#1{\ee\ee\ee\gdef\ee\ee\ee\vysld\ee\ee\ee{\ee\vysld\ee\printvysl\ee{\the\c@uloha}{#1}}}
\let\vysl\vyslplain

\def\locvysl#1{\ee\gdef\ee\locvysld\ee{\locvysld\item #1}}
\let\lv\locvysl

\newenvironment{ulohav}[1][]{\begin{uloha}[#1]\gdef\locvysld{\begin{enumerate*}}}{\ee\vyslplain\ee{\locvysld\end{enumerate*}}\end{uloha}}
\def\stitem{\@noitemargtrue\@item[$\star$ \@itemlabel]}

\makeatother

\def\atr{}
\def\basic{\def\atr{\llap{\mdseries$\sun$ }\gdef\atr{}}}
\def\interest{\def\atr{\llap{$\star$ }\gdef\atr{}}}
\def\iinterest{\def\atr{\llap{$\star\star$ }\gdef\atr{}}}


\begin{document}

% \tisk
% \resultssame

\section*{32. Pravděpodobnost aneb Jinak formulovaná kombinatorika}

\emph{Ve všech úlohách předpokládáme, že všechny volby, uspořádání atd. jsou stejně pravděpodobné.}%, pokud není řečeno jinak.}

\begin{ulohav}
Náhodně zamícháme karty s čísly $1, \dots, 8$. Jaká je pravděpodobnost, že ve výsledném pořadí 
\begin{enumerate}
    \item bude jednička na prvním místě?\lv{$\frac18$}
    \item bude jednička na jiném než prvním místě?\lv{$\frac78$}
    \item bude jednička na prvním místě a osmička na posledním?\lv{$\frac{1}{7\cdot8}$}
    \item bude jednička bezprostředně před dvojkou?\lv{$\frac18$}
    \item bude jednička bezprostředně následována sudým číslem?\lv{$\frac12$}
    \item bude jednička bezprostředně následována lichým číslem?\lv{$\frac38$}
    \item bude jednička před dvojkou (ne nutně bezprostředně)?\lv{$\frac12$}
    \stitem budou sudá i lichá čísla tvořit nepřerušované úseky?\lv{$\frac{1}{35}$}
    \stitem budou sudá čísla tvořit nepřerušený úsek?\lv{$\frac{1}{14}$}
\end{enumerate}
\end{ulohav}


\begin{ulohav}
V loterii se losuje 10 z celkem 50 čísel (každé může být vylosováno nejvýše jednou). Soutěžící si na losu zvolí dvanáct různých čísel. Jaká je pravděpodobnost, že
\begin{enumerate}
    \item žádné z čísel na losu nebude vylosováno?\label{los-nic}\lv{$\frac{\binom{38}{10}}{\binom{50}{10}} \doteq 0{,}046$}
    \item všechna vylosovaná čísla budou na losu?\lv{$\frac{\binom{12}{10}}{\binom{50}{10}} \doteq 6{,}425\cdot10^{-9}$}
    \item bude vylosováno alespoň jedno číslo z losu?\lv{$1 - \text{výsledek \ref{los-nic}}$}
    \item bude vylosováno právě jedno číslo z losu (je jedno které)?\lv{$\frac{12\cdot\binom{38}{9}}{\binom{50}{10}} \doteq 0{,}19$}
    \item budou vylosována právě dvě čísla z losu (je jedno která)?\lv{$\frac{\binom{12}{2}\cdot\binom{38}{8}}{\binom{50}{10}} \doteq 0{,}31$}
\end{enumerate}
\end{ulohav}


\begin{ulohav}\label{koule}
V pytli, do kterého nevidíme, je pět červených a osm zelených koulí. Karel postupně z pytle vytáhl tři koule (po vytažení je už nevracel). Jaká je pravděpodobnost, že
\begin{enumerate}
    \item první vytažená koule je červená?\lv{$\frac{5}{13}$}
    \item druhá vytažená koule je červená?\lv{$\frac{5}{13}$}
    \item jsou všechny vytažené koule červené?\lv{$\frac{\binom{5}{3}}{\binom{13}{3}} = \frac{5\cdot4\cdot3}{13\cdot12\cdot11} = \frac{5}{143}$}
    \item je právě jedna z vytažených koulí červená?\label{jedna-cervena}\lv{$\frac{5\cdot \binom82}{\binom{13}3} = \frac{3\cdot5\cdot8\cdot7}{13\cdot12\cdot11} = \frac{70}{143}$}
    \item \emph{pouze ta první} z vytažených koulí je červená?\lv{$\frac{5\cdot 8 \cdot 7}{13 \cdot 12 \cdot 11} = \frac{70}{429}$ nebo taky $\frac13 \cdot{}$výsledek~\ref{jedna-cervena}}
    \item \emph{pouze ta první} z vytažených koulí je zelená?\lv{$\frac{8 \cdot 5 \cdot 4}{13\cdot12\cdot11} = \frac{40}{429}$}
\end{enumerate}
\end{ulohav}


\interest
\begin{uloha}
Kolik by musel Karel vytáhnout z pytle koulí (stejná situace jako v Úloze \ref{koule}), aby pravděpodobnost, že mezi vytaženými budou právě tři červené, byla co nejvyšší? \hint{Prostě to spočtěte pro různá čísla.}\vysl{8, pravděpodobnost je $\frac{560}{1287} \doteq 0{,}435$}
\end{uloha}


\baselineskip=1.25\baselineskip
\setlist[enumerate]{label=\textbf{(\alph*)},itemjoin={\quad}}

\results
\parindent=0pt
\parskip=\smallskipamount
\rightskip=0pt plus1fil\relax
\def\printvysl#1#2{\textbf{#1.} #2\par}
\vysld
\endresults


\end{document}