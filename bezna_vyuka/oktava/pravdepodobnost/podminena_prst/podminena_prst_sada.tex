\documentclass[8pt,a5paper]{extarticle}
\usepackage[margin=.5cm,bottom=10mm]{geometry}
\usepackage[utf8]{inputenc}
\usepackage[IL2]{fontenc}
\usepackage[czech]{babel}
\usepackage{microtype}
\usepackage{amssymb}
\usepackage{amsthm}
\usepackage{amsmath}
\usepackage{graphicx}
\usepackage{wasysym}
\usepackage{multicol}
\usepackage{wrapfig}
\usepackage{pifont}
\newcommand{\cmark}{\ding{51}}%
\newcommand{\xmark}{\ding{55}}%

\multicolsep=\smallskipamount

\usepackage[inline]{enumitem}

\newcommand{\R}{\mathbb{R}}

\newcommand{\hint}[1]{{\color{gray}\footnotesize\noindent(Nápověda: #1)}}

\setlist[enumerate]{label={(\alph*)},topsep=\smallskipamount,itemsep=\smallskipamount,parsep=0pt,itemjoin={\quad}}
\setlist[itemize]{topsep=\smallskipamount,noitemsep}

\def\tisk{%
\newbox\shipouthackbox
\pdfpagewidth=2\pdfpagewidth
\let\oldshipout=\shipout
\def\shipout{\afterassignment\zdvojtmp \setbox\shipouthackbox=}%
\def\zdvojtmp{\aftergroup\zdvoj}%
\def\zdvoj{%
    \oldshipout\vbox{\hbox{%
        \copy\shipouthackbox
        \hskip\dimexpr .5\pdfpagewidth-\wd\shipouthackbox\relax
        \box\shipouthackbox
    }}%
}}%

\let\results\newpage
\let\endresults\relax

\def\resultssame{%
    \long\def\results##1\endresults{%
        \vfill
        \noindent\rotatebox{180}{\vbox{##1}}%
    }%
}


\newtheorem*{poz}{Pozorování}

\theoremstyle{definition}
\newtheorem{uloha}{\atr Úloha}
\newtheorem{suloha}[uloha]{\llap{$\star$ }Úloha}
\newtheorem*{bonus}{Bonus}
\newtheorem*{defn}{Definice}

\pagestyle{empty}

\let\ee\expandafter

\def\vysld{}
\let\printvysl\relax

\makeatletter
\long\def\vyslplain#1{\ee\ee\ee\gdef\ee\ee\ee\vysld\ee\ee\ee{\ee\vysld\ee\printvysl\ee{\the\c@uloha}{#1}}}
\let\vysl\vyslplain

\def\locvysl#1{\ee\gdef\ee\locvysld\ee{\locvysld\item #1}}
\let\lv\locvysl

\newenvironment{ulohav}[1][]{\begin{uloha}[#1]\gdef\locvysld{\begin{enumerate*}}}{\ee\vyslplain\ee{\locvysld\end{enumerate*}}\end{uloha}}
\def\stitem{\@noitemargtrue\@item[$\star$ \@itemlabel]}

\makeatother

\def\atr{}
\def\basic{\def\atr{\llap{\mdseries$\sun$ }\gdef\atr{}}}
\def\interest{\def\atr{\llap{$\star$ }\gdef\atr{}}}
\def\iinterest{\def\atr{\llap{$\star\star$ }\gdef\atr{}}}

\begin{document}

% \tisk
% \resultssame

\section*{34. Nezávislost a podmíněná pravděpodobnost}


\begin{uloha}
Na jisté vysoké škole 15\% studentů neprojde zkouškou z matematiky, 10\% zkouškou z fyziky a 5\% ani jednou ze zkoušek. Jsou jevy \uv{Student neudělá zkoušku z matematiky} a \uv{Student neudělá zkoušku z fyziky} nezávislé?\vysl{nejsou, protože $0{,}05 \neq 0{,}15 \cdot 0{,}10$}
\end{uloha}


\interest
\begin{uloha}
Dokažte, že pokud jsou $A$, $B$ nezávislé jevy, pak jsou nezávislé i dvojice $(A, B')$, $(A', B)$ a $(A', B')$.
\end{uloha}


\begin{ulohav}
Novákovi mají dvě děti.
\begin{enumerate}
    \item Pokud víme, že jedno dítě je dívka, s jakou pravděpodobností je i to druhé dívka?\lv{$\frac13$}
    \item Pokud víme, že to starší dítě je dívka, s jakou pravděpodobností je i to mladší dívka?\lv{$\frac12$}
\end{enumerate}
Předpokládejte, že pohlaví dětí jsou na sobě nezávislá a narození chlapce a dívky je stejně pravděpodobné.
\end{ulohav}


\begin{ulohav}
Hodíme dvěma spravedlivými mincemi.
\begin{enumerate}
    \item Pokud víme, že na jedné z nich padla panna, s jakou pravděpodobností padla i na druhé?\lv{$\frac13$}
    \item Pokud víme, že na první z nich padla panna, s jakou pravděpodobností padla i na druhé?\lv{$\frac12$}
\end{enumerate}
\end{ulohav}


\begin{ulohav}
Hodíme dvěma mincemi, na nichž panna padne s pravděpodobností $\frac35$.
\begin{enumerate}
    \item Pokud víme, že na jedné z nich padla panna, s jakou pravděpodobností padla i na druhé?\lv{$\frac{(\frac35)^2}{1-(\frac25)^2}=\frac{3}{7}$}
    \item Pokud víme, že na první z nich padla panna, s jakou pravděpodobností padla i na druhé?\lv{$\frac35$}
\end{enumerate}
\end{ulohav}


\begin{ulohav}
Hodíme dvěma spravedlivými šestistěnnými kostkami. Jaká je pravděpodobnost, že
\begin{enumerate}
    \item na první kostce padla trojka, pokud na druhé padla čtyřka?\lv{$\frac16$}
    \item na první kostce padla trojka, pokud součet čísel na kostkách je $4$?\lv{$\frac13$}
    \item součet padlých čísel je sudý, pokud je jejich součin sudý?\lv{$\frac13$}
    \item součin padlých čísel je sudý, pokud je jejich součet sudý?\lv{$\frac12$}
\end{enumerate}
\end{ulohav}


\begin{ulohav}
Náhodně seřadíme čísla $1,\dots,8$. Jaká je pravděpodobnost, že
\begin{enumerate}
    \item výsledné pořadí končí 8, pokud začíná 1?\lv{$\frac17$}
    \item výsledné pořadí končí 8, pokud \emph{nezačíná} 1?\lv{$\frac{6 \cdot 6!}{7 \cdot 7!} = \frac{6}{49}$}
    \item ve výsledém pořadí je trojka bezprostředně za dvojkou, pokud je ve výsledném pořadí dvojka bezprostředně za jedničkou?\lv{$\frac{6!}{7!} = \frac17$}
\end{enumerate}
\end{ulohav}


\def\rv#1{\mathrm{\ref{#1}}}
\begin{ulohav}
Marta hází spravedlivou mincí; pokud padne panna, tak druhý hod uskuteční tou samou, pokud orel, tak druhý hod provede mincí, na které padne panna s pravděpodobností $\frac45$.
\begin{enumerate}
    \item Určete pravděpodobnost, že v prvním hodu padne panna.\lv{$\frac12$}
    \item Určete pravděpodobnost, že v obou hodech padne panna.\lv{$\frac14$}\label{pp}
    \item Určete pravděpodobnost, že v obou hodech padne orel.\lv{$\frac{1}{10}$}\label{oo}
    \item Určete pravděpodobnost, že nejprve padne panna a potom orel.\lv{$\frac{1}{4}$}\label{po}
    \item Určete pravděpodobnost, že nejprve padne orel a potom panna.\lv{$\frac{2}{5}$}\label{op}
    \item Určete pravděpodobnost, že v druhém hodu padne orel.\lv{$\frac{7}{20} = \rv{oo} + \rv{po}$}\label{xo}
    \item Pokud v druhém hodu padl orel, s jakou pravděpodobností padla v prvním panna?\lv{$\frac57 = \frac{\rv{po}}{\rv{xo}}$}
    \item Pokud v druhém hodu padl orel, s jakou pravděpodobností padl v prvním orel?\lv{$\frac27 = \frac{\rv{oo}}{\rv{xo}}$}
    \item Pokud v druhém hodu padla panna, s jakou pravděpodobností padla v prvním panna?\lv{$\frac5{13} = \frac{\rv{pp}}{\rv{pp} + \rv{op}}$}
    \item Pokud v druhém hodu padla panna, s jakou pravděpodobností padl v prvním orel?\lv{$\frac8{13} = \frac{\rv{op}}{\rv{pp} + \rv{op}}$}
\end{enumerate}
\end{ulohav}


\baselineskip=1.25\baselineskip
\setlist[enumerate]{label=\textbf{(\alph*)},itemjoin={\quad}}

\results
\parindent=0pt
\parskip=\smallskipamount
\rightskip=0pt plus1fil\relax
\def\printvysl#1#2{\textbf{#1.} #2\par}
\begin{multicols}{2}
\vysld
\end{multicols}
\endresults



\end{document}