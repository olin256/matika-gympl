\documentclass[8pt,a5paper]{extarticle}
\usepackage[margin=.6cm]{geometry}
\usepackage[utf8]{inputenc}
\usepackage[IL2]{fontenc}
\usepackage[czech]{babel}
\usepackage{microtype}
\usepackage{amssymb}
\usepackage{amsthm}
\usepackage{amsmath}
\usepackage{xcolor}
\usepackage{graphicx}
\usepackage{wasysym}
\usepackage{multicol}
\usepackage{tikz}

\usepackage[inline]{enumitem}

\newcommand{\R}{\mathbb{R}}

\newcommand{\hint}[1]{{\color{gray}\footnotesize\noindent(Nápověda: #1)}}

\DeclareMathOperator{\tg}{tg}
\DeclareMathOperator{\cotg}{cotg}

\setlist[enumerate]{label={(\alph*)},topsep=\smallskipamount,itemsep=\smallskipamount,parsep=0pt,itemjoin={\quad}}
\setlist[itemize]{topsep=\smallskipamount,noitemsep}

\def\tisk{%
\newbox\shipouthackbox
\pdfpagewidth=2\pdfpagewidth
\let\oldshipout=\shipout
\def\shipout{\afterassignment\zdvojtmp \setbox\shipouthackbox=}%
\def\zdvojtmp{\aftergroup\zdvoj}%
\def\zdvoj{%
    \oldshipout\vbox{\hbox{%
        \copy\shipouthackbox
        \hskip\dimexpr .5\pdfpagewidth-\wd\shipouthackbox\relax
        \box\shipouthackbox
    }}%
}}%

\let\results\newpage
\let\endresults\relax

\def\resultssame{%
    \long\def\results##1\endresults{%
        %\vfill
        \noindent\rotatebox{180}{\vbox{##1}}%
    }%
}


\newtheorem*{poz}{Pozorování}

\theoremstyle{definition}
\newtheorem{uloha}{\atr Úloha}
\newtheorem{suloha}[uloha]{\llap{$\star$ }Úloha}
\newtheorem*{bonus}{Bonus}
\newtheorem*{defn}{Definice}

\pagestyle{empty}

\DeclareMathOperator{\arctg}{arctg}

\let\ee\expandafter

\def\vysld{}
\let\printvysl\relax

\makeatletter
\long\def\vyslplain#1{\ee\ee\ee\gdef\ee\ee\ee\vysld\ee\ee\ee{\ee\vysld\ee\printvysl\ee{\the\c@uloha}{#1}}}
\let\vysl\vyslplain

\def\locvysl#1{\ee\gdef\ee\locvysld\ee{\locvysld\item #1}}
\let\lv\locvysl

\newenvironment{ulohav}[1][]{\begin{uloha}[#1]\gdef\locvysld{\begin{enumerate*}}}{\ee\vyslplain\ee{\locvysld\end{enumerate*}}\end{uloha}}
\def\stitem{\@noitemargtrue\@item[$\star$ \@itemlabel]}

\makeatother

\def\atr{}
\def\basic{\def\atr{\llap{\mdseries$\sun$ }\gdef\atr{}}}
\def\interest{\def\atr{\llap{$\star$ }\gdef\atr{}}}
\def\iinterest{\def\atr{\llap{$\star\star$ }\gdef\atr{}}}
\let\mb\mathbf



\begin{document}

%\tisk
%\resultssame


\section*{30. Aplikace kombinačních čísel}

\begin{ulohav}
Martin má sedm knih, o které se zajímá Ivana; oproti tomu Ivana má deset knih, o které se zajímá Martin. Určete, kolika způsoby si mohou Martin a Ivana vyměnit \begin{enumerate*}\item dvě,\lv{$\binom 72 \cdot \binom{10}2 = 945$}\item \lv{$\binom 73 \cdot \binom{10}3 = 4200$}tři\end{enumerate*} knihy.
\end{ulohav}

\begin{uloha}\label{obdelniky}
Tentokrát už naposledy: kolik je na obrázku obdélníků? \hint{Každý obdélník je popsán dvojicí vodorovných a dvojicí svislých čar.}\vysl{$\binom82 \cdot \binom 52 = 280$}
\[ \begin{tikzpicture}[scale=.3] \draw (0,0) grid (7, 4); \node at(.5,.5) {\smiley}; \node at(6.5,3.5) {\sun}; \end{tikzpicture} \]
\end{uloha}

\begin{uloha}
Kolika způsoby se v tabulce v Úloze \ref{obdelniky} můžeme dostat z políčka \smiley\ do políčka \sun, jestliže jsou povoleny pouze tahy o jedna nahoru nebo o jedna doprava? \hint{Kolik uděláme celkem tahů? Kolik jich bude nahoru?}\vysl{$\binom{9}3=84$}
\end{uloha}

\begin{ulohav}\label{soucty-souciny}
Určete, kolika způsoby lze vybrat (neuspořádanou) čtveřici (různých) čísel z $1,2,\dots, 20$ tak, aby
\begin{enumerate*}
    \item jejich součin byl sudý,\lv{$\binom{20}4 - \binom{10}4 = 4635$}
    \item jejich součin byl lichý,\lv{$\binom{10}4 = 210$}
    \item jejich součet byl sudý,\lv{$2\cdot\binom{10}4 + \binom{10}2 \cdot \binom{10}2 = 2445$}\label{soucet-sudy}
    \item jejich součet byl lichý.\lv{$2 \cdot \binom{10}1 \cdot \binom{10}3 = 2400$}\label{soucet-lichy}
\end{enumerate*}
\end{ulohav}

\interest
\begin{uloha}
Najděte chybu v následující \uv{úvaze}, proč by v Úloze \ref{soucty-souciny} měly \ref{soucet-sudy} a \ref{soucet-lichy} vyjít stejně: \uv{Mezi čísly $1,\dots, 20$ je stejně sudých jako lichých, takže se ve čtveřicích budou sudá a lichá čísla vyskytovat stejně často, proto budou i součty stejně často sudé jako liché.}
\end{uloha}

\begin{ulohav}
Mějme šachovnici $8\times8$. Kolika způsoby na ní lze vybrat trojici políček, pokud
\begin{enumerate}
    \item (žádná další podmínka),\lv{$\binom{64}3 = 41\,664$}
    \item nesmí být všechna téže barvy,\lv{$\binom{64}3 - 2\cdot \binom{32}3 = 2 \cdot \binom{32}2 \cdot \binom{32}1 = 31\,744$}
    \item nesmí ležet všechna v jednom řádku,\lv{$\binom{64}3 - 8 \cdot \binom83 = 41\,216$}
    \item nesmí ležet všechna v jednom řádku ani v jednom sloupci,\lv{$\binom{64}3 - 8 \cdot \binom83 - 8 \cdot \binom83 = 40\,768$}
    \stitem žádná dvě nesmí být v témže řádku,\lv{$8^3 \cdot\binom83 = 28\,672$}
    \stitem žádná dvě nesmí být v témže řádku ani sloupci.\lv{$\frac1{3!} \cdot 64\cdot49\cdot36 = 8\cdot7\cdot6 \cdot \binom83 = 18\,816$}
\end{enumerate}
\end{ulohav}

\begin{ulohav}
V rovině se nachází $n \in \mathbb N$ bodů. Kolik je těmito body určeno
\begin{enumerate}
    \item přímek, jestliže žádné tři body neleží na jedné přímce?\lv{$\binom n2$}
    \item trojúhelníků, jestliže žádné tři neleží na jedné přímce?\lv{$\binom n3$}
    \item přímek, jestliže jich $p$ leží na jedné přímce a kromě nich už žádné tři neleží?\lv{$\binom n2 - \binom p2 + 1$}
    \item trojúhelníků, jestliže jich $p$ leží na jedné přímce a kromě nich už žádné tři neleží?\lv{$\binom n3 - \binom p3$}
\end{enumerate}
\end{ulohav}

\interest
\begin{uloha}
Nalezněte všechna $n \in \mathbb N$ s touto vlastností: $(n+1)$-prvková množina má o 8515 víc tříprvkových podmnožin, než kolik jich má $n$-prvková.\vysl{$n = 131$}
\end{uloha}

\iinterest
\begin{uloha}
Celkem $n$ pirátů uložilo svůj společný poklad do truhly. Na truhlu umístili celkem $\ell$ zámků, ke kterým si potom rozdali klíče, a to tak, že kdykoliv se sejde více jak $k$ pirátů, tak se budou moct do truhly dostat, zatímco když se jich sejde nejvýše $k$, tak truhlu neotevřou (tj. od nějakého zámku jim bude chybět klíč). Jaký je nejmenší počet zámků $\ell$, pro který lze totoho dosáhnout, a jak si mají od nich rozdat klíče?\vysl{$\ell=\binom nk$; každý zámek odpovídá jedné $k$-prvkové podmnožině pirátů, přičemž klíče od něj dostanou právě ti, kteří v oné množině nejsou}
\end{uloha}



\baselineskip=1.25\baselineskip
\setlist[enumerate]{label=\textbf{(\alph*)},itemjoin={\quad}}

\results
\parindent=0pt
\parskip=\smallskipamount
\rightskip=0pt plus1fil\relax
\def\printvysl#1#2{\textbf{#1.} #2\par}
\vysld
\endresults


\end{document}