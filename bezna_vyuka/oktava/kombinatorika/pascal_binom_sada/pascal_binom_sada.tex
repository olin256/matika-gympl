\documentclass[9pt,a5paper]{extarticle}
\usepackage[margin=.6cm]{geometry}
\usepackage[utf8]{inputenc}
\usepackage[IL2]{fontenc}
\usepackage[czech]{babel}
\usepackage{microtype}
\usepackage{amssymb}
\usepackage{amsthm}
\usepackage{amsmath}
\usepackage{xcolor}
\usepackage{graphicx}
\usepackage{wasysym}
\usepackage{multicol}
\usepackage{tikz}

\usetikzlibrary{math}

\usepackage[inline]{enumitem}

\newcommand{\R}{\mathbb{R}}

\newcommand{\hint}[1]{{\color{gray}\footnotesize\noindent(Nápověda: #1)}}

\setlist[enumerate]{label={(\alph*)},topsep=\smallskipamount,itemsep=\smallskipamount,parsep=0pt,itemjoin={\quad}}
\setlist[itemize]{topsep=\smallskipamount,noitemsep}

\def\tisk{%
\newbox\shipouthackbox
\pdfpagewidth=2\pdfpagewidth
\let\oldshipout=\shipout
\def\shipout{\afterassignment\zdvojtmp \setbox\shipouthackbox=}%
\def\zdvojtmp{\aftergroup\zdvoj}%
\def\zdvoj{%
    \oldshipout\vbox{\hbox{%
        \copy\shipouthackbox
        \hskip\dimexpr .5\pdfpagewidth-\wd\shipouthackbox\relax
        \box\shipouthackbox
    }}%
}}%

\let\results\newpage
\let\endresults\relax

\def\resultssame{%
    \long\def\results##1\endresults{%
        \vfill
        \noindent\rotatebox{180}{\vbox{##1}}%
    }%
}


\newtheorem*{poz}{Pozorování}

\theoremstyle{definition}
\newtheorem{uloha}{\atr Úloha}
\newtheorem{suloha}[uloha]{\llap{$\star$ }Úloha}
\newtheorem*{bonus}{Bonus}
\newtheorem*{defn}{Definice}

\pagestyle{empty}

\let\ee\expandafter

\def\vysld{}
\let\printvysl\relax

\makeatletter
\long\def\vyslplain#1{\ee\ee\ee\gdef\ee\ee\ee\vysld\ee\ee\ee{\ee\vysld\ee\printvysl\ee{\the\c@uloha}{#1}}}
\let\vysl\vyslplain

\def\locvysl#1{\ee\gdef\ee\locvysld\ee{\locvysld\item #1}}
\let\lv\locvysl

\newenvironment{ulohav}[1][]{\begin{uloha}[#1]\gdef\locvysld{\begin{enumerate*}}}{\ee\vyslplain\ee{\locvysld\end{enumerate*}}\end{uloha}}
\def\stitem{\@noitemargtrue\@item[$\star$ \@itemlabel]}

\makeatother

\def\atr{}
\def\basic{\def\atr{\llap{\mdseries$\sun$ }\gdef\atr{}}}
\def\interest{\def\atr{\llap{$\star$ }\gdef\atr{}}}
\def\iinterest{\def\atr{\llap{$\star\star$ }\gdef\atr{}}}


\begin{document}

%\tisk
%\resultssame

\section*{31. Pascalův trojúhelník a binomická věta}

\begin{uloha}
$11^0 = 1$, $11^1 = 11$, $11^2 = 121$, $11^3 = 1331$, vidíte to taky? Proč to funguje? Kdy se to poprvé pokazí? \hint{$11 = 10 + 1$, binomická věta}
\end{uloha}

\begin{ulohav}
Roznásobte (= dosaďte do binomické věty) a pokud to nějak jde, tak výsledek zjednodušte:
\begin{enumerate*}
    \item $(2 + a)^3$\lv{$a^3+6 a^2+12 a+8$}
    \item $\left(\frac 1x + 2x\right)^5$\lv{$32 x^5+\frac{1}{x^5}+80 x^3+\frac{10}{x^3}+80 x+\frac{40}{x}$}
    \item $\left(x - \sqrt x\right)^4$\lv{$x^4 - 4x^3\sqrt x + 6x^3 - 4x^2 \sqrt x + x^2$}
\end{enumerate*}
\end{ulohav}

\begin{ulohav}
\everymath{\displaystyle}
Spočtěte \emph{bez kalkulačky (!)}, kolik je
\begin{enumerate*}
    \item $\bigl(1+\sqrt2\bigr)^5$\lv{$29 \sqrt{2}+41$}
    \item $\bigl(1-\sqrt2\bigr)^5$\lv{$-29 \sqrt{2}+41$}
    \item $\bigl(\sqrt{2} + \sqrt[4]{2}\bigr)^4$\lv{$6+8\cdot 2^{\frac14} + 12 \cdot 2^{\frac12} + 8 \cdot 2^{\frac34}$}
\end{enumerate*}
\end{ulohav}

\begin{uloha}
Jaký je desátý člen binomického rozvoje $(2a + b)^{15}$?\vysl{$\binom{15}{9} (2a)^9 b^6 = 320320 a^9 b^6$}
\end{uloha}

\begin{uloha}
Určete $x\in \R$ tak, aby pátý člen binomického rozvoje $\left(\frac2x -\sqrt x\right)^9$ byl roven 2016.
\vysl{$\sqrt[3]2$}
\end{uloha}

\begin{uloha}
Který člen binomického rozvoje $(y^2 + y^{-1})^9$ obsahuje $y^3$?\vysl{šestý}
\end{uloha}

\begin{uloha}
Vypočítejte takový člen binomického rozvoje $\left(3\sqrt a - a^{-2}\right)^{10}$, který neobsahuje $a$.\vysl{třetí, $295\,245$}
\end{uloha}

\begin{uloha}
\everymath{\displaystyle}
Dokažte, že následující čísla jsou celá pro všechna $n \in \mathbb N$:
\begin{enumerate}
    \item $\left(1+\sqrt3\right)^n + \left(1-\sqrt3\right)^n$
    \item $\frac{1}{\sqrt5}\left( \left( \frac{1+\sqrt5}{2} \right)^n - \left( \frac{1-\sqrt5}{2} \right)^n  \right)$\quad (jde o vzorec pro $n$-té Fibonacciho číslo)
\end{enumerate}
\end{uloha}

\interest
\begin{uloha}
Dokažte, že pro všechna $n \in \mathbb N$ je $\dfrac{4^n - 1}{3}$ celé číslo. \hint{$4 = 3 + 1$.}
\end{uloha}

\interest
\begin{uloha}
Součty následujících \uv{šikmých příček} dávají jistou známou posloupnost. Proč tomu tak je?
\[\begin{tikzpicture}[yscale=-.35,xscale=.5]
\tikzmath{
    function binom(\nn, \kk) {
        if \kk == 0 then {
            return 1;
        } else { 
            if \kk == \nn then {
                return 1;
            } else {
                return binom(\nn-1, \kk-1) + binom(\nn-1, \kk);
            };
        };
    };
    int \nnm;
    for \nn in {0,...,6} {
        for \kk in {0,...,\nn} {
            \nnm = binom(\nn, \kk);
            {\node at (-.5*\nn+\kk,\nn) {\nnm};};
        };
    };
}
\foreach \nn in {1,...,6} {
    \draw[help lines] (-.5*\nn,\nn) -- (.25*\nn,.5*\nn);
}
\end{tikzpicture}\]
\end{uloha}

\interest
\begin{uloha}
Derivace součinu se spočte jako $(fg)' = f'g + fg'$. Jak se spočte druhá derivace součinu? A jak to bude pokračovat? Proč? Sedí to i pro \uv{nultou derivaci}?
\end{uloha}

\interest
\begin{uloha}
Dokažte následující tvrzení: je-li $p$ prvočíslo, pak jsou všechna čísla
\[ \binom p1, \binom p2, \dots, \binom p{p-1} \]
násobky $p$. \hint{Využijte $\binom nk = \frac{n!}{k!(n-k)!}$ a zvažte, kde se v tomto zlomku objeví $p$.}
\end{uloha}


\baselineskip=1.25\baselineskip
\setlist[enumerate]{label=\textbf{(\alph*)},itemjoin={\quad}}

\results
\parindent=0pt
\parskip=\smallskipamount
\rightskip=0pt plus1fil\relax
\def\printvysl#1#2{\textbf{#1.} #2\par}
\vysld
\endresults


\end{document}