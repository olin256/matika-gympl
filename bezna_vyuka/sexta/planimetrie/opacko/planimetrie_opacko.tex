\documentclass[10pt,a5paper]{extarticle}
\usepackage[utf8]{inputenc}
\usepackage[IL2]{fontenc}
\usepackage[czech]{babel}
\usepackage{amsmath}
\usepackage{amsthm}
\usepackage{amssymb}
\usepackage[margin=1cm,top=1.2cm]{geometry}
\usepackage[inline]{enumitem}
\usepackage{tikz}
\usepackage{multicol}

\usetikzlibrary{positioning}

\setlist[enumerate]{label={(\alph*)},itemjoin={\quad},topsep=\smallskipamount,parsep=0pt,itemsep=\smallskipamount}

\theoremstyle{definition}
\newtheorem{uloha}{Úloha}
\newtheorem{suloha}[uloha]{\llap{$\star$ }Úloha}
\newtheorem{ssuloha}[uloha]{\llap{$\star\star$ }Úloha}

\def\tisk{%
\newbox\shipouthackbox
\pdfpagewidth=2\pdfpagewidth
\let\oldshipout=\shipout
\def\shipout{\afterassignment\zdvojtmp \setbox\shipouthackbox=}%
\def\zdvojtmp{\aftergroup\zdvoj}%
\def\zdvoj{%
    \oldshipout\vbox{\hbox{%
        \copy\shipouthackbox
        \hskip\dimexpr .5\pdfpagewidth-\wd\shipouthackbox\relax
        \box\shipouthackbox
    }}%
}}%

\pagestyle{empty}

\newcommand{\stitem}{\item[\addtocounter{enumi}{1}$\star$ (\alph{enumi})]}
\newcommand{\ststitem}{\item[\addtocounter{enumi}{1}$\star\star$ (\alph{enumi})]}

\newcommand{\hint}[1]{{\footnotesize\textcolor{gray}{(Nápověda: #1)}}}

\def\bod#1#2#3{\node[circle,fill=black,inner sep=0pt,outer sep=0pt,minimum size=3pt,label=#2:{$#3$}] (#3) at (#1) {};}

\begin{document}

%\tisk

\section*{Opáčko planimetrie}

\hbox to\hsize{%
\vbox{\hsize=.6\hsize
\begin{uloha}
Do trojúhelníka $ABC$ vepište obdélník, jehož delší strana bude ležet na straně $AB$ a jehož délky stran budou v poměru $1:2$.
\end{uloha}
}%
\hfil
\vbox to0pt{\vss
\hbox{\begin{tikzpicture}[scale=1.25]
\bod{0,0}{left}{A}
\bod{3,0}{right}{B}
\bod{2,1.7}{above}{C}
\draw (A) -- (B) -- (C) -- (A);
\end{tikzpicture}%
}}
}


\begin{uloha}
Sestrojte trojúhelník $ABC$ (v rámci časové úspory nemusíte rýsovat, jen si rozmyslete, jak na to, kolik bude existovat řešení a jak by se zapsala konstrukce), jestliže %\par
\vskip-1.5\bigskipamount\vbox{}
\begin{multicols}{2}%
\begin{enumerate}
    \item $b = 8$, $t_b = 4{,}5$, $\gamma = 30^\circ$; %pet 77/18 b
    \item $c = 6$, $v_a = 3{,}5$, $v_b = 5{,}5$; %pet 77/18 p
    \item $t_a = 7{,}5$, $t_c = 6$, $\alpha = 45^\circ$; %pet 77/18 s pouzije se teziste
    \item $t_c = 4$, $\alpha = 45^\circ$, $\gamma = 60^\circ$; %pet 77/18 v bud stejnolehlost, nebo doplneni na rovnobeznik
    \item $a=6$, $t_c=5$, $v_b=4$; % variace r7.5
    \stitem $a = 4$, $v_a = 3$, $r = 3{,}5$ (poloměr kruž. opsané);
    \stitem $v_a = 3$, $v_b = 4$, $v_c = 5$; \hint{Jaké budou poměry délek stran?}
    \ststitem $a = 4$, $\beta = 50^\circ$, $b + c = 7$.
\end{enumerate}
\end{multicols}
\end{uloha}

\begin{uloha}
Je dána úsečka délky $x$ (níže); sestrojte úsečku délky \begin{enumerate*}\item $\sqrt 6\, x$, \item $\sqrt 7\, x$.\end{enumerate*}
\[\begin{tikzpicture} \draw (0,0) -- node[below] {$x$} (1.63,0);
\draw (0,-.1) -- (0,.1);
\draw (1.63,-.1) -- (1.63,.1);
\end{tikzpicture}\]
\end{uloha}

\begin{uloha}
Sestrojte obrazy kružnice $k$ (se středem $T$) a trojúhelníka $ABC$ ve stej\-no\-leh\-lo\-sti se středem $S$ a koeficientem \begin{enumerate*} \item $\frac32$, \item $-\frac23$.\end{enumerate*}
\vskip1.2cm
\hbox to\hsize{\hfil\hbox{%
\begin{tikzpicture}[scale=1.1]
\bod{0,.75}{below}{S}
\bod{-3,-1}{above left}{A}
\bod{-2,.5}{above}{B}
\bod{-1,-1.5}{below}{C}
\draw (A) -- (B) -- (C) -- (A);
\bod{-3.5,1.3}{above}{T}
\draw (T) circle (1);
\node[above=1.1] at (T) {$k$};
\end{tikzpicture}}%
\hfil\hskip1cm
}
\end{uloha}



\end{document}

