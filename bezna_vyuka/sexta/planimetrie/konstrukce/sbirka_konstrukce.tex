\documentclass[9pt,a5paper]{extarticle}
\usepackage[utf8]{inputenc}
\usepackage[IL2]{fontenc}
\usepackage[czech]{babel}
\usepackage{amsmath}
\usepackage{amsthm}
\usepackage{amssymb}
\usepackage[margin=.9cm]{geometry}
\usepackage[inline]{enumitem}
\usepackage{xcolor}
\usepackage{multicol}

\multicolsep=\smallskipamount

\setlist[enumerate]{label={(\alph*)},itemjoin={\quad},topsep=\smallskipamount,parsep=0pt,itemsep=\smallskipamount}

\theoremstyle{definition}
\newtheorem{uloha}{Úloha}
\newtheorem{suloha}[uloha]{\llap{$\star$ }Úloha}
\newtheorem{ssuloha}[uloha]{\llap{$\star\star$ }Úloha}

\def\tisk{%
\newbox\shipouthackbox
\pdfpagewidth=2\pdfpagewidth
\let\oldshipout=\shipout
\def\shipout{\afterassignment\zdvojtmp \setbox\shipouthackbox=}%
\def\zdvojtmp{\aftergroup\zdvoj}%
\def\zdvoj{%
    \oldshipout\vbox{\hbox{%
        \copy\shipouthackbox
        \hskip\dimexpr .5\pdfpagewidth-\wd\shipouthackbox\relax
        \box\shipouthackbox
    }}%
}}%

\pagestyle{empty}

\newcommand{\staritem}{\item[\addtocounter{enumi}{1}$\star$ (\alph{enumi})]}

\newcommand{\hint}[1]{{\footnotesize\textcolor{gray}{(Nápověda: #1)}}}

\begin{document}

% \tisk

\begin{center}
    \textbf{\LARGE (Nevyřešená) sbírka}\\[1mm]
    na\\[1mm]
    {\Large geometrické konstrukce}
\end{center}

% totální vykrádačka realisticky.cz


\begin{uloha}
Je dána přímka $p$ a bod $A$ na ní neležící. Zkonstruujte přímku $q$, která bude procházet bodem $A$ a s $p$ bude svírat úhel $50^\circ$.
\end{uloha}

\begin{uloha}
Sestrojte trojúhelník $ABC$, jestliže\par
\begin{multicols}{2}%
\begin{enumerate}[topsep=0pt]
    \item je dána úsečka $AB$, $|AB|=6$, a $v_c=4$, $t_c=6$; %r5.1
    \item je dána úsečka $AB$, $|AB|=6$, a $b=5$, $\gamma=90^\circ$; %r5.2
    \item $c = 6$, $b = 5$, $\gamma=90^\circ$; %r5.3
    \item $a = 6$, $b = 5$, $\beta=50^\circ$ (zkuste dvě možné konstrukce); %r5.5
    \item $a=3$, $\alpha=60^\circ$, $\gamma=90^\circ$; %pet 77/17 c
    \item je dána úsečka $BB_1$ (která je těžnicí $t_b$), $|BB_1| = 6$, a $\alpha=45^\circ$, $b=5$; %pet 77/14 a
    \item je dána úsečka $AA_1$ (která je těžnicí $t_a$), $|AA_1| = 5$, a $v_a=4{,}5$, $c=5{,}5$; %r6.1
    \item je dána úsečka $AA_0$ (která je výškou $v_a$), $|AA_0| = 5$, a $t_a=6$, $c=5{,}5$; %r6.2
    \item je dána úsečka $BC$, $|BC|=5$, a $t_c=5$, $v_c=4{,}5$ %r6.3
    \staritem je dána úsečka $AA_1$ (která je těžnicí $t_a$), $|AA_1| = 6$, a $t_b=3{,}9$, $\beta=70^\circ$; %r6.4
    \item $t_b=5$, $v_b=4$, $\gamma=110^\circ$; %r6.5
    \item $t_c=3{,}5$, $b=4$, $\gamma=90^\circ$; \hint{Jaký je v~pravoúhlém trojúhelníku vztah mezi délkami $t_c$ a $c$? Proč?} %r7.1
    \item je dána úsečka $AA_1$ (která je těžnicí $t_a$), $|AA_1| = 4$, a $b=5$, $c=4$; %r7.3
    \item $t_c=3{,}5$, $b=5$, $\gamma=65^\circ$; %r7.4
    \item $a=5$, $t_c=4{,}5$, $v_b=4$; %r7.5
    \item je dána úsečka $BC$, $|BC|=6$, a $v_a=4$, $t_b=4{,}5$; %r7.6
    \item $t_a=5$, $t_b=6$, $t_c=4$; %r8.1
    \item $r=4$ (poloměr kruž. opsané), $c=5$, $t_c=3{,}5$; %r8.2
    \staritem $v_c=5$, $b=6$, $\varrho=2$ (poloměr kruž. vepsané). %r8.
\end{enumerate}
\end{multicols}
\end{uloha}

\begin{uloha}
Sestrojte čtyřúhelník $ABCD$, jestliže
\begin{enumerate}
    \item $|BC|=5$, $|AC|=6$, $|CD|=4$, $|\sphericalangle DAB|=80^\circ$, $|\sphericalangle BCD|=100^\circ$; %r9.1
    \item je to rovnoběžník, je dána úsečka $AB$, $|AB| = 3$, $|BD| = 5{,}5$, $v_{AB} = 4$; %r9.2
    \item je to kosočtverec, $v_{AB}=4$, $|AC|=5$. %r9.3
\end{enumerate}
\end{uloha}

\begin{suloha}
Sestrojte všechny kružnice, které
\begin{enumerate}
    \item se dotýkají dvou daných rovnoběžných přímek a dané kružnice ležící uvnitř pásu;
    \item se dotýkají daného přímky $p$ a procházejí danými body $A$ ($A \in p$) a $B$ ($B \notin p$);
    \item mají poloměr $1$, dotýkají se dané přímky $p$ a prochází daným bodem $A$ ($A \notin p$) (rozeberte, za jakých okolností má úloha kolik řešení);
    \item mají poloměr $1$, dotýkají se dané kružnice $k$ a prochází daným bodem $A$ ($A \notin k$) (rozeberte, za jakých okolností má úloha kolik řešení).
\end{enumerate}
\end{suloha}

\begin{ssuloha}[Mascherionovské konstrukce] % https://prase.cz/library/KonstrukceKruzitkemMV/KonstrukceKruzitkemMV.pdf
Je dokázáno, že všechny konstrukce, které lze provést pravítkem (bez měřítka) a kružítkem, lze provést i \emph{jen kružítkem} (přičemž předpokládáme, že přímka či úsečka je \uv{zkonstruovaná}, pokud jsou zkonstruovány dva (krajní) body na ní). Jen pomocí kružítka proveďte: \begin{enumerate*}
\item k $AB$ nalezněte úsečku dvojnásobné délky;
\item k~bodu $C$ neležícím na $AB$ sestrojte bod osově symetrický podle $AB$;
\item ke kružnici $k$ se středem $O$ neležícím na přímce $AB$ sestrojte průsečík $k$ s $AB$;
\item k $AB$ nalezněte bod $C$ tak, aby platilo $AB \perp AC$;
\item sestrojte střed úsečky $AB$.
\end{enumerate*}\quad
\emph{(Pokračování třeba někdy příště.)}
\end{ssuloha}


\end{document}