\documentclass[10pt,a5paper]{extarticle}
\usepackage[margin=.8cm]{geometry}
\usepackage[utf8]{inputenc}
\usepackage[IL2]{fontenc}
\usepackage[czech]{babel}
\usepackage{microtype}
\usepackage{amssymb}
\usepackage{amsthm}
\usepackage{amsmath}
\usepackage{xcolor}
\usepackage{graphicx}
\usepackage{wasysym}
\usepackage{multicol}
\usepackage[inline]{enumitem}
\usepackage{pgfplots}

\pgfplotsset{compat=1.18}
\pgfplotsset{%
every tick label/.append style={font=\footnotesize},
grid=major,
xlabel={$x$},
ylabel={$y$},
axis lines=middle,
unit vector ratio=1 1,
xtick distance=1,
ytick distance=1}

\newcommand{\R}{\mathbb{R}}
\DeclareMathOperator{\relu}{relu}

\newcommand{\hint}[1]{{\color{gray}\footnotesize\noindent(Nápověda: #1)}}

\setlist[enumerate]{label={(\alph*)},topsep=\smallskipamount,itemsep=\smallskipamount,parsep=0pt,itemjoin={\quad}}
\setlist[itemize]{topsep=\smallskipamount,noitemsep}

\def\tisk{%
\newbox\shipouthackbox
\pdfpagewidth=2\pdfpagewidth
\let\oldshipout=\shipout
\def\shipout{\afterassignment\zdvojtmp \setbox\shipouthackbox=}%
\def\zdvojtmp{\aftergroup\zdvoj}%
\def\zdvoj{%
     \oldshipout\vbox{\hbox{%
        \copy\shipouthackbox
        \hskip\dimexpr .5\pdfpagewidth-\wd\shipouthackbox\relax
        \box\shipouthackbox
    }}%
}}%

\let\results\newpage
\let\endresults\relax

\def\resultssame{%
    \long\def\results##1\endresults{%
        \vfill
        \noindent\rotatebox{180}{\vbox{##1}}%
    }%
}


\newtheorem*{poz}{Pozorování}

\theoremstyle{definition}
\newtheorem{uloha}{\atr Úloha}
\newtheorem{suloha}[uloha]{\llap{$\star$ }Úloha}
\newtheorem*{bonus}{Bonus}
\newtheorem*{defn}{Definice}

\pagestyle{empty}

\let\ee\expandafter

\def\vysld{}
\let\printvysl\relax
\let\printalphvysl\relax

\makeatletter
\long\def\vyslplain#1{\ee\ee\ee\gdef\ee\ee\ee\vysld\ee\ee\ee{\ee\vysld\ee\printvysl\ee{\the\c@uloha}{#1}}}
\let\vysl\vyslplain

\def\locvysl#1{\ee\gdef\ee\locvysld\ee{\locvysld\item #1}}
\let\lv\locvysl

\newenvironment{ulohav}[1][]{\begin{uloha}[#1]\gdef\locvysld{\begin{enumerate*}}}{\ee\vyslplain\ee{\locvysld\end{enumerate*}}\end{uloha}}
\def\stitem{\@noitemargtrue\@item[$\star$ \@itemlabel]}

\makeatother

\def\atr{}
\def\basic{\def\atr{\llap{\mdseries$\sun$ }\gdef\atr{}}}
\def\interest{\def\atr{\llap{$\star$ }\gdef\atr{}}}
\def\iinterest{\def\atr{\llap{$\star\star$ }\gdef\atr{}}}

\newdimen\resgraphwidth
\resgraphwidth=5.5cm
\newcommand\smallgraph[3][]{$\vcenter{\hbox{\begin{tikzpicture}%
\begin{axis}[width=\resgraphwidth, xmin=-#2, xmax=#2, ymin=-#2, ymax=#2]
\addplot [thick, blue, domain=-#2:#2#1] {#3};%
\end{axis}\end{tikzpicture}}}$}
\newcommand\resgraph[3][]{\lv{\smallgraph[#1]{#2}{#3}}}


\begin{document}

%\tisk
%\resultssame

\section*{4. Funkce s absolutní hodnotou}

\begin{ulohav}\label{lin-grafy}
Načrtněte grafy funkcí
\begin{enumerate}
\item $f_1(x) = |x| - 1$ \resgraph{3}{abs(x) - 1}
\item $f_2(x) = |2x-1| - x$ \resgraph{3}{abs(2*x-1) - x}
\item $f_3(x) = |1-x| + |x-1|$ \resgraph{3}{2*abs(x-1)}
\item $f_4(x) = \frac12|x-2| - |x+1| + 2$ \resgraph{4}{abs(x-2)/2 - abs(x+1) + 2}
\item $f_5(x) = |x| - |x-2| + 2|x+1| - 3 - x$ \resgraph[,samples=100]{5}{abs(x) - abs(x-2) + 2*abs(x+1) - 3 - x}
\item $f_6(x) = |1-2x|-|1-3x|+|1-4x|$ \resgraph[,samples=100]{2}{abs(2*x-1) - abs(3*x-1) + abs(4*x-1)}
\end{enumerate}
\end{ulohav}

\begin{ulohav}
Podívejte se na graf $f_6(x)$ z úlohy \ref{lin-grafy} a určete:
\begin{enumerate}
\item všechny lokální extrémy $f_6$ a jejich typy (maximum/minimum? je ostré?),\lv{v $\frac14$ a $\frac12$ jsou ostrá lokální minima, v $\frac13$ je ostré lokální maximum}
\item všechny globální extrémy a jejich typy,\lv{pouze v~$\frac14$ je ostré globální minimum}
\item počet řešení rovnice $f_6(x) = c$ v závislosti na reálném parametru $c$ (jinak řečeno, pro jaké hodnoty bude mít tato rovnice kolik řešení),\lv{$c\in (-\infty; \frac14) \Rightarrow{}$0 řešení, $c=\frac14 \Rightarrow{}$1 řešení, $c \in (\frac14;\frac12) \cup (\frac23;\infty) \Rightarrow{}$2 řešení, $c \in \{\frac12;\frac23\}\Rightarrow{}$3 řešení, $c \in (\frac12;\frac23)\Rightarrow{}$4 řešení}
\item všechna řešení rovnice $f_6(x) = 10$ (tady bude potřeba i počítat, ale graf může pomoct -- na jakých intervalech může být řešení?).\lv{$K = \{-3;\frac{11}3\}$}
\end{enumerate}
\end{ulohav}


\begin{ulohav}
Ověřte (početně, tj. dosazením $-x$), že
\begin{enumerate}
\item funkce $h_1(x) = |x+6| + |x-6|$ je sudá,\lv{$h_1(-x) = |-x+6| + |{-x}-6| = |x-6| + |x+6| = h_1(x)$}
\item funkce $h_2(x) = |x+6| - |x-6| - x$ je lichá.\lv{$h_2(-x) = |{-x}+6| - |{-x}-6| + x = |x-6| - |x+6| + x = -h_2(x)$}
\end{enumerate}
\end{ulohav}

\interest
\begin{uloha}
Označme $\relu(x) = \frac12(x+|x|)$. Předpis funkce $g_1$ níže lze vyjádřit jako
\[ g_1\colon y = x - \relu(x+2) + 3\relu(x+1) - 4\relu(x-1).\]
Rozmyslete si, jak toto funguje, a najděte obdobný předpis pro další dvě funkce.

\hbox to\hsize{%
\hfil
$g_1\colon$\smallgraph{4}{5/2-2* abs(-1+x)+3*abs(1+x)/2 - abs(2+x)/2}\hfil
$g_2\colon$\smallgraph{4}{7/2+x/2-abs(-1+x)-abs(3+x)/2}\hfil
$g_3\colon$\smallgraph{4}{8-2*abs(-2+x)+abs(-1+x)/2+abs(1+x)/2-2*abs(2+x)}\hfil
}
\vysl{$g_2\colon y = 2x + 4 - \relu(x+3) - 2\relu(x-1)$, $g_3\colon y = 3 x + 8 - 4 \relu(x + 2) + \relu(x + 1) + \relu(x - 1) - 4 \relu(x - 2)$}
\end{uloha}


\begin{ulohav}
Načrtněte grafy funkcí
\multicolsep=\smallskipamount
\begin{multicols}{2}
\begin{enumerate}
\item $k_1(x) = |x^2 - 1|$ \resgraph{3}{abs(x^2 - 1)}
\item $k_2(x) = -|x^2 - 1| + 1$ \resgraph{3}{-abs(x^2 - 1) + 1}
\item $k_3(x) = x \cdot |x|$ \resgraph{4}{x*abs(x)}
\item $k_4(x) = x^2 - 2|x+1|$ \resgraph{4}{x*x - 2*abs(x+1)}
\item $k_5(x) = x \cdot |x+1|$ \resgraph{4}{x*abs(x+1)}
\item $k_6(x) = |x^2 - 1| + x$ \resgraph{4}{abs(x^2-1) + x}
\end{enumerate}
\end{multicols}
\end{ulohav}

\interest
\begin{ulohav}
Řešte rovnice ($\max(a,b) ={}$to větší z čísel $a$, $b$, $\min(a,b) ={}$to menší)
\begin{enumerate}
\item $\max(x+1, -2x) + 3 \min(x+2,-x+3) = 8$\lv{$K = \{\frac14;1\}$}
\item $\min(x+3, x^2) = 1$\lv{$K = \{-2;-1;1\}$}
\end{enumerate}
\end{ulohav}


\baselineskip=1.25\baselineskip
\setlist[enumerate]{label=\textbf{(\alph*)},itemjoin={\quad}}

\pgfplotsset{every tick label/.append style={font=\tiny}}
\resgraphwidth=5cm

\results
\subsubsection*{Výsledky}
\parindent=0pt
\parskip=\smallskipamount
\rightskip=0pt plus1fil\relax
\def\printvysl#1#2{\textbf{#1.} #2\par}
\vysld
\endresults


\end{document}