\documentclass[8pt,a5paper]{extarticle}
\usepackage[margin=.5cm]{geometry}
\usepackage[utf8]{inputenc}
\usepackage[IL2]{fontenc}
\usepackage[czech]{babel}
\usepackage{microtype}
\usepackage{amssymb}
\usepackage{amsthm}
\usepackage{amsmath}
\usepackage{xcolor}
\usepackage{graphicx}
\usepackage{wasysym}
\usepackage{multicol}
\usepackage[inline]{enumitem}
\usepackage{pgfplots}

\pgfplotsset{compat=1.18}
\pgfplotsset{%
every tick label/.append style={font=\footnotesize},
grid=major,
xlabel={$x$},
ylabel={$y$},
axis lines=middle,
unit vector ratio=1 1,
xtick distance=1,
ytick distance=1}

\newcommand{\R}{\mathbb{R}}
\newcommand{\N}{\mathbb{N}}

\newcommand{\hint}[1]{{\color{gray}\footnotesize\noindent(Nápověda: #1)}}

\setlist[enumerate]{label={(\alph*)},topsep=\smallskipamount,itemsep=\smallskipamount,parsep=0pt,itemjoin={\quad}}
\setlist[itemize]{topsep=\smallskipamount,noitemsep}

\def\tisk{%
\newbox\shipouthackbox
\pdfpagewidth=2\pdfpagewidth
\let\oldshipout=\shipout
\def\shipout{\afterassignment\zdvojtmp \setbox\shipouthackbox=}%
\def\zdvojtmp{\aftergroup\zdvoj}%
\def\zdvoj{%
    \oldshipout\vbox{\hbox{%
        \copy\shipouthackbox
        \hskip\dimexpr .5\pdfpagewidth-\wd\shipouthackbox\relax
        \box\shipouthackbox
    }}%
}}%

\let\results\newpage
\let\endresults\relax

\def\resultssame{%
    \long\def\results##1\endresults{%
        \vfill
        \noindent\rotatebox{180}{\vbox{##1}}%
    }%
}


\newtheorem*{poz}{Pozorování}

\theoremstyle{definition}
\newtheorem{uloha}{\atr Úloha}
\newtheorem{suloha}[uloha]{\llap{$\star$ }Úloha}
\newtheorem*{bonus}{Bonus}
\newtheorem*{defn}{Definice}

\pagestyle{empty}

\let\ee\expandafter

\def\vysld{}
\let\printvysl\relax
\let\printalphvysl\relax

\makeatletter
\long\def\vyslplain#1{\ee\ee\ee\gdef\ee\ee\ee\vysld\ee\ee\ee{\ee\vysld\ee\printvysl\ee{\the\c@uloha}{#1}}}
\let\vysl\vyslplain

\def\locvysl#1{\ee\gdef\ee\locvysld\ee{\locvysld\item #1}}
\let\lv\locvysl

\newenvironment{ulohav}[1][]{\begin{uloha}[#1]\gdef\locvysld{\begin{enumerate*}}}{\ee\vyslplain\ee{\locvysld\end{enumerate*}}\end{uloha}}
\def\stitem{\@noitemargtrue\@item[$\star$ \@itemlabel]}

\makeatother

\def\atr{}
\def\basic{\def\atr{\llap{\mdseries$\sun$ }\gdef\atr{}}}
\def\interest{\def\atr{\llap{$\star$ }\gdef\atr{}}}
\def\iinterest{\def\atr{\llap{$\star\star$ }\gdef\atr{}}}

\begin{document}


% \tisk
% \resultssame

\section*{8. Skládání funkcí}


\begin{ulohav}
Mějme funkce $f \colon y = x+2$, $g \colon y = 2x-1$, $h \colon y = x^2 - x$, $i\colon y = \frac1x$. Určete předpisy a definiční obory následujících funkcí:
\begin{enumerate*}
    \item $f \circ g$\lv{$y = (2x-1)+2 = 2x+1$,~$\R$}
    \item $g \circ f$\lv{$y = 2(x+2) - 1 = 2x+3$,~$\R$}
    \item $f \circ h$\lv{$y = x^2 - x + 2$,~$\R$}
    \item $h \circ f$\lv{$y = (x+2)^2 - (x+2)$,~$\R$}
    \item $i \circ h$\lv{$y = \frac1{x^2-x}$,~$\R\setminus\{0;1\}$}
    \item $h \circ i$\lv{$y = \left(\frac1x\right)^2 - \frac1x$,~$\R\setminus\{0\}$}
    \item $i \circ g \circ h \circ f$\lv{$y = \frac1{2((x+2)^2 - (x+2))-1}$,~$\R\setminus\bigl\{\frac12(-3\pm\sqrt3)\bigr\}$}
\end{enumerate*}
\end{ulohav}

\begin{ulohav}
O funkci $f$ víme to, že $D(f) = (-\infty; 1\rangle$ (a více vědět nepotřebujeme). Jaký definiční obor bude mít funkce $f \circ w$, jestliže
\begin{enumerate*}
    \item $w(x) = x^2$\lv{$\langle-1;1\rangle$}
    \item $w(x) = -x^2 + 1$\lv{$\R$}
    \item $w(x) = \frac1x$?\lv{$(-\infty;0) \cup \langle1;\infty)$}
\end{enumerate*}
\end{ulohav}


\begin{multicols}{2}
\begin{ulohav}\label{grafy}
Vizte grafy funkcí $f$ a $g$:

\hbox to\hsize{\hfil\begin{tikzpicture}\begin{axis}[
        width=\hsize,
        xmin=-4, xmax=4,
        ymin=-3, ymax=3,
        legend pos=north west,
        unbounded coords=discard,restrict y to domain=-15:15, samples=200, thick, domain=-4:4]
\addplot [blue] coordinates {(-3,-3) (-1,-1) (0,2) (1,-1) (3,-3)};
\addplot [red] coordinates {(-4,0) (-3,1) (-2,1) (-1,0) (0,0) (1,2) (3,2) (4,1)};
\addplot [blue, mark=*] coordinates {(-3,-3)};
\addplot [blue, mark=*] coordinates {(3,-3)};
\addplot [red, mark=*] coordinates {(-4,0)};
\addplot [red, mark=*] coordinates {(4,1)};
\legend{$f$, $g$}
\end{axis}\end{tikzpicture}\hfil}
\noindent
Určete hodnoty:
\begin{enumerate}
    \item $f(0)$\lv{$2$}
    \item $(f \circ f)(0)$\lv{$-2$}
    \item $(f \circ f \circ f)(0)$\lv{$-2$}
    \item $(f \circ f \circ f \circ f \circ f)(0)$\lv{$-2$}
    \item $(f \circ g)(3)$\lv{$-2$}
    \item $(g \circ f)(3)$\lv{$1$}
    \item $(f \circ g)\bigl(\frac32\bigr)$\lv{$-2$}
    \item $(g \circ g)\bigl(\frac32\bigr)$\lv{$2$}
    \item $(g \circ f \circ g)(-1)$\lv{$2$}
    \item $(g \circ f \circ g)(-2)$\lv{$0$}
\end{enumerate}
\end{ulohav}
\end{multicols}


\begin{uloha}
K funkci $f$ z Úlohy \ref{grafy} nalezněte takovou lineární funkci $\ell$ (tj. její předpis), aby platilo $(\ell \circ f)(1) = 0$ a $(\ell \circ f)\bigl(\tfrac23\bigr) = 2$.\vysl{$\ell(x) = 2x+2$}
\end{uloha}


\begin{uloha}
Uvažme funkci $h \colon y = x - 2$. Jak budou vypadat grafy funkcí $h \circ g$ a $g \circ h$, kde $g$ je z Úlohy \ref{grafy}?
\vysl{%
\begin{tikzpicture}\begin{axis}[
        width=.8\hsize,
        xmin=-4, xmax=6,
        ymin=-3, ymax=3,
        legend pos=outer north east,
        unbounded coords=discard,restrict y to domain=-15:15, samples=200, thick, domain=-4:4]
\addplot [red] coordinates {(-4,-2) (-3,-1) (-2,-1) (-1,-2) (0,-2) (1,0) (3,0) (4,-1)};
\addplot [cyan] coordinates {(-2,0) (-1,1) (0,1) (1,0) (2,0) (3,2) (5,2) (6,1)};
\addplot [red, mark=*] coordinates {(-4,-2)};
\addplot [red, mark=*] coordinates {(4,-1)};
\addplot [cyan, mark=*] coordinates {(-2,0)};
\addplot [cyan, mark=*] coordinates {(6,1)};
\legend{$h \circ g$, $g \circ h$}
\end{axis}\end{tikzpicture}%
}
\end{uloha}


\begin{ulohav}
Nalezněte všechny lineární funkce $\ell$ takové, že
\begin{enumerate*}
    \item $(\ell \circ \ell)(x) = 4x+2$,\lv{dvě řešení: $\ell_1(x) = 2x+\frac23$ a $\ell_2(x) = -2x-2$}
    \item $(\ell \circ \ell)(x) = -x+2$.\lv{taková $\ell$ neexistuje}
\end{enumerate*}
\hint{Vyjděte z obecného předpisu lineární funkce $y = kx+q$, který jen \uv{složíte se sebou samým}.}
\end{ulohav}


\begin{ulohav}
Označme $a(x) = x+1$ a $b(x) = 2x$.
\begin{enumerate}
    \item Jaký předpis bude mít (lineární!) funkce $\underbrace{a \circ a \circ \cdots \circ a}_{100\times}$?\lv{$y = x+100$}
    \item Jaký předpis bude mít (lineární!) funkce $\underbrace{b \circ b \circ \cdots \circ b}_{100\times}$?\lv{$y = 2^{100}x$}
    \stitem Jaký předpis bude mít (lineární!) funkce $\underbrace{c \circ c \circ \cdots \circ c}_{100\times}$, kde $c = a \circ b$?\lv{$y = 2^{100} x + 2^{100} - 1$}
\end{enumerate}
\end{ulohav}


\begin{ulohav}
Rozhodněte, pro které z následujících vlastností $\heartsuit$ platí výrok \uv{Složení dvou $\heartsuit$ funkcí je vždy~$\heartsuit$.}
\begin{enumerate*}
    \item lineární\lv{ano}
    \item kvadratická\lv{ne}
    \item prostá\lv{ano}
    \item rostoucí\lv{ano}
    \item klesající\lv{ne,~bude rostoucí}
    \item sudá\lv{ano}
    \item lichá.\lv{ano}
\end{enumerate*}
\end{ulohav}


\baselineskip=1.25\baselineskip
\setlist[enumerate]{label=\textbf{(\alph*)},itemjoin={\quad}}
\setlength{\columnseprule}{.4pt}

\results
\parindent=0pt
\parskip=\smallskipamount
\rightskip=0pt plus1fil\relax
\def\printvysl#1#2{\textbf{#1.} #2\par}
\begin{multicols}{2}
\vysld
\end{multicols}
\endresults

\end{document}