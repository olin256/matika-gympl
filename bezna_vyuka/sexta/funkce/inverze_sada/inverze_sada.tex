\documentclass[9pt,a5paper]{extarticle}
\usepackage[margin=.8cm]{geometry}
\usepackage[utf8]{inputenc}
\usepackage[IL2]{fontenc}
\usepackage[czech]{babel}
\usepackage{microtype}
\usepackage{amssymb}
\usepackage{amsthm}
\usepackage{amsmath}
\usepackage{xcolor}
\usepackage{graphicx}
\usepackage{wasysym}
\usepackage{multicol}
\usepackage[inline]{enumitem}
\usepackage{pgfplots}

\pgfplotsset{compat=1.18}
\pgfplotsset{every tick label/.append style={font=\tiny},
width=.45\hsize,
grid=major,
xmin=-5, xmax=5, xtick distance=1,
ymin=-5, ymax=5, ytick distance=1,
xlabel={$x$},ylabel={$y$},axis lines=middle,unit vector ratio=1 1
}

\newcommand{\R}{\mathbb{R}}

\newcommand{\hint}[1]{{\color{gray}\footnotesize\noindent(Nápověda: #1)}}

\setlist[enumerate]{label={(\alph*)},topsep=\smallskipamount,itemsep=\smallskipamount,parsep=0pt,itemjoin={\quad}}
\setlist[itemize]{topsep=\smallskipamount,noitemsep}

\def\tisk{%
\newbox\shipouthackbox
\pdfpagewidth=2\pdfpagewidth
\let\oldshipout=\shipout
\def\shipout{\afterassignment\zdvojtmp \setbox\shipouthackbox=}%
\def\zdvojtmp{\aftergroup\zdvoj}%
\def\zdvoj{%
    \oldshipout\vbox{\hbox{%
        \copy\shipouthackbox
        \hskip\dimexpr .5\pdfpagewidth-\wd\shipouthackbox\relax
        \box\shipouthackbox
    }}%
}}%

\let\results\newpage
\let\endresults\relax

\def\resultssame{%
    \long\def\results##1\endresults{%
        %\vfill
        \noindent\rotatebox{180}{\vbox{##1}}%
    }%
}


\newtheorem*{poz}{Pozorování}

\theoremstyle{definition}
\newtheorem{uloha}{\atr Úloha}
\newtheorem{suloha}[uloha]{\llap{$\star$ }Úloha}
\newtheorem*{bonus}{Bonus}
\newtheorem*{defn}{Definice}

\pagestyle{empty}

\let\ee\expandafter

\def\vysld{}
\let\printvysl\relax
\let\printalphvysl\relax

\makeatletter
\long\def\vyslplain#1{\ee\ee\ee\gdef\ee\ee\ee\vysld\ee\ee\ee{\ee\vysld\ee\printvysl\ee{\the\c@uloha}{#1}}}
\let\vysl\vyslplain

\def\locvysl#1{\ee\gdef\ee\locvysld\ee{\locvysld\item #1}}
\let\lv\locvysl

\newenvironment{ulohav}[1][]{\begin{uloha}[#1]\gdef\locvysld{\begin{enumerate*}}}{\ee\vyslplain\ee{\locvysld\end{enumerate*}}\end{uloha}}
\def\stitem{\@noitemargtrue\@item[$\star$ \@itemlabel]}

\makeatother

\def\atr{}
\def\basic{\def\atr{\llap{\mdseries$\sun$ }\gdef\atr{}}}
\def\interest{\def\atr{\llap{$\star$ }\gdef\atr{}}}
\def\iinterest{\def\atr{\llap{$\star\star$ }\gdef\atr{}}}


\begin{document}

%\tisk
%\resultssame

\section*{2. Inverzní funkce}

\begin{uloha}\label{grafy}
Dokreslete k zadaným grafům funkcí grafy k ním inverzních funkcí.

\hbox to\hsize{\hfil
\hbox{\begin{tikzpicture}\begin{axis}
\addplot [thick, blue, domain=-5:5, smooth] {1-.2*x^3};
\end{axis}\end{tikzpicture}}%
\hfil
\hbox{\begin{tikzpicture}\begin{axis}
\addplot [thick, blue] coordinates {(-4,-5) (-3,-4) (-2,-1) (0,2) (3,3) (5,4)};
\end{axis}\end{tikzpicture}}%
\hfil
\hbox{\begin{tikzpicture}\begin{axis}
\addplot [thick, blue] coordinates {(-5,3) (-3,4) (0,5)};
\addplot [thick, blue] coordinates {(1,-1) (2,2) (5,3)};
\addplot [thick, blue, mark=*] coordinates {(-5,3)};
\addplot [thick, blue, mark=*] coordinates {(0,5)};
\draw[blue,fill=white, thick] (5,3) circle [radius=.2];
\draw[blue,fill=white, thick] (1,-1) circle [radius=.2];
\end{axis}\end{tikzpicture}}%
\hfil}
\vysl{%
\begin{tikzpicture}\begin{axis}
\addplot [thick, blue, domain=-5:5, smooth] {1-.2*x^3};
\addplot [thick, red, domain=-5:5, smooth] ({1-.2*x^3},{x});
\end{axis}\end{tikzpicture}
\begin{tikzpicture}\begin{axis}
\addplot [thick, blue] coordinates {(-4,-5) (-3,-4) (-2,-1) (0,2) (3,3) (5,4)};
\addplot [thick, red] coordinates {(-5,-4) (-4,-3) (-1,-2) (2,0) (3,3) (4,5)};
\end{axis}\end{tikzpicture}
\begin{tikzpicture}\begin{axis}
\addplot [thick, blue] coordinates {(-5,3) (-3,4) (0,5)};
\addplot [thick, blue] coordinates {(1,-1) (2,2) (5,3)};
\addplot [thick, blue, mark=*] coordinates {(-5,3)};
\addplot [thick, blue, mark=*] coordinates {(0,5)};
\draw[blue,fill=white, thick] (5,3) circle [radius=.2];
\draw[blue,fill=white, thick] (1,-1) circle [radius=.2];
%
\addplot [thick, red] coordinates {(3,-5) (4,-3) (5,0)};
\addplot [thick, red] coordinates {(-1,1) (2,2) (3,5)};
\addplot [thick, red, mark=*] coordinates {(3,-5)};
\addplot [thick, red, mark=*] coordinates {(5,0)};
\draw[red,fill=white, thick] (3,5) circle [radius=.2];
\draw[red,fill=white, thick] (-1,1) circle [radius=.2];
\end{axis}\end{tikzpicture}
}
\end{uloha}


\begin{uloha}
Nalezněte zcela obecně předpis inverzní funkce k lineární funkci $f(x) = kx+q$, kde $k$ a $q$ jsou reálná čísla (výsledkem bude předpis funkce, ve kterém budou vystupovat $k$ a $q$). Bude inverzní funkce existovat vždy?\vysl{Inverzní funkce existuje, právě když $k \neq 0$. Její předpis je $f^{-1}(x) = \frac1k x - \frac qk$}
\end{uloha}

\begin{uloha}
Funkce $f$ je definována po částech:
\[f(x) = \begin{cases}
x+2 & \text{pro }x < -1\\
\tfrac12 x + \tfrac32 & \text{pro }x\geq -1
\end{cases}\]
Načrtněte graf funkce, ujistěte se, že jde o prostou funkci, a nalezněte obdobný předpis (\uv{po částech}) pro $f^{-1}$.
\vysl{\begin{tikzpicture}\begin{axis}[legend pos=south east]
\addplot [thick, blue] coordinates {(-5,-3) (-1,1) (5,4)};
\addplot [thick, red] coordinates {(-3,-5) (1,-1) (4,5)};
\legend{$\scriptstyle f$,$\scriptstyle f^{-1}$}
\end{axis}\end{tikzpicture}
$\displaystyle f^{-1}(x) = \begin{cases} x-2 & \text{pro }x<1 \\ 2x - 3 & \text{pro } x \geq 1\end{cases}$}
\end{uloha}


\begin{ulohav}
Rozhodněte, které výroky platí pro každou funkci $f$:
\begin{enumerate}
    \item Je-li $f$ prostá, pak je $f^{-1}$ také prostá.\lv{ano}
    \item Je-li $f$ rostoucí, pak je $f^{-1}$ také rostoucí.\lv{ano}
    \item Je-li $f$ klesající, pak je $f^{-1}$ také klesající.\lv{ano}
    \item Je-li $f$ neklesající a prostá, pak už musí být rostoucí.\lv{ano}
\end{enumerate}
\end{ulohav}

\begin{ulohav}
Rozhodněte, v jakém vztahu jsou funkce $f^{-1}$ a $g^{-1}$, je-li $f$ libovolná prostá funkce a $g$ je zadaná předpisem níže; např. pokud je $g(x) = f(x)+1$, tak $g^{-1}(x) = f^{-1}(x-1)$ (proč?).
\begin{enumerate}
    \item $g(x) = f(x+1)$\lv{$g^{-1}(x) = f^{-1}(x)-1$}
    \item $g(x) = -f(x)$\lv{$g^{-1}(x) = f^{-1}(-x)$}
    \item $g(x) = 2f(x)$\lv{$g^{-1}(x) = f^{-1}\bigl(\frac12 x\bigr)$}
    \item $g(x) = f(2x)$.\lv{$g^{-1}(x) = \frac12 f^{-1}(x)$}
\end{enumerate}
Můžete na to jít tak, že si představíte (či rovnou načrtnete) grafy $f$ a $g$ a co se s nimi stane po přechodu k inverzním funkcím.
\end{ulohav}


\interest
\begin{uloha}
Rozhodněte, co musí splňovat čtveřice reálných čísel $a$, $b$, $c$, $d$, aby funkce s předpisem
\[ y = \frac{ax+b}{cx+d} \]
byla prostá.\vysl{Musí platit $ad \neq bc$.}
\end{uloha}

\interest
\begin{uloha}
Známe již tři \uv{překlápěcí operace} s grafem funkce: $f(x) \to -f(x)$, $f(x) \to f(-x)$ a $f(x) \to f^{-1}(x)$. Kolik nejvíc různých grafů funkcí můžeme aplikováním těchto operací dostat z jednoho grafu funkce?\vysl{8 (se započítáním i toho původního)}
\end{uloha}


\baselineskip=1.25\baselineskip
\setlist[enumerate]{label=\textbf{(\alph*)},itemjoin={\quad}}

\results
\parindent=0pt
\parskip=\smallskipamount
\rightskip=0pt plus1fil\relax
\def\printvysl#1#2{\textbf{#1.} #2\par}
\vysld
\endresults

\end{document}