\documentclass[9pt,a5paper]{extarticle}
\usepackage[margin=.8cm]{geometry}
\usepackage[utf8]{inputenc}
\usepackage[IL2]{fontenc}
\usepackage[czech]{babel}
\usepackage{microtype}
\usepackage{amssymb}
\usepackage{amsthm}
\usepackage{amsmath}
\usepackage{xcolor}
\usepackage{graphicx}
\usepackage{wasysym}
\usepackage{multicol}
\usepackage[inline]{enumitem}
\usepackage{pgfplots}

\pgfplotsset{compat=1.18}

\newcommand{\R}{\mathbb{R}}

\newcommand{\hint}[1]{{\color{gray}\footnotesize\noindent(Nápověda: #1)}}

\DeclareMathOperator{\tg}{tg}
\DeclareMathOperator{\cotg}{cotg}

\setlist[enumerate]{label={(\alph*)},topsep=\smallskipamount,itemsep=\smallskipamount,parsep=0pt,itemjoin={\quad}}
\setlist[itemize]{topsep=\smallskipamount,noitemsep}

\def\tisk{%
\newbox\shipouthackbox
\pdfpagewidth=2\pdfpagewidth
\let\oldshipout=\shipout
\def\shipout{\afterassignment\zdvojtmp \setbox\shipouthackbox=}%
\def\zdvojtmp{\aftergroup\zdvoj}%
\def\zdvoj{%
    \oldshipout\vbox{\hbox{%
        \copy\shipouthackbox
        \hskip\dimexpr .5\pdfpagewidth-\wd\shipouthackbox\relax
        \box\shipouthackbox
    }}%
}}%

\let\results\newpage
\let\endresults\relax

\def\resultssame{%
    \long\def\results##1\endresults{%
        %\vfill
        \noindent\rotatebox{180}{\vbox{##1}}%
    }%
}


\newtheorem*{poz}{Pozorování}

\theoremstyle{definition}
\newtheorem{uloha}{\atr Úloha}
\newtheorem{suloha}[uloha]{\llap{$\star$ }Úloha}
\newtheorem*{bonus}{Bonus}
\newtheorem*{defn}{Definice}

\pagestyle{empty}

\let\ee\expandafter

\def\vysld{}
\let\printvysl\relax
\let\printalphvysl\relax

\makeatletter
\long\def\vyslplain#1{\ee\ee\ee\gdef\ee\ee\ee\vysld\ee\ee\ee{\ee\vysld\ee\printvysl\ee{\the\c@uloha}{#1}}}
\let\vysl\vyslplain

\def\locvysl#1{\ee\gdef\ee\locvysld\ee{\locvysld\item #1}}
\let\lv\locvysl

\newenvironment{ulohav}[1][]{\begin{uloha}[#1]\gdef\locvysld{\begin{enumerate*}}}{\ee\vyslplain\ee{\locvysld\end{enumerate*}}\end{uloha}}
\def\stitem{\@noitemargtrue\@item[$\star$ \@itemlabel]}

\makeatother

\def\atr{}
\def\basic{\def\atr{\llap{\mdseries$\sun$ }\gdef\atr{}}}
\def\interest{\def\atr{\llap{$\star$ }\gdef\atr{}}}
\def\iinterest{\def\atr{\llap{$\star\star$ }\gdef\atr{}}}


\begin{document}

%\tisk
%\resultssame

\section*{1. Kvadratické funkce}

\begin{uloha}\label{grafy}
Určete předpisy kvadratických funkcí $f$, $g$, $h$, jejichž grafy jsou níže.
\[ \begin{tikzpicture}\begin{axis}[
        width=.8\hsize,
        grid=major,
        xmin=-8, xmax=8, xtick distance=1,
        ymin=-4, ymax=4, ytick distance=1,
        xlabel={$x$},ylabel={$y$},axis lines=middle,unit vector ratio=1 1]
\addplot [thick, blue, domain=-7:-3, smooth] {2*(x+5)^2-1};
\addplot [thick, red, domain=-2:4, smooth] {3*(x-1)^2-2};
\addplot [thick, black, domain=-2:7, smooth] {-((x-4)^2)/2-1};
\legend{$f$, $g$, $h$}
\end{axis}\end{tikzpicture} \]
\vysl{$f(x) = 2(x+5)^2 - 1$,\quad $g(x) = 3(x-1)^2 - 2$,\quad $h(x) = -\frac12(x-4)^2-1$}
\end{uloha}


\begin{uloha}
U funkcí z úlohy \ref{grafy} určete definiční obor a obor hodnot.
\vysl{def. obor je vždy $\R$, $H(f) = \langle-1;\infty)$, $H(g) = \langle-2;\infty)$, $H(h) = (-\infty;-1\rangle$}
\end{uloha}


\begin{ulohav}
Pohledem na graf rozhodněte, kolik řešení budou mít rovnice
\begin{enumerate}
\item $f(x)=0$\lv{2}
\item $g(x)=-3$\lv{0}
\item $h(x) = -1$\lv{1}
\item $g(x) = x$\lv{2}
\item $f(x) = x$.\lv{0}
\end{enumerate}
\end{ulohav}


\begin{uloha}
O kvadratické funkci víme, že její graf protíná osu $x$ v bodech $[7; 0]$ a $[17; 0]$. O následujících údajích rozhodněte, zda je už umíme jednoznačně určit, a pokud ano, určete je:
\begin{itemize}
\item $x$-ová souřadnice vrcholu paraboly,
\item $y$-ová souřadnice vrcholu paraboly,
\item $x$-ová souřadnice průsečíku s osou $y$,
\item $y$-ová souřadnice průsečíku s osou $y$.
\end{itemize}
\vysl{Umíme určit $x$-ovou souřadnici vrcholu paraboly -- ten je totiž vždy horizontálně přesně mezi průsečíky s osou $x$, takže ona souřadnice je $(7+17)/2 = 12$. Dále $x$-ová souřadnice průsečíku s osou $y$ je vždy $0$. Zbývající dvě hodnoty určit nedokážeme: kvadratické funkce $f_1(x) = (x-7)(x-17)$ a $f_2(x) = 2(x-7)(x-17)$ mají obě zadané průsečíky s osou $x$, ale různou pozici vrcholu i průsečíku s osou $y$.}
\end{uloha}

\begin{uloha}
Může existovat kvadratická funkce $f$ taková, že by současně platilo $f(0) = f(2)$ a $f(1) = f(3)$?\vysl{nemůže; grafem je vždy parabola, což je osově souměrná křivka, přičemž $f(0) = f(2)$ by znamenalo, že bude osově souměrná podle přímky $x = 1$, zatímco $f(1) = f(3)$ by vedlo k~osové souměrnosti podle přímky $x = 2$}
\end{uloha}

\begin{ulohav}\label{predpisy}
Nalezněte předpis kvadratické funkce $f$, jejíž graf prochází následujícími body:
\begin{enumerate}
\item $[0;-3]$, $[1; 0]$, $[-1; -4]$,\lv{$f(x) = x^2 + 2x - 3$}
\item $[1;2]$, $[2;7]$, $[-1;4]$.\lv{$f(x) = 2x^2 - x + 1$}
\end{enumerate}
\end{ulohav}

\begin{uloha}
U funkcí z úlohy \ref{predpisy} určete souřadnice vrcholu a načrtněte jejich grafy (výsledek si můžete zkontrolovat např. ve Photomathu).
\end{uloha}

\begin{ulohav}
Projektil vypálený ze země v čase $t=0$ se v čase $t=1$ nacházel ve výšce 400\,m a v čase $t=5$ ve výšce 600\,m. Určete
\begin{enumerate*}
\item jaké maximální výšky projektil dosáhl,\lv{$\frac{11045}{14}$\,m}
\item v jakém čase to bylo,\lv{$\frac{47}{14}$}
\item kdy dopadl na zem.\lv{$\frac{47}{7}$}
\end{enumerate*}
Předpokládejte \uv{obvyklé fyzikální předpoklady} (bez odporu vzduchu, rovná zem, projektil letí po parabole\dots)
\end{ulohav}

\begin{uloha}
Parabolická nosná konstrukce mostu přes řeku má vrchol 6\,m nad vodorovnou vozovkou 24\,m dlouhou. Svislé nosné traverzy jsou rozmístěny pravidelně ve vzdálenostech 3\,m od sebe. Vypočítejte délky všech traverz.\vysl{$\frac{21}{8}$, $\frac{9}{2}$, $\frac{45}{8}$, 6 a pak ty samé}
\end{uloha}

\interest
\begin{uloha}
Reálná čísla $a$, $b$ splňují $2a+3b = 13$. Jaké maximální hodnoty může nabývat součin $a \cdot b$?\vysl{$\frac{169}{24}$ (pro $a = \frac{13}{4}$, $b = \frac{13}{6}$)}
\end{uloha}


\baselineskip=1.25\baselineskip
\setlist[enumerate]{label=\textbf{(\alph*)},itemjoin={\quad}}

\results
\parindent=0pt
\parskip=\smallskipamount
\rightskip=0pt plus1fil\relax
\def\printvysl#1#2{\textbf{#1.} #2\par}
\vysld
\endresults


\end{document}