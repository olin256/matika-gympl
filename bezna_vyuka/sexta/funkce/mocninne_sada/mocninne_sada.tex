\documentclass[8pt,a5paper]{extarticle}
%\documentclass[9pt,a5paper]{extarticle}
%\usepackage[margin=.75in]{geometry}
\usepackage[margin=.75cm]{geometry}
%\usepackage[margin=1in]{geometry}
\usepackage[utf8]{inputenc}
\usepackage[IL2]{fontenc}
\usepackage[czech]{babel}
\usepackage{microtype}
\usepackage{amssymb}
\usepackage{amsthm}
\usepackage{amsmath}
\usepackage{xcolor}
\usepackage{graphicx}
\usepackage{wasysym}
\usepackage{multicol}
\usepackage[inline]{enumitem}
\usepackage{pgfplots}

\pgfplotsset{compat=1.18}
\pgfplotsset{%
every tick label/.append style={font=\footnotesize},
grid=major,
xlabel={$x$},
ylabel={$y$},
axis lines=middle,
unit vector ratio=1 1,
xtick distance=1,
ytick distance=1}

\newcommand{\R}{\mathbb{R}}
\newcommand{\N}{\mathbb{N}}

\newcommand{\hint}[1]{{\color{gray}\footnotesize\noindent(Nápověda: #1)}}

\DeclareMathOperator{\tg}{tg}
\DeclareMathOperator{\relu}{relu}

\setlist[enumerate]{label={(\alph*)},topsep=\smallskipamount,itemsep=\smallskipamount,parsep=0pt,itemjoin={\quad}}
\setlist[itemize]{topsep=\smallskipamount,noitemsep}

\def\tisk{%
\newbox\shipouthackbox
\pdfpagewidth=2\pdfpagewidth
\let\oldshipout=\shipout
\def\shipout{\afterassignment\zdvojtmp \setbox\shipouthackbox=}%
\def\zdvojtmp{\aftergroup\zdvoj}%
\def\zdvoj{%
    \oldshipout\vbox{\hbox{%
        \copy\shipouthackbox
        \hskip\dimexpr .5\pdfpagewidth-\wd\shipouthackbox\relax
        \box\shipouthackbox
    }}%
}}%

\let\results\newpage
\let\endresults\relax

\def\resultssame{%
    \long\def\results##1\endresults{%
        \vfill
        \noindent\rotatebox{180}{\vbox{##1}}%
    }%
}


\newtheorem*{poz}{Pozorování}

\theoremstyle{definition}
\newtheorem{uloha}{\atr Úloha}
\newtheorem{suloha}[uloha]{\llap{$\star$ }Úloha}
\newtheorem*{bonus}{Bonus}
\newtheorem*{defn}{Definice}

\pagestyle{empty}

\let\ee\expandafter

\def\vysld{}
\let\printvysl\relax
\let\printalphvysl\relax

\makeatletter
\long\def\vyslplain#1{\ee\ee\ee\gdef\ee\ee\ee\vysld\ee\ee\ee{\ee\vysld\ee\printvysl\ee{\the\c@uloha}{#1}}}
\let\vysl\vyslplain

\def\locvysl#1{\ee\gdef\ee\locvysld\ee{\locvysld\item #1}}
\let\lv\locvysl

\newenvironment{ulohav}[1][]{\begin{uloha}[#1]\gdef\locvysld{\begin{enumerate*}}}{\ee\vyslplain\ee{\locvysld\end{enumerate*}}\end{uloha}}
\def\stitem{\@noitemargtrue\@item[$\star$ \@itemlabel]}

\makeatother

\def\atr{}
\def\basic{\def\atr{\llap{\mdseries$\sun$ }\gdef\atr{}}}
\def\interest{\def\atr{\llap{$\star$ }\gdef\atr{}}}
\def\iinterest{\def\atr{\llap{$\star\star$ }\gdef\atr{}}}


\begin{document}

% \tisk
% \resultssame


\section*{6. Mocninné funkce s celočíselnými exponenty}

\begin{uloha}
Jaký je definiční obor funkce $f\colon y = x^0$? Jak vypadá její graf?
\vysl{$D(f) = \R\setminus\{0\}$, graf je konstantí jednička, ale pro $x=0$ je to nedefinované (tj. \uv{prázdné kolečko})}
\end{uloha}

\begin{multicols}{2}

\begin{uloha}
Přiřaďte funkcím předpisy $y = x^1$, $x^2$, $x^5$, $x^6$, $x^9$.

\hbox to\hsize{\hfil\begin{tikzpicture}\begin{axis}[
        width=1.5\hsize,
        xmin=-3, xmax=3,
        ymin=-5, ymax=5,
        legend pos=south east,
        unbounded coords=discard,restrict y to domain=-40:40,samples=150]
\addplot [thick, black, densely dashed, domain=-8:8] {x^6};
\addplot [thick, black, densely dotted, domain=-8:8,] {x^9};
\addplot [thick, red, domain=-8:8] {x^2};
\addplot [thick, blue, domain=-8:8] {x};
\addplot [thick, black, domain=-8:8,] {x^5};
\legend{$f$, $g$, $h$, $i$, $j$}
\end{axis}\end{tikzpicture}\hfil}
\vysl{$f(x) = x^6$, $g(x) = x^9$, $h(x) = x^2$, $i(x) = x^1$, $j(x) = x^5$}
\end{uloha}

\begin{ulohav}
V této úloze uvažujeme jen mocninné funkce s \textbf{kladným celočíselným} exponentem.
Rozhodněte:\let\vysl\lv
\begin{enumerate}
\item Pro která $n \in \N$ je funkce s předpisem $y = x^n$ (shora/zdola) omezená?\vysl{pro sudá $n$ je zdola omezená, pro lichá ani shora ani zdola}
\item Pro která $n \in \N$ je funkce s předpisem $y = x^n$ rostoucí/klesající? Pokud není, je rostoucí/klesající alespoň na nějakých intervalech?\vysl{pro lichá $n$ je rostoucí, pro sudá $n$ je klesající na $(-\infty;0\rangle$ a rostoucí na $\langle0;\infty)$}
\item Jaké má funkce $y = x^n$ extrémy (v~závislosti na $n \in \N$)?\vysl{pro lichá $n$ žádné, pro sudá $n$ má ostré globální minimum v $x=0$}
\item Jaký má funkce $y = x^n$ obor hodnot (v~závislosti na $n \in \N$)?\vysl{pro lichá $n$ je to celé $\R$, pro sudá $n$ to je $\rangle0;\infty)$}
\end{enumerate}
\end{ulohav}

\begin{uloha}
Načrtněte grafy funkcí
\begin{enumerate*}
\item $y = x^3 + 1$
\item $y = (x+1)^3$
\item $y = (x+1)^3 + 1$
\item $y = 2x^3$
\item $y = 2(x+1)^3$
\item $y = 2(x+1)^3 - 1$
\item $y = -\frac12 x^3$
\item $y = |x^3 - 1|$
\end{enumerate*}
\end{uloha}

\begin{uloha}
Jaký definiční obor bude mít funkce $f \colon y = x^n$, jestliže $n$ je celé \textbf{záporné} číslo? Jaký obor hodnot?\vysl{$D(f) = \R\setminus\{0\}$, pro sudá $n$ je $H(f) = (0;\infty)$, pro lichá $n$ pak $H(f) = \R\setminus\{0\}$}
\end{uloha}

\end{multicols}




\begin{uloha}
Přiřaďte funkcím předpisy $y = x^{-1}$, $x^{-2}$, $x^{-5}$, $x^{-6}$, $x^{-9}$.

\hbox to\hsize{\hfil\begin{tikzpicture}\begin{axis}[
        width=.65\hsize,
        xmin=-5, xmax=5,
        ymin=-4, ymax=4,
        legend pos=south east,
        unbounded coords=discard,restrict y to domain=-10:10,samples=151]
\addplot [thick, black, densely dashed, domain=-8:8] {x^-6};
\addplot [thick, black, densely dotted, domain=-8:8,] {x^-9};
\addplot [thick, red, domain=-8:8] {x^-2};
\addplot [thick, blue, domain=-8:8] {x^-1};
\addplot [thick, black, domain=-8:8,] {x^-5};
\legend{$f$, $g$, $h$, $i$, $j$}
\end{axis}\end{tikzpicture}\hfil}
\vysl{$f(x) = x^{-6}$, $g(x) = x^{-9}$, $h(x) = x^{-2}$, $i(x) = x^{-1}$, $j(x) = x^{-5}$}
\end{uloha}


\baselineskip=1.25\baselineskip
\setlist[enumerate]{label=\textbf{(\alph*)},itemjoin={\quad}}



\results
\parindent=0pt
\parskip=\smallskipamount
\rightskip=0pt plus1fil\relax
\def\printvysl#1#2{\textbf{#1.} #2\par}
\vysld
\endresults


\end{document}