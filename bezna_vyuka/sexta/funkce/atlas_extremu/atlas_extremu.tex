\documentclass[12pt,a4paper]{article}

\usepackage{amsmath}
\usepackage{amssymb}
\usepackage[margin=.75in,bottom=1in]{geometry}
\usepackage[utf8]{inputenc}
\usepackage[IL2]{fontenc}
\usepackage[czech]{babel}
\usepackage{graphicx}
\usepackage{enumitem}
\usepackage{microtype}

\setlist[itemize]{noitemsep}

\DeclareMathOperator{\DO}{D}
\def\R{\mathbb R}

\parindent 0pt
\def\graf#1{\includegraphics[scale=.65]{grafy/#1.pdf}}

\begin{document}

\begin{center}
{\LARGE\bfseries Atlas extrémů}\\[1mm]
\Large\emph{aneb}\\[1mm]
{\Large Kterak minima a maxima rozeznávati.}
\end{center}

\section{Ostré globální maximum}
\begin{itemize}
    \item Též známé jen jako ostré maximum.
    \item Funkční hodnoty ve \textbf{všech} ostatních bodech definičního oboru jsou \textbf{menší}.
    \item Může existovat nanejvýš jedno pro jednu funkci.
    \item \uv{Ostrost} nemá nic společného s tím, jestli je v tom bodě opravdu \uv{ostrá špička}.
\end{itemize}
\[ \graf{glob_ostre_1}\quad\graf{glob_ostre_2} \]


\section{Ostré lokální maximum}
\begin{itemize}
    \item Funkční hodnoty ve \textbf{všech} ostatních bodech na \textbf{nějakém intervalu okolo toho bodu} jsou \textbf{menší} (tento interval obsahuje jak body \emph{vpravo}, tak body \emph{vlevo} od zkoumaného bodu!).
    \item Může jich existovat mnoho, ale nemohou tvořit souvislý interval; \uv{obvykle} jde o izolované body.
    \item Opět, \uv{ostrost} nemá nic společného s tím, jestli je v tom bodě opravdu \uv{ostrá špička}.
\end{itemize}
\[ \graf{lok_ostre_1}\quad\graf{lok_ostre_2} \]


\section{Globální maximum}
\begin{itemize}
    \item Též známé jen jako maximum.
    \item Funkční hodnoty ve \textbf{všech} ostatních bodech definičního oboru jsou \textbf{menší nebo stejně velké}.
    \item Může jich pro jednu funkci existovat mnoho, klidně mohou tvořit i interval.
\end{itemize}
\[ \graf{glob_neostre_1}\quad\graf{glob_neostre_2} \]


\section{Lokální maximum}
\begin{itemize}
    \item Funkční hodnoty ve \textbf{všech} ostatních bodech na \textbf{nějakém intervalu okolo toho bodu} jsou \textbf{menší nebo stejně velké}.
    \item Může jich pro jednu funkci existovat mnoho, klidně mohou tvořit i intervaly; takové intervaly mohou být různě otevřené, uzavřené, z různých stran různě\dots
\end{itemize}
\[ \graf{lok_neostre_1}\quad\graf{lok_neostre_2} \]

\section{Varování}
Pozor! Bod může být současně ostrým lokálním maximem a globálním maximem, a přitom \emph{nebýt} ostrým globálním maximem. Příkladem jsou zde vyznačené body:
\[ \graf{warn} \]
Nejde o ostrá globální maxima, protože to může být jen jedno. Nikde jinde větší hodnota není, takže jde alespoň o globální maxima. Konečně, když se podíváme jen na dostatečně malé intervaly okolo těchto bodů, zjistíme, že jde o ostrá lokální maxima.


\section{Minima}

Všechno to bude naprosto analogicky, pro vizuální příklady stačí otočit stránky o $180^\circ$.


\end{document}