\documentclass[10pt,a4paper,twocolumn]{extarticle}
\usepackage[margin=.8cm]{geometry}
\usepackage[utf8]{inputenc}
\usepackage[IL2]{fontenc}
\usepackage[czech]{babel}
\usepackage{microtype}
\usepackage{amssymb}
\usepackage{amsthm}
\usepackage{amsmath}
\usepackage{xcolor}
\usepackage{graphicx}
\usepackage{wasysym}
\usepackage{multicol}

\usepackage{tikz,pgfplots}
\pgfplotsset{compat=1.18}

\microtypesetup{nopatch=item}

\usepackage[inline]{enumitem}

\newcommand{\R}{\mathbb{R}}

\newcommand{\hint}[1]{{\color{gray}\footnotesize\noindent(Nápověda: #1)}}

\setlist[enumerate]{label={(\alph*)},topsep=\smallskipamount,itemsep=\smallskipamount,parsep=0pt,itemjoin={\quad}}
\setlist[itemize]{topsep=\smallskipamount,noitemsep}

\def\tisk{%
\newbox\shipouthackbox
\pdfpagewidth=2\pdfpagewidth
\let\oldshipout=\shipout
\def\shipout{\afterassignment\zdvojtmp \setbox\shipouthackbox=}%
\def\zdvojtmp{\aftergroup\zdvoj}%
\def\zdvoj{%
    \oldshipout\vbox{\hbox{%
        \copy\shipouthackbox
        \hskip\dimexpr .5\pdfpagewidth-\wd\shipouthackbox\relax
        \box\shipouthackbox
    }}%
}}%

\let\results\newpage
\let\endresults\relax

\def\resultssame{%
    \long\def\results##1\endresults{%
        %\vfill
        \noindent\rotatebox{180}{\vbox{##1}}%
    }%
}


\newtheorem*{poz}{Pozorování}

\theoremstyle{definition}
\newtheorem{uloha}{\atr Úloha}
\newtheorem{suloha}[uloha]{\llap{$\star$ }Úloha}
\newtheorem*{bonus}{Bonus}
\newtheorem*{defn}{Definice}

\pagestyle{empty}

\let\ee\expandafter

\def\vysld{}
\let\printvysl\relax
\let\printalphvysl\relax

\makeatletter
\long\def\vyslplain#1{\ee\ee\ee\gdef\ee\ee\ee\vysld\ee\c\ee{\ee\vysld\ee\printvysl\ee{\the\c@uloha}{#1}}}
\let\vysl\vyslplain

\def\locvysl#1{\ee\gdef\ee\locvysld\ee{\locvysld\item #1}}
\let\lv\locvysl

\newenvironment{ulohav}[1][]{\begin{uloha}[#1]\gdef\locvysld{\begin{enumerate*}}}{\ee\vyslplain\ee{\locvysld\end{enumerate*}}\end{uloha}}
\def\stitem{\@noitemargtrue\@item[$\star$ \@itemlabel]}

\makeatother

\def\atr{}
\def\basic{\def\atr{\llap{\mdseries$\sun$ }\gdef\atr{}}}
\def\interest{\def\atr{\llap{$\star$ }\gdef\atr{}}}
\def\iinterest{\def\atr{\llap{$\star\star$ }\gdef\atr{}}}


\begin{document}

%\tisk
%\resultssame

\begin{uloha}
Graf znázorňuje vývoj počtu obyvatel Lotyšska a Litvy v čase. Načrtněte do téhož grafu vývoj celkové populace Lotyšska a Litvy dohromady.

\hbox{\begin{tikzpicture}\begin{axis}[
        width=\hsize,
        grid=major,
        xmin=1950, xmax=2020, xtick distance=10,% xlabel={rok},
        ymin=0, ymax=7, ytick distance=1, ylabel={populace (v mil. lidí)},
        legend pos=south east,
        /pgf/number format/.cd,
        use comma,
        1000 sep={},
        fixed]
\addplot [smooth, thick, blue] table {lotyssko.dat};
\addplot [smooth, thick, red] table {litva.dat};
\legend{Lotyšsko, Litva}
\end{axis}
\end{tikzpicture}}
\end{uloha}


\begin{uloha}
Graf znázorňuje vývoj hrubého domácího produktu na jednoho obyvatele (tzv. \emph{per capita}) Česka a Slovenska. Načrtněte do téhož grafu vývoj HDP per capita obou zemí dohromady; předpokládejte pro jednoduchost, že ČR má oproti Slovensku dvakrát více obyvatel.

\hbox{\begin{tikzpicture}\begin{axis}[
        width=\hsize,
        grid=major,
        xmin=1995, xmax=2020, xtick distance=5,% xlabel={rok},
        ymin=10000, ymax=44000, ytick distance=4000, ylabel={HDP per capita (v USD)},
        legend pos=south east,
        /pgf/number format/.cd,
        use comma,
        1000 sep={},
        fixed]
\addplot [smooth, thick, blue] table {cesko_hdp.dat};
\addplot [smooth, thick, red] table {slovensko_hdp.dat};
\legend{ČR, SR}
\end{axis}
\end{tikzpicture}}
\end{uloha}



\begin{uloha}
Graf znázorňuje vývoj průměrné teploty naměřené v Praze na Karlově během roku, a to za rok 2021 a průměr z~let 1960--2021. Načrtněte do téhož grafu graf rozdílu teploty r.~2021 oproti dlouhodobému průměru.

\hbox{\begin{tikzpicture}\begin{axis}[
        width=\hsize,
        grid=major,
        xmin=1, xmax=12, xtick distance=1,% xlabel={rok},
        ymin=-2, ymax=22, ytick distance=2, ylabel={teplota (ve $^\circ$C)},
        legend pos=north west,
        /pgf/number format/.cd,
        use comma,
        1000 sep={},
        fixed]
\addplot [thick, blue] table {teplota_2021.dat};
\addplot [thick, red] table {teplota_prum.dat};
\legend{2021, průměr}
\end{axis}
\end{tikzpicture}}
\end{uloha}


\begin{ulohav}\label{deff}
Vizte graf funkce $f$.

\hbox{\begin{tikzpicture}\begin{axis}[
        width=\hsize,
        grid=major,
        xmin=-5, xmax=5, xtick distance=1,
        ymin=-4, ymax=4, ytick distance=1,
        xlabel={$x$},ylabel={$y$},axis lines=middle,unit vector ratio=1 1]
\addplot [thick, blue, domain=-5:5, smooth] {9*x^3/200 + x^2/10 - 123*x/200 - 153/100};
\end{axis}
\end{tikzpicture}}
\begin{enumerate}
    \item Z grafu odhadněte, jaká $x$ jsou řešení rovnice $f(x) = -1$.\lv{cca $-4{,}6$, $-0{,}8$ a $3{,}2$}
    \item Kolik řešení bude mít rovnice $f(x)=0$?\lv{2}
    \item Uveďte příklad čísla $c$ takového, že rovnice $f(x) = c$ bude mít jen jedno řešení.\lv{např. $c = 1$}
    \item Ke grafu funkce $f$ načrtněte grafy funkcí $g_1$, $g_2$ a $g_3$ dané předpisy \begin{itemize*} \item $g_1(x) = f(x)+1$, \item $g_2(x) = f(x)-2$, \item $g_3(x) = -f(x)$.\end{itemize*}
\end{enumerate}
\hbox{\begin{tikzpicture}\begin{axis}[
        width=\hsize,
        grid=major,
        xmin=-5, xmax=5, xtick distance=1,
        ymin=-4, ymax=4, ytick distance=1,
        xlabel={$x$},ylabel={$y$},axis lines=middle,unit vector ratio=1 1]
\end{axis}\end{tikzpicture}}
\end{ulohav}


\begin{ulohav}
Vizte graf funkce $h$.

\hbox{\begin{tikzpicture}\begin{axis}[
        width=\hsize,
        grid=major,
        xmin=-5, xmax=5, xtick distance=1,
        ymin=-4, ymax=4, ytick distance=1,
        xlabel={$x$},ylabel={$y$},axis lines=middle,unit vector ratio=1 1]
\addplot [thick, red, domain=-5:5] coordinates  {(-5,0) (-3,2) (3,-2) (5,0)};
\end{axis}
\end{tikzpicture}}

\begin{enumerate}
\item Načrtněte grafy funkcí $i_1, \dots, i_5$, jejichž předpisy budou
    \begin{itemize*}%[noitemsep]
    \item $i_1(x) = 2\cdot h(x)$,
    \item $i_2(x) = \frac12 \cdot h(x)$,
    \item $i_3(x) = f(x) + h(x)$, 
    \item $i_4(x) = f(x) \cdot h(x)$,
    \item $i_5(x) = f(h(x))$,
    \end{itemize*}
    kde $f$ je funkce z~úlohy \ref{deff}.
\item Kolik $x$ z intervalu $\langle-5;5\rangle$ bude splňovat $f(x) = h(x)$? Které to cca budou?
\end{enumerate}
\end{ulohav}

\end{document}