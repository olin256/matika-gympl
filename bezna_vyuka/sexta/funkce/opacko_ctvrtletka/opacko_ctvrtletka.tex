\documentclass[9pt,a5paper]{extarticle}
\usepackage[margin=.8cm,bottom=10mm]{geometry}
\usepackage[utf8]{inputenc}
\usepackage[IL2]{fontenc}
\usepackage[czech]{babel}
\usepackage{microtype}
\usepackage{amssymb}
\usepackage{amsthm}
\usepackage{amsmath}
\usepackage{xcolor}
\usepackage{graphicx}
\usepackage{wasysym}
\usepackage{multicol}
\usepackage[inline]{enumitem}
\usepackage{pgfplots}


\pgfplotsset{compat=1.18}
\pgfplotsset{%
every tick label/.append style={font=\footnotesize},
grid=major,
xlabel={$x$},
ylabel={$y$},
axis lines=middle,
unit vector ratio=1 1,
ytick distance=1,
xtick distance=1}

\newcommand{\R}{\mathbb{R}}
\newcommand{\N}{\mathbb{N}}

\newcommand{\hint}[1]{{\color{gray}\footnotesize\noindent(Nápověda: #1)}}

\setlist[enumerate]{label={(\alph*)},topsep=\smallskipamount,itemsep=\smallskipamount,parsep=0pt,itemjoin={\quad}}
\setlist[itemize]{topsep=\smallskipamount,noitemsep}

\def\tisk{%
\newbox\shipouthackbox
\pdfpagewidth=2\pdfpagewidth
\let\oldshipout=\shipout
\def\shipout{\afterassignment\zdvojtmp \setbox\shipouthackbox=}%
\def\zdvojtmp{\aftergroup\zdvoj}%
\def\zdvoj{%
    \oldshipout\vbox{\hbox{%
        \copy\shipouthackbox
        \hskip\dimexpr .5\pdfpagewidth-\wd\shipouthackbox\relax
        \box\shipouthackbox
    }}%
}}%

\let\results\newpage
\let\endresults\relax

\def\resultssame{%
    \long\def\results##1\endresults{%
        \vfill
        \noindent\rotatebox{180}{\vbox{##1}}%
    }%
}


\newtheorem*{poz}{Pozorování}

\theoremstyle{definition}
\newtheorem{uloha}{\atr Úloha}
\newtheorem{suloha}[uloha]{\llap{$\star$ }Úloha}
\newtheorem*{bonus}{Bonus}
\newtheorem*{defn}{Definice}

\pagestyle{empty}

\let\ee\expandafter

\def\vysld{}
\let\printvysl\relax
\let\printalphvysl\relax

\makeatletter
\long\def\vyslplain#1{\ee\ee\ee\gdef\ee\ee\ee\vysld\ee\ee\ee{\ee\vysld\ee\printvysl\ee{\the\c@uloha}{#1}}}
\let\vysl\vyslplain

\def\locvysl#1{\ee\gdef\ee\locvysld\ee{\locvysld\item #1}}
\let\lv\locvysl

\newenvironment{ulohav}[1][]{\begin{uloha}[#1]\gdef\locvysld{\begin{enumerate*}}}{\ee\vyslplain\ee{\locvysld\end{enumerate*}}\end{uloha}}
\def\stitem{\@noitemargtrue\@item[$\star$ \@itemlabel]}

\makeatother

\def\atr{}
\def\basic{\def\atr{\llap{\mdseries$\sun$ }\gdef\atr{}}}
\def\interest{\def\atr{\llap{$\star$ }\gdef\atr{}}}
\def\iinterest{\def\atr{\llap{$\star\star$ }\gdef\atr{}}}

\def\lm#1{\lv{$\{#1\}$}}

\begin{document}

% \tisk
% \resultssame

\section*{15. Opáčko před čtvrtletkou}

\begin{multicols}{2}

\begin{ulohav}
\everymath{\displaystyle}
Nalezněte všechna řešení těchto velmi zajímavých rovnic:
\begin{enumerate}
    \item $3\cdot \frac{3^{x+2}}{9^{2x}} = \sqrt{3^{-x} \cdot 27}$
        \lm{\frac35}
    \item $4^x = 13 - 4^{x+2}$
        \lm{\log_4\frac{13}{17}}
    \item $2^{x+1} - 7\cdot\bigl(\sqrt2\bigr)^x + 3 = 0$
        \lm{-2; \log_{\sqrt2} 3}
    \item $2^x - 6 \cdot 2^{-x} = -1$
        \lm{1}
    \item $2 - \log_3 (x-2) = \log_3 (2x+3)$
        \lm{3}
    \item $\log_5 \bigl(\log_2 x\bigr) = 1$
        \lm{32}
    \item $2 \log(x+1) = 2 + \log(2x+23)$
        \lm{209}
    \item $\log _5\bigl(x^3\bigr)+2 \log _5(25 x)-\log _5\bigl(\sqrt{x}\bigr)=13$
        \lm{25}
\end{enumerate}
\end{ulohav}


\begin{uloha}
Vyjádřete $x$ pomocí $a$, $b$, $c$, $d$ (předpokládáme, že jsou všechna čísla kladná), je-li
\[ \log x = 3 \bigl( \log a - \log \sqrt b \bigr) - 2 + \tfrac12 \log c - \log(2d). \]
\vysl{$x = \frac{1}{200} a^3 b^{-\frac32} c^{\frac12} d^{-1}$}
\end{uloha}


\begin{uloha}
Nalezněte předpisy těchto (posunutých) exponenciálních funkcí:

\hbox to\hsize{\hfil\begin{tikzpicture}\begin{axis}[
        thick,
        width=1.15\hsize,
        xmin=-6, xmax=6,
        ymin=-3, ymax=5,
        legend pos=south east,
        domain=-8:8,
        unbounded coords=discard,restrict y to domain=-10:10,samples=100]
\addplot [black, densely dashed] {2^(x-3) + 1}; %f
\addplot [black, densely dotted] {(1/3)^x - 2}; %g
\addplot [red] {5^((x+1)/2)}; %h
\addplot [blue] {3-4^(x+4)}; %i
\legend{$f$, $g$, $h$, $i$}
\end{axis}\end{tikzpicture}\hfil}
\vysl{$f(x) = 2^{x-3}+1$;\quad $g(x) = \bigl(\frac13\bigr)^x - 2$;\quad $h(x) = \bigl(\sqrt5\bigr)^{x+1}$;\quad $i(x) = 3-4^{x+4}$}
\end{uloha}


\begin{ulohav}
Nalezněte předpis funkce inverzní k~funkci o předpisu
\begin{enumerate*}
    \item $y = 3^{x+3} + 7$,\lv{$y = \log_3(x-7) - 3$}
    \item $y = \log_2(2x-1) - 3$.\lv{$y = \frac12\bigl(2^{x+3}+1\bigr)$}
\end{enumerate*}
\end{ulohav}


\begin{ulohav}
Ve 12:00 se ve zkoumaném vzorku nacházelo 5000 bakterií; ve 13:00 jich tam bylo 13000. Předpokládáme, že se velikost populace bakterií zvětšuje exponenciálně.
\begin{enumerate}
    \item Odhadněte velikost populace v 14:00.\lv{33\,800}
    \item Odhadněte velikost populace v 12:30.\lv{cca 8062}
    \item Odhadněte, za jak dlouho se populace zdvojnásobí.\lv{$\log_{13/5} 2$ hodin, což je cca 43,5 minuty}
    \item Odhadněte, za jak dlouho se populace ztrojnásobí.\lv{$\log_{13/5} 3$ hodin, což je cca 69 minut}
    \item Odhadněte, v kolik hodin bude v populaci 100\,000 bakterií.\lv{nastane to za $\log_{13/5} \frac{100\,000}{5\,000}$ hodin, tedy cca v 15:08}
\end{enumerate}
\end{ulohav}


\begin{ulohav}
Připojíme-li nabitý kondenzátor ke spotřebiči se stálým odporem, bude jeho náboj klesat v čase exponenicálně (jak snadno vyplyne z~příslušné diferenciální rovnice).
Předpokládejme, že se kondenzátor vybije na polovinu své kapacity za 0,25\,s.
\begin{enumerate}
    \item Za jak dlouho se vybije na setinu své kapacity?\lv{$0{,}25 \cdot \log_2 100 \doteq 1{,}66\,\mathrm{s}$}
    \item Z kolika \% své kapacity byl nabit před jednou sekundou, jestliže je v tuto chvíli nabit ze 3\,\%?\lv{$3\,\% \cdot 2^4 = 48\,\%$}
\end{enumerate}
\end{ulohav}

\interest
\begin{ulohav}
Vyřešte v $\R$ následující ještě mnohem zajímavější rovnice či nerovnice:
\begin{enumerate}
    \item $2^x - 3^x = 2^{x-1} + 5 \cdot 3^{x-1}$
        \lm{\log_{2/3}\frac{16}3}
    \item $7 \cdot 6^x - 2 \cdot 4^x = 6 \cdot 9^x$
        \lm{-1; \log_{2/3} 2}
    \item $\bigl(\tfrac12\bigr)^{-x} - \bigl(\tfrac12\bigr)^{x-1} \geq 1$
        \lv{$\langle 1;\infty)$}
    \item $2^x - 3^x > 2^{x+1} - 3^{x+1}$
        \lv{$(\log_{2/3} 2; \infty)$}
\end{enumerate}
\end{ulohav}

\end{multicols}

\results
\parindent=0pt
\parskip=\smallskipamount
\rightskip=0pt plus1fil\relax
\def\printvysl#1#2{\textbf{#1.} #2\par}
\vysld
\endresults


\end{document}