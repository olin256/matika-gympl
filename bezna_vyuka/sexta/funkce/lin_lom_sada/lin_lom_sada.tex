\documentclass[8pt,a5paper]{extarticle}
\usepackage[margin=.5cm]{geometry}
\usepackage[utf8]{inputenc}
\usepackage[IL2]{fontenc}
\usepackage[czech]{babel}
\usepackage{microtype}
\usepackage{amssymb}
\usepackage{amsthm}
\usepackage{amsmath}
\usepackage{xcolor}
\usepackage{graphicx}
\usepackage{wasysym}
\usepackage{multicol}
\usepackage[inline]{enumitem}
\usepackage{pgfplots}

\pgfplotsset{compat=1.18}
\pgfplotsset{%
every tick label/.append style={font=\footnotesize},
grid=major,
xlabel={$x$},
ylabel={$y$},
axis lines=middle,
unit vector ratio=1 1,
xtick distance=1,
ytick distance=1}

\newcommand{\R}{\mathbb{R}}

\newcommand{\hint}[1]{{\color{gray}\footnotesize\noindent(Nápověda: #1)}}

\setlist[enumerate]{label={(\alph*)},topsep=\smallskipamount,itemsep=\smallskipamount,parsep=0pt,itemjoin={\quad}}
\setlist[itemize]{topsep=\smallskipamount,noitemsep}

\def\tisk{%
\newbox\shipouthackbox
\pdfpagewidth=2\pdfpagewidth
\let\oldshipout=\shipout
\def\shipout{\afterassignment\zdvojtmp \setbox\shipouthackbox=}%
\def\zdvojtmp{\aftergroup\zdvoj}%
\def\zdvoj{%
    \oldshipout\vbox{\hbox{%
        \copy\shipouthackbox
        \hskip\dimexpr .5\pdfpagewidth-\wd\shipouthackbox\relax
        \box\shipouthackbox
    }}%
}}%

\let\results\newpage
\let\endresults\relax

\def\resultssame{%
    \long\def\results##1\endresults{%
        \vfill
        \noindent\rotatebox{180}{\vbox{##1}}%
    }%
}


\newtheorem*{poz}{Pozorování}

\theoremstyle{definition}
\newtheorem{uloha}{\atr Úloha}
\newtheorem{suloha}[uloha]{\llap{$\star$ }Úloha}
\newtheorem*{bonus}{Bonus}
\newtheorem*{defn}{Definice}

\pagestyle{empty}

\let\ee\expandafter

\def\vysld{}
\let\printvysl\relax
\let\printalphvysl\relax

\makeatletter
\long\def\vyslplain#1{\ee\ee\ee\gdef\ee\ee\ee\vysld\ee\ee\ee{\ee\vysld\ee\printvysl\ee{\the\c@uloha}{#1}}}
\let\vysl\vyslplain

\def\locvysl#1{\ee\gdef\ee\locvysld\ee{\locvysld\item #1}}
\let\lv\locvysl

\newenvironment{ulohav}[1][]{\begin{uloha}[#1]\gdef\locvysld{\begin{enumerate*}}}{\ee\vyslplain\ee{\locvysld\end{enumerate*}}\end{uloha}}
\def\stitem{\@noitemargtrue\@item[$\star$ \@itemlabel]}

\makeatother

\def\atr{}
\def\basic{\def\atr{\llap{\mdseries$\sun$ }\gdef\atr{}}}
\def\interest{\def\atr{\llap{$\star$ }\gdef\atr{}}}
\def\iinterest{\def\atr{\llap{$\star\star$ }\gdef\atr{}}}

\begin{document}

%\tisk
%\resultssame

\section*{5. Lineární lomené funkce}

\begin{ulohav}
\everymath{\displaystyle}%
Upravte následující předpisy lineárních lomených funkcí do středového tvaru:
\begin{enumerate*}
\item $y = \frac{2x+1}{x+1}$\lv{$y = -\frac{1}{x+1}+2$}
\item $y = \frac{1+x}{1-x}$\lv{$y=-\frac{2}{x-1}-1$}
\item $y = \frac{2x+1}{3x+1}$\lv{$y = \frac{\frac19}{x+\frac13} + \frac23$}
\item $y = \frac{-x+2}{2x-3}$.\lv{$y = \frac{\frac14}{x-\frac32}-\frac12$}
\end{enumerate*}
\end{ulohav}

\begin{uloha}\label{grafy}
Určete předpisy lineárních lomených funkcí $f$, $g$, $h$, $i$, jejichž grafy jsou níže.

\hbox to\hsize{\hfil\begin{tikzpicture}\begin{axis}[
        width=.8\hsize,
        xmin=-8, xmax=8,
        ymin=-5, ymax=5,
        legend pos=south east,
        extra x ticks={5/2},
        extra x tick labels={$\frac52$},
        unbounded coords=discard,restrict y to domain=-15:15]
\addplot [thick, blue, domain=-8:8, samples=201] {1/(x+6)-1};
\addplot [thick, red, domain=-8:8, samples=201] {-2/(x+3)+2};
\addplot [thick, black, densely dashed, domain=-8:8, samples=251] {1/(2*(x-1))};
\addplot [thick, black, domain=-8:8, samples=251] {-1/(x-5/2)-3};
\legend{$f$, $g$, $h$, $i$}
\end{axis}\end{tikzpicture}\hfil}
\vysl{$f(x) = \frac{1}{x+6}-1$,\quad $g(x) = -\frac{2}{x+3}+2$,\quad $h(x) = \frac{1}{2(x-1)}$,\quad $i(x) = -\frac{1}{x-\frac52}-3$}
\end{uloha}

\begin{uloha}
U funkcí z úlohy \ref{grafy} určete definiční obory a obory hodnot.
\vysl{$D(f) = \R\setminus\{-6\}$,~$H(f) = \R\setminus\{-1\}$,\quad
$D(g) = \R\setminus\{-3\}$,~$H(g) = \R\setminus\{2\}$,\quad
$D(h) = \R\setminus\{1\}$,~$H(h) = \R\setminus\{0\}$,\quad
$D(i) = \R\setminus\{\frac52\}$,~$H(i) = \R\setminus\{-3\}$}
\end{uloha}


\begin{uloha}
K funkcím z úlohy \ref{grafy} určete předpisy inverzních funkcí. (Možná si při tom něčeho všimnete, čímž se to stane početně velmi jednoduchým...)
\vysl{$f^{-1}(x) = \frac{1}{x+1}-6$,\quad $g^{-1}(x) = -\frac{2}{x-2}-3$,\quad $h^{-1}(x) = \frac1{2x}+1$,\quad $i^{-1}(x) = -\frac1{x+3}+\frac52$}
\end{uloha}

\begin{uloha}
K funkcím z úlohy \ref{grafy} určete souřadnice průsečíků s osami soustavy souřadnic.
\vysl{%
$f$: $P_y[0;-\frac56]$, $P_x[-5;0]$;\quad
$g$: $P_y[0;\frac43]$, $P_x[-2;0]$;\quad
$h$: $P_y[0;-\frac12]$, $P_x$ neexistuje;\quad
$i$: $P_y[0;-\frac{13}5]$, $P_x[\frac{13}6;0]$}
\end{uloha}


\begin{uloha}
\everymath{\displaystyle}%
Načrtněte grafy následujících funkcí; výsledky si můžete zkontrolovat v GeoGebře nebo Photomathu atd. (nebo se mě zeptejte).
\begin{enumerate*}
    \item $y = \frac{1}{x-1}$
    \item $y = \frac{x}{x+2}$
    \item $y = \frac{2x-1}{2x+1}$
    \item $y = \frac{1}{2|x|+1}-1$
    \item $y = \left|\frac{1}{2x+1}-1\right|$
    \item $y = \frac{1}{x + |x| + 1} - 1$
    \item $y = \frac{x}{|x| + 1}$.
\end{enumerate*}
\end{uloha}

\begin{ulohav}
Nalezněte předpis lineární lomené funkce, jejíž graf prochází body
\begin{enumerate*}
\item $[0;-1]$, $[1;1]$, $[2;2]$\lv{$y = -\frac{12}{x+2}+5$}
\item $[0;0]$, $[1;1]$, $[2;2]$.\lv{neexistuje}
\end{enumerate*}
\end{ulohav}


\begin{ulohav}
Rozhodněte následující:
\begin{enumerate}
    \item Jaké mají lineární lomené funkce extrémy?\lv{žádné}
    \item Mohou být lineární lomené funkce liché? Sudé? Prosté? Shora/zdola omezené?\lv{liché ano (např. $y=\frac1x$), sudé nikdy, prosté vždy, shora ani zdola omezené nejsou nikdy}
\end{enumerate}
\end{ulohav}

\interest
\begin{uloha}
Funkce $y = \frac1x$ je příkladem lineární lomené funkce, která je sama sobě inverzní funkcí. Nalezněte příklad(y) dalších takových lineárních lomených funkcí, případně je nalezněte \emph{všechny}.\vysl{např $y = \frac1{x-1}+1$; obecně jde o funkce tvaru $y = \frac{c}{x-d}+d$, kde $c \neq 0$}
\end{uloha}





\baselineskip=1.25\baselineskip
\setlist[enumerate]{label=\textbf{(\alph*)},itemjoin={\quad}}

\results
\parindent=0pt
\parskip=\smallskipamount
\rightskip=0pt plus1fil\relax
\def\printvysl#1#2{\textbf{#1.} #2\par}
\vysld
\endresults


\end{document}