\documentclass[10pt,a5paper]{extarticle}
\usepackage[margin=.5cm]{geometry}
\usepackage[utf8]{inputenc}
\usepackage[IL2]{fontenc}
\usepackage[czech]{babel}
\usepackage{microtype}
\usepackage{amssymb}
\usepackage{amsthm}
\usepackage{amsmath}
\usepackage{xcolor}
\usepackage{graphicx}
\usepackage{wasysym}
\usepackage{multicol}
\usepackage[inline]{enumitem}

\multicolsep=\smallskipamount

\newcommand{\R}{\mathbb{R}}
\newcommand{\N}{\mathbb{N}}

\newcommand{\hint}[1]{{\color{gray}\footnotesize\noindent(Nápověda: #1)}}

\setlist[enumerate]{label={(\alph*)},topsep=\smallskipamount,itemsep=\smallskipamount,parsep=0pt,itemjoin={\quad}}
\setlist[itemize]{topsep=\smallskipamount,noitemsep}

\def\tisk{%
\newbox\shipouthackbox
\pdfpagewidth=2\pdfpagewidth
\let\oldshipout=\shipout
\def\shipout{\afterassignment\zdvojtmp \setbox\shipouthackbox=}%
\def\zdvojtmp{\aftergroup\zdvoj}%
\def\zdvoj{%
    \oldshipout\vbox{\hbox{%
        \copy\shipouthackbox
        \hskip\dimexpr .5\pdfpagewidth-\wd\shipouthackbox\relax
        \box\shipouthackbox
    }}%
}}%

\let\results\newpage
\let\endresults\relax

\def\resultssame{%
    \long\def\results##1\endresults{%
        \vfill
        \noindent\rotatebox{180}{\vbox{##1}}%
    }%
}


\newtheorem*{poz}{Pozorování}

\theoremstyle{definition}
\newtheorem{uloha}{\atr Úloha}
\newtheorem{suloha}[uloha]{\llap{$\star$ }Úloha}
\newtheorem*{bonus}{Bonus}
\newtheorem*{defn}{Definice}

\pagestyle{empty}

\let\ee\expandafter

\def\vysld{}
\let\printvysl\relax
\let\printalphvysl\relax

\makeatletter
\long\def\vyslplain#1{\ee\ee\ee\gdef\ee\ee\ee\vysld\ee\ee\ee{\ee\vysld\ee\printvysl\ee{\the\c@uloha}{#1}}}
\let\vysl\vyslplain

\def\locvysl#1{\ee\gdef\ee\locvysld\ee{\locvysld\item #1}}
\let\lv\locvysl

\newenvironment{ulohav}[1][]{\begin{uloha}[#1]\gdef\locvysld{\begin{enumerate*}}}{\ee\vyslplain\ee{\locvysld\end{enumerate*}}\end{uloha}}
\def\stitem{\@noitemargtrue\@item[$\star$ \@itemlabel]}

\makeatother

\def\atr{}
\def\basic{\def\atr{\llap{\mdseries$\sun$ }\gdef\atr{}}}
\def\interest{\def\atr{\llap{$\star$ }\gdef\atr{}}}
\def\iinterest{\def\atr{\llap{$\star\star$ }\gdef\atr{}}}


\begin{document}


% \tisk
% \resultssame

\section*{7. Mocniny s racionálními exponenty}


\begin{uloha}
Je-li $a$ kladné reálné číslo, vyjádřete součin $\sqrt[3]{a} \cdot \sqrt[4]{a^3} \cdot\sqrt[6]{a^3}$ ve tvaru jediné mocniny $a$ (exponent nemusí být celý) a také pomocí odmocniny (bez použití neceločíselného exponentu).\vysl{$a^{\frac{19}{12}} = \sqrt[12]{a^{19}}$}
\end{uloha}

\begin{ulohav}
Vypočtěte následující hodnoty (ideálně s určitým zapojením vlastního intelektu, jinak souhlasím s tím, že to zvládne každá rozumná kalkulačka):
\begin{enumerate*}
    \item $\bigl(\frac{8}{125}\bigr)^{\frac13}$\lv{$\frac{2}{5}$}
    \item $243^{\frac15}$\lv{3}
    \item $1000^{-\frac23}$\lv{$\frac1{100}$}
    \item $1^{-\frac54}$\lv{$1$}
    \item $-1^{\frac54}$\lv{$-1$}
    \item $8 \cdot \bigl(\frac14\bigr)^{\frac12}$\lv{4}
    \item $16^{1{,75}}$\lv{$2^7 = 128$}
\item $5{,}6^{\frac12} : \bigl(\frac75\bigr)^{\frac12}$\lv{2}
\end{enumerate*}
\end{ulohav}


\begin{ulohav}% pet 8.7 51
Vyjádřete jako jedinou mocninu dvojky:
\begin{enumerate*}
    \item $2^{\frac14} \cdot 2^{\frac23}$\lv{$2^{\frac{11}{12}}$}
    \item $2^{\frac34} : 2^{\frac12}$\lv{$2^{\frac{1}{4}}$}
    \item $\biggl(\bigl(2^{\frac12}\bigr)^{\frac14}\biggr)^{\frac34}$\lv{$2^{\frac{3}{32}}$}
    \item $\bigl(2 \cdot 2^{\frac12}\bigr)^{\frac12} : 2^{\frac78}$\lv{$2^{-\frac{1}{8}}$}
    \item $2^{\frac12} \cdot 4^{\frac14} \cdot 8^{\frac18} \cdot 16^{\frac{10}{16}}$\lv{$2^{\frac{31}{8}}$}
    \item $\bigl(2^{\frac13}\bigr)^4\cdot 2^{(\frac13)^4}$\lv{$2^{\frac{109}{81}}$}
\end{enumerate*}
\end{ulohav}

\begin{uloha}
Určete hodnotu výrazu $\bigl[a^{-\frac32} \cdot b(ab^{-2})^{-\frac12} \cdot(a^{-1})^{-\frac23}\bigr]^3$ pro $a = \dfrac{\sqrt2}{2}$ a $b = \dfrac{1}{\sqrt[3]2}$. (Doporučuji nejprve výraz zjednodušit.)\vysl{1}
\end{uloha}

\begin{ulohav}
\everymath{\displaystyle}
Zjednodušte tyto výrazy:

\medskip
\noindent
\begin{enumerate*}
    \item $\frac{\bigl(y^{\frac12}\bigr)^3 \cdot (y^2)^{\frac13}}{y \cdot y^{\frac23}}$\lv{$y^{\frac12}$}
    \item $\frac{(xy)^{\frac12}\cdot(x^2 y)^{-\frac13}}{(xy^2)^{-\frac23}}$\lv{$x^{\frac12}y^{\frac32}$}
    \item $\left(\frac{x^{\frac25}}{y^{\frac32}}\right)^{-2}\cdot \frac{(y^{-1}x^{-2})^{-\frac12}}{(xy^2)^{\frac1{10}}}$\lv{$x^{\frac1{10}}y^{\frac{33}{10}}$}
\end{enumerate*}
\end{ulohav}


\begin{ulohav} % Petáková
\everymath{\displaystyle}
Zjednodušte (užitím racionálních exponentů):
\begin{enumerate*}
    \item $\sqrt{3 \cdot \sqrt3}$\lv{$3^{\frac34}$}
    \item $\sqrt{8 \cdot \sqrt{4 \cdot \sqrt2}}$\lv{$2^{\frac{17}8}$}
    \item $\sqrt[3]{5^2 \cdot \sqrt5}$\lv{$5^{\frac56}$}
    \item $\sqrt{5 \cdot \sqrt[3]{\tfrac15}\cdot \sqrt[4]5}$\lv{$5^{\frac{11}{24}}$}
    \item $\frac{\sqrt[3]{2 \cdot \sqrt8}}{\sqrt{2 \cdot \sqrt[3]4}}$\lv{$1$}
    \item $\frac{\sqrt[5]{u\cdot \sqrt[6]{u^2}}}{\sqrt u}$\lv{$u^{-\frac7{30}}$}
    \item $\sqrt[6]{\frac{b^4}{\sqrt b}} \cdot \sqrt[3]{\frac{b^3}{\sqrt b}} \cdot \sqrt b$\lv{$b^{\frac{23}{12}}$}
\end{enumerate*}
\end{ulohav}


\begin{ulohav}
Řešte následující rovnice s neznámou $x \in \mathbb R$:
\begin{multicols}{3}
\begin{enumerate}
    \item $x^{\frac13} = 0{,}4$\lv{$\frac{8}{125} = 0{,}064$}
    \item $\sqrt{x^3} = 8$\lv{$4$}
    \item $x^{\frac23} = 0{,}25$\lv{$2^{-3} = 0{,}125$}
    \item $\sqrt[5]{\sqrt[3]{x^5}} = 3$\lv{$3^3 = 27$}
    \item $64x^6 - 1 = 0$\lv{$\pm \frac12$}
    \item $x^{-\frac13} = 4 x^{\frac 14}$\lv{$2^{-\frac{24}7}$}
\end{enumerate}
\end{multicols}
\end{ulohav}

\interest
\begin{uloha}
(Žádná zajímavá úloha na racionální exponenty mě nenapadla, tak sem dám tuto.)
Matematik Pěnkava se jednoho dne rozhodl, že bude zkoumat čísla tvaru $2^{2^n} - 1$, kde $n$ je přirozené číslo. Po dosazení jedničky mu vyšlo $2^{2^1} - 1 = 2^2 - 1 = 3$, z čehož usoudil, že výsledek bude asi vždycky prvočíslo. Vaším úkolem je dokázat, že ze všech Pěnkavových čísel je trojka ve skutečnosti jako jediná prvočíslo.
\end{uloha}


%\renewcommand{\baselinestretch}{1.25}
\baselineskip=1.25\baselineskip
\setlist[enumerate]{label=\textbf{(\alph*)},itemjoin={\quad}}


\results
%\subsubsection*{Výsledky}
\parindent=0pt
\parskip=\smallskipamount
\rightskip=0pt plus1fil\relax
\def\printvysl#1#2{\textbf{#1.} #2\par}
\def\printalphvysl#1#2#3{\textbf{#1}(#2)\ #3\par}
\vysld
\endresults
%}

%\copy0

%\vfil

%\box0
%\eject
%\unvbox0



\end{document}




