\documentclass[12pt,a5paper]{extarticle}
\usepackage[margin=1cm]{geometry}
\usepackage[utf8]{inputenc}
\usepackage[IL2]{fontenc}
\usepackage[czech]{babel}
\usepackage{microtype}
\usepackage{amssymb}
\usepackage{amsthm}
\usepackage{amsmath}
\usepackage{graphicx}
\usepackage{nccmath}

\usepackage[inline]{enumitem}

\newcommand{\R}{\mathbb{R}}

\setlist[enumerate]{label={(\alph*)},topsep=\smallskipamount,noitemsep}

\def\tisk{%
\newbox\shipouthackbox
\pdfpagewidth=2\pdfpagewidth
\let\oldshipout=\shipout
\def\shipout{\afterassignment\zdvojtmp \setbox\shipouthackbox=}%
\def\zdvojtmp{\aftergroup\zdvoj}%
\def\zdvoj{%
    \oldshipout\vbox{\hbox{%
        \copy\shipouthackbox
        \hskip\dimexpr .5\pdfpagewidth-\wd\shipouthackbox\relax
        \box\shipouthackbox
    }}%
}}%


\theoremstyle{definition}
\newtheorem{uloha}{Úloha}

\pagestyle{empty}

\begin{document}

% \tisk

\begin{uloha}[2 body]
V závislosti na parametru $p \in \R$ řešte rovnici
\[ 2 x p + p (1 - x) = 3 p - 4 + 2 x. \]
\end{uloha}


\begin{uloha}[3 body]
V závislosti na parametru $k \in \R$ řešte rovnici
\[ \frac{k^2(x-1)}{kx - 2} = 2. \]
\end{uloha}


\begin{uloha}[3 body]
V závislosti na parametru $p \in \R$ řešte soustavu rovnic
\useshortskip
\begin{alignat*}{2}
3x + 2y &= 6 \\
px + 4y &= 2p.
\end{alignat*}
\end{uloha}


\begin{uloha}[2 body]
Nalezněte všechny hodnoty parametru $m \in \R$ takové že rovnice
\[ mx^2 + 2mx + m = x + 2 \]
má právě jedno reálné řešení, a toto řešení nalezněte.
\end{uloha}


\begin{uloha}
Je dána rovnice
\[ x + m x + 3 m x^2 + 1 = 5 x^2 \]
s parametrem $m \in \R$.
\begin{enumerate}
    \item V závislosti na parametru $m$ určete, kolik má rovnice reálných řešení. (3~body)
    \item Nalezněte všechny hodnoty $m$, pro které je $x = 1$ řešením rovnice.  (1~bod)
    \item Nalezněte všechny hodnoty $m$, pro které je součet kořenů rovnice roven $3$. (2~body)
\end{enumerate}
\end{uloha}



\end{document}