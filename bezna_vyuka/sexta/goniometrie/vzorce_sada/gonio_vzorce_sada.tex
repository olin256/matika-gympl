\documentclass[10pt,a4paper]{extarticle}
\usepackage[margin=.8cm,bottom=10mm]{geometry}
\usepackage[utf8]{inputenc}
\usepackage[IL2]{fontenc}
\usepackage[czech]{babel}
\usepackage{microtype}
\usepackage{amssymb}
\usepackage{amsthm}
\usepackage{amsmath}
\usepackage{xcolor}
\usepackage{graphicx}
\usepackage{wasysym}
\usepackage{multicol}
\usepackage[inline]{enumitem}

\multicolsep=\smallskipamount

\newcommand{\R}{\mathbb{R}}
\newcommand{\N}{\mathbb{N}}
\newcommand{\Z}{\mathbb{Z}}

\newcommand{\hint}[1]{{\color{gray}\footnotesize\noindent(Nápověda: #1)}}

\DeclareMathOperator{\tg}{tg}
\DeclareMathOperator{\cotg}{cotg}

\setlist[enumerate]{label={(\alph*)},topsep=\smallskipamount,itemsep=\smallskipamount,parsep=0pt,itemjoin={\quad}}
\setlist[itemize]{topsep=\smallskipamount,noitemsep}

\def\tisk{%
\newbox\shipouthackbox
\pdfpagewidth=2\pdfpagewidth
\let\oldshipout=\shipout
\def\shipout{\afterassignment\zdvojtmp \setbox\shipouthackbox=}%
\def\zdvojtmp{\aftergroup\zdvoj}%
\def\zdvoj{%
    \oldshipout\vbox{\hbox{%
        \copy\shipouthackbox
        \hskip\dimexpr .5\pdfpagewidth-\wd\shipouthackbox\relax
        \box\shipouthackbox
    }}%
}}%

\let\results\newpage
\let\endresults\relax

\def\resultssame{%
    \long\def\results##1\endresults{%
        \vfill
        \noindent\rotatebox{180}{\vbox{##1}}%
    }%
}


\newtheorem*{poz}{Pozorování}

\theoremstyle{definition}
\newtheorem{uloha}{\atr Úloha}
\newtheorem{suloha}[uloha]{\llap{$\star$ }Úloha}
\newtheorem*{bonus}{Bonus}
\newtheorem*{defn}{Definice}

\pagestyle{empty}

\DeclareMathOperator{\arctg}{arctg}

\let\ee\expandafter

\def\vysld{}
\let\printvysl\relax
\let\printalphvysl\relax

\makeatletter
\long\def\vyslplain#1{\ee\ee\ee\gdef\ee\ee\ee\vysld\ee\ee\ee{\ee\vysld\ee\printvysl\ee{\the\c@uloha}{#1}}}
\let\vysl\vyslplain

\def\locvysl#1{\ee\gdef\ee\locvysld\ee{\locvysld\item #1}}
\let\lv\locvysl

\newenvironment{ulohav}[1][]{\begin{uloha}[#1]\gdef\locvysld{\begin{enumerate*}}}{\ee\vyslplain\ee{\locvysld\end{enumerate*}}\end{uloha}}
\def\stitem{\@noitemargtrue\@item[$\star$ \@itemlabel]}

\makeatother

\def\atr{}
\def\basic{\def\atr{\llap{\mdseries$\sun$ }\gdef\atr{}}}
\def\sinterest{\def\atr{\llap{$(\star)$ }\gdef\atr{}}}
\def\interest{\def\atr{\llap{$\star$ }\gdef\atr{}}}
\def\iinterest{\def\atr{\llap{$\star\star$ }\gdef\atr{}}}

\catcode`<=\active
\catcode`>=\active
\def<{\left\langle}
\def>{\right\rangle}


\def\st{^\circ}

\begin{document}

% \tisk
% \resultssame


\section*{17. Goniometrické vzorce}

\begin{ulohav}
Aniž určíte hodnotu $x$,
určete hodnoty zbývajících goniometrických funkcí z množiny $\{\sin, \cos, \tg, \cotg\}$, víte-li, že
\begin{multicols}{3}
\begin{enumerate}
    \item $\cos x = \frac45$ a $x \in <0; \pi>$
        \lv{$\sin x = \frac35$, $\tg x = \frac34$, $\cotg x = \frac43$}
    \item $\cos x = \frac45$ a $x \in <-\pi; 0>$
        \lv{$\sin x = -\frac35$, $\tg x = -\frac34$, $\cotg x = -\frac43$}
    \item $\sin x = -\frac27$ a $x \in <\frac\pi2; \frac{3\pi}2>$
        \lv{$\cos x = -\frac{3\sqrt5}{7}$, $\tg x = \frac{2\sqrt5}{15}$, $\cotg x = \frac{3\sqrt5}{2}$}
    \item $\sin x = \frac{12}{13}$ a $x \in <-\frac\pi2; \frac\pi2>$
        \lv{$\cos x = \frac{5}{13}$, $\tg x = \frac{12}{5}$, $\cotg x = \frac5{12}$}
    \item $\tg x = \frac12$ a $x \in <0;\pi>$
        \lv{$\cotg x = 2$, $\sin x = \frac{\sqrt5}5$, $\cos x = \frac{2\sqrt5}5$}
    \item $\tg x = \frac12$ a $x \in <\pi;2\pi>$
        \lv{$\cotg x = 2$, $\sin x = -\frac{\sqrt5}5$, $\cos x = -\frac{2\sqrt5}5$}
    \item $\tg x = -\sqrt7$ a $x \in <\frac{\pi}2; \frac{3\pi}2>$
        \lv{$\cotg x = -\frac{\sqrt7}{7}$, $\sin x = \frac{\sqrt{14}}{4}$, $\cos x = -\frac{\sqrt2}{4}$}
    \item $\cotg x = 10$ a $x \in <0;\pi>$
        \lv{$\tg x = \frac1{10}$, $\sin x = \frac{\sqrt{101}}{101}$, $\cos x = \frac{10\sqrt{101}}{101}$}
    \item $\cotg x = \frac23$ a $x \in <\pi; 2\pi>$
        \lv{$\tg x = \frac32$, $\sin x = -\frac{3\sqrt{13}}{13}$, $\cos x = -\frac{2\sqrt{13}}{13}$}
\end{enumerate}
\end{multicols}
\end{ulohav}



\begin{ulohav}\label{dvanactiny}
Určete přesně následující hodnoty:
\begin{enumerate*}
    \item $\cos \frac{5\pi}{12}$ \lv{$\frac{\sqrt2}{4}(\sqrt3-1)$}
    \item $\sin \frac{5\pi}{12}$ \lv{$\frac{\sqrt2}{4}(\sqrt3+1)$}
    \item $\tg \frac{5\pi}{12}$ \lv{$\frac{\sqrt3+1}{\sqrt3-1} = \sqrt3 + 2$}
    \item $\cos \frac{\pi}{12}$ \lv{$\frac{\sqrt2}{4}(\sqrt3+1)$}
    \item $\sin \frac{\pi}{12}$ \lv{$\frac{\sqrt2}{4}(\sqrt3-1)$}
    \item $\tg \frac{\pi}{12}$ \lv{$\frac{\sqrt3-1}{\sqrt3+1} = 2-\sqrt3$}
\end{enumerate*}
\hint{$\frac5{12} = \frac14 + \frac16$, $\frac1{12} = \frac14 - \frac16$.}
\end{ulohav}


\begin{uloha}
Ověřte pomocí součtových vzorců platnost následujících vzorců (jejichž platnost je ale \uv{zřejmá} z pouhé úvahy o jednotkové kružnici):
\begin{enumerate*}
    \item $\sin (x+\pi) = -\sin x$
    \item $\sin \bigl(x + \frac\pi2\bigr) = \cos x$
    \item $\cos \bigl(\frac\pi2 - x) = \sin x$
    \item $\sin \bigl(\frac\pi2 - x) = \cos x$.
\end{enumerate*}
\end{uloha}


\begin{ulohav}
Dokažte, že pro všechna $x \in \R$ platí
\begin{multicols}{2}
\begin{enumerate}
    \item $\sin x + \sin \bigl(x+\frac{2\pi}3\bigr) + \sin \bigl(x+\frac{4\pi}3\bigr) = 0$,
        \lv{$\sin\bigl(x+\frac{2\pi}3\bigr) = \sin x \cos \frac{2\pi}3 + \cos x \sin \frac{2\pi}3 = -\frac12 \sin x + \frac{\sqrt3}{2} \cos x$, dále $\sin\bigl(x+\frac{4\pi}3\bigr) = \sin x \cos \frac{4\pi}3 + \cos x \sin \frac{4\pi}3 = -\frac12 \sin x - \frac{\sqrt3}{2} \cos x$, všechno se to sečte na $0$.}
    \item $\cos x + \cos \bigl(x+\frac{2\pi}3\bigr) + \cos \bigl(x+\frac{4\pi}3\bigr) = 0$.
        \lv{$\cos \bigl(x+\frac{2\pi}3\bigr) = \cos x \cos \frac{2\pi}3 - \sin x \sin \frac{2\pi}3 = -\frac12 \cos x - \frac{\sqrt3}{2} \sin x$, dále $\cos \bigl(x+\frac{4\pi}3\bigr) = \cos x \cos \frac{4\pi}3 - \sin x \sin \frac{4\pi}3 = -\frac12 \cos x + \frac{\sqrt3}{2} \sin x$, všechno se to sečte na $0$.}
\end{enumerate}
\end{multicols}
\noindent
$\star$ Najdete \emph{geometrický} důvod pro tyto rovnosti? \hint{Uvažte tři body na jednotkové kružnici.}
\end{ulohav}

% \interest
\begin{ulohav}\label{vzorce}
Dokažte následující \uv{tabulkové} vzorce:
\begin{enumerate}
    \item $\sin 2x = 2 \sin x \cos x$\quad \hint{Do součtových vzorců dosaďte $x$ za $y$.}
        \lv{$\sin2x = \sin(x+x) = \sin x \cos x + \cos x \sin x = 2 \sin x \cos x$}
    \item $\cos 2x = \cos^2 x - \sin^2 x = 2 \cos^2 x - 1 = 1 - 2 \sin^2 x$\label{cos2x}
        \lv{$\cos 2x = \cos(x+x) = \cos x \cos x - \sin x \sin x = \cos^2 x - \sin^2 x$, další dva vztahy pomocí $\sin^2 x + \cos^2 x = 1$}
    \item $\sin^2 x = \frac12(1-\cos 2x)$\quad \hint{vyjádřete z \ref{cos2x}}\label{sinsquare}
    \item $\cos^2 x = \frac12(1+\cos 2x)$
    \item $\bigl|\sin \frac x2\bigr| = \sqrt{\frac{1-\cos x}{2}}$ \label{sinhalf}\quad \hint{dosaďte v \ref{sinsquare} $\frac x2$ za $x$ a odmocněte}
    \item $\bigl|\cos \frac x2\bigr| = \sqrt{\frac{1+\cos x}{2}}$\label{coshalf}
    \item $\tg \frac x2 = \frac{\sin x}{1 \textcolor{red}{+} \cos x} = \frac{1 \textcolor{red}{-} \cos x}{\sin x}$; první rovnost platí pro $x \neq 2k \pi$, druhá pro $x \neq k\pi$ ($k \in \Z$)\quad \hint{Místo pro $\frac x2$ a $x$ to raději dokazujte pro $x$ a $2x$.}
\end{enumerate}
\end{ulohav}


\begin{uloha}
Pomocí vzorců \ref{sinhalf} a \ref{coshalf} z Úlohy \ref{vzorce} spočtěte hodnotu $\cos \frac\pi{12}$ a $\sin \frac\pi{12}$. Je to ten samý výsledek jako v Úloze \ref{dvanactiny}?\vysl{Vyjde $\cos \frac\pi{12} = \frac12\sqrt{2 + \sqrt3}$ a $\sin \frac\pi{12} = \frac12\sqrt{2 - \sqrt3}$; že to je stejný výsledek se ověří umocněním na druhou.}
\end{uloha}



\begin{ulohav}
Aniž určíte hodnotu $x$, určete hodnoty $\sin 2x$, $\cos 2x$, $\sin \frac x2$ a $\cos \frac x2$, jestliže platí níže uvedené. Jak se bude určovat znaménko u $\sin \frac x2$ a $\cos \frac x2$?
\begin{enumerate*}
    \item $\cos x = \frac45$ a $x \in <0; \pi>$
        % \lv{$\sin x = \frac35$, $\tg x = \frac34$, $\cotg x = \frac43$}
        \lv{$\sin 2x = \frac{24}{25}$, $\cos 2x = \frac{7}{25}$, $\sin \frac x2 = \frac{\sqrt{10}}{10}$, $\cos \frac x2 = \frac{3\sqrt{10}}{10}$}
    \item $\cos x = \frac45$ a $x \in <-\pi; 0>$
        % \lv{$\sin x = -\frac35$, $\tg x = -\frac34$, $\cotg x = -\frac43$}
        \lv{$\sin 2x = -\frac{24}{25}$, $\cos 2x = \frac{7}{25}$, $\sin \frac x2 = -\frac{\sqrt{10}}{10}$, $\cos \frac x2 = \frac{3\sqrt{10}}{10}$}
    \item $\sin x = -\frac27$ a $x \in <\frac\pi2; \frac{3\pi}2>$
        % \lv{$\cos x = -\frac{3\sqrt5}{7}$, $\tg x = \frac{2\sqrt5}{15}$, $\cotg x = \frac{3\sqrt5}{2}$}
        \lv{$\sin 2x = \frac{12\sqrt5}{49}$, $\cos 2x = \frac{41}{\textcolor{red}{49}}$, $\sin \frac x2 = \sqrt{\frac{7-3\sqrt5}{14}}$, $\cos \frac x2 = -\sqrt{\frac{7+3\sqrt5}{14}}$}
    \item $\sin x = -\frac{12}{13}$ a $x \in <-\frac\pi2; \frac\pi2>$.
        % \lv{$\cos x = \frac{5}{13}$, $\tg x = \frac{12}{5}$, $\cotg x = \frac5{12}$}
        \lv{$\sin 2x = -120\frac{169}{}$, $\cos 2x = -\frac{119}{169}$, $\sin \frac x2 = -\frac{2\sqrt{13}}{13}$, $\cos \frac x2 = \frac{3\sqrt{13}}{13}$}
\end{enumerate*}
\end{ulohav}


\begin{ulohav}
Aniž určíte hodnoty $x$ a $y$, vypočítejte $\sin(x+y)$ a $\cos(x-y)$, jestliže
\begin{enumerate*}
    \item $\cos x = \frac57$, $\sin y = \frac15$, $x \in <0;\frac\pi2>$ a $y \in <\frac\pi2; \pi>$,
        \lv{$\sin(x+y) = -\frac{19}{35}$, $\cos(x-y) = -\frac{8\sqrt6}{35}$}
    \item $\tg x = 3$, $\cotg y = -2$, $x \in <\pi; \frac{3\pi}2>$ a $y \in <\frac\pi2; \pi>$.
        \lv{$\sin(x+y) = \frac{\sqrt2}2$, $\cos(x-y) = -\frac{\sqrt2}{10}$}
\end{enumerate*}
\end{ulohav}


\begin{ulohav}
Vyjádřete uvedený výraz ve tvaru $a \sin (x + b)$, kde $a$ a $b$ jsou nějaká vhodná reálná čísla.
\begin{multicols}{3}
\begin{enumerate}
    \item $\frac12 \sin x - \frac{\sqrt3}{2} \cos x$
        \lv{$\sin\bigl(x - \frac\pi3\bigr)$}
    \item $\cos x - \sqrt 3 \sin x$
        \lv{$2 \sin \bigl(x + \frac{5\pi}{6}\bigr)$}
    \item $\sin x + \cos x$
        \lv{$\sqrt 2 \sin \bigl(x + \frac{\pi}{4}\bigr)$}
    \item $\sin x + \sin \bigl(x + \frac{\pi}{3}\bigr)$
        \lv{$\sqrt3 \sin\bigl(x + \frac{\pi}{6}\bigr)$}
    \item $\sin x + 2 \cos x$
        \lv{$\sqrt 5 \sin \bigl(x + \arcsin\frac{2\sqrt5}5 \bigr)$}
    \item $3 \sin x - 4 \cos x$\lv{$5 \sin \bigl(x - \arcsin\frac45 \bigr)$}
\end{enumerate}
\end{multicols}
\end{ulohav}


% \begin{uloha}
% Zjednodušte následující výrazy a určete podmínky, za kterých mají smysl:
% \begin{multicols}{3}
% \begin{enumerate}
    % \item $(1-\cos x)(1+\cos x)$
% \end{enumerate}
% \end{multicols}
% \end{uloha}

% \interest
% \begin{enumerate}
% Na zamyšlení: Jestliže známe hodnotu $\sin x$ nebo $\cos x$, tak hodnotu $\sin 2x$ už známe až na znaménko, zatímco hodnotu $\cos 2x$ známe přesně. Jaký je \uv{geometrický} důvod pro 
% \end{enumerate}


% \pgfplotsset{every tick label/.append style={font=\tiny}}
% \resgraphwidth=5cm
\setlist[enumerate]{label={{\bfseries(\alph*)}},itemjoin={\quad}}

\results
%\subsubsection*{Výsledky}
\parindent=0pt
\parskip=\smallskipamount
\rightskip=0pt plus1fil\relax
\def\printvysl#1#2{\textbf{#1.} #2\par}
\def\printalphvysl#1#2#3{\textbf{#1}(#2)\ #3\par}
% \begin{multicols}{2}
\vysld
% \end{multicols}
\endresults
%}

%\copy0

%\vfil

%\box0
%\eject
%\unvbox0



\end{document}




