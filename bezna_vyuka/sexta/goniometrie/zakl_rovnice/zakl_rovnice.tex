\documentclass[10pt,a4paper]{extarticle}
\usepackage[margin=.8cm,bottom=10mm]{geometry}
%\usepackage[margin=1in]{geometry}
\usepackage[utf8]{inputenc}
\usepackage[IL2]{fontenc}
\usepackage[czech]{babel}
\usepackage{microtype}
\usepackage{amssymb}
\usepackage{amsthm}
\usepackage{amsmath}
\usepackage{xcolor}
\usepackage{graphicx}
\usepackage{wasysym}
\usepackage{multicol}
\usepackage[inline]{enumitem}

\multicolsep=\smallskipamount

\newcommand{\R}{\mathbb{R}}
\newcommand{\N}{\mathbb{N}}

\newcommand{\hint}[1]{{\color{gray}\footnotesize\noindent(Nápověda: #1)}}

\DeclareMathOperator{\tg}{tg}
\DeclareMathOperator{\cotg}{cotg}

\setlist[enumerate]{label={(\alph*)},topsep=\smallskipamount,itemsep=\smallskipamount,parsep=0pt,itemjoin={\quad}}
\setlist[itemize]{topsep=\smallskipamount,noitemsep}

\def\tisk{%
\newbox\shipouthackbox
\pdfpagewidth=2\pdfpagewidth
\let\oldshipout=\shipout
\def\shipout{\afterassignment\zdvojtmp \setbox\shipouthackbox=}%
\def\zdvojtmp{\aftergroup\zdvoj}%
\def\zdvoj{%
    \oldshipout\vbox{\hbox{%
        \copy\shipouthackbox
        \hskip\dimexpr .5\pdfpagewidth-\wd\shipouthackbox\relax
        \box\shipouthackbox
    }}%
}}%

\let\results\newpage
\let\endresults\relax

\def\resultssame{%
    \long\def\results##1\endresults{%
        \vfill
        \noindent\rotatebox{180}{\vbox{##1}}%
    }%
}


\newtheorem*{poz}{Pozorování}

\theoremstyle{definition}
\newtheorem{uloha}{\atr Úloha}
\newtheorem{suloha}[uloha]{\llap{$\star$ }Úloha}
\newtheorem*{bonus}{Bonus}
\newtheorem*{defn}{Definice}

\pagestyle{empty}

\DeclareMathOperator{\arctg}{arctg}

\let\ee\expandafter

\def\vysld{}
\let\printvysl\relax
\let\printalphvysl\relax

\makeatletter
\long\def\vyslplain#1{\ee\ee\ee\gdef\ee\ee\ee\vysld\ee\ee\ee{\ee\vysld\ee\printvysl\ee{\the\c@uloha}{#1}}}
\let\vysl\vyslplain

\def\locvysl#1{\ee\gdef\ee\locvysld\ee{\locvysld\item #1}}
\let\lv\locvysl

\newenvironment{ulohav}[1][]{\begin{uloha}[#1]\gdef\locvysld{\begin{enumerate*}}}{\ee\vyslplain\ee{\locvysld\end{enumerate*}}\end{uloha}}
\def\stitem{\@noitemargtrue\@item[$\star$ \@itemlabel]}

\makeatother

\def\atr{}
\def\basic{\def\atr{\llap{\mdseries$\sun$ }\gdef\atr{}}}
\def\sinterest{\def\atr{\llap{$(\star)$ }\gdef\atr{}}}
\def\interest{\def\atr{\llap{$\star$ }\gdef\atr{}}}
\def\iinterest{\def\atr{\llap{$\star\star$ }\gdef\atr{}}}

\def\lm#1{\lv{$\left\{#1\right\}$}}
% \let\ra\rangle
% \let\la\langle
\catcode`<=\active
\catcode`>=\active
\def<{\left\langle}
\def>{\right\rangle}

\def\st{^\circ}

\begin{document}

% \tisk
% \resultssame

\section*{16. Základní goniometrické rovnice a jejich aplikace}

\begin{ulohav}
Nejzákladnější goniometrické rovnice s \uv{hezkými} výsledky:
\begin{multicols}{3}
\begin{enumerate}
    \item $\cos x = \frac12$ \lm{\pm\frac\pi3 + 2k\pi}
    \item $\sin x = \frac12$ \lm{\frac\pi6+2k\pi; \frac{5\pi}6+2k\pi}
    \item $\sin x = -\frac{\sqrt3}2$ \lm{\frac{4\pi}3+2k\pi; \frac{5\pi}{3}+2k\pi}
    \item $\cos x = 0$ \lm{\frac\pi2 + k\pi}
    \item $\sin x = -1$ \lm{\frac{3\pi}2+2k\pi}
    \item $\cos x = -\frac{\sqrt2}2$ \lm{\pm\frac{3\pi}{4}+2k\pi}
    \item $\tg x = 1$ \lm{\frac\pi4 + k\pi}
    \item $\tg x = -\sqrt3$ \lm{\frac{2\pi}3 + k\pi}
    \item $\cotg x = \frac{\sqrt3}3$ \lm{\frac{\pi}3 + k\pi}
\end{enumerate}
\end{multicols}
\end{ulohav}


\begin{ulohav}
Nejzákladnější goniometrické rovnice s ne tak hezkými výsledky:
\begin{multicols}{3}
\begin{enumerate}
    \item $\cos x = \frac35$ \lm{\pm\arccos\frac35 + 2k\pi}
    \item $\sin x = -\sqrt{\frac13}$ \lm{\arcsin\left(-\sqrt{\frac13}\right)+2k\pi; \pi - \arcsin\left(-\sqrt{\frac13}\right)+2k\pi}
    \item $\tg x = \pi$ \lm{\arctg \pi + k\pi}
\end{enumerate}
\end{multicols}
\end{ulohav}


\begin{ulohav}
Rozličné (ale stále celkem \uv{základní}) goniometrické rovnice:
\begin{multicols}{2}
\begin{enumerate}
    \item $\cos 2x = 0$ \lm{\frac\pi4+\frac{k\pi}2}
    \item $\sin \bigl(x + \frac{7\pi}{6} \bigr) = -\frac{\sqrt2}2$ \lm{\frac{\pi}{12}+2k\pi; \frac{7\pi}{12}+2k\pi}
    \item $\sin \frac13x = \frac45$ \lm{3(\arcsin \frac45 + 2k\pi); 3(\pi-\arcsin \frac45 + 2k\pi)}
    \item $\cos\bigl(-2x + \frac\pi3\bigr) = 1$ \lm{\frac\pi6 + k\pi}
    \item $\bigl(\sin x - \frac12\bigr)\bigl(\sin x - \frac13\bigr)(\sin x - 2)=0$ \lm{\frac\pi6+2k\pi; \frac{5\pi}6+2k\pi; \arcsin \frac13 + 2k\pi; \pi - \arcsin \frac13 + 2k\pi}
    \item $\cos x - \cos^3 x = 0$ \lm{\frac{k\pi}{2}}
    \item $\bigl|\sin x + \frac12 \bigr| = \frac12$ \lm{k\pi; \frac{3\pi}{2}+2k\pi}
    \item $3 \cos^2 x + 4 \cos x = 4$ \lm{\pm \arccos \frac23 + 2k\pi}
\end{enumerate}
\end{multicols}
\end{ulohav}


\begin{ulohav}
Vymyslete goniometrickou rovnici tvaru $\sin(\text{něco}) = \text{něco}$ nebo $\cos(\text{něco}) = \text{něco}$, jejíž množinou řešení bude ta uvedená; u úloh s $\star$ vezměte v potaz, že $x$ v argumentu může něčím násobit, něco k němu přičítat\dots
\begin{multicols}{3}
\begin{enumerate}
    \item $\left\{\frac\pi3+2k\pi; \frac{2\pi}3+2k\pi\right\}$ \lv{$\sin x = \frac{\sqrt3}{2}$}
    \item $\left\{\frac{2\pi}{3}+2k\pi; \frac{4\pi}{3}+2k\pi\right\}$ \lv{$\cos x = -\frac{1}{2}$}
    \item $\left\{3\pi + 2k\pi\right\}$ \lv{$\cos x = -1$}
    \stitem $\left\{\frac\pi4 + 2k\pi\right\}$ \lv{$\cos(x+\frac\pi4) = 1$}
    \stitem $\left\{\frac\pi2 + 4k\pi; \frac{3\pi}2 + 4k\pi\right\}$ \lv{$\sin \frac x2 = \frac{\sqrt2}{2}$}
    \stitem $\left\{-\frac\pi4 + k\pi; \frac{\pi}{12} + k\pi\right\}$ \lv{$\cos\bigl(2x + \frac\pi6\bigr) = \frac12$}
\end{enumerate}
\end{multicols}
\end{ulohav}


\begin{ulohav}
Nalezněte všechna řešení rovnice na zadaném intervalu:
\begin{multicols}{3}
\begin{enumerate}
    \item $\cos x = 0$ na $<0;2\pi>$ \lm{\frac\pi2; \frac{3\pi}2}
    \item $\cos x = 0$ na $<-\pi;\pi>$ \lm{\pm\frac\pi2}
    \item $\sin x = \frac12$ na $<-\frac\pi2; \frac\pi2>$ \lm{\frac\pi6}
    \item $\cos x = -\frac{\sqrt2}2$ na $<-\pi;0>$ \lm{-\frac{3\pi}4}
    \item $\sin x = 1$ na $<0; 4\pi>$ \lm{\frac\pi2; \textcolor{red}{\frac{5\pi}2}}
    \item $\cos x = \frac14$ na $<0;2\pi>$ \lm{\arccos \frac14; -\arccos\frac14 + 2\pi}
    \item $\sin x = \frac34$ na $<0; 2\pi>$ \lm{\arcsin \frac34; \pi - \arcsin \frac34}
    \item $\sin x = \frac34$ na $<-\pi; \pi>$ \lm{\arcsin \frac34; \pi - \arcsin \frac34}
    \item $\sin x = \frac34$ na $<-2\pi; 0>$ \lm{\arcsin \frac34 -2\pi; -\pi - \arcsin \frac34}
    \item $\mathopen|\cos x\mathclose| = \frac{\sqrt3}2$ na $<0;2\pi>$ \lm{\frac\pi6; \frac{5\pi}6; \frac{7\pi}6; \frac{11\pi}{6}}
    \item $\sin 2x = \frac12$ na $<0; 2\pi>$ \lm{\frac\pi{12}; \frac{5\pi}{12}; \frac{13\pi}{12}; \frac{17\pi}{12}}
    \item $\cos \frac x3 = -\frac12$ na $<0; 2\pi>$ \lm{2\pi}
    \item $\sin \bigl(x - \frac{2\pi}3\bigr) = 0$ na $<\pi; 2\pi>$ \lm{\frac{5\pi}3}
    \item $\sin \bigl(2x + \frac\pi4\bigr) = \frac{\sqrt{2}}2$ \\ na $<-\pi;\pi>$ \lm{-\pi; -\frac{3\pi}{4}; 0; \frac\pi4; \pi}
\end{enumerate}
\end{multicols}
\end{ulohav}


\begin{ulohav}
Na pružině je zavěšeno závaží, které kmitá nahoru a dolů tak, že vzdálenost nejvyššího a nejnižšího bodu je 10\,cm. Jeden celý kmit (úplně nahoru\,+\,úplně dolů) trvá 1\,s. V čase $t = 0$ se závaží nacházelo přesně uprostřed a pohybovalo se směrem nahoru.
\begin{enumerate}
    \item Uveďte předpis funkce, která udává závislost vertikální polohy závaží $y$ na čase $t$; nechť výchozí (prostřední) poloze odpovídá hodnota $y = 0$, nejvyšší $y = 5$ a nejnižší $y = -5$. \lv{$y = 5\sin(2\pi t)$}
    \item Určete, kde (tj. $y$) se bude závaží nacházet v $t = 0{,}75$. \lv{$5\sin \frac{3\pi}{2} = -5$, tj. úplně dole}
    \item Určete, kde se bude závaží nacházet v $t = \frac13$. \lv{$5\sin \frac{2\pi}{3} = \frac{5\sqrt3}2$}
    \item Do výšky $y = 1$ umístíme senzor, který zaznamenává, že se tam závaží objeví. V jaký čas se tam závaží objeví poprvé, podruhé a potřetí?
        \lv{poprvé $\frac1{2\pi}\arcsin\frac15 \doteq 0{,}032\,\mathrm{s}$, podruhé $\frac1{2\pi}(\pi-\arcsin\frac15) \doteq 0{,}468\,\mathrm{s}$, potřetí o 1\,s později než poprvé, tj. cca $1{,}032\,\mathrm{s}$}
\end{enumerate}
\end{ulohav}


\begin{uloha}
Vzdálenost Země od Slunce v průběhu roku je dána přibližným vztahem $d = 1-0{,}01672 \cos\bigl(0{,}0172(t-4)\bigr)$, kde $t$ je čas od začátku roku v dnech a $d$ je ona vzdálenost v astronomických jednotkách. Určete, kdy během roku je Země vzdálená $1{,}01$\,AU od Slunce.\vysl{cca 132. den a 240. den}
\end{uloha}



\interest
\begin{uloha}
Rozmyslete si, jak budou vypadat grafy následujících funkcí; výsledky si můžete ověřit např. v GeoGebře.
\begin{multicols}{4}
\begin{enumerate}
    \item $y = \cos(\arccos x)$
    \item $y = \arccos(\cos x)$
    \item $y = \sin(\arcsin x)$
    \item $y = \arcsin(\sin x)$
    \item $y = \arccos(\sin x)$
    \item $y = \arcsin(\cos x)$
    \item $y = \cos(\arcsin x)$
    \item $y = \sin(\arccos x)$
\end{enumerate}
\end{multicols}
\end{uloha}


\results
\parindent=0pt
\parskip=\smallskipamount
\rightskip=0pt plus1fil\relax
\def\printvysl#1#2{\textbf{#1.} #2\par}
\vysld
\endresults


\end{document}